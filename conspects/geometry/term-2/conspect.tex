\input{preamble.tex}

\begin{document}
	\Header

	\BeginConspect

	\Section{Аналитическая геометрия}{}{Илья Дудников}

	\Subsection{Системы координат}
	
	\Subsubsection{Аффинные системы координат}

	\begin{Def}
		Аффинной системой координат на прямой называется взаимно-однозначное соответствие $l \longleftrightarrow \R$. \\
		
		\begin{figure*}[h]
			\centering
			\def\svgwidth{0.3\columnwidth}
			\input{img/as_line.pdf_tex}
		\end{figure*}
		
		Она определяется выбором точки $O$  и выбором вектора $\overline{e}$. АСК = $\{O, \{\overline{e}\}\}$.
	\end{Def}

	\begin{Def}
		АСК на плоскости называется биекция $\pi \longleftrightarrow \R^2$. 

		\begin{figure*}[h]
			\centering
			\def\svgwidth{0.3\columnwidth}
			\input{img/as_plane.pdf_tex}
		\end{figure*}

		Она определяется выбором точки $O$ и векторов $\overline{e}_1, \overline{e}_2 \neq \overline{e}, \overline{e}_1 \not\parallel \overline{e}_2$. 
		АСК = $\{O, \{\overline{e}_1, \overline{e}_2\}\}$.
	\end{Def}

	\begin{Def}
		Если $|\overline{e}_1| = |\overline{e}_2| = 1, \overline{e}_1 \perp \overline{e}_2$, то АСК называется декартовой системой координат. 
	\end{Def}

	\begin{Def}
		АСК в пространстве называется биекция $M \longleftrightarrow \R^3$ .
		Она определяется выбором точки $O$ и векторов $\overline{e}_1, \overline{e}_2, \overline{e}_3 \neq \overline{0}$ -- не компланарны. АСК = $\{O, \{\overline{e}_1, \overline{e}_2, \overline{e}_3\}\}$.
	\end{Def}

	\begin{Def}
		Упорядоченная тройка векторов $(\overline{u}, \overline{v}, \overline{w})$ называется \textbf{правой} , 
		если из конца вектор $\overline{w}$ поворот то $\overline{u}$ к $\overline{v}$ по наименьшему углу выглядит происходящим против часовой стрелки,
		и \textbf{левой} -- в противном случае. 
	\end{Def}

	\Pagebreak
	\Subsubsection{Криволинейные системы координаты}

	\begin{Def}
		Выберем точку $O$ и построим из неё луч $p$, который назовем \textit{полярной осью}. Возьмем теперь произвольную точку $M$ на плоскости и измерим две величины:
		расстояние от $M$ до $O$ и угол между вектором $\overline{OM}$ и полярной осью. Обозначим расстояние за $r$, а угол за $\PHI$.
		Тогда, чтобы избежать неоднозначности, будем считать, что $r > 0, \PHI \in [0, 2\pi)$, и если $r = 0$, то $\PHI = 0$.    
		Такая система координат называется \textbf{полярной}.		

		\begin{figure*}[h]
			\centering
			\def\svgwidth{0.3\columnwidth}
			\input{img/polar_system.pdf_tex}
		\end{figure*}
	\end{Def}

	\begin{Def}
		Полярная система координат, где $r \in \R, \PHI \in \R$, то она называется \textit{обобщенной} полярной системой координат.
	\end{Def}

	\begin{figure*}[h!]
		\centering
		\def\svgwidth{0.3\columnwidth}
		\input{img/coordinate_net.pdf_tex}
		\caption{Координатная сеть полярной системы координат}
	\end{figure*}

	\begin{Def}
		Цилиндрической системой координат называют трёхмерную систему координат, являющуюся расширением полярной системы координат путём добавления третьей координаты (обычно обозначаемой ${\displaystyle z}$), которая задаёт высоту точки над плоскостью.
	\end{Def}

	\begin{Def}
		Сферическая система координат — трёхмерная система координат, в которой каждая точка пространства определяется тремя числами, где r — расстояние до начала координат, а $\theta$ и $\varphi$ — зенитный и азимутальный углы соответственно.
	\end{Def}

	\Subsubsection{Параметризации}

	Построим декартову систему координат. Теперь возьмем какую-то новую систему координат $x', y', z'$.
	Проведем через $x', y'$ плоскость. Если $z'$ не совпадает с $z$, то эта плоскость пересекает плоскость $(x, y)$ по какой-то прямой.
	Отсчитает от вектора $x$ до этой прямой угол $\PHI$. Угол между $z$ и $z'$ обозначим за $\psi$.
	Теперь, мы можем эту прямую поворачивать вокруг оси $z'$ на угол $\delta$, пока она не совпадет с $x'$.
	
	\begin{figure*}
		\centering
		\def\svgwidth{0.3\columnwidth}
		\input{img/parametrisation.pdf_tex}
	\end{figure*}

	Таким образом, мы совместили исходную систему координат с новой СК. То есть
	мы построили соответствие между ( $\psi, \PHI, \delta$ ).
	
	\Subsection{Понятие вектора}

	Пусть $E$ -- евклидово пространство.
	\begin{Def}
		Закрепленный вектор -- упорядоченная пара точек в евклидовом пространстве.
		Обозначение: $\overrightarrow{AB}$, модуль $|\overrightarrow{AB}|$ -- расстояние между точками $A$ и $B$.
	\end{Def}

	\begin{Def}
		Пусть $\{(A, B), A, B \in E\}$ -- множество закрепленных векторов. Введём на нём отношение равенства:
		$(A, B) = (C, D) \EQ$:
		\begin{MyList}
			\item $|\overrightarrow{AB}| = |\overrightarrow{CD}|$ 
			\item $(A, B) || (C, D)$ либо совпадают.
			\item $\overrightarrow{AB} \upuparrows \overrightarrow{CD}$. 
		\end{MyList}
	\end{Def}

	\begin{Rem}
		$\forall A, B \to (A, A) = (B, B)$.
	\end{Rem}

	\begin{Prop}
		Отношение, введённое в прошлом определении -- отношение эквивалентности.
	\end{Prop}

	\begin{proof}
		\begin{MyList}
			\item Рефлексивность: $(A, B) = (A, B)$ -- верно.
			\item Симметричность -- очевидно.
			\item Транзитивность: $(A, B) = (C, D), (C, D) = (F, G) \SO (A, B) = (F, G)$ -- верно.
		\end{MyList}
		Значит множество закрепленных векторов разбивается на классы эквивалентности.
	\end{proof}

	\begin{Def}
		Класс эквивалентности называется \textbf{свободным вектором}.
	\end{Def}

	\Subsection{Сложение и умножение на число}

	Пусть $\overline{a}, \overline{b} \in V$ -- классы.

	\begin{Def}
		Сложение векторов: $V \times V \to V$.
		$[\overrightarrow{OO''}] = \overline{a} + \overline{b}$ 
	\end{Def}

	\begin{Def}
		Пусть $\overline{a} \in V, \lambda \in \R$. Умножение на число на число: $\R \times V \to V$.
	\end{Def}
	
	$(V, +, \cdot)$. Свойства:
	\begin{MyList}
		\item $\forall \overline{a}, \overline{b} \in V \ \overline{a} + \overline{b} = \overline{b} + \overline{a}$.
		\item $\forall \overline{a}, \overline{b}, \overline{c} \in V \ (\overline{a} + \overline{b}) + \overline{c} = \overline{a} + (\overline{b} + \overline{c})$.
		\item $\exists \overline{0} : \forall \overline{a} \ \overline{a} + \overline{0} = \overline{0} + \overline{a} = \overline{a}$.
		\item $\forall \overline{a} \ \exists -\overline{a} : \overline{a} + (-\overline{a}) = \overline{0}$.
		\item $\forall \lambda \in \R, \overline{a}, \overline{b} \in V \ \lambda(\overline{a} + \overline{b}) = \lambda \overline{a} + \lambda \overline{b}$.
		\item $\forall \lambda, \mu \in \R, \overline{a} \in V \ (\lambda + \mu) \overline{a} = \lambda \overline{a} + \mu \overline{a}$.
		\item $\forall \overline{a} \in V \ 1 \cdot \overline{a} = \overline{a}$.
		\item $\forall \lambda, \mu \in \R, \overline{a} \in V \ \lambda(\mu \overline{a}) = (\lambda \mu) \overline{a}$.
	\end{MyList} 

	\begin{Def}
		Множество $(V, +, \cdot)$, удовлетворяющее свойствам 1-8, называется \textbf{векторным пространством}. Элементы -- векторы.
	\end{Def}

	\Subsection{ЛЗ, ЛНЗ, Базис, размерность}

	\begin{Def}
		$\lambda_1 \overline{a}_1 + ... + \lambda_n \overline{a}_n$ -- линейная комбинация. Если $(\lambda_1, ..., \lambda_n) \neq (0, ..., 0)$ -- нетривиальная ЛК.  
	\end{Def}

	\begin{Def}
		$\{\overline{a}_i\}_{i = 1}^n$ -- линейно зависимый, если $\exists$ нетривиальная ЛК $\{\lambda_i\}_{i = 1}^n : \sum_{i = 1}^n \lambda_i \overline{a}_i = 0$   
	\end{Def}

	\begin{Def}
		$\{\overline{a}_i\}_{i = 1}^n$ -- ЛНЗ, если он не ЛЗ. 
	\end{Def}

	Свойства:
	\begin{MyList}
		\item $\{\overline{a} \neq \overline{0}\}$ -- ЛНЗ.
		\item $\{\overline{0}\}$ -- ЛЗ.
		\item $\{\overline{a_1}, ..., \overline{a}_n, \overline{0}\}$ -- ЛЗ.
		\item Пусть $\{\overline{a}_i\}$ -- ЛЗ. Тогда $\{\overline{a}_i, \overline{a}_j\}_{i = 1, j = 1}^{n, m}$ -- ЛЗ.
	\end{MyList}

	\begin{Def}
		$\{\overline{a}_\alpha\}_{\alpha \in \Lambda}$ -- ЛЗ, если в нем $\exists$ ЛЗ конечный поднабор.
	\end{Def}

	\begin{Def}
		ЛНЗ -- набор, который не является ЛЗ.
	\end{Def}

	\begin{Def}
		$\{\overline{a}_\alpha\}_{\alpha \in \Lambda}$ -- полный, если $\forall \overline{v} \in V \ \exists \{\alpha_i\}_{i = 1}^n, \{\lambda_i\}_{i = 1}^n \ \overline{v} = \lambda_1 \overline{a}_{\alpha_1} + ... + \lambda_n \overline{a}_{\alpha_n}$.
	\end{Def}

	\begin{Def}
		$\{\overline{a}_\alpha\}_{\alpha \in \Lambda}$ -- базис $V$, если он полный и ЛНЗ.
	\end{Def}

	\begin{Def}
		Размерность $V$ ( $\dim V$ ) -- мощность базиса.
	\end{Def}

	\begin{Def}
		Векторное пространство $V$ называется конечномерным, если $\exists$ конечный полный набор.
	\end{Def}

    \gdef\AuthorName{Дарья Гольденберг}
	\Subsection{Скалярное умножение}

	Будем определять скалярное произведение для элементов векторного пространства $V$.
	
	\begin{Def}
	  $(\overline{a}, \overline{b})$ -- скалярное произведение: $V \times V \to \R$
	\end{Def}
	
	Свойства:
	\begin{MyList}
		\item Свойства 1-8, необходимые для существования векторного пространства.
		\item $\forall \overline{a} \in V \ (\overline{a}, \overline{a}) \geqslant 0$ -- положительная определённость. \\
		Кроме того, $(\overline{a}, \overline{a}) = 0 \Leftrightarrow \overline{a} = \overline{0}$ -- невырожденность.
		\item $\forall \overline{a}, \overline{b}, \overline{c} \in V \ (\overline{a} + \overline{b}, \overline{c}) = (\overline{a}, \overline{c}) + (\overline{b}, \overline{c})$ -- аддитивность.\\
		$\forall \lambda \in \R, \overline{a}, \overline{b} \in V \ (\lambda \overline{a}, \overline{b}) = \lambda(\overline{a}, \overline{b})$ -- однородность.
		\item $\forall \overline{a}, \overline{b} \in V \ (\overline{a}, \overline{b}) = (\overline{b}, \overline{a})$. -- коммутативность.
	\end{MyList}
  
	\begin{Example}
	  $\R^n = \{ (x_1, x_2, ..., x_n): x_i \in \R \}$ \\
	  $\overline{v} = (x_1, ..., x_n), \overline{w} = (y_1, ..., y_n)$ \\
	  Тогда скалярное произведение: $(\overline{v}, \overline{w}) = x_1 y_1 + ... + x_n y_n.$ \\
	  Проверим свойства: 
	  \begin{MyList}
		\item $(\overline{v}, \overline{v}) = x_1^2 + ... + x_n^2 \geqslant 0$.\\
		$(\overline{v}, \overline{v}) = 0 \Leftrightarrow \forall i \ x_i = 0$.
		\item Пусть $ \overline{z} = (z_1, ..., z_n)$, тогда $(\overline{v} + \overline{w}, \overline{z}) = (x_1 + y_1)z_1 + ... + (x_n + y_n)z_n = x_1 z_1 + ... + x_n y_z + y_1 z_1 + ... + y_n z_n = (\overline{v}, \overline{z}) + (\overline{w}, \overline{z})$.\\
		$(\lambda \overline{v}, \overline{w}) = \lambda x_1 y_1 + ... + \lambda x_n y_n = \lambda (x_1 y_1 + ... + x_n y_n) = \lambda(\overline{v}, \overline{w})$.
		\item $(\overline{v}, \overline{w}) = x_1 y_1 + ... + x_n y_n = y_1 x_1 + ... + y_n x_n = (\overline{w}, \overline{v})$.
	  \end{MyList}
	\end{Example}
  
	\begin{Example}
	  $C[0, 1]$ -- непрерывные функции на отрезке $[0, 1]$.\\
	  Пусть $f, g, q \in C[0, 1]$ -- функции: $(f, g) = \int_0^1 fg \,dx$.
	  \begin{MyList}
		\item $(f, f) = \int_0^1 f^2 \,dx \geqslant 0. \\
		(f, f) = 0  \Leftrightarrow f = 0$.
		\item $(f + q, g) = \int_0^1 (f + q)g \,dx = \int_0^1 (fg + qg) \,dx = \int_0^1 fg \,dx + \int_0^1 qg \,dx = (f, g) + (q, g)$. \\
		$(\lambda f, g) =\int_0^1 \lambda fg \,dx  = \lambda \int_0^1 fg \,dx = \lambda(f, g)$.
		\item $(f, g) =\int_0^1 fg \,dx  = \int_0^1 gf \,dx = (g, f)$.
	  \end{MyList}
	  Таким образом, это скалярное произведение непрерывных на $[0, 1]$ функций.
	\end{Example}
  
	Пусть есть конечномерное векторное пространство $V$, на нём задано скалярное произведение $(,)$, выберем базис векторного пространства $\{\overline{e}_i\}$, рассмотрим векторы $\overline{v} = (x_i), \overline{w} = (y_i)$, 
	тогда их скалярное произведение $(\overline{v}, \overline{w}) = (x_1 \overline{e}_1 + ... + x_n \overline{e}_n, y_1 \overline{e}_1 + ... + y_n\overline{e}_n)$, т.е.
	$$(\overline{v}, \overline{w}) = \sum_{i, j}^n x_i y_j (\overline{e}_i, \overline{e}_j)$$
	Либо же запись вида:
	\[(\overline{v}, \overline{w}) = \left(\begin{array}{cccc}
	  x_1 & x_2 & \cdots & x_n
	  \end{array}\right)
	   \left(\begin{array}{cccc}
	  (\overline{e}_1, \overline{e}_1) & (\overline{e}_1, \overline{e}_2) & \cdots & (\overline{e}_1, \overline{e}_n) \\ 
	  \vdots & \vdots & \ddots & \vdots \\
	  (\overline{e}_n, \overline{e}_1) & (\overline{e}_n, \overline{e}_2) & \cdots & (\overline{e}_n, \overline{e}_n)
	  \end{array}\right) 
	  \left(\begin{array}{c}
		  y_1 \\ 
		  y_2 \\ 
		  \vdots \\ 
		  y_n
		  \end{array}\right)\] 
	  где $G = ((\overline{e}_i, \overline{e}_j)), {1 \leqslant i \leqslant j \leqslant n}$ -- матрица Грама сколярного произведения.
	  
	  Тогда скалярное произведение можно записать в следующем виде: $(\overline{v}, \overline{w}) = \overline{v}^T G \overline{w}$.
  
	  В силу коммутативности скалярного произведения $G^T = G$.
	  
	  \begin{Thm}[Критерий Сильвестра]
		 \[\forall k = 1,..., n \ \ \det(G_k) > 0\]
		 где $G_n$ -- миноры главной диагонали.
	  \end{Thm}
  
	  \begin{Prop}
		  Если взять $R^n, G$, то $G$ -- матрица Грама $\Leftrightarrow G^T = G$, которая удовлетворяет критерию Сильвестра.
	  \end{Prop}
  
	  \begin{Def}
		  Если базис обладает свойством:
		  $(e_i, e_j) = \begin{cases}
		  1, i \neq j\\
		  1, i = j
	  \end{cases} \Rightarrow G = E$,
	  тогда он называется ортонормированным базисом (ОРБ).
	  \end{Def} 
  
	  \begin{Thm}[Теорема Грама-Шмидта]
		  В $\forall V^n$ со скалярным произведением $(,) \ \exists$ ОНБ. 
	  \end{Thm}
  
	  \begin{Def}
		  $V$ -- векторное пространство, $(,)$ -- скалярное произведение на нём, тогда \textbf{модуль} (\textbf{длина}) $|\overline{a}| = \sqrt{(\overline{a}, \overline{a})}, |\overline{a}| = 0 \Leftrightarrow \overline{a} = 0$.
	  \end{Def}
	  
	  \begin{Def}
		  Величина угла между векторами -- число $\alpha \in [0; \pi] \in R: \cos \alpha = \frac{(\overline{a}, \overline{b})}{|\overline{a}||\overline{b}|}, \overline{a} \neq 0, \overline{b} \neq 0$.
	  \end{Def}
  
	  \begin{Thm}[Неравенство Коши-Буняковского]
		  $$(\overline{a}, \overline{b})^2 \leqslant \overline{a}^2 \overline{b}^2$$
	  \end{Thm}
  
	  \begin{proof}
		  По свойству скалярного произведения $(\overline{a} + t\overline{b})^2$ всегда невырожденная величина, т.е. $(\overline{a} + t\overline{b})^2 \geqslant 0 \Rightarrow \overline{a}^2 + 2t(\overline{a}, \overline{b}) + t^2 \overline{b}^2 \geqslant 0$, 
		  тогда его дискриминант не положительный, т.к. $t$ -- любое число, то $$(\overline{a}, \overline{b})^2 - \overline{a}^2 \overline{b}^2 \leqslant 0 \Rightarrow (\overline{a}, \overline{b})^2 \leqslant \overline{a}^2 \overline{b}^2$$.
	  \end{proof}
  
	  \Subsection{Векторное умножение}
  
	  Векторное умножение определяется только для трёхмерного пространства $V^3$, кроме того, необходимо, чтобы пространство было ориентированным, выберем в нём правый ОНБ $(\overline{i}, \overline{j}, \overline{k})$.
	  
	  \begin{figure*}[h]
		  \centering
		  \def\svgwidth{0.2\columnwidth}
		  \input{img/right-oriented.pdf_tex}
	  \end{figure*}
	  \begin{Def}
		  Пусть $\overline{v} = (x_1, x_2, x_3), \overline{w} = (y_1, y_2, y_3)$. Тогда \textbf{векторное произедение} \\
		  $\overline{v} \times \overline{w} = \left|\begin{array}{cccc}
			  \overline{i} & \overline{j} & \overline{k} \\ 
			  x_1 & x_2 & x_3 \\ 
			  y_1 & y_2 & y_3
			  \end{array}\right| = \overline{i}(x_2 y_3 - x_3y_2) - \overline{j}(x_3 y_1 - x_1 y_3) + \overline{k}(x_1 y_2 - x_2 y_1)$.
	  \end{Def}
	  
	  Свойства:
	  \begin{MyList}
		  \item $\overline{v} \times \overline{w} = - \overline{w} \times \overline{v}$ -- косокоммутативность.
		  
		  \item $\overline{v} \times \overline{v} = \overline{0}$.
		  
		  \item $(\overline{v} + \overline{w}) \times \overline{z} = 
			  \left|\begin{array}{cccc}
			  \overline{i} & \overline{j} & \overline{k} \\ 
			  x_1 + y_1 & x_2 + y_2 & x_3 + y_3 \\ 
			  z_1 & z_2 & z_3
			  \end{array}\right| =
			  \left|\begin{array}{cccc}
			  \overline{i} & \overline{j} & \overline{k} \\ 
			  x_1 & x_2 & x_3 \\ 
			  z_1 & z_2 & z_3
			  \end{array}\right| +
			  \left|\begin{array}{cccc}
			  \overline{i} & \overline{j} & \overline{k} \\ 
			  y_1 & y_2 & y_3 \\ 
			  z_1 & z_2 & z_3
			  \end{array}\right| = \overline{v} \times \overline{z} + \overline{w} \times \overline{z}$ -- аддитивность.
		  
		  \item $(\lambda \overline{v}) \times \overline{w} = 
			  \left|\begin{array}{cccc}
			  \overline{i} & \overline{j} & \overline{k} \\ 
			  \lambda x_1 & \lambda x_2 & \lambda x_3 \\ 
			  y_1 & y_2 & y_3
			  \end{array}\right| = \lambda \overline{v} \times \overline{w}$.
  
		  \item $\overline{v} \times \overline{w} \perp \overline{v}, \overline{w}\\
		  (\overline{v}, \overline{v} \times \overline{w}) = 
			  (\left|\begin{array}{cccc}
			  \overline{i} & \overline{j} & \overline{k} \\ 
			  x_1 & x_2 & x_3 \\ 
			  y_1 & y_2 & y_3
			  \end{array}\right|, \overline{v}) = 
			  \left|\begin{array}{cccc}
				  x_1 &  x_2 & x_3 \\ 
				  x_1 &  x_2 & x_3 \\ 
				  y_1 & y_2 & y_3
				  \end{array}\right| = 0$.
		  \item $\overline{v} \times \overline{w} = 0 \Leftrightarrow \overline{v} \parallel \overline{w} \\
		  (\overline{v}, \overline{w}, \overline{v} \times \overline{w}) =
		  \left|\begin{array}{cccc}
			  x_1 & x_2 & x_3 \\ 
			  y_1 & y_2 & y_3 \\ 
			  (x_2 y_3 - x_3 y_2) & (x_3 y_1 - x_1 y_3) & (x_1 y_2 - x_2 y_1)
			  \end{array}\right| = (x_2 y_3 - x_3 y_2)^2 + (x_3 y_1 - x_1 y_3)^2 + (x_1 y_2 - x_2 y_1)^2 \geqslant 0 \Rightarrow (\overline{v}, \overline{w}, \overline{v} \times \overline{w}) = 0 \Leftrightarrow \frac{x_2}{y_2} = \frac{x_3}{y_3}, \frac{x_3}{y_3} = \frac{x_1}{y_1}, \frac{x_1}{y_1} = \frac{x_2}{y_2} \Rightarrow \frac{x_1}{y_1} = \frac{x_2}{y_2} = \frac{x_3}{y_3}$.
		  \item $\overline{v} \nparallel \overline{w}  \Rightarrow (\overline{v}, \overline{w}, \overline{v} \times \overline{w})$ -- правая.
		  \item $\overline{i} \times \overline{j} = \overline{k}$. Получим таблицу умножения: $i \to j, j \to k, k \to i$.
		  \item $(\overline{a} \times \overline{b})^2 = (x_2 y_3 - y_2 x_3)^2 + (...)^2 + (...)^2 \\
		  \overline{a}^2 \overline{b}^2 - (\overline{a}, \overline{b})^2 = (x_1^2 + x_2^2 + x_3^2)(y_1^2 + y_2^2 + y_3^2) - (x_1y_1 +x_2y_2 +x_3y_3)^2 \\
		  (\overline{a} \times \overline{b})^2 = \overline{a}^2 \overline{b}^2 - (\overline{a}, \overline{b})^2$ -- упражнение.\\
		  $\overline{a}^2 \overline{b}^2 - (\overline{a}, \overline{b})^2 = S^2 = |\overline{a}|^2 |\overline{b}|^2 \sin^2\alpha = |\overline{a}|^2 \overline{b}|^2 (1 - \cos^2 \alpha) = |\overline{a}|^2 |\overline{b}|^2 - (\overline{a}, \overline{b})^2$, т.к. $\cos^2 \alpha = \frac{(a, b)^2}{|a|^2|b|^2}$.\\
	  \begin{figure*}[h]
		  \centering
		  \def\svgwidth{0.3\columnwidth}
		  \input{img/liner_transformation.pdf_tex}
	  \end{figure*}
	  Следствие: $|\overline{a}| = |\overline{b}| = 1,  (\overline{a}, \overline{b}) \Rightarrow |\overline{a} \times \overline{b}| = 1$.
  \end{MyList}
  
	Рассмотрим $V^3$, зафиксируем ОНБ $(\overline{i}, \overline{j}, \overline{k})$, зададим векторное произведение $\times: V \times V \to V$.

	Выберем $\overline{a}, \overline{b}, \overline{c}$ -- правый ОНБ, таким образом, если взять любой ОНБ, можно получить таблицу умножения: $a \to b, b \to c, c \to a$.

	$\overline{v} = (\lambda_1, \lambda_2, \lambda_3), \ \overline{w} = (\mu_1, \mu_2, \mu_3) \Rightarrow \overline{v} \times \overline{w} = (\lambda_1 \overline{a} + \lambda_2 \overline{b} + \lambda_3 \overline{c})\times (\mu_1 \overline{a} + \mu_2 \overline{b} + \mu_3 \overline{c}) = \\
	= \lambda_1 \mu_1 \overline{a} \times \overline{a} + \lambda_1 \mu_2 \overline{a} \times \overline{b} + \lambda_1 \mu_3 \overline{a} \times \overline{c}
	+ \lambda_2 \mu_1 \overline{b} \times \overline{a} + \lambda_2 \mu_2 \overline{b} \times \overline{b} + \lambda_2 \mu_3 \overline{b} \times \overline{c}
	+ \lambda_3 \mu_1 \overline{c} \times \overline{a} + \lambda_3 \mu_2 \overline{c} \times \overline{b} + \lambda_3 \mu_3 \overline{c} \times \overline{c} = \\
	= \overline{c}(\lambda_1 \mu_2 - \lambda_2 \mu_1) - \overline{b}(\lambda_1 \mu_3 - \lambda_3 \mu_1) + \overline{a}(\lambda_2 \mu_3 - \lambda_3 \mu_2) =  \left|\begin{array}{cccc}
		\overline{a} & \overline{b} & \overline{c} \\ 
		\lambda_1 & \lambda_2 & \lambda_3 \\ 
		\mu_1 & \mu_2 & \mu_3
		\end{array}\right|$.  
	
	\Subsection{Смешанное умножение}

	Рассмотрим трёхмерное ориентированное векторное пространство $V^3$, в котором зафиксирован базис $\{\overline{i}, \overline{j}, \overline{k}\}$ и на котором задано векторное умножение: $\times$.

	Зададим новую операцию -- смешанное произведение $(,,): V \times V \times V \to \R$. 

	\begin{Def}
		Смешанное произведение трёх векторов $\overline{a}, \overline{b}, \overline{c}$ -- скалярное произведение вектора $\overline{a}$ и векторного произведения векторов $\overline{b}$ и $\overline{c}$. $(\overline{a}, \overline{b}, \overline{c}) = (\overline{a}, \overline{b} \times \overline{c}).$
	\end{Def}

	\begin{Example}
		Что подразумевает под собой смешанное произведение?
		
		Рассмотрим $\overline{a} = (a_1, a_2, a_3), \overline{b} = (b_1, b_2, b_3), \overline{c} = (c_1, c_2, c_3) $.
		
		$\overline{b} \times \overline{c} = 
		\left|\begin{array}{cccc}
			\overline{i}  & \overline{j}  & \overline{k}  \\ 
			b_1 &  b_2 & b_3 \\ 
			c_1 & c_2 & c_3
			\end{array}\right| = \overline{i} 
		\left|\begin{array}{cccc}
			b_2 & b_3 \\ 
			c_2 & c_3
			\end{array}\right| - \overline{j}
		\left|\begin{array}{cccc}
			b_1 & b_3 \\ 
			c_1 & c_3
			\end{array}\right| + \overline{k}
		\left|\begin{array}{cccc}
			b_1 & b_2 \\ 
			c_1 & c_2
			\end{array}\right| $
		
			$(\overline{a}, \overline{b} \times \overline{c}) = a_1
		\left|\begin{array}{cccc}
		b_2 & b_3 \\ 
		c_2 & c_3
		\end{array}\right| - 
		a_2 \left|\begin{array}{cccc}
			b_1 & b_3 \\ 
			c_1 & c_3
			\end{array}\right| + 
		a_3  \left|\begin{array}{cccc}
			b_1 & b_2 \\ 
			c_1 & c_2
			\end{array}\right| = 
		\left|\begin{array}{cccc}
			a_1  & a_2  & a_3  \\ 
			b_1 &  b_2 & b_3 \\ 
			c_1 & c_2 & c_3
			\end{array}\right| = (\overline{a}, \overline{b}, \overline{c})$

		Таким образом, $(\overline{a}, \overline{b}, \overline{c}) = \left|\begin{array}{cccc}
			a_1  & a_2  & a_3  \\ 
			b_1 &  b_2 & b_3 \\ 
			c_1 & c_2 & c_3
			\end{array}\right|$
	\end{Example}

	Свойства смешанного произведения:
	\begin{MyList}
		\item Линейность по каждому аргументу, как композиция линейных отображений.
		\item $(\overline{a}, \overline{b}, \overline{c}) = - (\overline{a}, \overline{c}, \overline{b})$ по свойству векторного произведения.
		
			  $(\overline{a}, \overline{b}, \overline{c}) = - (\overline{b}, \overline{a}, \overline{c}); (\overline{a}, \overline{b}, \overline{c}) = (\overline{b}, \overline{c}, \overline{a})$.

			  Знак меняется в зависимости от чётности перестановки в силу свойств определителя.
		\item Геометрический смысл для трёх некомпланарных.

			Если $\overline{a}$ сходит туда же, куда и $\overline{b} \times \overline{c}$, то $(\overline{a}, \overline{b}, \overline{c})$ -- правая тройка. Если $\overline{a}$ "смотрит" в другую плоскость, то тройка -- левая.

			$\overline{b} \times \overline{c} = S$, тогда $(\overline{a}, \overline{b} \times \overline{c}) = S \underbrace{|\overline{a}| \cos \alpha}_{h} = S\cdot h$.
			Таким образом, $(\overline{a}, \overline{b}, \overline{c}) = S\cdot h = V_{\text{пар}}$. 

			Вывод: смешанное произведение равно $\pm V$ параллелепипеда (знак зависит от ориентации).
	\end{MyList}
	
	\begin{figure*}[h]
		\centering
		\def\svgwidth{0.5\columnwidth}
		\input{img/triple-product.pdf_tex}
	\end{figure*}

	\Subsection{Двойное векторное умножение. Тождество Якоби}

	Рассмотрим ориентированное $V^3$ и $\overline{a} \times (\overline{b} \times \overline{c})$, выражение имеет смысл, поскольку и $\overline{a}$ -- вектор, и $(\overline{b} \times \overline{c})$ -- вектор. 

	\begin{Prop}[Формула "бац минус цаб"]
		$$\overline{a} \times (\overline{b} \times \overline{c}) = \overline{b}(\overline{a}, \overline{c}) - \overline{c}(\overline{a}, \overline{b}).$$
	\end{Prop}
		
	\begin{proof}
		И справа, и слева знака равенства линейные выражения. Представим, что есть функция с операцией из трёх аргументов $f(\overline{a}, \overline{b}, \overline{c})$, где каждый из векторов может быть расписан по базису, т.е. $f(\overline{a}, \overline{b}, \overline{c}) = f(a_1 \overline{i} + a_2 \overline{j} +a_3 \overline{k}, b_1 \overline{i} + b_2 \overline{j} + b_3 \overline{k}, c_1 \overline{i} + c_2 \overline{j} + c_3 \overline{k})
		= a_1 b_1 c_1 f(\overline{i}, \overline{i}, \overline{i}) + a_1 b_1 c_2 f(\overline{i}, \overline{i}, \overline{j}) + ... + a_3 b_3 c_3 f(\overline{i}, \overline{j}, \overline{k})$, в силу линейности мы вынесли числа за скобки и получили все возможные наборы базисных элементов. В качестве первого, второго и третьего аргумента может быть один из трёх векторов: $\overline{i}, \overline{j}, \overline{k}$, т.е. 27 базисных наборов, в данном выражении 27 слагаемых. 

		Чтобы вычислить значение трилинейной функции на каком-то наборе векторов, достаточно знать координаты этих векторов и значения отображения функции 27 базисных наборов.
		Тогда для доказательства выражения достаточно проверить, совпадают ли две трилинейные функции на 27 базисных наборах, значит, они совпадают везде.\\
		Проверим для базисного набора "$\overline{i}, \overline{i}, \overline{i}$": $\overline{i} \times \underbrace{(\overline{i} \times \overline{i})}_{0} = \overline{i} \underbrace{(\overline{i}, \overline{i})}_{1} - \overline{i}\underbrace{(\overline{i}, \overline{i})}_{1}$\\
		Для набора "$\overline{i}, \overline{i}, \overline{j}$": $\overline{i} \times (\overline{i} \times \overline{j}) = \overline{i}(\overline{i}, \overline{j}) - \overline{j}(\overline{i}, \overline{i})$, используем таблицу умножения: $i \to j, j \to k, k \to i$, тогда $\overline{i} \times \underbrace{(\overline{i} \times j)}_{\overline{i} \times \overline{k} = - \overline{j} } = \overline{i}\underbrace{(\overline{i}, \overline{j})}_0 - \overline{j}\underbrace{(\overline{i}, \overline{i})}_1\\$
		Аналогично для остальных 25 базисных наборов.
	\end{proof}

	\begin{Thm}[Тождество Якоби]
		$$\overline{a} \times (\overline{b} \times \overline{c}) + \overline{b}\times(\overline{c} \times \overline{a}) + \overline{c} \times (\overline{a}\times \overline{b}) = 0$$
	\end{Thm}

	\begin{proof}
		$\overline{b}(\overline{a},\overline{c}) - \overline{c}(\overline{a}, \overline{b}) + \overline{c}(\overline{a}, \overline{b}) - \overline{a}(\overline{b}, \overline{c}) + \overline{a}(\overline{c}, \overline{b}) - \overline{b}(\overline{a}, \overline{c}) = 0$
	\end{proof}

	Пусть $V$ -- векторное пространство, на нём есть бинарная операция $[,]: V \times V \to V$, которая обладает свойствами:
	\begin{MyList}
		\item Билинейность
		\item Косокоммутативность: $[\overline{a}, \overline{b}] = -[\overline{b}, \overline{a}]$
		\item Удовлетворяет тождеству Якоби: $[\overline{a}, [\overline{b}, \overline{c}]] + [\overline{b}, [\overline{c}, \overline{a}]] + [\overline{c}, [\overline{a}, \overline{b}]] = 0$
	\end{MyList}
	
	\begin{Def}
		Если выполняются все свойства, то операция $[,]$ называется скобка Ли, а векторное пространство $(V, [,])$ -- алгебра Ли.
	\end{Def}

	\Subsection{Уравнение прямой на плоскости}

	Возьмём точку $M$ и вектор $\overline{v}$, который отложим от точки $M$, проведём прямую, кторая содержит эти два объекта. Отметим точку $M'$, которая записывается как $M' = M + t \overline{v}$.  

	\begin{Def}
		$M' = M + t \overline{v}$ -- параметрическое задание. 
	\end{Def}

	Выберем начало координат $O$, вектор $\overline{r}_0$, который соответсвует точке $M$ и вектор $\overline{r}$, который соответсвует произольной точке $M'$. Тогда вектор $\overline{r}$ в зависимости от $t$ представляется как $\overline{r}(t) = \overline{r}_0 + t \overline{v}$
	
	\begin{Def}
		$$\overline{r}(t) = \overline{r}_0 + t \overline{v}$$ -- параметрическое уравнение прямой.
	\end{Def}

	Обозначим координаты вектора $\overline{r}$ как $(x, y)$, вектора $\overline{r}_0$ как $(x_0, y_0)$, вектора $\overline{v}$ как $(a, b)$. Тогда запишем это уравнение с каждой координатой.

	\begin{Def}
		$ \begin{cases}
			x = x_0 + ta\\
			y = y_0 + tb
		\end{cases} $ -- параметрическое уравнение в координатах.
	\end{Def}

	\begin{figure*}[h]
		\centering
		\def\svgwidth{0.5\columnwidth}
		\input{img/line_in_Cartesian_coordinates.pdf_tex}
		\caption{Изображение вектора $\overline{v}$ на плоскости}
	\end{figure*}
	Выразим из обоих уравнений $t$, из первого уравнения получаем $t = \frac{x - x_0}{a}$, из второго $t = \frac{y - y_0}{b}$, тогда верно $\frac{x-x_0}{a} = \frac{y - y_0}{b}$.

	\begin{Def}
		$$\frac{x-x_0}{a} = \frac{y - y_0}{b}$$ -- каноническое уравнение прямой на плоскости.
	\end{Def}

	\begin{Example}
		$2x = y - 1$, можно записать данное выражение как: $\frac{x - 0}{1} = \frac{y - 1}{2}$
	\end{Example}

	Пусть нам известны координаты начала вектора $(x_0, y_0)$ и координаты конца $(x_1, y_1)$, тогда можно записать каноническое уравнение прямой в другом виде.

	\begin{figure*}[h]
		\centering
		\def\svgwidth{0.2\columnwidth}
		\input{img/canonical_equation_of_line.pdf_tex}
	\end{figure*}

	\begin{Def}
		$$\frac{x-x_0}{x_1 - x_0} = \frac{y - y_0}{y_1 - y_0}$$ -- каноническое уравнение прямой на плоскости.
	\end{Def}

	Приведём первый вариант канонического уравнения прямой к следующему виду: \\
	$bx - x_0b = ay - ay_0 \Rightarrow bx - ay + (ay_0 -bx_0) = 0$.\\
	Введём обозначения, пусть $A$ -- коэффициент при $x$, $B$ -- коэффициент при $y$, а $C$ -- свободный член.

	\begin{Def}
		$A^2 + B^2 \neq 0$
		$$Ax + By + C = 0$$ -- общее уравнение прямой на плоскости. \\
		$(-B, A)$ -- направляющий вектор.\\
		$(-\frac{C}{A}, 0)$ -- точка.\\
		$(-B, A)$ -- перпендикуляр к прямой (вектор нормали). 
	\end{Def}

	$B \neq 0$, тогда $y = -\frac{A}{B} - \frac{C}{B}$, введём обозначения.
	\begin{Def}
		$$y = kx + b$$ -- уравнение прямой с угловым коэффициентом, где $k = \tg \alpha$.
	\end{Def}

	Пусть вектор $\overline{n}$ с координатами $(A,B)$ перпендикулярен вектору  $\overline{v}$ (рис. 2), т.е. $\overline{MM'} \perp \overline{n})$, тогда $(\overline{n}, \overline{MM'}) = 0$.

	\begin{Def}
		$$(\overline{n}, \overline{r} - \overline{r}_0) = 0$$ -- векторное уравнение прямой на плоскости.
	\end{Def}

	Раскроем скобки: $(\overline{n}, \overline{r}) - (\overline{n} - \overline{n}_0) = 0 \Rightarrow (\overline{n}, \overline{r}) = (\overline{n}, \overline{r}_0)$, обозначим $(\overline{n}, \overline{r}_0) = \alpha$, поскольку векторы $\overline{n}$ и $\overline{r}_0$ зафиксированы.

	\begin{Def}
		$$(\overline{n}, \overline{r}) = \alpha$$ -- векторное уравнение прямой на плоскости, где $\overline{n}$ -- перпендикуляр к исходной прямой.
	\end{Def}

	\Subsection{Уравнение плоскости в пространстве}
	
	Возьмём точку $M_0$ в $\R^3$ и зададим плоскость двумя неколлинеарными векторами $\overline{a}$ и $\overline{b}$, т.е. $\overline{a} \nparallel \overline{b}$. Любая точка $M$ этой плоскости является линейной комбинацией: $M = M_0 + \alpha \overline{a} + \beta \overline{b}$.
	
	\begin{Def}
		$M = M_0 + \alpha \overline{a}  + \beta \overline{b}$ -- параметрическое задание точек пространства $\R^3$.
	\end{Def}
	
	Аналогично, как в уравнении прямой на плоскости, возьмём начало координат $O$ и запишем через параметры радиус-вектор $\overline{r}$ точки $M$.

	\begin{Def}
		$$\overline{r} (\alpha, \beta) = \overline{r}_0 + \alpha \overline{a} + \beta \overline{b}$$ -- параметрическое уравнение плоскости в пространстве.
	\end{Def}

	Выберем систему координат, тогда у точки $M$ будут координаты $(x, y, z)$. Запишем параметрическое уравнение в координатах.

	\begin{Def}
		$ \begin{cases}
			x = x_0 + \alpha a_1 +  \beta b_1\\
			y = y_0 + \alpha a_2 + \beta b_2 \\
			z = z_0 + \alpha a_3 + \beta b_3
		\end{cases} $ -- параметрическое уравнение плоскости в координатах.
	\end{Def}

	Рассмотрим правоориентированное векторное пространство $\R^3$. Если есть вектора $\overline{a}$ и $\overline{b}$, то их векторное произведение -- перпендикуляр $\overline{n}$ к плоскости, которая содержит эти вектора $\overline{a}$ и $\overline{b}$. С помощью этого перпендикуляра можно записать уравнение плоскости: $(\overline{n}, \overline{M_0M}) = 0$.

	\begin{Def}
		$$(\overline{n}, \overline{r} - \overline{r_0}) = 0$$ -- векторное уравнение плоскости.
	\end{Def}

	\begin{figure*}[h]
		\centering
		\def\svgwidth{0.4\columnwidth}
		\input{img/equation_of_a_plane.pdf_tex}
	\end{figure*}

	Если координаты вектора $\overline{n}$ -- $(A, B, C)$, то можно записать скалярное произведение в другом виде.

	\begin{Def}
		$$A(x- x_0) + B(y - y_0) + C(z - z_0) = 0$$ -- общее уравнение плоскости.
	\end{Def}

	$x_0,  y_0, z_0$ -- координаты конкретной точки, от которой можно отойти. Раскроем скобки, обозначим свободный член за $D$.
	
	\begin{Def}
		$$Ax + By + Cz + D = 0$$ -- общее уравнение плоскости.
	\end{Def}

	Векторов, перпендикулярных плоскости, содержащей векторы $\overline{a}$ и $\overline{b}$, бесконечное множество, один из них -- векторное произведение векторов $\overline{a}$ и $\overline{b}$. Если "отойти"  от параметров, то получим общее уравнение, значит, коэффициенты $A, B, C$ будут пропорциональны векторному произведению.  

	Сопоставляя общее уравнение плоскости и параметрическое можно прийти к другому виду общего уравнения.

	\begin{Def}
		$$\left|\begin{array}{cccc}
			a_2 & a_3 \\ 
			b_2 & b_3
			\end{array}\right| (x-x_0) - 
		\left|\begin{array}{cccc}
			a_1 & a_3 \\ 
			b_1 & b_3
			\end{array}\right| (y- y_0) + 
		\left|\begin{array}{cccc}
			a_1 & a_2 \\ 
			b_1 & b_3
			\end{array}\right| (z-z_0) = 0$$ -- общее уравнение плоскости.
	\end{Def}

	Плоскость можно задать тремя точками, не лежащими на одной прямой. Пусть их координаты $(x_1, y_1, z_1), \ (x_2, y_2, z_2), \ (x_3, y_3, z_3)$. Предположим, что необходимо найти плоскость через общее уравнение плоскости, т.е. $Ax + By + Cz + D = 0$. Подставим координаты трёх точек в это уравнение:
	$ \begin{cases}
		Ax_1 + By_1 + Cz_1 + D = 0\\
		Ax_2 + By_2 + Cz_2 + D = 0\\
		Ax_3 + By_3 + Cz_3 + D = 0
	\end{cases} $ \\
	Необоходимо решить эту систему, чтобы найти значения коэффициентов $A, B, C, D$. Три уравнения, четыре переменные, значит, решений у такой системы много.
	Добавим ещё одну точку $M(x, y, z)$, тогда система принимает вид:
	$ \begin{cases}
		Ax + By + Cz + D = 0 \\
		Ax_1 + By_1 + Cz_1 + D = 0\\
		Ax_2 + By_2 + Cz_2 + D = 0\\
		Ax_3 + By_3 + Cz_3 + D = 0
	\end{cases} $ \\
	При каком условии у такой системы найдётся решение? Если эта новая точка $M$ лежит в одной плоскости с заданными трёмя точками, то решение есть, иначе -- нет.

	В системе четыре уравнения, четыре неизвестных, все свободные члены равны нулю, получается, эта СЛУ однородная. Что значит, что эта система разрешима? У неё единственное решение, если эта система невырожденная, т.е. $(0, 0, 0, 0)$, это решение не подходит. \\
	Если точка $M$ принадлежит искомой плоскости, то решение существует, причем решение системым должно быть не тождественный нуль, значит, эта система имеет вырожденную матрицу:
	$\left|\begin{array}{cccc}
		x & y & z & 1 \\ 
		x_1 & y_1 & z_1 & 1 \\
		x_2 & y_2 & z_2 & 1 \\
		x_3 & y_3 & z_3 & 1 \\
		\end{array}\right| = 0 \Leftrightarrow M \in \pi$, где $\pi$ -- плоскость.

	\begin{Def}
		$$\left|\begin{array}{cccc}
			x & y & z & 1 \\ 
			x_1 & y_1 & z_1 & 1 \\
			x_2 & y_2 & z_2 & 1 \\
			x_3 & y_3 & z_3 & 1 
			\end{array}\right| = 0$$ -- матричное уравнение плоскости в пространстве.
	\end{Def}

	\Pagebreak
	\Subsection{Уравнение прямой в пространстве}

	\begin{Def}
		$M' = M + t \overline{v}$ -- параметрическое задание. 
	\end{Def}
	
	\begin{Def}
		$$\overline{r}(t) = \overline{r}_0 + t \overline{v}$$ -- параметрическое уравнение прямой.
	\end{Def}

	Введём декартову систему координат.

	\begin{figure*}[h]
		\centering
		\def\svgwidth{0.25\columnwidth}
		\input{img/equation_of_line.pdf_tex}
	\end{figure*}

	\begin{Def}
		$$ \begin{cases}
			x = x_0 + ta\\
			y = y_0 + tb \\
			z = z_0 + tc
		\end{cases} $$ -- параметрическое уровнение прямой в координатах, где $a, b, c$ -- координаты направляющего вектора $\overline{v}$.
	\end{Def}

	"Избавимся" от параметра: 
	\begin{Def}
		$$\frac{x - x_0}{a} = \frac{y - y_0}{b} = \frac{z - z_0}{c}$$ -- каноническое уравнение прямой в пространстве.
	\end{Def}
	
	Если есть произвольная точка $M_1$, лежащая на прямой, координаты которой мы знаем.

	\begin{Def}
		$$\frac{x - x_0}{x_1 - x_0} = \frac{y - y_0}{y_1 - y_0} = \frac{z - z_0}{z_1 - z_0}$$ -- каноническое уравнение прямой через две точки.
	\end{Def}

	Можно переписать данное выражение в виде системы: 
	$ \begin{cases}
		\frac{x - x_0}{x_1 - x_0} = \frac{y - y_0}{y_1 - y_0} \\
		\frac{y - y_0}{y_1 - y_0} = \frac{z - z_0}{z_1 - z_0}
	\end{cases} $, что эквивалентно другой системе:
	$ \begin{cases}
		Ax + By + Cz + D = 0 \\
		A'x + B'y + C'z + D' = 0
	\end{cases} $. Каждое из уравнений в этой системе -- уравнение плоскости, значит, прямая записана как пересечение двух плоскостей.

	\begin{Def}
		$$ \begin{cases}
			Ax + By + Cz + D = 0 \\
			A'x + B'y + C'z + D' = 0
		\end{cases} $$ -- уравнение прямой как линия пересечения двух плоскостей.
	\end{Def}

	Значения $A, B, C$ -- координаты вектора нормали к плоскости. 
	Пусть есть плоскость $\alpha$ с вектором нормали $\overline{n}_1$ и плоскость $\beta$ с вектором нормали $\overline{n}_2$. Тогда линия пересечения плоскостей -- необходимая прямая. 
	
	\begin{figure*}[h]
		\centering
		\def\svgwidth{0.25\columnwidth}
		\input{img/plane_intersection.pdf_tex}
	\end{figure*}

	Чтобы эту прямую явно задать каноническим способом, нужно знать направляющий вектор и точку. \\
	Точку можно найти следующим образом: можно любую из координат "положить" нуль, например, $x = 0$, тогда решаем СЛУ стандартным образом.\\
	Как найти направляющий вектор? Этот вектор -- векторное произведение двух нормалей плоскостей $\alpha$ и $\beta: \overline{n}_1 \times \overline{n}_2$.
	
	\begin{Example}
		Допустим, координаты точки $(0, y_0, z_0)$. Тогда можно записать каноническое уравнение как:
		$$\frac{x}{
			\left|\begin{array}{cccc}
				B & C \\ 
				B' & C'
				\end{array}\right| 
		} = \frac{y - y_0}{
			\left|\begin{array}{cccc}
				C & A \\ 
				C' & A'
				\end{array}\right| 
		} = \frac{z - z_0}{
			\left|\begin{array}{cccc}
				A & B \\ 
				A' & B'
				\end{array}\right| 
		}
		$$
	\end{Example}

	\begin{Rem}
		Можно судить о пересечении двух плоскостей; если вектора нормали неколлинеарны, то это заведомо прямая, иначе -- нужно судить по свободным членам, 
		если все коэффициенты пропорциональны, решение у этой системы -- вся плоскость (т.е. плоскости совпадают), если не пропорциональны, то решение -- $\varnothing$.
	\end{Rem}

	Рассмотрим некоторые векторы в ориентированном векторном пространстве $V^3$: зафиксированные $\overline{a}$ и $\overline{b}$ и переменный вектор $\overline{r}$ -- 
	в выражении $\overline{a}\times \overline{r} = \overline{b}$. 
	
	\begin{figure*}[h]
		\centering
		\def\svgwidth{0.4\columnwidth}
		\input{img/geometric_equation.pdf_tex}
	\end{figure*}

	Чтоб решение у этого выражения существовало, необходимо задать, что $\overline{b} \perp \overline{a}$, 
	значит, вектор $\overline{r}$ лежит в плоскости, перпендикулярной к $\overline{b}$. Разделим эту плоскость на две полуплоскости по отношению к $\overline{a}$, 
	вектор $\overline{r}$ должен лежать так, чтобы (по правилу буравчика) от вектора $\overline{a}$ давать вектор $\overline{b}$. 
	Каков геометрический смысл векторного произведения? $|\overline{a} \times \overline{r}| = S_{\text{пар}}$, кроме того, $S = |\overline{b}|$. 
	С другой стороны, $S = |\overline{a}| h$, где $h$ -- расстроение от "кончика" вектора $\overline{r}$ до прямой, содержащей $\overline{a}$, поскольку вектор $\overline{b}$ фиксирован,
	отчего фиксирована и величина $S$, то и все концы возможных векторов $\overline{r}$ должны лежать на одинаковом удалении от $\overline{a}$, равном $h$. 
	Значит, все решения лежат на прямой, параллельной той, на которой лежит вектор $\overline{a}$, на расстоянии $h$ по определённую полуплоскость.
	Таким образом, решение уравнения $\overline{a}\times \overline{r} = \overline{b}, \overline{b} \perp \overline{a}$ всегда есть, и им является прямая в пространстве.\\ 
	
	\begin{Def}
		$$\overline{a} \times \overline{r} = \overline{b}, \ \overline{b} \perp \overline{a}$$ -- векторное уравнение прямой в пространстве.
	\end{Def}

	Найдём точку, которая лежит на данной прямой, поскольку мы уже знаем, что это за прямая.
	Изобразим расстояние (перпендикуляр) от данной прямой, обозначим точкой $M_0$, до точки $O$. Из описанного ранее известно, что $h = \frac{S}{|\overline{a}|}$. 
	Тогда $\overline{OM_0}$ это результат векторного произведения $\overline{a} \times \overline{b}$, тогда направление данного вектора это $\overline{e} = \frac{\overline{a} \times \overline{b}}{|\overline{a} \times \overline{b}|}$,
	но нам необходим вектор, умноженный на $h$, тогда этот вектор $\overline{r}_0 = \frac{\overline{a} \times \overline{b}}{|\overline{a} \times \overline{b}|} \cdot \frac{|\overline{b}|}{|\overline{a}|}$ -- частное решение уравнения, где направляющий вектор $\overline{a}$. 

	\begin{figure*}[h]
		\centering
		\def\svgwidth{0.4\columnwidth}
		\input{img/private_solution.pdf_tex}
	\end{figure*}

	\Section{Кривые и поверхности второго порядка}{}{Дарья Гольденберг}

	\Subsection{Кривые второго порядка}

	\begin{Def}
    Кривая второго порядка (КВП) -- уравнение на плоскости $\R^2$, где введена декартова система координат $xOy$, вида: $$\underbrace{a_{11} x^2 + 2 a_{12}xy + a_{22} y^2}_{\text{квадратичная часть}} + \overbrace{2 a_{13}x + 2 a_{23}y + \underbrace{a_{33}}_{\text{св.коэф.}}}^{\text{линейная часть}} = 0$$ 
    где $a_{11}^2 + a_{12}^2 + a_{22}^2 \neq 0$.
  \end{Def}

  Поскольку это уравнение, разумен вопрос "Что является решением этого уравнения?"\ Решением такого уравнения является множество точек на плоскости, координаты которых удовлетворяют этому уравнению.
  Какова связь между точками множества и уравнением? Уравнение пораждает множество точек. Если есть уравнение, то есть множество его решений, которое может быть пустым. Если есть множество, то уравнений, которые пораждают это множество, "необозримое количество". 
  Поэтому однозначной связи между уравнением и множеством нет. По этой причине КВП определяем как уравнение.

  \begin{Example}
    $x^2 + y^2 + 1 = 0$. Существуют ли на плоскости точки, координаты которых удовлетворяют этому уравнению? Нет, поскольку $x^2 + y^2 \geqslant 0$, то есть решение -- $\varnothing$. Это кривая второго порядка, но мы её не видим, потому что множество точек -- пустое. 
    Рассмотрим другое уравнение: $x^2 + 1 = 0$, по ананлогичной причине решение -- $\varnothing$. Тогда верно ли, что эти кривые второго порядка совпадают? Как множества -- конечно, множества решений совпадают, но это совершенно разные КВП, нельзя никакой заменой координат свести одно уравнение к другому.
  \end{Example}

  \begin{Rem}
    Шесть коэффициентов определяют кривую второго порядка. Если уравнение в определении КВП умножить на $\lambda$, получится другой набор коэффициентов, но множество решений этих двух уравнений ничем не будет отличаться.
    Возникает тот же вопрос, что и в описанном ранее примере, множество решений совпадают, но одинаковые ли будут уравнения? Это уже вопрос о том, с точностью до чего рассматриваются уравнения, с точностью до каких преобразований плоскости.
    Поэтому ответ на вопрос, одинаковые ли уравнение и уравнение, домноженное на $\lambda$, зависит от той группы преобразований, которые допускаем, приводя одно уравнение к другому.
  \end{Rem}

  Изобразим набор коэффициентов в виде матрицы.
  $\left( \begin{array}{cccc}
    a_{11} & a_{12} & a_{13}\\
    a_{12} & a_{22} & a_{23} \\
    a_{13} & a_{23} & a_{33}
    \end{array}\right)$ -- симметричная матрица.
  Умножим эту матрицу  на другие матрицы:
  $\underbrace{\left( \begin{array}{cccc}
    x & y & 1
    \end{array}\right)
  \left( \begin{array}{cccc}
    a_{11} & a_{12} & a_{13}\\
    a_{12} & a_{22} & a_{23} \\
    a_{13} & a_{23} & a_{33}
    \end{array}\right)}_
      {\left( \begin{array}{cccc}
      ... & ... & ... 
      \end{array}\right)}
  \left( \begin{array}{cccc}
    x \\
    y \\
    1
    \end{array}\right) = 0 $. В результате умножения матриц получится матрица $1 \times 1$.

  \begin{Ex}
    Посчитать произведение трёх матриц и убедиться, что оно равно уравнению в определению кривой второго порядка.
  \end{Ex}

  \begin{Example}
    Гипербола -- выражение вида $y = \frac{a}{x}$, приведём к другому виду $xy - a = 0$. Является ли она кривой второго порядка? Да, гипербола -- это КВП, где $a_{12} = \frac{1}{2}, a_{33} = -a$.
  \end{Example}

  \Subsubsection{Конические сечения}

  Посмотрим на матричное уравнение расширенно. $\left( \begin{array}{cccc}
    x & y & z
    \end{array}\right)
  \left( \begin{array}{cccc}
    a_{11} & a_{12} & a_{13}\\
    a_{12} & a_{22} & a_{23} \\
    a_{13} & a_{23} & a_{33}
    \end{array}\right)
  \left( \begin{array}{cccc}
    x \\
    y \\
    z
    \end{array}\right) = 0$.\\ Формально необходимо рассматривать в трёхмерном пространстве и декартовой системе координат $xOyz$.
  Тогда уравнение в определении КВП будет иметь вид: \\
  $a_{11} x^2 + 2 a_{12}xy + a_{22} y^2 + 2 a_{13}xz + 2 a_{23}yz + a_{33}z^2 = 0$.

  С точки зрения алгебры это многочлен от трёх переменных, где все слагаемые второй степени.

	\begin{Def}
    $$a_{11} x^2 + 2 a_{12}xy + a_{22} y^2 + 2 a_{13}xz + 2 a_{23}yz + a_{33}z^2 = 0$$ -- однородный многочлен второй степени от трёх переменных.
  \end{Def}

  Однородность подразумевает под собой, что если заменить значения переменной следующим образом: $x = \lambda x, y = \lambda y, z = \lambda z$, то при подстановке в многочлен выражение примет вид:\\
  $a_{11} \lambda^2 x^2 + 2 a_{12} \lambda x \lambda y + a_{22} \lambda^2 y^2 + 2 a_{13} \lambda x \lambda z + 1 a_{23} \lambda y \lambda z + a_{33} \lambda^2z^2 = \lambda^2(a_{11} x^2 + 2 a_{12}xy + a_{22} y^2 + 2 a_{13}xz + 2 a_{23}yz + a_{33}z^2) = 0$.
  Замена переменных на переменную, умноженную на константу, не меняет самого уравнения, поскольку константа выносится.

  Геометрический смысл однородности в трёхмерном пространстве:

  Заметим, что $x = 0, y = 0, z = 0$ удовлетворяет уравнению.

  Предположим, что есть точка с координатами $(x_0, y_0, z_0)$, которая является решением. Что означает однородность? Какое бы число $\lambda$ ни взяли, то в силу однородности это $\lambda$ выносится, и равенство нулю подтверждается.
  $(x_0, y_0, z_0)$ -- координаты радиус-вектора, заменили на $(\lambda x_0, \lambda y_0, \lambda z_0)$, каждую координату умножили на $\lambda$, этот радиус-вектор стал с новыми координатами. Но $\lambda$ -- любое, значит, это множество точек -- прямая, проходящая через начало координат.


  Если какая-то точка является решением матричного уравнения, то и вся прямая, которая проходит через эту точку и начало координат, является решением уравнения.

  Что собой представляет решение однородного уравнения? Оно собой представляет точку $O$ и совокупность прямых, проходящих через точку $O$.

  \begin{Def}
    Множество точек пространства, которое состоит из фиксированной точки и некоторых совокупностей прямых, проходящих через точку, называется \textbf{конус}, где $O$ -- вершина.
  \end{Def}

  \begin{Example}
    $x^2  + y^2 + z^2 = 0$. Единственное решение - $(0, 0, 0)$, что есть только одна точка? начало координат. 
  \end{Example}

  Вернёмся к исходному виду:

  $\begin{cases}
    
  
  \left( \begin{array}{cccc}
    x & y & z
    \end{array}\right)
  \left( \begin{array}{cccc}
    a_{11} & a_{12} & a_{13}\\
    a_{12} & a_{22} & a_{23} \\
    a_{13} & a_{23} & a_{33}
    \end{array}\right)
  \left( \begin{array}{cccc}
    x \\
    y \\
    z
    \end{array}\right) = 0 \\
  z = 1    
  \end{cases}$, где первое уравнение -- уравнение конуса в пространстве, а второе -- линейное уравнение в пространстве, которое задаёт плоскость, параллельную $xOy$.
  
  В геометрическом смысле система -- пересечение множества решений. У первого уравнения решение -- некий конус, у второго -- плоскость, тогда их пересечение -- кривая второго порядка, что есть коническое сечение.

  \begin{Rem}
    Любая КВП является коническим сечением, но в обратную сторону бывает только с оговорками. Конус обязательно должен быть задан как однородный многочлен второй степени от трёх переменных.
  \end{Rem}

  \Subsection{Центральные КВП}
  
\end{document}

\end{document}