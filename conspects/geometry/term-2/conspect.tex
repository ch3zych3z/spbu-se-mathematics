\input{preamble.tex}

\begin{document}
	\Header

	\BeginConspect

	\Section{Аналитическая геометрия}{}{Илья Дудников}

	\Subsection{Системы координат}
	
	\Subsubsection{Аффинные системы координат}

	\begin{Def}
		Аффинной системой координат на прямой называется взаимно-однозначное соответствие $l \longleftrightarrow \R$. \\
		
		\begin{figure*}[h]
			\centering
			\def\svgwidth{0.3\columnwidth}
			\input{img/as_line.pdf_tex}
		\end{figure*}
		
		Она определяется выбором точки $O$  и выбором вектора $\overline{e}$. АСК = $\{O, \{\overline{e}\}\}$.
	\end{Def}

	\begin{Def}
		АСК на плоскости называется биекция $\pi \longleftrightarrow \R^2$. 

		\begin{figure*}[h]
			\centering
			\def\svgwidth{0.3\columnwidth}
			\input{img/as_plane.pdf_tex}
		\end{figure*}

		Она определяется выбором точки $O$ и векторов $\overline{e}_1, \overline{e}_2 \neq \overline{e}, \overline{e}_1 \not\parallel \overline{e}_2$. 
		АСК = $\{O, \{\overline{e}_1, \overline{e}_2\}\}$.
	\end{Def}

	\begin{Def}
		Если $|\overline{e}_1| = |\overline{e}_2| = 1, \overline{e}_1 \perp \overline{e}_2$, то АСК называется декартовой системой координат. 
	\end{Def}

	\begin{Def}
		АСК в пространстве называется биекция $M \longleftrightarrow \R^3$ .
		Она определяется выбором точки $O$ и векторов $\overline{e}_1, \overline{e}_2, \overline{e}_3 \neq \overline{0}$ -- не компланарны. АСК = $\{O, \{\overline{e}_1, \overline{e}_2, \overline{e}_3\}\}$.
	\end{Def}

	\begin{Def}
		Упорядоченная тройка векторов $(\overline{u}, \overline{v}, \overline{w})$ называется \textbf{правой} , 
		если из конца вектор $\overline{w}$ поворот то $\overline{u}$ к $\overline{v}$ по наименьшему углу выглядит происходящим против часовой стрелки,
		и \textbf{левой} -- в противном случае. 
	\end{Def}

	\Pagebreak
	\Subsubsection{Криволинейные системы координаты}

	\begin{Def}
		Выберем точку $O$ и построим из неё луч $p$, который назовем \textit{полярной осью}. Возьмем теперь произвольную точку $M$ на плоскости и измерим две величины:
		расстояние от $M$ до $O$ и угол между вектором $\overline{OM}$ и полярной осью. Обозначим расстояние за $r$, а угол за $\PHI$.
		Тогда, чтобы избежать неоднозначности, будем считать, что $r > 0, \PHI \in [0, 2\pi)$, и если $r = 0$, то $\PHI = 0$.    
		Такая система координат называется \textbf{полярной}.		

		\begin{figure*}[h]
			\centering
			\def\svgwidth{0.3\columnwidth}
			\input{img/polar_system.pdf_tex}
		\end{figure*}
	\end{Def}

	\begin{Def}
		Полярная система координат, где $r \in \R, \PHI \in \R$, то она называется \textit{обобщенной} полярной системой координат.
	\end{Def}

	\begin{figure*}[h!]
		\centering
		\def\svgwidth{0.3\columnwidth}
		\input{img/coordinate_net.pdf_tex}
		\caption{Координатная сеть полярной системы координат}
	\end{figure*}

	\begin{Def}
		Цилиндрической системой координат называют трёхмерную систему координат, являющуюся расширением полярной системы координат путём добавления третьей координаты (обычно обозначаемой ${\displaystyle z}$), которая задаёт высоту точки над плоскостью.
	\end{Def}

	\begin{Def}
		Сферическая система координат — трёхмерная система координат, в которой каждая точка пространства определяется тремя числами, где r — расстояние до начала координат, а $\theta$ и $\varphi$ — зенитный и азимутальный углы соответственно.
	\end{Def}

	\Subsubsection{Параметризации}

	Построим декартову систему координат. Теперь возьмем какую-то новую систему координат $x', y', z'$.
	Проведем через $x', y'$ плоскость. Если $z'$ не совпадает с $z$, то эта плоскость пересекает плоскость $(x, y)$ по какой-то прямой.
	Отсчитает от вектора $x$ до этой прямой угол $\PHI$. Угол между $z$ и $z'$ обозначим за $\psi$.
	Теперь, мы можем эту прямую поворачивать вокруг оси $z'$ на угол $\delta$, пока она не совпадет с $x'$.
	
	\begin{figure*}
		\centering
		\def\svgwidth{0.3\columnwidth}
		\input{img/parametrisation.pdf_tex}
	\end{figure*}

	Таким образом, мы совместили исходную систему координат с новой СК. То есть
	мы построили соответствие между ( $\psi, \PHI, \delta$ ).
	
	\Subsection{Понятие вектора}

	Пусть $E$ -- евклидово пространство.
	\begin{Def}
		Закрепленный вектор -- упорядоченная пара точек в евклидовом пространстве.
		Обозначение: $\overrightarrow{AB}$, модуль $|\overrightarrow{AB}|$ -- расстояние между точками $A$ и $B$.
	\end{Def}

	\begin{Def}
		Пусть $\{(A, B), A, B \in E\}$ -- множество закрепленных векторов. Введём на нём отношение равенства:
		$(A, B) = (C, D) \EQ$:
		\begin{MyList}
			\item $|\overrightarrow{AB}| = |\overrightarrow{CD}|$ 
			\item $(A, B) || (C, D)$ либо совпадают.
			\item $\overrightarrow{AB} \upuparrows \overrightarrow{CD}$. 
		\end{MyList}
	\end{Def}

	\begin{Rem}
		$\forall A, B \to (A, A) = (B, B)$.
	\end{Rem}

	\begin{Prop}
		Отношение, введённое в прошлом определении -- отношение эквивалентности.
	\end{Prop}

	\begin{proof}
		\begin{MyList}
			\item Рефлексивность: $(A, B) = (A, B)$ -- верно.
			\item Симметричность -- очевидно.
			\item Транзитивность: $(A, B) = (C, D), (C, D) = (F, G) \SO (A, B) = (F, G)$ -- верно.
		\end{MyList}
		Значит множество закрепленных векторов разбивается на классы эквивалентности.
	\end{proof}

	\begin{Def}
		Класс эквивалентности называется \textbf{свободным вектором}.
	\end{Def}

	\Subsection{Сложение и умножение на число}

	Пусть $\overline{a}, \overline{b} \in V$ -- классы.

	\begin{Def}
		Сложение векторов: $V \times V \to V$.
		$[\overrightarrow{OO''}] = \overline{a} + \overline{b}$ 
	\end{Def}

	\begin{Def}
		Пусть $\overline{a} \in V, \lambda \in \R$. Умножение на число на число: $\R \times V \to V$.
	\end{Def}
	
	$(V, +, \cdot)$. Свойства:
	\begin{MyList}
		\item $\forall \overline{a}, \overline{b} \in V \ \overline{a} + \overline{b} = \overline{b} + \overline{a}$.
		\item $\forall \overline{a}, \overline{b}, \overline{c} \in V \ (\overline{a} + \overline{b}) + \overline{c} = \overline{a} + (\overline{b} + \overline{c})$.
		\item $\exists \overline{0} : \forall \overline{a} \ \overline{a} + \overline{0} = \overline{0} + \overline{a} = \overline{a}$.
		\item $\forall \overline{a} \ \exists -\overline{a} : \overline{a} + (-\overline{a}) = \overline{0}$.
		\item $\forall \lambda \in \R, \overline{a}, \overline{b} \in V \ \lambda(\overline{a} + \overline{b}) = \lambda \overline{a} + \lambda \overline{b}$.
		\item $\forall \lambda, \mu \in \R, \overline{a} \in V \ (\lambda + \mu) \overline{a} = \lambda \overline{a} + \mu \overline{a}$.
		\item $\forall \overline{a} \in V \ 1 \cdot \overline{a} = \overline{a}$.
		\item $\forall \lambda, \mu \in \R, \overline{a} \in V \ \lambda(\mu \overline{a}) = (\lambda \mu) \overline{a}$.
	\end{MyList} 

	\begin{Def}
		Множество $(V, +, \cdot)$, удовлетворяющее свойствам 1-8, называется \textbf{векторным пространством}. Элементы -- векторы.
	\end{Def}

	\Subsection{ЛЗ, ЛНЗ, Базис, размерность}

	\begin{Def}
		$\lambda_1 \overline{a}_1 + ... + \lambda_n \overline{a}_n$ -- линейная комбинация. Если $(\lambda_1, ..., \lambda_n) \neq (0, ..., 0)$ -- нетривиальная ЛК.  
	\end{Def}

	\begin{Def}
		$\{\overline{a}_i\}_{i = 1}^n$ -- линейно зависимый, если $\exists$ нетривиальная ЛК $\{\lambda_i\}_{i = 1}^n : \sum_{i = 1}^n \lambda_i \overline{a}_i = 0$   
	\end{Def}

	\begin{Def}
		$\{\overline{a}_i\}_{i = 1}^n$ -- ЛНЗ, если он не ЛЗ. 
	\end{Def}

	Свойства:
	\begin{MyList}
		\item $\{\overline{a} \neq \overline{0}\}$ -- ЛНЗ.
		\item $\{\overline{0}\}$ -- ЛЗ.
		\item $\{\overline{a_1}, ..., \overline{a}_n, \overline{0}\}$ -- ЛЗ.
		\item Пусть $\{\overline{a}_i\}$ -- ЛЗ. Тогда $\{\overline{a}_i, \overline{a}_j\}_{i = 1, j = 1}^{n, m}$ -- ЛЗ.
	\end{MyList}

	\begin{Def}
		$\{\overline{a}_\alpha\}_{\alpha \in \Lambda}$ -- ЛЗ, если в нем $\exists$ ЛЗ конечный поднабор.
	\end{Def}

	\begin{Def}
		ЛНЗ -- набор, который не является ЛЗ.
	\end{Def}

	\begin{Def}
		$\{\overline{a}_\alpha\}_{\alpha \in \Lambda}$ -- полный, если $\forall \overline{v} \in V \ \exists \{\alpha_i\}_{i = 1}^n, \{\lambda_i\}_{i = 1}^n \ \overline{v} = \lambda_1 \overline{a}_{\alpha_1} + ... + \lambda_n \overline{a}_{\alpha_n}$.
	\end{Def}

	\begin{Def}
		$\{\overline{a}_\alpha\}_{\alpha \in \Lambda}$ -- базис $V$, если он полный и ЛНЗ.
	\end{Def}

	\begin{Def}
		Размерность $V$ ( $\dim V$ ) -- мощность базиса.
	\end{Def}

	\begin{Def}
		Векторное пространство $V$ называется конечномерным, если $\exists$ конечный полный набор.
	\end{Def}

    \gdef\AuthorName{Дарья Гольденберг}
	\Subsection{Скалярное умножение}

	Будем определять скалярное произведение для элементов векторного пространства $V$.
	
	\begin{Def}
	  $(\overline{a}, \overline{b})$ -- скалярное произведение: $V \times V \to \R$
	\end{Def}
	
	Свойства:
	\begin{MyList}
		\item Свойства 1-8, необходимые для существования векторного пространства.
		\item $\forall \overline{a} \in V \ (\overline{a}, \overline{a}) \geqslant 0$ -- положительная определённость. \\
		Кроме того, $(\overline{a}, \overline{a}) = 0 \Leftrightarrow \overline{a} = \overline{0}$ -- невырожденность.
		\item $\forall \overline{a}, \overline{b}, \overline{c} \in V \ (\overline{a} + \overline{b}, \overline{c}) = (\overline{a}, \overline{c}) + (\overline{b}, \overline{c})$ -- аддитивность.\\
		$\forall \lambda \in \R, \overline{a}, \overline{b} \in V \ (\lambda \overline{a}, \overline{b}) = \lambda(\overline{a}, \overline{b})$ -- однородность.
		\item $\forall \overline{a}, \overline{b} \in V \ (\overline{a}, \overline{b}) = (\overline{b}, \overline{a})$. -- коммутативность.
	\end{MyList}
  
	\begin{Example}
	  $\R^n = \{ (x_1, x_2, ..., x_n): x_i \in \R \}$ \\
	  $\overline{v} = (x_1, ..., x_n), \overline{w} = (y_1, ..., y_n)$ \\
	  Тогда скалярное произведение: $(\overline{v}, \overline{w}) = x_1 y_1 + ... + x_n y_n.$ \\
	  Проверим свойства: 
	  \begin{MyList}
		\item $(\overline{v}, \overline{v}) = x_1^2 + ... + x_n^2 \geqslant 0$.\\
		$(\overline{v}, \overline{v}) = 0 \Leftrightarrow \forall i \ x_i = 0$.
		\item Пусть $ \overline{z} = (z_1, ..., z_n)$, тогда $(\overline{v} + \overline{w}, \overline{z}) = (x_1 + y_1)z_1 + ... + (x_n + y_n)z_n = x_1 z_1 + ... + x_n y_z + y_1 z_1 + ... + y_n z_n = (\overline{v}, \overline{z}) + (\overline{w}, \overline{z})$.\\
		$(\lambda \overline{v}, \overline{w}) = \lambda x_1 y_1 + ... + \lambda x_n y_n = \lambda (x_1 y_1 + ... + x_n y_n) = \lambda(\overline{v}, \overline{w})$.
		\item $(\overline{v}, \overline{w}) = x_1 y_1 + ... + x_n y_n = y_1 x_1 + ... + y_n x_n = (\overline{w}, \overline{v})$.
	  \end{MyList}
	\end{Example}
  
	\begin{Example}
	  $C[0, 1]$ -- непрерывные функции на отрезке $[0, 1]$.\\
	  Пусть $f, g, q \in C[0, 1]$ -- функции: $(f, g) = \int_0^1 fg \,dx$.
	  \begin{MyList}
		\item $(f, f) = \int_0^1 f^2 \,dx \geqslant 0. \\
		(f, f) = 0  \Leftrightarrow f = 0$.
		\item $(f + q, g) = \int_0^1 (f + q)g \,dx = \int_0^1 (fg + qg) \,dx = \int_0^1 fg \,dx + \int_0^1 qg \,dx = (f, g) + (q, g)$. \\
		$(\lambda f, g) =\int_0^1 \lambda fg \,dx  = \lambda \int_0^1 fg \,dx = \lambda(f, g)$.
		\item $(f, g) =\int_0^1 fg \,dx  = \int_0^1 gf \,dx = (g, f)$.
	  \end{MyList}
	  Таким образом, это скалярное произведение непрерывных на $[0, 1]$ функций.
	\end{Example}
  
	Пусть есть конечномерное векторное пространство $V$, на нём задано скалярное произведение $(,)$, выберем базис векторного пространства $\{\overline{e}_i\}$, рассмотрим векторы $\overline{v} = (x_i), \overline{w} = (y_i)$, 
	тогда их скалярное произведение $(\overline{v}, \overline{w}) = (x_1 \overline{e}_1 + ... + x_n \overline{e}_n, y_1 \overline{e}_1 + ... + y_n\overline{e}_n)$, т.е.
	$$(\overline{v}, \overline{w}) = \sum_{i, j}^n x_i y_j (\overline{e}_i, \overline{e}_j)$$
	Либо же запись вида:
	\[(\overline{v}, \overline{w}) = \left(\begin{array}{cccc}
	  x_1 & x_2 & \cdots & x_n
	  \end{array}\right)
	   \left(\begin{array}{cccc}
	  (\overline{e}_1, \overline{e}_1) & (\overline{e}_1, \overline{e}_2) & \cdots & (\overline{e}_1, \overline{e}_n) \\ 
	  \vdots & \vdots & \ddots & \vdots \\
	  (\overline{e}_n, \overline{e}_1) & (\overline{e}_n, \overline{e}_2) & \cdots & (\overline{e}_n, \overline{e}_n)
	  \end{array}\right) 
	  \left(\begin{array}{c}
		  y_1 \\ 
		  y_2 \\ 
		  \vdots \\ 
		  y_n
		  \end{array}\right)\] 
	  где $G = ((\overline{e}_i, \overline{e}_j)), {1 \leqslant i \leqslant j \leqslant n}$ -- матрица Грама сколярного произведения.
	  
	  Тогда скалярное произведение можно записать в следующем виде: $(\overline{v}, \overline{w}) = \overline{v}^T G \overline{w}$.
  
	  В силу коммутативности скалярного произведения $G^T = G$.
	  
	  \begin{Thm}[Критерий Сильвестра]
		 \[\forall k = 1,..., n \ \ \det(G_k) > 0\]
		 где $G_n$ -- миноры главной диагонали.
	  \end{Thm}
  
	  \begin{Prop}
		  Если взять $R^n, G$, то $G$ -- матрица Грама $\Leftrightarrow G^T = G$, которая удовлетворяет критерию Сильвестра.
	  \end{Prop}
  
	  \begin{Def}
		  Если базис обладает свойством:
		  $(e_i, e_j) = \begin{cases}
		  1, i \neq j\\
		  1, i = j
	  \end{cases} \Rightarrow G = E$,
	  тогда он называется ортонормированным базисом (ОРБ).
	  \end{Def} 
  
	  \begin{Thm}[Теорема Грама-Шмидта]
		  В $\forall V^n$ со скалярным произведением $(,) \ \exists$ ОНБ. 
	  \end{Thm}
  
	  \begin{Def}
		  $V$ -- векторное пространство, $(,)$ -- скалярное произведение на нём, тогда \textbf{модуль} (\textbf{длина}) $|\overline{a}| = \sqrt{(\overline{a}, \overline{a})}, |\overline{a}| = 0 \Leftrightarrow \overline{a} = 0$.
	  \end{Def}
	  
	  \begin{Def}
		  Величина угла между векторами -- число $\alpha \in [0; \pi] \in R: \cos \alpha = \frac{(\overline{a}, \overline{b})}{|\overline{a}||\overline{b}|}, \overline{a} \neq 0, \overline{b} \neq 0$.
	  \end{Def}
  
	  \begin{Thm}[Неравенство Коши-Буняковского]
		  $$(\overline{a}, \overline{b})^2 \leqslant \overline{a}^2 \overline{b}^2$$
	  \end{Thm}
  
	  \begin{proof}
		  По свойству скалярного произведения $(\overline{a} + t\overline{b})^2$ всегда невырожденная величина, т.е. $(\overline{a} + t\overline{b})^2 \geqslant 0 \Rightarrow \overline{a}^2 + 2t(\overline{a}, \overline{b}) + t^2 \overline{b}^2 \geqslant 0$, 
		  тогда его дискриминант не положительный, т.к. $t$ -- любое число, то $$(\overline{a}, \overline{b})^2 - \overline{a}^2 \overline{b}^2 \leqslant 0 \Rightarrow (\overline{a}, \overline{b})^2 \leqslant \overline{a}^2 \overline{b}^2$$.
	  \end{proof}
  
	  \Subsection{Векторное умножение}
  
	  Векторное умножение определяется только для трёхмерного пространства $V^3$, кроме того, необходимо, чтобы пространство было ориентированным, выберем в нём правый ОНБ $(\overline{i}, \overline{j}, \overline{k})$.
	  
	  \begin{figure*}[h]
		  \centering
		  \def\svgwidth{0.2\columnwidth}
		  \input{img/right-oriented.pdf_tex}
	  \end{figure*}
	  \begin{Def}
		  Пусть $\overline{v} = (x_1, x_2, x_3), \overline{w} = (y_1, y_2, y_3)$. Тогда \textbf{векторное произедение} \\
		  $\overline{v} \times \overline{w} = \left|\begin{array}{cccc}
			  \overline{i} & \overline{j} & \overline{k} \\ 
			  x_1 & x_2 & x_3 \\ 
			  y_1 & y_2 & y_3
			  \end{array}\right| = \overline{i}(x_2 y_3 - x_3y_2) - \overline{j}(x_3 y_1 - x_1 y_3) + \overline{k}(x_1 y_2 - x_2 y_1)$.
	  \end{Def}
	  
	  Свойства:
	  \begin{MyList}
		  \item $\overline{v} \times \overline{w} = - \overline{w} \times \overline{v}$ -- косокоммутативность.
		  
		  \item $\overline{v} \times \overline{v} = \overline{0}$.
		  
		  \item $(\overline{v} + \overline{w}) \times \overline{z} = 
			  \left|\begin{array}{cccc}
			  \overline{i} & \overline{j} & \overline{k} \\ 
			  x_1 + y_1 & x_2 + y_2 & x_3 + y_3 \\ 
			  z_1 & z_2 & z_3
			  \end{array}\right| =
			  \left|\begin{array}{cccc}
			  \overline{i} & \overline{j} & \overline{k} \\ 
			  x_1 & x_2 & x_3 \\ 
			  z_1 & z_2 & z_3
			  \end{array}\right| +
			  \left|\begin{array}{cccc}
			  \overline{i} & \overline{j} & \overline{k} \\ 
			  y_1 & y_2 & y_3 \\ 
			  z_1 & z_2 & z_3
			  \end{array}\right| = \overline{v} \times \overline{z} + \overline{w} \times \overline{z}$ -- аддитивность.
		  
		  \item $(\lambda \overline{v}) \times \overline{w} = 
			  \left|\begin{array}{cccc}
			  \overline{i} & \overline{j} & \overline{k} \\ 
			  \lambda x_1 & \lambda x_2 & \lambda x_3 \\ 
			  y_1 & y_2 & y_3
			  \end{array}\right| = \lambda \overline{v} \times \overline{w}$.
  
		  \item $\overline{v} \times \overline{w} \perp \overline{v}, \overline{w}\\
		  (\overline{v}, \overline{v} \times \overline{w}) = 
			  (\left|\begin{array}{cccc}
			  \overline{i} & \overline{j} & \overline{k} \\ 
			  x_1 & x_2 & x_3 \\ 
			  y_1 & y_2 & y_3
			  \end{array}\right|, \overline{v}) = 
			  \left|\begin{array}{cccc}
				  x_1 &  x_2 & x_3 \\ 
				  x_1 &  x_2 & x_3 \\ 
				  y_1 & y_2 & y_3
				  \end{array}\right| = 0$.
		  \item $\overline{v} \times \overline{w} = 0 \Leftrightarrow \overline{v} \parallel \overline{w} \\
		  (\overline{v}, \overline{w}, \overline{v} \times \overline{w}) =
		  \left|\begin{array}{cccc}
			  x_1 & x_2 & x_3 \\ 
			  y_1 & y_2 & y_3 \\ 
			  (x_2 y_3 - x_3 y_2) & (x_3 y_1 - x_1 y_3) & (x_1 y_2 - x_2 y_1)
			  \end{array}\right| = (x_2 y_3 - x_3 y_2)^2 + (x_3 y_1 - x_1 y_3)^2 + (x_1 y_2 - x_2 y_1)^2 \geqslant 0 \Rightarrow (\overline{v}, \overline{w}, \overline{v} \times \overline{w}) = 0 \Leftrightarrow \frac{x_2}{y_2} = \frac{x_3}{y_3}, \frac{x_3}{y_3} = \frac{x_1}{y_1}, \frac{x_1}{y_1} = \frac{x_2}{y_2} \Rightarrow \frac{x_1}{y_1} = \frac{x_2}{y_2} = \frac{x_3}{y_3}$.
		  \item $\overline{v} \nparallel \overline{w}  \Rightarrow (\overline{v}, \overline{w}, \overline{v} \times \overline{w})$ -- правая.
		  \item $\overline{i} \times \overline{j} = \overline{k}$. Получим таблицу умножения: $i \to j, j \to k, k \to i$.
		  \item $(\overline{a} \times \overline{b})^2 = (x_2 y_3 - y_2 x_3)^2 + (...)^2 + (...)^2 \\
		  \overline{a}^2 \overline{b}^2 - (\overline{a} \overline{b})^2 = (x_1^2 + x_2^2 + x_3^2)(y_1^2 + y_2^2 + y_3^2) - (x_1y_1 +x_2y_2 +x_3y_3)^2 \\
		  (\overline{a} \times \overline{b})^2 = \overline{a}^2 \overline{b}^2 - (\overline{a} \overline{b})^2$ -- упражнение.\\
		  $\overline{a}^2 \overline{b}^2 - (\overline{a} \overline{b})^2 = S^2 = |\overline{a}|^2 |\overline{b}|^2 \sin^2\alpha = a^2 b^2 (1 - \cos^2 \alpha) = \overline{a}^2 \overline{b}^2 - (\overline{a} \overline{b})^2$, т.к. $\cos^2 \alpha = \frac{(a, b)^2}{|a||b|}$.\\
	  \begin{figure*}[h]
		  \centering
		  \def\svgwidth{0.3\columnwidth}
		  \input{img/liner_transformation.pdf_tex}
	  \end{figure*}
	  Следствие: $|\overline{a}| = |\overline{b}| = 1,  (\overline{a}, \overline{b}) \Rightarrow |\overline{a} \times \overline{b}| = 1$.
  \end{MyList}
  
	Рассмотрим $V^3$, фиксируем ОНБ $(\overline{i}, \overline{j}, \overline{k})$, зададим векторное произведение $\times: V \times V \to V$.

	Выберем $\overline{a}, \overline{b}, \overline{c}$ -- правый ОНБ, таким образом, если взять любой ОНБ, можно получить таблицу умножения: $a \to b, b \to c, c \to a$.

	$\overline{v} = (\lambda_1, \lambda_2, \lambda_3), \ \overline{w} = (\mu_1, \mu_2, \mu_3) \Rightarrow \overline{v} \times \overline{w} = (\lambda_1 \overline{a} + \lambda_2 \overline{b} + \lambda_3 \overline{c})\times (\mu_1 \overline{a} + \mu_2 \overline{b} + \mu_3 \overline{c}) = \\
	= \lambda_1 \mu_1 \overline{a} \times \overline{a} + \lambda_1 \mu_2 \overline{a} \times \overline{b} + \lambda_1 \mu_3 \overline{a} \times \overline{c}
	+ \lambda_2 \mu_1 \overline{b} \times \overline{a} + \lambda_2 \mu_2 \overline{b} \times \overline{b} + \lambda_2 \mu_3 \overline{b} \times \overline{c}
	+ \lambda_3 \mu_1 \overline{c} \times \overline{a} + \lambda_3 \mu_2 \overline{c} \times \overline{b} + \lambda_3 \mu_3 \overline{c} \times \overline{c} = \\
	= \overline{c}(\lambda_1 \mu_2 - \lambda_2 \mu_1) - \overline{b}(\lambda_1 \mu_3 - \lambda_3 \mu_1) + \overline{a}(\lambda_2 \mu_3 - \lambda_3 \mu_2) =  \left|\begin{array}{cccc}
		\overline{a} & \overline{b} & \overline{c} \\ 
		\lambda_1 & \lambda_2 & \lambda_3 \\ 
		\mu_1 & \mu_2 & \mu_3
		\end{array}\right|$.   

\end{document}