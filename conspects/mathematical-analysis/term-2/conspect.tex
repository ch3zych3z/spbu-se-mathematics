\input{preamble.tex}

\begin{document}

\Header

\BeginConspect

\Section{Интегральное исчисление}{}{Илья Дудников}

\Subsection{Неопределенный интеграл}
\begin{Def}
    $f : \langle A, B\rangle \to \R, F : \langle A, B\rangle \to \R$ называется первообразной функцией $f$, если $F$ дифференцируема на $\langle A, B\rangle, F'(x) = f(x) \ \forall x \in \langle A, B\rangle$.
\end{Def}

\begin{Thm}
    Пусть $f, F, G : \langle A, B\rangle \to \R, F$ -- первообразная $f$. Тогда
    $G$ -- первообразная $f \EQ \exists c \in \R : F(x) + c = G(x)$.
\end{Thm}

\begin{proof}
	$\SO$.  
    Пусть $H(x) = F(x) - G(x)$. Тогда
	\[H'(x) = F'(x) - G'(x) = f(x) - f(x) = 0 \EQ H'(x) = 0 \SO H(x) \equiv \const\]
    $\Leftarrow$. $(F(x) + c)' = (G(x))' \EQ f(x) = F'(x) = G'(x) \SO G$ -- первообразная.  
\end{proof}

\begin{Def}
    $f : \langle A, B\rangle \to \R, F$ -- первообразная $f$. Множество функций 
    $\{F(x) + c, c \in \R\}$ называется неопределенным интегралом $f$.
    \[\int f(x) \,dx = F(x) + c, c \in \R \] 
\end{Def}

Далее, $f : \langle A, B\rangle \to \R$. 

\begin{MyList}
	\item Дифференцирование
	\[\left(\int f(x) \,dx\right)' = f(x), x \in \langle A, B\rangle\]

	\item Арифметические действия: 
	\[\int f(x) \,dx + \int g(x) \,dx = \left\{F(x) + G(x) + c, c \in \R\right\}\]
	\[\int f(x) \,dx + H(x) = \left\{F(x) + H(x) + c, c \in \R\right\}\]
	\[\lambda \int f(x) \,dx = \left\{\lambda F(x) + c, c \in \R\right\}, \lambda \neq 0, \lambda \in \R\]
\end{MyList}

\begin{Prop}
	Если функция $f$ непрерывна на $\langle A, B\rangle$, то у неё есть первообразная на~$\langle A, B\rangle$.
\end{Prop}

\begin{Ex}
	$f(x) = \begin{cases}
		1, x \geqslant 0 \\
		-1, x < 0
	\end{cases}$. Есть ли первообразная у этой функции?
\end{Ex}

\begin{Def}
	$E \subset \R, f : E \to \R$. Если $F$ дифференцируема на $E$ и $F'(x) = f(x)$ на $E$, то $F$ -- первообразная $f$ на множестве $E$.
\end{Def}

\Pagebreak
Таблица неопределенных интегралов

\begin{multicols}{2}
	\begin{MyList}
		\item $\int a \,dx = ax + c, a \in \R$ 
		\item $\int x^a \,dx = \frac{x^{a + 1}}{a + 1} + c, a \neq -1$ 
		\item $\int \frac{1}{x}\,dx = \ln |x| + c$ 
		\item $\int e^x \,dx = e^x + c$ 
		\item $\int a^x \,dx = \frac{a^x}{\ln a} + c, a > 0, a \neq 1$ 
		\item $\int \sin x \,dx = -\cos x + c$
		\item $\int \cos x \,dx = \sin x + c$ 
		\item $\int \frac{1}{\cos^2 x}\,dx = \tg x + c$ 
		\item $\int \frac{1}{\sin^2 x}\,dx = -\ctg x + c$ 
		\item $\int \frac{\,dx}{x^2 + a^2} = \frac{1}{a} \arctg \frac{x}{a} + c, a \neq 0$
		\item $\int \frac{\,dx}{\sqrt{a^2 - x^2}} = \arcsin \frac{x}{a} + c, a > 0$
		\item $\int \frac{\,dx}{x^2 - a^2} = \frac{1}{2a}\ln \left| \frac{x - a}{x + a}\right| + c, a \neq 0$ 
		\item $\int \frac{\,dx}{\sqrt{x^2 + a}} = \ln \left| x + \sqrt{x^2 + a}\right| + c, a \in \R$   
	\end{MyList}	
\end{multicols}

\begin{proof}
	Дифференцирование
\end{proof}

\begin{Example}
	$\int \frac{\sin x}{x} \,dx$ -- неберущийся интеграл.
	$\Si(x)$ -- интегральный синус (одна из первообразных, закрепленная при $x \to 0+$ ).
	\[(\Si(x))' = \frac{\sin x}{x}\]
\end{Example}

\begin{Thm}[Линейность неопределенного интеграла]
	$f, g : \langle A, B\rangle \to \R$, имеют первообразные на $\langle A, B\rangle$.
	Тогда $\forall \alpha, \beta \in \R : \alpha, \beta \neq 0$
	\[\int (\alpha f(x) + \beta g(x))\,dx = \alpha \int f(x) \,dx + \beta \int g(x) \,dx\] 
\end{Thm}

\begin{proof}
	Пусть $F$ и $G$ -- первообразные $f$ и $g$ на $\langle A, B\rangle$.
	Правая часть равенства: $\left\{\alpha F(x) + \beta G(x) + c, c \in \R\right\}$.
	\[\left(\alpha F(x) + \beta G(x) + c\right)' = \alpha F'(x) + \beta G'(x) = \alpha f(x) + \beta g(x)\]
\end{proof}

\begin{Thm}[Замена переменной]
	$f : \langle A, B\rangle \to \R, F$ -- первообразная $f$ на $\langle A, B\rangle$, $\PHI : \langle C, D\rangle \to \langle A, B\rangle$ -- дифференцируемая функция.
	Тогда 
	\[\int f(\PHI(x))\PHI'(x) \,dx = F(\PHI(x)) + c\]
\end{Thm}

\begin{proof}
	\[\left(F(\PHI(x)) + c\right)' = F'(\PHI(x)) \cdot \PHI'(x) = f(\PHI(x)) \cdot \PHI'(x)\]
\end{proof}

\begin{Rem}
	$\PHI'(x) \,dx = d \PHI(x)$. Пусть $y = \PHI (x)$ 
	\[\int f(y) dy = F(y) + c = F(\PHI(x)) + c\] 
\end{Rem}

\begin{Example}
	$\int \frac{\ln x}{x}\,dx = \int \ln x \cdot \frac{1}{x}\,dx$. Пусть $y = \ln x \SO dy = \frac{1}{x} \,dx$ 
	\[\SO \int \frac{\ln x}{x}\,dx = \int y \,dy = \frac{y^2}{2} + c = \frac{\ln^2 x}{2} + c\]
\end{Example}

\begin{Cons}
	Пусть в условиях теоремы $\PHI$ имеет обратную функцию $\psi : \langle A, B\rangle \to \langle C, D\rangle$. Если $G(x)$ -- первообразная функции $(f \circ \PHI(x)) \cdot \PHI'(x)$, то \[\int f(x) \,dx = G(\psi (x)) + c\]
\end{Cons}

\begin{proof}
	Пусть $F$ -- первообразная $f$ на $\langle A, B\rangle$.
	$F(\PHI(x))$ -- первообразная $f(\PHI(y)) \PHI'(y)$~(по теореме).
	Рассмотрим $G(x) - F(\PHI(x))$ -- постоянная (т.к. производная равна нулю).
	$y = \PHI(x) \EQ x = \psi(y)$. Тогда
	\[G(\psi(y)) - F(y) = \const \SO \int f(y) \,dy = G(\psi(y)) + c \] 
\end{proof}

\begin{Example}
	$\int \frac{dx}{1 + \sqrt{x}}$. Пусть $t = \sqrt{x}, t > 0 \EQ t^2 = x \SO dx = dt^2 = 2t \,dt$. Тогда
	\[\int \frac{dx}{1 + \sqrt{x}} = \int \frac{2t \,dt}{1 + t} = \int \left( \frac{2t + 2}{t + 1} - \frac{2}{t + 1}\right) \,dt = \int \left(2 - \frac{2}{t + 1}\right) \,dt = 2\int dt - 2\int \frac{dt}{t + 1} =\] 
	\[= 2t - \int \frac{d(t + 1)}{t + 1} = 2t - 2\ln |t + 1| + c = 2\sqrt{x} - 2\ln (\sqrt{x} + 1) + c\]
\end{Example}

\begin{Example}
	$\int \sin x \cos x \,dx = \int \sin x \,d\sin x = \frac{\sin^2 x }{2} + c$. \\
	Иначе: $\int \sin x \cos x \,dx = -\int \cos x \,d\cos x = - \frac{\cos^2 x}{2} + c$. \\
	Иначе: $\int \sin x \cos x \,dx = \frac{1}{2} \int \sin 2x \,dx = \frac{1}{2} \cdot \frac{1}{2} \int \sin 2x \,d(2x) = \frac{-\cos 2x}{4} + c$. \\
	Мораль сей басни такова: константы разные, а не $ \frac{\sin^2 x}{2} = - \frac{\cos^2 x}{2} = - \frac{\cos 2x}{4}$. 
\end{Example}

\begin{Thm}[Формула интегрирования по частям]
	$f, g \in C^1 \langle A, B\rangle$. Тогда 
	\[\int f(x)g'(x) \,dx = f(x) g(x) - \int f'(x) g(x) \,dx\] 
\end{Thm}

\begin{proof}
	$H$ -- первообразная $g \cdot f'$. Тогда 
	\[(f(x) g(x) - H(x))' = f'(x) g(x) + f(x)g'(x) - H'(x) = f(x)g'(x)\]
\end{proof}

\begin{Rem}
	$\int u \,dv = uv - \int v \,du$
\end{Rem}

\begin{Example}
	$\int x e^x \,dx$. Пусть $u = x, u' = 1, v' = e^x, v = e^x$
	\[\int x e^x \,dx = xe^x - \int 1 \cdot e^x \,dx = x e^x - e^x + c\] 
\end{Example}

\begin{Example}
	$\int \ln x \,dx$. Пусть $u = \ln x, u' = \frac{1}{x}, v' = 1, v = x$.
	\[\int \ln x \,dx = x \ln x - \int \frac{1}{x} \cdot x \,dx = x \ln x - x + c\]  
\end{Example}

\begin{Ex}
	$\int e^x \cdot \sin x \,dx$
	Пусть $f = \sin x, g = e^x$. Тогда
	\[\int f \, dg = fg - \int g \, df \EQ \int e^x \sin x = e^x\sin x - \int e^x \cos x\]
	Пусть теперь $f = \cos x, g = e^x$. Тогда
	\[\int f \, dg = fg - \int g \, df \EQ \int e^x \cos x = e^x \cos x + \int e^x \sin x\]
	Отсюда 
	\[\int e^x \sin x = e^x\sin x - e^x\cos x - \int e^x \sin x \EQ \int e^x\sin x = \frac{e^x}{2}(\sin x - \cos x)\]
\end{Ex}

\begin{Example}
	Пусть $a \in \R, a \neq 0, I_n = \int \frac{dx}{(x^2 + a)^n}, n \in \N$. Выразим интеграл $I_{n + 1}$ через $I_n$ для произвольного натурального $n$.
	
	Обозначим $f(x) = \frac{1}{(x^2 + a)^n}$ и $g(x) = x$. Тогда
	\[df(x) = \left(\frac{1}{(x^2 + a)^n}\right)' \, dx = -\frac{2nx}{(x^2 + a)^{n + 1}} \, dx, dg(x) = dx\]
	По формуле интегрирования по частям:
	\begin{align*}
		I_n &= \frac{x}{(x^2 + a)^n} + 2n \int \frac{x^2}{(x^2 + a)^{n + 1}} \, dx = \frac{x}{(x^2 + a)^n} + 2n \int \frac{x^2 + a - a}{(x^2 + a)^{n + 1}} \, dx \\
		&= \frac{x}{(x^2 + a)^n} + 2n \int \frac{dx}{(x^2 + a)^n} - 2na \int \frac{dx}{(x^2 + a)^{n + 1}} = \frac{x}{(x^2 + a)^n} + 2n I_n - 2na I_{n + 1} \\
	\end{align*}
	Откуда
	\[2na I_{n + 1} = (2n - 1)I_n + \frac{x}{(x^2 + a)^n}\]
\end{Example} 

\begin{Prop}
	Любая рациональная функция имеет элементарную первообразную.
\end{Prop}

Рассмотрим простешие дроби:

\begin{MyList}
	\item $\frac{a}{(x + p)^n}, n \in \N, a, p \in \R$ 
	\item $ \frac{ax + b}{(x^2 + px + q)^n}$ 
\end{MyList}

Интегралы от простейших дробей первого рода вычисляются по таблице. Для простейших дробей второго рода используется следующий алгоритм:

\begin{MyList}
	\item Если $p \neq 0$, то выделим полный квадрат и выполним замену $y = x + \frac{p}{2}$. Если $p = 0$, тогда
	\[\int \frac{ax + b}{(x^2 + px + q)^n} = a\int \frac{x \,dx}{(x^2 + q)^n} + b \int \frac{dx}{(x^2 + q)^n}\]
	\item Интеграл $\int \frac{x \, dx}{(x^2 + q)^n}$ можно вычислить с помощью замены $y = x^2 + q$, т.к. $dy = 2x \, dx$.
	\item Применяя к интегралу $I_n = \int \frac{dx}{(x^2 + q)^n}$ формулу понижения $n - 1$ раз сведем его к интегралу $I_1$, который является табличным.
\end{MyList}

\begin{Example}[12 и 13 из таблицы]
	\[\int \frac{dx}{x^2 - 4} = \int \left( \frac{\frac{1}{4}}{x - 2} + \frac{-\frac{1}{4}}{x + 2}\right) \,dx = \frac{1}{4} \left(\ln |x - 2| - \ln |x + 2|\right) + c\]	
\end{Example}

\begin{Example}
	$\int \frac{dx}{\sqrt{x^2 + 1}}$. Пусть $x = \sh t, dx = \ch t dt $. Тогда
	\[\int \frac{\ch t dt}{\sqrt{1 + \sh^2 t}} = \int \frac{\ch t}{\ch t} dt = \int dt = t + c\]   
\end{Example}

\begin{Ex}
	Найди формулу для $(\sh t)^{-1}$ 
\end{Ex}

Неберущиеся интегралы:

\begin{multicols}{2}
	\begin{MyItemize}
		\item $\int \frac{\sin x}{x} \,dx$ 
		\item $\int \frac{\cos x}{x} \,dx$ 
		\item $\int \frac{\,dx}{\ln x}$ 
		\item $\int \frac{e^x}{x} \,dx$ 
		\item $\int \sin x^2 \,dx$ 
		\item $\int \cos x^2 \,dx$ 
		\item $\int e^{-x^2}\,dx$ 
	\end{MyItemize}
\end{multicols}

\Subsection{Определенный интеграл Римана}

\begin{Def}
	$[a, b], a < b$. Набор точек $\tau = \{x_k\}_{k = 0}^n : x_0 = a < x_1 < x_2 < ... < x_n = b$ -- разбиение (дробление) отрезка $[a, b]$, $\Delta x_k = x_{k + 1} - x_k$ -- длина отрезка $[x_k, x_{k + 1}]$ . 
	$\lambda = \lambda_\tau  = \max_{k \in [0, n - 1]} \Delta x_k$ -- ранг дробления (мелкость), $\xi = \{\xi_k\}_{k = 0}^{n - 1} : \xi_k \in [x_k, x_{k + 1}]$ -- оснащение дробления $\tau$.
	Пара $(\tau, \xi)$ называется оснащенным дроблением.  
\end{Def}

\begin{Def}
	$f : [a, b] \to \R, \sigma_\tau = \sum_{k = 0}^{n - 1} f(\xi_k) \Delta x_k$ -- суммы Римана (интегральные суммы). 
\end{Def}

\begin{figure*}[h]
	\centering
	\def\svgwidth{0.5\columnwidth}
	\input{img/riemann_sum.pdf_tex}
\end{figure*}

\begin{Def}
	$f : [a, b] \to \R$. Число $I \in \R$ называют пределом интегральных сумм при ранге $\to 0 :$
	\[I = \lim_{\lambda_\tau \to 0} \sigma_\tau (f, \xi) \quad (I = \lim_{\lambda \to 0} \sigma)\]
	если $\forall \varepsilon > 0 \ \exists \delta > 0 : \forall \tau : \lambda_\tau < \delta$
	\[|\sigma_\tau (f, \xi) - I| < \varepsilon\] 
\end{Def}

\begin{Rem}
	Последовательность оснащенных дроблений $\{(\tau^{(i)}, \xi^{(i)})\}_{i = 1}^\infty : \lambda^{(i)} \to 0$.
	$\forall \{\tau^{(i)}, \xi^{(i)}\} : \lambda^{(i)} \to 0 \ \sigma_{\tau^{(i)}}(f, \xi^{(i)}) \to I$.  
\end{Rem}

\begin{Def}[Интеграл Римана]
	$f : [a, b] \to \R$. Если $\exists \lim_{\lambda \to 0} \sigma = I $, то $f$ называется интегрируемой по Риману на $[a, b]$, а число $I$ называется интегралом $f$ по $[a, b]$. \\
	$R[a, b]$ -- класс функций, интегрируемых по Риману на $[a, b]$.
	\[\int_a^b f(x)\,dx\]
\end{Def}

\Subsection{Суммы Дарбу}

\begin{Def}
	$f : [a, b] \to \R, \tau = \{x_k\}_{k = 0}^n$ -- дробление $[a, b]$.
	\[M_k = \displaystyle{\sup_{x \in [x_k, x_{k + 1}]}} f(x), m_k = \displaystyle{\inf_{x \in [x_k, x_{k + 1}]}}f(x)\] 
	Суммы
	\[S = S_\tau (f) = \displaystyle{\sum_{k = 0}^{n - 1}} M_k \Delta x_k, s = s_\tau (f) = \displaystyle{\sum_{k = 0}^{n - 1}} m_k \Delta x_k\]
	называются верхними и нижними интегральными суммами.
\end{Def}

\begin{Rem}
	Если $f$ -- непрерывна на $[a, b]$, то это две частные суммы из сумм Римана.
\end{Rem}

\begin{Rem}
	$f$ ограничена сверху $\EQ$ $S$ ограничена.
\end{Rem}

Свойства сумм Дарбу:

\begin{MyList}
	\item $S_\tau (f) = \displaystyle{\sup_\xi \sigma_\tau (f, \xi)}, s_\tau = \displaystyle{\inf_\xi \sigma_\tau (f, \xi)}$ 
	\begin{proof}
		$M_k \geqslant f(\xi_k), k = 0, ..., n - 1$. Тогда $M_k \Delta x_k \geqslant f(\xi_k) \Delta x_k \EQ \sum_{k = 0}^{n - 1} M_k \Delta x_k \geqslant \sum_{k = 0}^{n - 1} f(\xi_k) \Delta x_k \SO S_\tau (f) \geqslant \sigma_\tau $, т.е. $S_\tau$ -- верхняя граница. Докажем, что она является точной верхней границей. \\
		Если $f$ ограничена на $[a, b]$. Фиксируем $\varepsilon > 0$. На каждом кусочке разбиения $\exists \xi_k^* \in [x_k, x_{k + 1}] : f(\xi_k^*) > M_k - \frac{\varepsilon}{b - a}$.
		Тогда $\sigma^* = \sum_{k = 0}^{n - 1} f(\xi_k^*) \Delta x_k > S - \frac{\varepsilon}{b - a}\sum_{k = 0}^{n - 1} \Delta x_k = S - \varepsilon$. \\
		Если $f$ не ограничена на $[a, b] \SO$ не ограничена на каком-то кусочке $[x_l, x_{l + 1}]$. 
		Фиксируем $A > 0$ и выберем $\xi_k^*$ при $k \neq l$ произвольно, а для $\xi_l^*$
		\[f(\xi_l^*) > \frac{1}{\Delta x_l}\left(A - \displaystyle{\sum_{k \neq l} f(\xi_k^*) \Delta x_k}\right)\]

		Тогда 
		\[\sigma^* = \displaystyle{\sum_{k = 0}^{n - 1} f(\xi_k^*) \Delta x_k} > A \SO \sup_\xi \sigma = +\infty = S\] 
	\end{proof}

	\item При добавлении новых точек дробления верхняя сумма не увеличится, а нижняя не уменьшится.
	\begin{proof}
		Докажем для верхних сумм при добавлении одной точки.
		$\tau : \{x_k\}_{k = 0}^{n - 1}$. Добавим точку $c$ в $[x_l, x_{l + 1}] - T$ -- новое дробление. \\
		\[S_\tau = \sum_{k = 0}^{l - 1} M_k \Delta x_k + M_l \Delta x_l + \sum_{k = l + 1}^{n - 1} M_k \Delta x_k\]
		\[S_T = \sum_{k = 0}^{l - 1} M_k \Delta x_k + (c - x_l) \cdot M' + (x_{l + 1} - c) M'' + \sum_{k = l + 1}^{n - 1} M_k \Delta x_k\]
		где $M' = \sup_{x \in [x_l, c]} f, M'' = \sup_{x \in [c, x_{l + 1}]} f$. $M_l \geqslant M', M_l \geqslant M''$, т.к. $[x_l, c] \subset [x_l, x_{l + 1}], [c, x_{l + 1}] \subset [x_l, x_{l + 1}]$.

		Рассмотрим $S_\tau - S_T = M_l \Delta x_l - (c - x_l) M' - (x_{l + 1} - c) M'' \geqslant M_l (x_{l + 1} - x_l - c + x_l - x_{l + 1} + c) = 0$.
		
		Добавить больше точек можно по индукции.
	\end{proof}

	\item Каждая нижняя сумма Дарбу не превосходит каждой верхней.
	\begin{proof}
		$\tau_1, \tau_2$ -- разные дробления $[a, b]$. Докажем, что $s_{\tau_1} \leqslant S_{\tau_2}$. Возьмем $\tau = \tau_1 \cup \tau_2$. Тогда $s_{\tau_1} \leqslant s_\tau \leqslant S_\tau \leqslant S_{\tau_2}$ (по свойству 2).
	\end{proof}
\end{MyList}

\begin{Prop}
	$f \in R[a, b] \SO f$ ограничена на $[a, b]$.
\end{Prop}

\begin{proof}
	Пусть $f$ не ограничена на $[a, b]$ сверху. Тогда $\forall \tau \SO \sup_\xi \sigma_\tau (f, \xi) = +\infty$. Тогда 
	$\forall \tau$ и числа $I \ \exists$ оснащение $\xi' : \sigma_\tau (\xi') > I + 1 \SO$ никакое число $I$ не является пределом интегральных сумм.   
\end{proof}

\begin{Def}
	$f : [a, b] \to \R$. Возьмем
	\[I^* = \inf_\tau S_\tau \qquad I_* = \sup_\tau s_\tau\]
	где $I^*$ -- верхний интеграл Дарбу, $I_*$ -- нижний интеграл Дарбу.
\end{Def}

\begin{Rem}
	$I^* \geqslant I_*$.
\end{Rem}

\begin{Rem}
	$f$ ограничена сверху $\EQ I^*$ ограничена. 
\end{Rem}

\begin{Thm}[Критерий интегрируемости функции]
	Пусть $f : [a, b] \to \R$. Тогда $f \in R[a, b] \EQ S_\tau (f) - s_\tau (f) \xrightarrow[\lambda \to 0]{} 0$, т.е.
	\[\forall \varepsilon > 0 \ \exists \delta > 0 : \forall \tau : \lambda_\tau < \delta \ S_\tau(f) - s_\tau(f) < \varepsilon\] 
\end{Thm}

\begin{proof}
	$\SO$. Пусть $f \in R[a, b]$. Обозначим $I = \int_a^b f$. Возьмем $\varepsilon > 0$, подберем $\delta > 0 :$
	\[I - \frac{\varepsilon}{3} < \sigma_\tau (f, \xi) < I + \frac{\varepsilon}{3}\]
	Переходя к супремуму и инфимуму, получим
	\[I - \frac{\varepsilon}{3} \leqslant s_\tau \leqslant S_\tau \leqslant I + \frac{\varepsilon}{3}\]
	откуда $S_\tau - s_\tau \leqslant I + \frac{\varepsilon}{3} - I + \frac{\varepsilon}{3} = \frac{2\varepsilon}{3} < \varepsilon$.

	$\Leftarrow$. Пусть $S_\tau - s_\tau \xrightarrow[\lambda \to 0]{} 0 \SO$ все суммы Дарбу конечны.
	\[s_\tau \leqslant I_* \leqslant I^* \leqslant S_\tau \SO 0 \leqslant I^* - I_* \leqslant S_\tau - s_\tau\]
	$\SO I^* = I_*$ (т.к. это числа). Обозначим $I = I^* = I_*$.
	\[s_\tau \leqslant I \leqslant S_\tau, s_\tau \leqslant \sigma_\tau \leqslant S_\tau \SO |I - \sigma_\tau| \leqslant S_\tau - s_\tau\]
	$\SO \forall \varepsilon > 0 \ \exists \delta > 0 : \forall \tau : \lambda_\tau < \delta \ |I - \sigma_\tau| < \varepsilon$.   
\end{proof}

\begin{Rem}
	Если $f \in R[a, b] \SO s_\tau \leqslant \int_a^b f \leqslant S_\tau$. 
\end{Rem}

\begin{Cons}
	$f \in R[a, b] \SO \lim_{\lambda \to 0} S_\tau = \lim_{\lambda \to 0} s_\tau = \int_a^b f$ 
\end{Cons}

\begin{proof}
	$0 \leqslant S_\tau - \int_a^b f \leqslant S_\tau - s_\tau$, $0 \leqslant \int_a^b f - s_\tau \leqslant S_\tau - s_\tau$.
\end{proof}

\begin{Rem}
	$\lim_{\lambda \to 0} S_\tau = I^*, \lim_{\lambda \to 0} s_\tau = I_*$. 
\end{Rem}

\begin{Prop}[Критерий Дарбу интегрируемости функции по Риману]
	$f \in R[a, b] \EQ f$ ограничена на $[a, b]$ и $I_* = I^*$. 
\end{Prop}

\begin{Prop}[Критерий Римана интегрируемости]
	$f \in R[a, b] \EQ \forall \varepsilon > 0 \ \exists \tau \ S_\tau(f) - s_\tau (f) < \varepsilon$. 
\end{Prop}

\begin{Def}
	$f : D \to \R$. Величина
	\[\omega (f)_D = \sup_{x, y \in D} (f(x) - f(y))\]
	называется колебанием $f$ на $D$. Из определений граней функции ясно, что
	\[\omega (f)_D = \sup_{x \in D} f(x) - \inf_{y \in D} f(y)\]

	Если задано $\tau$ отрезка $[a, b]$, то 
	\[\omega_k (f) = M_k - m_k\]
\end{Def}

Тогда теорему можно записать:
\[f \in R[a, b] \EQ \lim_{\lambda \to 0} \sum_{k = 0}^{n - 1} \omega_k (f) \Delta x_k = 0\]

\begin{Thm}[Интегрируемость непрерывной функции]
	$f : [a, b] \to \R, f \in C[a, b] \SO f \in R[a, b]$.
\end{Thm}

\begin{proof}
	По теореме Кантора $f \in C[a, b] \SO f$ равномерна непрерывна на $[a, b]$.
	\[\forall \varepsilon > 0 \ \exists \delta > 0 : \forall t', t'' \in [a, b] : |t' - t''| < \delta \ |f(t') - f(t'')| < \frac{\varepsilon}{b - a}\]
	По теореме Вейерштрасса $f$ достигает наибольшего и наименьшего значения на любом отрезке, содержащемся в $[a, b]$.
	Поэтому колебание $f$ на всяком отрезке, длина которого меньше $\delta$, будет меньше $\frac{\varepsilon}{b - a}$. Значит, $\forall \tau : \lambda_\tau < \delta$ 
	\[\sum_{k = 0}^{n - 1} \omega_k(f) \Delta x_k < \sum_{k = 0}^{n - 1} \frac{\varepsilon}{b - a} \Delta x_k\]
\end{proof}

\begin{Thm}[Интегрируемость монотонной функции]
	$f$ монотонна на $[a, b] \SO f \in R[a, b]$. 
\end{Thm}

\begin{proof}
	Пусть $f$ монотонно возрастает на $[a, b]$. Если $f(a) = f(b) \SO f$ постоянна $\SO f \in C[a, b] \SO f \in R[a, b]$. \\
	Если $f(a) < f(b)$. $\forall \varepsilon > 0$ возьмем $\delta = \frac{\varepsilon}{f(b) - f(a)}$. Возьмем произвольное $\tau : \lambda_\tau < \delta$ на $[x_k, x_{k + 1}]$. В силу монотонности $f$ верно $\omega_k(f) = f(x_{k + 1}) - f(x_k)$.
	\[\sum_{k = 0}^{n - 1} \omega_k(f) \Delta_k = \sum_{k = 0}^{n - 1} (f(x_{k + 1}) - f(x_k)) \Delta x_k < \sum_{k = 0}^n (f(x_{k + 1}) - f(x_k)) \cdot \frac{\varepsilon}{f(b) - f(a)} = \varepsilon\]   
\end{proof}

\begin{Rem}
	$f \in R[a, b]$. Если изменить значение $f$ в конечном числе точек, то интегрируемость не нарушится и интеграл не изменится.
\end{Rem}

\begin{proof}
	$\widetilde{f}$ -- отличается от $f$ в точках $t_1, t_2, ..., t_m$.
	$|f|$ ограничена на $[a, b] \SO |\widetilde{f}|$ ограничена.
	$|f| \leqslant A$, возьмем $\widetilde{A} = \max \{A, |\widetilde{f}(t_1)|, |\widetilde{f}(t_2)|, ..., |\widetilde{f}(t_m)|\}$.
	В интегральных суммах для $f$ и $\widetilde{f}$ отличаются не более $2m$ слагаемых, поэтому
	\[|\sigma_\tau(f, \xi) - \sigma_\tau(\widetilde{f}, \xi)| \leqslant 2m(A + \widetilde{A}) \lambda_\tau \xrightarrow[\lambda_\tau]{}0\]  
	Поэтому предел $\sigma_\tau (\widetilde{f}, \xi)$ существует и равен пределу $\sigma_\tau (f, \xi)$.  
\end{proof}

\begin{Thm}[Интегрируемость функции и её сужения]
	\begin{MyList}
		\item $f \in R[a, b], [\alpha, \beta] \subset [a, b] \SO f \in R[\alpha, \beta]$
		\item Если $a < c < b, f : [a, b] \to \R$ и $f \in R[a, c], f \in R[c, b]$, то $f \in R[a, b]$.   
	\end{MyList}
\end{Thm}

\begin{proof}
	\begin{MyList}
		\item Возьмем $\varepsilon > 0$, подберем $\delta > 0$ из критерия интегрируемости на $[a, b]$. \\
		$\tau_0$ -- дробление $[\alpha, \beta], \lambda_{\tau_0} < \delta$. Добавим точек до дробления $[a, b]$. Получим $\tau (\lambda_\tau < \delta)$.
		\[S_{\tau_0} - s_{\tau_0} = \sum_{k = l}^{m - 1} \omega_k (f) \Delta x_k \leqslant \sum_{k = 0}^{n - 1} \omega_k (f) \Delta x_k < \varepsilon\]

		\item Пусть $f$ не постоянна, т.е. $\omega(f)_{[a, b]} > 0$.
		Возьмем $\varepsilon > 0$, подберем $\delta_1, \delta_2 : \forall \tau_1 : \lambda_{\tau_1} < \delta_1, \forall tau_2 : \lambda_{\tau_2} < \delta_2$
		\[S_{\tau_1} - s_{\tau_1} < \frac{\varepsilon}{3}, S_{\tau_2} - s_{\tau_2} < \frac{\varepsilon}{3}\]
		$\delta = \min \{\delta_1, \delta_2, \frac{\varepsilon}{3 \omega}\}$. Пусть $\tau$ -- дробление $[a, b], \lambda_\tau < \delta$.
		Точка $c \in [x_l, x_{l + 1})$. Обозначим $\tau' = \tau \cup \{c\}, \tau_1 = \tau' \cap [a, c], \tau_2 = \tau' \cap [c, b]$
		\[S_\tau - s_\tau \leqslant S_{\tau_1} - s_{\tau_1} + S_{\tau_2} - s_{\tau_1} + \omega_l (f) \delta < \varepsilon\]   
	\end{MyList}
\end{proof}

\end{document}