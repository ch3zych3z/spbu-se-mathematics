\documentclass[12pt]{article}

% Автор стиля: Сергей Копелиович
% Автор конспекта: Илья Дудников

\usepackage{cmap}
\usepackage[T2A]{fontenc}
\usepackage[utf8]{inputenc}
\usepackage[russian]{babel}
\usepackage{graphicx}
\usepackage{amsthm,amsmath,amssymb}
\usepackage{listings}
\usepackage{color}
\usepackage{xcolor}
\usepackage{array}
\usepackage{mathrsfs}
\usepackage{epigraph}

\usepackage[russian,colorlinks=true,urlcolor=red,linkcolor=blue]{hyperref}
\usepackage{enumerate}
\usepackage{datetime}
\usepackage{fancyhdr}
\usepackage{lastpage}
\usepackage{verbatim}
\usepackage{tikz}
\usetikzlibrary{arrows,decorations.markings,decorations.pathmorphing}
\usepackage{pgfplots}

\usepackage{ifthen}
\usepackage{mathtools}

%\usepackage{tabls}
%\usepackage{tabularx}
%\usepackage{xifthen}
%\listfiles

\usepackage{import}
\usepackage{MnSymbol}
\usepackage{xifthen}
\usepackage{pdfpages}
\usepackage{transparent}

\def\NAME{Лекция, 10/27/21}

\sloppy
\voffset=-20mm
\textheight=235mm
\hoffset=-22mm
\textwidth=180mm
\headsep=12pt
\footskip=20pt

\parskip=0em
\parindent=0em

\setlength\epigraphwidth{.8\textwidth}

\newlength{\tmplen}
\newlength{\tmpwidth}
\newcounter{listcounter}

% Список с маленькими отступами
\newenvironment{MyList}[1][4pt]{
  \begin{enumerate}[1.]
  \setlength{\parskip}{0pt}
  \setlength{\itemsep}{#1}
}{       
  \end{enumerate}
}
% Вложенный список с маленькими отступами
\newenvironment{InnerMyList}[1][0pt]{
  \vspace*{-0.5em}
  \begin{enumerate}[(a)]
  \setlength{\parskip}{-0pt}
  \setlength{\itemsep}{#1}
}{       
  \end{enumerate}
  \vspace*{-0.5em}
}
% Список с маленькими отступами
\newenvironment{MyItemize}[1][4pt]{
  \begin{itemize}
  \setlength{\parskip}{0pt}
  \setlength{\itemsep}{#1}
}{       
  \end{itemize}
}

% Основные математические символы
\def\TODO{{\color{red}\bf TODO}}
\def\Q{\mathbb{Q}}
\def\N{\mathbb{N}}       %
\def\R{\mathbb{R}}       %
\def\F2{\mathbb{F}_2}    %
\def\Z{\mathbb{Z}}       %
\def\INF{\t{+}\infty}    % +inf
\def\EPS{\varepsilon}    %
\def\EMPTY{\varnothing}  %
\def\PHI{\varphi}        %
\def\SO{\Rightarrow}     % =>
\def\EQ{\Leftrightarrow} % <=>
\def\t{\texttt}          % mono font
\def\c#1{{\rm\sc{#1}}}   % font for classes NP, SAT, etc
\def\O{\mathcal{O}}      %
\def\NO{\t{\#}}          % #
\def\XOR{\text{ {\raisebox{-2pt}{\ensuremath{\Hat{}}}} }}
\renewcommand{\le}{\leqslant}
\renewcommand{\ge}{\geqslant}
\newcommand{\q}[1]{\langle #1 \rangle}               % <x>
\newcommand\URL[1]{{\footnotesize{\url{#1}}}}        %
% \newcommand{\sfrac}[2]{{\scriptscriptstyle\frac{#1}{#2}}}  % Очень маленькая дробь
% \newcommand{\mfrac}[2]{{\scriptstyle\frac{#1}{#2}}}    % Небольшая дробь
\newcommand{\sfrac}[2]{{\scriptstyle\frac{#1}{#2}}}  % Очень маленькая дробь
\newcommand{\mfrac}[2]{{\textstyle\frac{#1}{#2}}}    % Небольшая дробь

\newcommand{\fix}[1]{{\color{fixcolor}{#1}}} % \underline
\def\bonus{\t{\red{(*)}}}
\def\ifbonus#1{\ifthenelse{\equal{#1}{}}{}{\bonus}}
\def\smallsquare{$\scalebox{0.5}{$\square$}$}

\newlength{\myItemLength}
\setlength{\myItemLength}{0.3em}
\def\ItemSymbol{\smallsquare}
\def\Item{\vspace*{\myItemLength}\ItemSymbol \ \ }

\newcommand{\LET}{%
  % [line width=0.6pt]
  \begin{tikzpicture}%
  \draw(0.8ex,0) -- (0.8ex,1.6ex);%
  \draw(0,1.6ex) -- (0.8ex,1.6ex);%
  \end{tikzpicture}%
  \hspace*{0.1em}%
}

% Отступы
\def\makeparindent{\hspace*{\parindent}\unskip}
\def\up{\vspace*{-0.5em}}%{\vspace*{-\baselineskip}}
\def\down{\vspace*{0.5em}}
\def\LINE{\vspace*{-1em}\noindent \underline{\hbox to 1\textwidth{{ } \hfil{ } \hfil{ } }}}
\def\BOX#1{\mbox{\fbox{\bf{#1}}}}
\def\Pagebreak{\pagebreak\vspace*{-1.5em}}

% Мелкий заголовок
\newcommand{\THEE}[1]{
  \vspace*{0.5em}
  \noindent{\bf \underline{#1}}%\hspace{0.5em}
  \vspace*{0.2em}
}
% Другой тип мелкого заголовка
\newcommand{\THE}[1]{
  \vspace*{0.5em} $\bullet$
  \noindent{\bf #1}%\hspace{0.5em}
  \vspace*{0.2em}
}

\newenvironment{MyTabbing}{
  \t\bgroup
  \vspace*{-\baselineskip}
  \begin{tabbing}
    aaaa\=aaaa\=aaaa\=aaaa\=aaaa\=aaaa\kill
}{
  \end{tabbing}
  \t\egroup
}

% Код с правильными отступами
\lstnewenvironment{code}{
  \lstset{}
%  \vspace*{-0.2em}
}%
{
%  \vspace*{-0.2em}
}
\lstnewenvironment{codep}{
  \lstset{language=python}
}%
{
}

% Формулы с правильными отступами
\newenvironment{smallformula}{
 
  \vspace*{-0.8em}
}{
  \vspace*{-1.2em}
  
}
\newenvironment{formula}{
 
  \vspace*{-0.4em}
}{
  \vspace*{-0.6em}
  
}

% Большая квадратная скобка
\makeatletter
\newenvironment{sqcases}{%
  \matrix@check\sqcases\env@sqcases
}{%
  \endarray\right.%
}
\def\env@sqcases{%
  \let\@ifnextchar\new@ifnextchar
  \left\lbrack
  \def\arraystretch{1.2}%
  \array{@{}l@{\quad}l@{}}%
}
\makeatother

\DeclareMathOperator{\sort}{sort}

% Определяем основные секции: \begin{Lm}, \begin{Thm}, \begin{Def}, \begin{Rem}
\renewcommand{\qedsymbol}{$\blacksquare$}
\theoremstyle{definition} % жирный заголовок, плоский текст
\newtheorem{Thm}{\underline{Теорема}}[subsection] % нумерация будет "<номер subsection>.<номер теоремы>"
\newtheorem{Lm}[Thm]{\underline{Lm}} % Нумерация такая же, как и у теорем
\newtheorem{Ex}[Thm]{Упражнение} % Нумерация такая же, как и у теорем
\newtheorem{Example}[Thm]{Пример} % Нумерация такая же, как и у теорем
\newtheorem{Solution}[Thm]{Решение}
\newtheorem{Code}[Thm]{Код} % Нумерация такая же, как и у теорем
\theoremstyle{plain} % жирный заголовок, курсивный текст
\newtheorem{Def}[Thm]{Def.} % Нумерация такая же, как и у теорем
\theoremstyle{remark} % курсивный заголовок, плоский текст
\newtheorem{Cons}[Thm]{Следствие} % Нумерация такая же, как и у теорем
\newtheorem{Conj}[Thm]{Гипотеза} % Нумерация такая же, как и у теорем
\newtheorem{Prop}[Thm]{Утверждение} % Нумерация такая же, как и у теорем
\newtheorem{Rem}[Thm]{Замечание} % Нумерация такая же, как и у теорем
\newtheorem{Remark}[Thm]{Замечание} % Нумерация такая же, как и у теорем
\newtheorem{Algo}[Thm]{Алгоритм} % Нумерация такая же, как и у теорем

% Определяем ЗАГОЛОВКИ
\def\SectionName{Матанализ}
\def\AuthorName{Илья Дудников}

\newlength{\sectionvskip}
\setlength{\sectionvskip}{0.5em}
\newcommand{\Section}[4][]{
  % Заголовок
  \pagebreak
%  \ifthenelse{\isempty{#1}}{
    \refstepcounter{section}
%  }{}
  \vspace{0.5em}
%  \ifthenelse{\isempty{#1}}{
%    \addtocontents{toc}{\protect\addvspace{-5pt}}%
    \addcontentsline{toc}{section}{\arabic{section}. #2}
%  }{}
  


  % Запомнили название и автора главы
  \gdef\SectionName{#2}
  \gdef\AuthorName{#4}

  % Заголовок страницы
  \lhead{Конспект, \NAME}
  \chead{}
  \rhead{\SectionName}
  \renewcommand{\headrulewidth}{0.4pt}
    
  \lfoot{}
  \cfoot{\thepage\t{/}\pageref*{LastPage}}
  \rfoot{Автор: \AuthorName}
  \renewcommand{\footrulewidth}{0.4pt}
}

\newcommand{\Subsection}[2][]{
  \refstepcounter{subsection}
  \vspace*{1em}
  \ifthenelse{\equal{#1}{}}
    {\addcontentsline{toc}{subsection}{\arabic{section}.\arabic{subsection}. #2}}
    {\addcontentsline{toc}{subsection}{\arabic{section}.\arabic{subsection}. \bonus\,#2}}
  {\color{blue}\bf\large \arabic{section}.\arabic{subsection}. \ifbonus{#1}\,{#2}} 
  \vspace*{0.5em}
  \makeparindent
}
\newcommand{\Subsubsection}[2][]{
  \refstepcounter{subsubsection}
  \vspace*{1em}
  \ifthenelse{\equal{#1}{}}
    {\addcontentsline{toc}{subsubsection}{\arabic{section}.\arabic{subsection}.\arabic{subsubsection}. #2}}
    {\addcontentsline{toc}{subsubsection}{\arabic{section}.\arabic{subsection}.\arabic{subsubsection}. \bonus\,#2}}
  {\color{blue}\bf\large \arabic{section}.\arabic{subsection}.\arabic{subsubsection}. \ifbonus{#1}\,#2}
  \vspace*{0.5em}
  \makeparindent
}

\newcommand{\Header}{
  \pagestyle{empty}
  \renewcommand{\dateseparator}{--}
  \begin{center}
    {\Large\bf 
    Матанализ 1 семестр ПИ,\\
    \vspace{0.3em}
     \NAME}\\
    \vspace{0.7em}
    {Собрано {\today} в {\currenttime}}
  \end{center}

  \LINE
  \vspace{0em}

  \renewcommand{\baselinestretch}{0.98}\normalsize
  \tableofcontents
  \renewcommand{\baselinestretch}{1.0}\normalsize
  \pagebreak
}

\newcommand{\BeginConspect}{
  \pagestyle{fancy}
  \setcounter{page}{1}
}

\definecolor{mygray}{rgb}{0.7,0.7,0.7}
\definecolor{ltgray}{rgb}{0.9,0.9,0.9}
\definecolor{fixcolor}{rgb}{0.7,0,0}
\definecolor{red2}{rgb}{0.7,0,0}
\definecolor{dkred}{rgb}{0.4,0,0}
\definecolor{dkblue}{rgb}{0,0,0.6}
\definecolor{dkgreen}{rgb}{0,0.6,0}
\definecolor{brown}{rgb}{0.5,0.5,0}

\newcommand{\green}[1]{{\color{green}{#1}}}
\newcommand{\black}[1]{{\color{black}{#1}}}
\newcommand{\red}[1]{{\color{red}{#1}}}
\newcommand{\dkred}[1]{{\color{dkred}{#1}}}
\newcommand{\blue}[1]{{\color{blue}{#1}}}
\newcommand{\dkgreen}[1]{{\color{dkgreen}{#1}}}

\newcommand{\Mod}[1]{\ (\mathrm{mod}\ #1)}

\begin{document}

\Header

\BeginConspect

\Section{Замечательные пределы}{}{Илья Дудников}

\begin{Def}
    Первый замечательный предел:
    \[\lim_{x \to \infty} \frac{\sin x}{x} = 1\]    
\end{Def}

\begin{proof}
    $\sin x < x < \tg x$ на $\left(0, \frac{\pi}{2}\right)$, $\cos x < \frac{\sin x}{x} < 1$ \\
    $\cos x, \frac{\sin x}{x}, 1$ -- четные функции, значит верно и для $x \in \left(-\frac{\pi}{2}, 0\right)$. Перейдем к пределу при $x \to 0$
    \[1 \leqslant \lim_{x \to 0} \frac{\sin x}{x} \leqslant 1 \SO \exists \lim_{x \to 0} \frac{\sin x}{x} = 1\]
\end{proof}

\begin{Cons}
    $\lim_{x \to 0} \frac{1 - \cos x}{x^2} = \frac{1}{2}$ 
\end{Cons}

\begin{proof}
    $\cos 2\alpha = 1 - 2\sin^2 \alpha \EQ \sin^2 \alpha = \frac{1 - \cos 2\alpha}{2}$ 
    \[ \frac{1 - \cos x}{2} = \frac{2 \sin^2 \frac{x}{2}}{x^2} = \frac{1}{2} \cdot \frac{\sin^2 \frac{x}{2}}{\left(\frac{x}{2}\right)^2} = \frac{1}{2} \left( \frac{\sin \frac{x}{2}}{\frac{x}{2}}^2 \to \frac{1}{2}\right)\]
\end{proof}

\begin{Cons}
    $\lim_{x \to 0} \frac{\tg x}{x} = 1$ 
\end{Cons}

\begin{proof}
    \[\frac{\tg x}{x} = \frac{\sin x}{x} \cdot \frac{1}{\cos x} \xrightarrow[x \to 0]{} 1\]
\end{proof}

\begin{Cons}
    $\lim_{x \to 0} \frac{\arcsin x}{x} = 1$ 
\end{Cons}

\begin{proof}
    $\frac{\sin x}{x} = \frac{y}{\arcsin y}$, $y = \sin x$ в окрестности $x \in (-\varepsilon, \varepsilon) \SO \arcsin y = x$. 
    $\arcsin x$ непрерывна в нуле, в $0$ равен $0$. $ \frac{\sin x}{x}$ непрерывна в $(-\varepsilon, \varepsilon) \setminus \{0\}$  

    \[g(x) = \begin{cases}
        \frac{\sin x}{x}, x \neq 0 \\
        1, x = 0
    \end{cases} \text { -- непрерывна на } \R\]
    $\SO$ по теореме о непрерывности композиции $\frac{y}{\arcsin y} \xrightarrow[y \to 0]{} 1$  
\end{proof}

\begin{Cons}
    $\lim_{x \to 0} \frac{\arctg x}{x} = 1$ 
\end{Cons}

\begin{proof}
    Аналогично предыдущему следствию.
\end{proof}

\begin{Def}
    Второй замечательный предел
    \begin{align*}
        &\lim_{x \to +\infty} \left(1 + \frac{1}{x}\right)^x = e \\
        &\lim_{x \to -\infty} \left(1 + \frac{1}{x}\right)^x = e \\
        &\lim_{x \to 0}(1 + x)^{\frac{1}{x}} = e
    \end{align*}
\end{Def}

\begin{proof}
    $f(x) = \left(1 + \frac{1}{x}\right)^x$ задана на $\R \setminus [-1, 0]$. Пусть $x_n \to +\infty$. Нужно доказать, что $f(x_n) \to e$.
    
    \begin{MyList}
        \item Рассмотрим $\{x_n\}$ из $\N$. $f(x_n) \to e$ как подпоследовательность.
        \item $\{x_n\}$ из $\R$. Начиная с некоторого номера $x_n \geqslant 1$.
        \[\left(1 + \frac{1}{[x_n] + 1}\right)^{[x_n]} \leqslant \left(1 + \frac{1}{x_n}\right)^{x_n} \leqslant \left(1 + \frac{1}{[x_n]}\right)^{[x_n] + 1}\]
        Очевидно, $[x_n] \leqslant x_n \leqslant [x_n] + 1$. Тогда
        \[\frac{1}{1 + \frac{1}{[x_n] + 1}} \cdot f([x_n] + 1) \leqslant f(x_n) \leqslant f([x_n]) \cdot \left(1 + \frac{1}{[x_n]}\right)\] 
        $\{[x_n]\}_{n = 1}^\infty$ -- последовательность из $\N$. Выполним предельный переход в неравенстве.
        \[e \leqslant \lim_{n \to \infty} f(x_n) \leqslant e \SO \exists \lim_{n \to \infty} f(x_n) = e\]  
    \end{MyList}
\end{proof}

\begin{Def}
    Третий замечательный предел (обычно не нумеруется).
    \[\lim_{x \to 0} \frac{\log_a (1 + x)}{x} = \frac{1}{\ln a}, a > 0, a \neq 1\]
\end{Def}

\begin{proof}
    $\log_a (1 + x) = \frac{\ln (1 + x)}{\ln a}$ 
    \[\lim_{x \to 0} \frac{\ln (1 + x)}{x} = \lim_{x \to 0} \ln (1 + x)^{\frac{1}{x}} = \ln e = 1\]
\end{proof}

\begin{Def}
    Четвертый замечательный предел (обычно не нумеруется)
    \[\lim_{x \to 0} \frac{(1 + x)^\alpha - 1}{x} = \alpha, \alpha \in \R\]
\end{Def}

\begin{proof}
    $\alpha = 0$ тривиально. \\
    $\alpha \neq 0$. $x_n \to 0, x_n \neq 0, |x_n| < 1 \ \forall n$. Обозначим 
    \[y_n = (1 + x_n)^\alpha - 1 \xrightarrow[n \to \infty]{} 0, y_n \neq 0 \SO \alpha \ln(1 + x_n) = \ln(1 + y_n)\]   
    Тогда
    \[ \frac{(1 + x_n)^\alpha - 1}{x_n} = \frac{y_n}{x_n} = \frac{y_n}{\ln (1 + y_n)} \cdot \frac{\alpha \ln(1 + x_n)}{x_n} \to \alpha\]
\end{proof}

\begin{Def}
    Пятый замечательный предел (обычно не нумеруется)
    $\lim_{x \to 0} \frac{a^x - 1}{x} = \ln a, a > 0$ 
\end{Def}

\begin{proof}
    $a = 1$ тривиально. \\
    $a \neq 1$. $x_n \to 0, x_n \neq 0$
    \[y_n = a^{x_n} - 1 \to 0, y_n \neq 0, \ln (1 + y_n) = x_n \cdot \ln a\]  
    \[ \frac{a^{x_n} - 1}{x_n} = \frac{y_n}{x_n} = \frac{y_n}{\ln(1 + y_n)} \cdot \frac{x_n \ln a}{x_n} \xrightarrow[n \to \infty]{} \ln a\]
\end{proof}

\Subsection{Сравнение функций}

\begin{Def}
    $f, g : D \to \R, D \subset \R, x_0$ -- предельная точка $D$ и $\exists \PHI : D \to \R : f(x) = \PHI (x) \cdot g(x) $ в $\dot{V}(x_0) \cap D$.
    
    \begin{MyList}
        \item Если $\PHI(x)$ ограничена на $\dot{V}(x_0) \cap D$, то говорят, что $f$ ограничена по сравнению с $g$ при $x \to x_0$
        \[f(x) = O(g(x)), x \to x_0\]
        \item Если $\PHI(x) \xrightarrow[x \to x_0]{} 0$, то говорят, что $f$ бесконечно малая по сравнению с $g$ при $x \to x_0$
        \[f(x) = o(g(x)), x \to x_0\]
        \item Если $\PHI(x) \xrightarrow[x \to x_0]{} 1$, то говорят, что $f$ и $g$ асимптотически равны.
        \[f(x) \thicksim g(x), x \to x_0\]
    \end{MyList}
\end{Def}

\begin{Rem}
    \begin{MyList}
        \item $ \frac{f(x)}{g(x)}$ ограничена в $\dot{V} \cap D$
        \item $\lim_{x \to x_0} \frac{f(x)}{g(x)} = 0$
        \item $\lim_{x \to x_0} \frac{f(x)}{g(x)} = 1$    
    \end{MyList}
\end{Rem}

\end{document}