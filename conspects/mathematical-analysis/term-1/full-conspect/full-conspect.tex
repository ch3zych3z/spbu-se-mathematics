\documentclass[12pt]{article}

% Автор: Игорь Смирнов
% Автор стиля: Илья Дудников

\usepackage{cmap}
\usepackage[T2A]{fontenc}
\usepackage[utf8]{inputenc}
\usepackage[russian]{babel}
\usepackage{graphicx}
\usepackage{amsthm,amsmath,amssymb}
\usepackage{listings}
\usepackage{color}
\usepackage{xcolor}
\usepackage{array}
\usepackage{epigraph}

\usepackage[russian,colorlinks=true,urlcolor=red,linkcolor=blue]{hyperref}
\usepackage{enumerate}
\usepackage{datetime}
\usepackage{fancyhdr}
\usepackage{lastpage}
\usepackage{verbatim}
\usepackage{tikz}
\usepackage{MnSymbol}
\usetikzlibrary{arrows,decorations.markings,decorations.pathmorphing}
\usepackage{pgfplots}

\usepackage{ifthen}
\usepackage{mathtools}

%\usepackage{tabls}
%\usepackage{tabularx}
%\usepackage{xifthen}
%\listfiles

\def\NAME{Лекции}
\def\SEASON{Конспект лекций по матанализу, ПИ, 1 семестр}

\sloppy
\voffset=-20mm
\textheight=235mm
\hoffset=-22mm
\textwidth=180mm
\headsep=12pt
\footskip=20pt

\parskip=0em
\parindent=0em

\setlength\epigraphwidth{.8\textwidth}

\newlength{\tmplen}
\newlength{\tmpwidth}
\newcounter{listcounter}

% Список с маленькими отступами
\newenvironment{MyList}[1][4pt]{
  \begin{enumerate}[1.]
  \setlength{\parskip}{0pt}
  \setlength{\itemsep}{#1}
}{       
  \end{enumerate}
}
% Вложенный список с маленькими отступами
\newenvironment{InnerMyList}[1][0pt]{
  \vspace*{-0.5em}
  \begin{enumerate}[(a)]
  \setlength{\parskip}{-0pt}
  \setlength{\itemsep}{#1}
}{       
  \end{enumerate}
  \vspace*{-0.5em}
}
% Список с маленькими отступами
\newenvironment{MyItemize}[1][4pt]{
  \begin{itemize}
  \setlength{\parskip}{0pt}
  \setlength{\itemsep}{#1}
}{       
  \end{itemize}
}

% Основные математические символы
\def\TODO{{\color{red}\bf TODO}}
\def\C{\mathbb{C}}       %
\def\Q{\mathbb{Q}}       %
\def\N{\mathbb{N}}       %
\def\R{\mathbb{R}}       %
\def\F2{\mathbb{F}_2}    %
\def\Z{\mathbb{Z}}       %
\def\INF{\t{+}\infty}    % +inf
\def\EPS{\varepsilon}    %
\def\EMPTY{\varnothing}  %
\def\PHI{\varphi}        %
\def\SO{\Rightarrow}     % =>
\def\EQ{\Leftrightarrow} % <=>
\def\t{\texttt}          % mono font
\def\c#1{{\rm\sc{#1}}}   % font for classes NP, SAT, etc
\def\O{\mathcal{O}}      %
\def\NO{\t{\#}}          % #
\def\XOR{\text{ {\raisebox{-2pt}{\ensuremath{\Hat{}}}} }}
\renewcommand{\le}{\leqslant}
\renewcommand{\ge}{\geqslant}
\newcommand{\q}[1]{\langle #1 \rangle}               % <x>
\newcommand\URL[1]{{\footnotesize{\url{#1}}}}        %
% \newcommand{\sfrac}[2]{{\scriptscriptstyle\frac{#1}{#2}}}  % Очень маленькая дробь
% \newcommand{\mfrac}[2]{{\scriptstyle\frac{#1}{#2}}}    % Небольшая дробь
\newcommand{\sfrac}[2]{{\scriptstyle\frac{#1}{#2}}}  % Очень маленькая дробь
\newcommand{\mfrac}[2]{{\textstyle\frac{#1}{#2}}}    % Небольшая дробь

\newcommand{\fix}[1]{{\color{fixcolor}{#1}}} % \underline
\def\bonus{\t{\red{(*)}}}
\def\ifbonus#1{\ifthenelse{\equal{#1}{}}{}{\bonus}}
\def\smallsquare{$\scalebox{0.5}{$\square$}$}

\newlength{\myItemLength}
\setlength{\myItemLength}{0.3em}
\def\ItemSymbol{\smallsquare}
\def\Item{\vspace*{\myItemLength}\ItemSymbol \ \ }

\newcommand{\LET}{%
  % [line width=0.6pt]
  \begin{tikzpicture}%
  \draw(0.8ex,0) -- (0.8ex,1.6ex);%
  \draw(0,1.6ex) -- (0.8ex,1.6ex);%
  \end{tikzpicture}%
  \hspace*{0.1em}%
}

% Отступы
\def\makeparindent{\hspace*{\parindent}\unskip}
\def\up{\vspace*{-0.5em}}%{\vspace*{-\baselineskip}}
\def\down{\vspace*{0.5em}}
\def\LINE{\vspace*{-1em}\noindent \underline{\hbox to 1\textwidth{{ } \hfil{ } \hfil{ } }}}
\def\BOX#1{\mbox{\fbox{\bf{#1}}}}
\def\Pagebreak{\pagebreak\vspace*{-1.5em}}

% Мелкий заголовок
\newcommand{\THEE}[1]{
  \vspace*{0.5em}
  \noindent{\bf \underline{#1}}%\hspace{0.5em}
  \vspace*{0.2em}
}
% Другой тип мелкого заголовка
\newcommand{\THE}[1]{
  \vspace*{0.5em} $\bullet$
  \noindent{\bf #1}%\hspace{0.5em}
  \vspace*{0.2em}
}

\newenvironment{MyTabbing}{
  \t\bgroup
  \vspace*{-\baselineskip}
  \begin{tabbing}
    aaaa\=aaaa\=aaaa\=aaaa\=aaaa\=aaaa\kill
}{
  \end{tabbing}
  \t\egroup
}

% Код с правильными отступами
\lstnewenvironment{code}{
  \lstset{}
%  \vspace*{-0.2em}
}%
{
%  \vspace*{-0.2em}
}
\lstnewenvironment{codep}{
  \lstset{language=python}
}%
{
}

% Формулы с правильными отступами
\newenvironment{smallformula}{
 
  \vspace*{-0.8em}
}{
  \vspace*{-1.2em}
  
}
\newenvironment{formula}{
 
  \vspace*{-0.4em}
}{
  \vspace*{-0.6em}
  
}

% Большая квадратная скобка
\makeatletter
\newenvironment{sqcases}{%
  \matrix@check\sqcases\env@sqcases
}{%
  \endarray\right.%
}
\def\env@sqcases{%
  \let\@ifnextchar\new@ifnextchar
  \left\lbrack
  \def\arraystretch{1.2}%
  \array{@{}l@{\quad}l@{}}%
}
\makeatother

% Определяем основные секции: \begin{Lm}, \begin{Thm}, \begin{Def}, \begin{Rem}
\renewcommand{\qedsymbol}{$\blacksquare$}
\theoremstyle{definition} % жирный заголовок, плоский текст
\newtheorem{Thm}{\underline{Теорема}}[subsection] % нумерация будет "<номер subsection>.<номер теоремы>"
\newtheorem{Lm}[Thm]{\underline{Lm}} % Нумерация такая же, как и у теорем
\newtheorem{Ex}[Thm]{Упражнение} % Нумерация такая же, как и у теорем
\newtheorem{Example}[Thm]{Пример} % Нумерация такая же, как и у теорем
\newtheorem{Code}[Thm]{Код} % Нумерация такая же, как и у теорем
\theoremstyle{plain} % жирный заголовок, курсивный текст
\newtheorem{Def}[Thm]{Def} % Нумерация такая же, как и у теорем
\theoremstyle{remark} % курсивный заголовок, плоский текст
\newtheorem{Cons}[Thm]{Следствие} % Нумерация такая же, как и у теорем
\newtheorem{Conj}[Thm]{Гипотеза} % Нумерация такая же, как и у теорем
\newtheorem{Prop}[Thm]{Утверждение} % Нумерация такая же, как и у теорем
\newtheorem{Rem}[Thm]{Замечание} % Нумерация такая же, как и у теорем
\newtheorem{Remark}[Thm]{Замечание} % Нумерация такая же, как и у теорем
\newtheorem{Algo}[Thm]{Алгоритм} % Нумерация такая же, как и у теорем

% Определяем ЗАГОЛОВКИ
\def\SectionName{unknown}
\def\AuthorName{unknown}

\newlength{\sectionvskip}
\setlength{\sectionvskip}{0.5em}
\newcommand{\Section}[4][]{
  % Заголовок
  \pagebreak
%  \ifthenelse{\isempty{#1}}{
    \refstepcounter{section}
%  }{}
  \vspace{0.5em}
%  \ifthenelse{\isempty{#1}}{
%    \addtocontents{toc}{\protect\addvspace{-5pt}}%
    \addcontentsline{toc}{section}{\arabic{section}. #2}
%  }{}
  \begin{center}
    {\Large \bf Раздел \NO{\arabic{section}}: #2} \\ 
    \vspace{\sectionvskip}
    \ifthenelse{\equal{#3}{}}{}{{\large #3}\\}
  \end{center}

  \LINE

  % Запомнили название и автора главы
  \gdef\SectionName{#2}
  \gdef\AuthorName{#4}

  % Заголовок страницы
  \lhead{\SEASON}
  \chead{}
  \rhead{\SectionName}
  \renewcommand{\headrulewidth}{0.4pt}

  \lfoot{Глава \NO{\arabic{section}}.}
  \cfoot{\thepage\t{/}\pageref*{LastPage}}
  \rfoot{Автор: \AuthorName}
  \renewcommand{\footrulewidth}{0.4pt}
}

\newcommand{\Subsection}[2][]{
  \refstepcounter{subsection}
  \vspace*{1em}
  \ifthenelse{\equal{#1}{}}
    {\addcontentsline{toc}{subsection}{\arabic{section}.\arabic{subsection}. #2}}
    {\addcontentsline{toc}{subsection}{\arabic{section}.\arabic{subsection}. \bonus\,#2}}
  {\color{blue}\bf\large \arabic{section}.\arabic{subsection}. \ifbonus{#1}\,{#2}} 
  \vspace*{0.5em}
  \makeparindent
}
\newcommand{\Subsubsection}[2][]{
  \refstepcounter{subsubsection}
  \vspace*{1em}
  \ifthenelse{\equal{#1}{}}
    {\addcontentsline{toc}{subsubsection}{\arabic{section}.\arabic{subsection}.\arabic{subsubsection}. #2}}
    {\addcontentsline{toc}{subsubsection}{\arabic{section}.\arabic{subsection}.\arabic{subsubsection}. \bonus\,#2}}
  {\color{blue}\bf\large \arabic{section}.\arabic{subsection}.\arabic{subsubsection}. \ifbonus{#1}\,#2}
  \vspace*{0.5em}
  \makeparindent
}

\newcommand{\Header}{
  \pagestyle{empty}
  \renewcommand{\dateseparator}{--}
  \begin{center}
    {\Large\bf 
     Матанализ 1 семестр ПИ,\\
    \vspace{0.3em}
    \NAME}\\
    \vspace{0.7em}
    {Собрано {\today} в {\currenttime}}
  \end{center}

  \LINE
  \vspace{0em}

  \renewcommand{\baselinestretch}{0.98}\normalsize
  \tableofcontents
  \renewcommand{\baselinestretch}{1.0}\normalsize
  \pagebreak
}

\newcommand{\BeginConspect}{
  \pagestyle{fancy}
  \setcounter{page}{1}
}

\definecolor{mygray}{rgb}{0.7,0.7,0.7}
\definecolor{ltgray}{rgb}{0.9,0.9,0.9}
\definecolor{fixcolor}{rgb}{0.7,0,0}
\definecolor{red2}{rgb}{0.7,0,0}
\definecolor{dkred}{rgb}{0.4,0,0}
\definecolor{dkblue}{rgb}{0,0,0.6}
\definecolor{dkgreen}{rgb}{0,0.6,0}
\definecolor{brown}{rgb}{0.5,0.5,0}

\newcommand{\green}[1]{{\color{green}{#1}}}
\newcommand{\black}[1]{{\color{black}{#1}}}
\newcommand{\red}[1]{{\color{red}{#1}}}
\newcommand{\dkred}[1]{{\color{dkred}{#1}}}
\newcommand{\blue}[1]{{\color{blue}{#1}}}
\newcommand{\dkgreen}[1]{{\color{dkgreen}{#1}}}

\newcommand{\Mod}[1]{\ (\mathrm{mod}\ #1)}

\DeclareMathOperator{\Real}{Re}
\DeclareMathOperator{\Imag}{Im}
\DeclareMathOperator{\lcm}{lcm}
\DeclareMathOperator{\sign}{sign}

\begin{document}

\Header

\BeginConspect

\Section{Аксиомы вещественных чисел}{}{Вячеслав Бучин}

\Subsection{Аксиомы сложения ($\R \times \R \rightarrow \R$)}

\begin{MyList}
	\item $\forall a, b \in \R \rightarrow a + b = b + a$ (коммутативность сложения)
	\item $ \forall a, b, c \in \R \rightarrow (a + b) + c = a + (b + c) $ (ассоциативность сложения)
	\item $ \exists 0 \in \R: \forall a \in \R \rightarrow a + 0 = a$ (существование нуля)
	\item $ \forall a \in \R \rightarrow \exists (-a) \in \R: a + (-a) = 0 $ ($(-a)$ --- противоположное число для $a$)
\end{MyList}

\Subsection{Аксиомы умножения ($\R \times \R \rightarrow \R$)}

\begin{MyList}
	\item $ \forall a, b \in \R \rightarrow a \cdot b = b \cdot a $ (коммутативность умножения)
	\item $ \forall a, b, c \in \R \rightarrow (a \cdot b) \cdot c = a \cdot (b \cdot c) $ (ассоциативность умножения)
	\item $ \exists 1 \in \R, 1 \neq 0:  \forall x \in \R \rightarrow 1 \cdot x = x $ (существование единицы)
	\item $ \forall a \in \R, a \neq 0 \rightarrow \exists \frac{1}{a} : a \cdot \frac{1}{a} = 1 $ ($\frac{1}{a}$ --- обратное число для $a$)
\end{MyList}

\Subsection{Дистрибутивность умножения относительно сложения}

$ \forall a, b, c \in \R \rightarrow (a + b) \cdot c = a \cdot c + b \cdot c $

\Subsection{Аксиомы порядка ($ \forall a, b \in \R $ установлено отношение $ a \leq b $ или $ b \leq a $)}

\begin{MyList}
	\item $\forall a \in \R \rightarrow a \leq a$ (рефлексивность)
	\item $\forall a, b, c \in \R : a \leq b , b \leq c \rightarrow a \leq c$ (транзитивность)
	\item $\forall a, b \in \R : a \leq b, b \leq a \rightarrow a = b$ (антисимметричность)
	\item $\forall a, b \in \R \rightarrow a \leq b \text{ или } b \leq a$
	\item $\forall a, b, c \in \R : a \leq b \rightarrow a + c \leq b + c$
	\item $\forall a, b \in R: 0 \leq a, 0 \leq b \rightarrow 0 \leq a \cdot b$
\end{MyList}

\Subsection{Ещё несколько определений}

\begin{MyItemize}
	\item $ a \leq b \EQ b \geq a $ (определение $\geq$)
	\item $ a < b \EQ a \leq b \text{ и } a \neq b $ (определение $<$)
	\item $ a > b \EQ b < a $ (определение $>$)
\end{MyItemize}

\Subsection{Аксиома полноты}

$ \forall A, B \subset \R: A \neq \EMPTY , B \neq \EMPTY : \forall x \in A, \forall y \in B, x \leq y \rightarrow \exists c \in \R : x \leq c \leq y $

\Pagebreak
\Subsection{Следствия из аксиом множества действительных чисел}
\begin{Cons}
	Число 0 единственно
	\begin{proof}
	Предположим обратное: $\exists 0' \neq 0$, тогда рассмотрим следующее:
	$$
		0' + 0 = 0'
	$$
	$$
		0 + 0' = 0
	$$
	Теперь заметим, что левые части равны по аксиоме о коммутативности сложения $\SO 0' = 0$, что противоречит предполагаемому. 
	\end{proof}
\end{Cons}

\begin{Cons}
	Число 1 единственно
	
	Доказательство аналогично доказательству единственности нуля, только используется умножение вместо сложения.
\end{Cons}

\begin{Cons}
	$\forall a, b, c \in \R : a = b \EQ a + c = b + c$
	\begin{proof}
		$$a = b \SO a \leq b \SO a + c \leq b + c$$
		$$a = b \SO b \leq a \SO b + c \leq a + c$$
		$$a + c \leq b + c,\text{ } b + c \leq a + c \SO a + c = b + c$$
		
		В обратную сторону аналогично:
		
		$$ a + c = b + c \SO a + c \leq b + c \SO a \leq b$$
		$$ a + c = b + c \SO b + c \leq a + c \SO b \leq a$$
		$$ a \leq b,\text{ } b \leq a \SO a = b$$
		
	\end{proof}

\end{Cons}

\begin{Cons}
	$\forall a \in \R$ $(-a)$ единственно.
	\begin{proof}
		Пусть верно обратное: $\exists a \in \R : \exists (-a)_1, (-a)_2 \in \R : (-a)_1 \neq (-a)_2$
		$$a + (-a)_1 = a + (-a)_2 = 0$$
		Добавим к обеим частям $(-a)_1$:
		$$(-a)_1 = (-a)_2$$
		Пришли к противоречию.
		
		
	\end{proof}
\end{Cons}

\begin{Cons}
	$\forall a, b \in \R: a \leq b \rightarrow -b \leq -a$
	\begin{proof}
		$$a + ((-a) + (-b)) \leq b + ((-a) + (-b))$$
		$$-b \leq -a$$
	\end{proof}
\end{Cons}

\begin{Cons}
	$\forall x \in \R \SO 0 \cdot x = 0$
	\begin{proof}
		$$0 \cdot x = 0 \cdot x + 0 \cdot x + (-0\cdot x) = x \cdot (0 + 0) + (-0 \cdot x) = 0 \cdot x + (-0\cdot x) = 0$$
	\end{proof}
\end{Cons}

\begin{Cons}
	$\forall x \in \R \rightarrow (-x) = (-1) \cdot x$
	\begin{proof}
		Предположим обратное: $\exists x \in \R: (-1) \cdot x = b, b \neq (-x)$
		$$(-1) \cdot x + 1 \cdot x = b + 1 \cdot x$$
		$$x \cdot (1 + (-1)) = b + 1 \cdot x$$
		$$b + x = 0$$
		$$b = -x$$
		Противоречие.
	\end{proof}
\end{Cons}

\begin{Cons}
	0 < 1
	\begin{proof}
		Предположим обратное: $0 \geq 1$, вариант $0 = 1$ сразу отпадает из-за аксиомы о существовании единицы, значит $0 > 1 \EQ -1 > 0$
		
		Пусть $x \in \R$, $x > 0$, тогда $(-1) \cdot x \geq 0 \EQ (-1) \cdot x + 1 \cdot x \geq 1 \cdot x \EQ 0 \geq x$
		
		Противоречие.
	\end{proof}
\end{Cons}

\begin{Thm}[Теорема о вложенных отрезка]
    Пусть $[a_1, b_1] \supset [a_2, b_2] \supset [a_3, b_3] \supset ... \supset [a_n, b_n]$. Тогда $$\exists a \in \bigcap_{n = 1}^{\infty}[a_n, b_n]$$
\end{Thm}

\begin{proof}
    $a_1 \leqslant a_2 \leqslant a_3 \leqslant ... \leqslant a_n \leqslant b_n \leqslant b_{n - 1} \leqslant ... \leqslant b_1$
    
    Значит, $\forall k, m \to a_k \leqslant b_m$ \\
    Пусть $A = \{a_n\}, B = \{b_n\}$. По аксиоме полноты $\exists c \in \mathbb{R} : \forall k, m \in \mathbb{N} \to a_k \leqslant c \leqslant b_m \Rightarrow c \in \bigcap_{n = 1}^{\infty} [a_n, b_n]$  
\end{proof}
Замечания: 1) $\bigcap_{n = 1}^{\infty} \left(0; \frac{1}{n}\right] = \varnothing$. Важно, что именно отрезки, а не интервалы или полуинтервалы.  \\
2) $\bigcap_{n = 1}^{\infty} [n; +\infty) = ?$ \\
3) Без аксиомы полноты не работает. Например 
\[[1.4; 1.5] \supset [1.41; 1.42] \supset ...\]
\[\bigcap_{n = 1}^{\infty}[a_n, b_n] = \{\sqrt{2}\}, \text{но не в } \mathbb{Q}\]

\Section{Принцип математической индукции}{}{Илья Дудников}
$\{P_n\}_{n = 1}^{\infty}$ - утверждения. Если
\begin{enumerate}
    \item $P_1$ верно - база
    \item $\forall n \in \mathbb{N} \to P_n \Rightarrow P_{n + 1}$ - индукционный переход  
\end{enumerate}
Тогда $\forall n \in \mathbb{N} \to P_n$.
\begin{Def}
    $M \subset \mathbb{R}$ - индуктивное, если $1 \in M \wedge (x \in M \Rightarrow x + 1 \in M)$.  
\end{Def} 

\begin{Def}
    $\mathbb{N}$ - минимальное индуктивное подмножество $\mathbb{R}$  
\end{Def}

\begin{Def}[Сдвиг индекса суммирования]
    $$\sum_{n = m}^k a_n = \sum_{j = m + p}^{k + p}{a_{j - p}}, p \in \mathbb{Z}$$
\end{Def}

\begin{Def}
    $k!!$ - произведение целых чисел до $k$ включительно одной четности с $k$.
\end{Def}

\begin{Def}[Биномиальные коэффициенты]
    $$C_n^k = \frac{n!}{k!(n - k)!} = \binom{n}{k}$$
\end{Def}

\begin{Thm}[Формула бинома Ньютона]
    Пусть $n \in \mathbb{Z}, x, y, \in \mathbb{R}$. Тогда 
    \[(x + y)^n = \sum_{k = 0}^{n}{C_n^k x^k y^{n - k}}\]
\end{Thm}

\begin{proof}
    $n = 0 \to 1 = 1$, верно
    
    Индукционный переход: \\
    \begin{align*}
        &(x + y)^{n + 1} = (x + y)(x + y)^n = (x + y)(\sum_{k = 0}^{n}{C_n^k x^k y^{n - k}}) = \\
        &=\sum_{k = 0}^{n}{C_n^k x^{k + 1}y^{n - k}} + \sum_{k = 0}^{n}C_n^k x^k y^{n - k + 1} = \sum_{j = 1}^{n + 1}C_n^{j - 1}x^j y^{n + 1 - j} + \sum_{k = 0}^{n}C_n^k x^k y^{n - k + 1} = \\
        &=\sum_{k = 1}^{n + 1}C_n^{k - 1}x^k y^{n + 1 - k} + \sum_{k = 0}^{n}C_n^k x^k y^{n - k + 1} = C_n^n x^{n + 1} y^0 + \sum_{k = 1}^{n} \left(C_n^{k - 1} x^k y^{n + 1 - k} + C_n^k x^k y^{n + 1 - k}\right) + C_n^0 x^0 y^{n + 1} = \\
        &=C_{n + 1}^{n + 1} x^{n + 1} y^0 + \sum_{k = 1}^{n} \left(C_n^{k - 1} + C_n^k\right)x^k y^{n + 1 - k} + C_{n + 1}^0 x^0 y^{n + 1} = \sum_{k = 0}^{n + 1}C_{n + 1}^k x^k y^{n + 1 - k} 
    \end{align*}
\end{proof}

\Section{Супремум и инфимум}{}{Илья Дудников}


\begin{Def}
    $E \subset \mathbb{R}$ - ограниченное сверху, если $\exists A : \forall x \in E \to x \leqslant A$ 
\end{Def}

\begin{Def}
    $E \subset \mathbb{R}$ - ограниченное снизу, если $\exists B : \forall x \in E \to x \geqslant B$  
\end{Def}

\begin{Def}
    $E \subset \mathbb{R}$ - ограниченное, если оно ограничено и снизу, и сверху.
\end{Def}

\begin{Def}
    $M \in \mathbb{R}$ называется максимумом мн-ва $E$, если $\forall x \in E \to x \leqslant M \wedge M \in E$ 
\end{Def}    

\begin{Def}
    $K \in \mathbb{R}$ называется минимумом мн-ва $E$, если $\forall x \in E \to x \geqslant K \wedge K \in E$ 
\end{Def}

\begin{Thm}[Существование минимума и максимума у конечного множества из $\mathbb{R}$ ]
    Во всяком конечном непустом подмножестве $\mathbb{R}$ есть наибольший и наименьший элементю
\end{Thm}

\begin{proof}
    $n = 1$ - количество элементов (База) \\
    Индукционный переход: $\exists \max\{x_1, x_2, ..., x_n\} = C$ \\
    Добавим $x_{n + 1}$: если $x_{n + 1} > C \Rightarrow \max\{x_1, ..., x_{n + 1}\} = x_{n + 1}$ \\
    если $x_{n + 1} \leqslant C \Rightarrow \max\{x_1, ..., x_{n + 1}\} = C$     
\end{proof}

\begin{Cons}
    $\forall E \neq \varnothing \wedge E \subset \mathbb{Z}\wedge E \text{ - огр.} \to \exists \max E \wedge \min E $ 
\end{Cons}

\begin{Cons}
    $\forall E \subset \mathbb{N}, E \neq \varnothing \to \exists \min E$ 
\end{Cons}

Далее везде $E \subset \mathbb{R}, E \neq \varnothing$ \\

\begin{Def}
    Пусть $E$ ограничено сверху, тогда $\sup{E}$ - наименьшаяя из верхних границ. (точная верхняя граница)
\end{Def}

\begin{Def}
    Пусть $E$ ограничено снизу, тогда $\inf{E}$ - наибольшая из нижних границ. (точная нижняя граница)
\end{Def}

\begin{Thm}
    $E \neq \varnothing$. Если $E$ ограничено снизу, то $\exists ! \inf{E}$ 
\end{Thm}

\begin{proof}
    Пусть $A$ - множество всех нижних границ $E (A \neq \varnothing)$ \\
    $\forall a \in A, b \in E \to a \leqslant b$ \\
    Тогда по аксиоме полноты $\Rightarrow \exists c \in \mathbb{R}: a \leqslant c \leqslant b \text{ } \forall a \in A, b \in E \Rightarrow \\ \Rightarrow \begin{cases}
        c \leqslant b \text{ } \forall b \in E \text{ - } c \text{ - нижняя граница}, \\
        c \geqslant a \text{ } \forall a \in A \text{ - } c \text{ - наибольшее}
    \end{cases}$ 
\end{proof}

\begin{Def}
    \begin{align*}
        &l = \sup{E} \Leftrightarrow \begin{cases}
            \forall x \in E \to x \leqslant l \\
            \forall \varepsilon > 0 \to \exists y \in E : y > l - \varepsilon
        \end{cases}\\
        &m = \inf{E} \Leftrightarrow \begin{cases}
            \forall x \in E \to x \geqslant m \\
            \forall \varepsilon > 0 \to \exists y \in E: y < l + \varepsilon
        \end{cases}
    \end{align*}
        
\end{Def}

Если $E$ не ограничено сверху, то $\sup{E} = +\infty$ \\
Если $E = \varnothing$, то чаще всего $\sup{E} \text{ и } \inf{E}$ не определены, но иногда $\sup{\varnothing} = -\infty, \inf{\varnothing} = +\infty$  

\begin{Prop}
    $\varnothing \neq B \subset A \subset \mathbb{R}$. Тогда если $A$ ограничено снизу, то $\inf{A} \leqslant \inf{B}$ 
\end{Prop}

\begin{proof}
    Если $C$ - нижняя граница $A$, то $\forall x \in A \to C \leqslant x \Rightarrow \forall y \in B \to C \leqslant y \Rightarrow C$ - нижняя граница $B \Rightarrow \inf{A}$ - тоже нижняя граница $B \Rightarrow \inf{A} \leqslant \inf{B}$  
\end{proof}

\begin{Prop}
    $\varnothing \neq B \subset A \subset \mathbb{R}$. Тогда если $A$ ограничено сверху, то $\sup{A} \geqslant \sup{B}$  
\end{Prop}
\newpage

\Section{Отображения}{}{Илья Дудников}
$f: A \to B$ 
$f(x) = y \\
y - \text{образ элемента } X \\ 
x - \text{прообраз } y \\
f(A) - \text{образ множества } A \\
f^{-1}(B) - \text{прообраз множества } B
$ \\
$G_f = \{(x, y) : x \in A, y = f(x)\}$ 

\begin{Def}
    $f: A \to B$. Если $f(A) = B$, то $f$ сюръективно.
\end{Def}
\begin{Def}
    $f: A \to B$. Если $(x_1 \neq x_2 \in A) \Leftrightarrow (f(x_1) \neq f(x_2))$, то $f$ инъективно.
\end{Def}
\begin{Def}
    Биекция - $f$ инъективно и сюръективно. 
\end{Def}
\begin{Def}[Композиция]
    $g(x), f(x). \\h(x) = g \circ f(x) = g(f(x))$ \\
    $f: X \to Y$
    $g: Y_0 \to Z, f(x) \subset Y_0$   
\end{Def}
\begin{Def}
    $id_x $ - тождественное отображение: $f(x) = x$  
\end{Def}
\begin{Def}
    $f: X \to Y, X_0 \subset X \\ f |_{X_0}$ - сужение отображения $f$ на $X_0$ 
\end{Def}


\Section{Последовательности}{}{Илья Дудников}
\begin{Def}
    Последовательность --- это отображение $f: \mathbb{N} \to \mathbb{R}$ 
\end{Def}

\begin{Example}
    $x_n = n^2 : x_n = \{1, 4, 9, ...\}$ 
\end{Example}

\Subsection{Предел последовательности и его свойства}

\begin{Def}
    Предел последовательности - это такое число $l = lim_{n \to \infty}x_n$, что 
    \[\forall \varepsilon > 0 \to \exists N \in \mathbb{N} : \forall n \geqslant N \to |x_n - l| < \varepsilon\] 

    Также говорят, что вне любого интервала, содержащего $l$, лежит лишь конечно число элементов $\{X_n\}_{n = 1}^{\infty}$ 
\end{Def}

\begin{Example}
    $x_n = \frac{1}{n}, \frac{1}{n} < \varepsilon \Leftrightarrow n > \frac{1}{\varepsilon}$. Тогда 
    $N_\varepsilon = \left[\frac{1}{\varepsilon} + 1\right]$  
\end{Example}

\begin{Rem}
    $N$ необязательно наименьшее.
\end{Rem}

\begin{Def}
    Последовательность называется \textbf{сходящейся}, если она имеет конечный предел.
\end{Def}

\begin{Def}
    $\lim_{n \to \infty}x_n = +\infty \Leftrightarrow \forall E \in \mathbb{R} \to \exists N \in \mathbb{N} : \forall n \geqslant N \to x_n > E$ 
\end{Def}

\begin{Def}
    $\lim_{n \to \infty}x_n = -\infty \Leftrightarrow \forall E \in \mathbb{R} \to \exists N \in \mathbb{N} : \forall n \geqslant N \to x_n < E$ 
\end{Def}

\begin{Def}[Беззнаковая бесконечность]
    $$\lim_{n \to \infty}x_n = \infty \Leftrightarrow \forall E \in \mathbb{R} \to \exists N \in \mathbb{N} : \forall n \geqslant N \to |x_n| > E$$
\end{Def}

\begin{Def}
    Последовательность называется бесконечно большой, если она стремится к бесконечности 
\end{Def}

\begin{Def}
    Последовательность называется бесконечно малой, если она стремится к нулю
\end{Def}

Свойства пределов последовательности:
\begin{MyList}
    \item Последовательность не может иметь двух различных пределов.
    
    \begin{proof}
        Пусть $a \neq b$ - пределы, $a < b$. Возьмем $\varepsilon = \left(\frac{b - a}{3}\right)$. Тогда по определнию предела вне $\varepsilon$-окрестности $a$ лежит конечно число членов последовательности, и
        вне $\varepsilon$-окрестности $b$ лежит конечно число членов последовательности $\Rightarrow$ сама последовательности конечна !?    
    \end{proof}
    \item Сходящаяся последовательность ограничена.
    \begin{proof}
        Пусть $\lim \{x_n\} = a$. По определению предела для $\varepsilon = 1$ найдем номер $N$ такой, что при всех $n \geqslant N$ имеет место неравенство $|x_n - a| < 1$.
        Так как модуль суммы не превосходит суммы модулей, то 
        \[|x_n| = |x_n - a + a| \leqslant |x_n - a| + |a|\]

        Поэтому при всех $n \geqslant N$ выполняется неравенство
        \[|x_n| < 1 + |a|\]
        Положим $M = \max(1 + |a|, |x_1|, ..., |x_{N - 1}|)$. Тогда $|x_n| \leqslant M \ \forall n \in \N$   
    \end{proof}

    \begin{Prop}
        Пусть $\lim x_n = a, \lim y_n = b$. Тогда $$\forall \varepsilon > 0 \to \exists N \in \N : \forall n \geqslant N \to \begin{cases}
            |x_n - a| < \varepsilon \\
            |y_n - b| < \varepsilon
        \end{cases}$$

        \begin{proof}
            $N_1$ - номер из определения $\lim x_n = a$ \\
            $N_2$ - номер из определения $\lim y_n = b$ \\
            $N = \max \{N_1; N_2\}$ 
        \end{proof}
    \end{Prop}
    \item Пусть $\lim x_n = a, \lim y_n = b, \forall n \in \N \to x_n \leqslant y_n$. Тогда $a \leqslant b$ (предельный переход в неравенстве).
    \begin{proof}
        Пусть $a > b$. Тогда $\exists N$, начиная с которого в $\varepsilon$-окрестности $b$ лежит бесконечное число членов $y_n$, а $\varepsilon$-окрестности $a$ лежит бесконечное число 
        членов $x_n$. Но тогда, если бы возьмем $\varepsilon = \frac{a - b}{3}$, то $\exists y_n \in (b - \varepsilon; b + \varepsilon) : y_n < x_n, x_n \in (a - \varepsilon; a + \varepsilon)$    
    \end{proof}
    \begin{Cons}
        \begin{MyList}
            \item $\lim x_n = a, \forall n \to x_n \leqslant b \Rightarrow a \leqslant b$ 
            \item $\lim y_n = b, \forall n \to y_n \geqslant a \Rightarrow a \leqslant b$  
        \end{MyList} 
    \end{Cons}
    \item \begin{Thm}[Теорема о сжатой последовательности, теорема о двух милиционерах]
        Пусть $\forall n \in \N \to x_n \leqslant z_n \leqslant y_n \wedge \lim x_n = \lim y_n = a$. Тогда $\lim z_n = a$ 
    \end{Thm}
    \begin{proof}
        $\forall \varepsilon > 0 \to \exists N \in \N : \forall n \geqslant N \to x_n, y_n \in (a - \varepsilon; a + \varepsilon)$. Т.к. $x_n \leqslant z_n \leqslant y_n \Rightarrow z_n \in (a - \varepsilon; a + \varepsilon)$  
    \end{proof}
\end{MyList}

\Subsection{Монотонные последовательности}

\begin{Def}
    Последовательность называется возрастающей, если $x_1 \leqslant x_2 \leqslant x_3 \leqslant ...$ 
\end{Def}
\begin{Def}
    Последовательность называется убывающей, если $x_1 \geqslant x_2 \geqslant x_3 \geqslant ...$ 
\end{Def}

\begin{Def}
    Последовательность называется монотонной, если она возрастающая или убывающая.
\end{Def}

\begin{Thm}[О монотонной ограниченной последовательности]
    \begin{MyList}
        \item Возрастающая и ограниченная сверху последовательность сходится
        \item Убывающая и ограниченная снизу последовательность сходится
    \end{MyList}
\end{Thm}

\begin{proof}
    Пусть множество $E = \{x_1, x_2, ...\}, c = \sup E$ 

    $\forall n \in \N \to x_n \leqslant c$. Тогда
    \[\forall \varepsilon > 0 \to \exists N \in \N : x_N > c - \varepsilon\]
    Т.к. $x_n$ возрастает, то \[\forall n > N \to x_n \geqslant x_N > c - \varepsilon \wedge x_n < c + \varepsilon \Leftrightarrow |x_n - c| < \varepsilon\] 
\end{proof}
\begin{Rem}
    \begin{MyList}
        \item Возрастающая и неограниченная сверху последовательность стремится к $+\infty$.
        \item Убывающая и неограниченная снизу последовательность стремится к $-\infty$
    \end{MyList}
    \begin{proof}
        $\forall E \to \exists N : x_N > E$ и $x_n \geqslant x_N$ $\forall n \geqslant N$  
    \end{proof}
\end{Rem}

\Pagebreak
\Subsection{Теорема об арифметических действиях с пределами}
\begin{Thm}[Теорема об арифметических действиях с пределами]
    Пусть $\lim x_n = a, \lim y_n = b, a, b \in \R$. Тогда
    \begin{MyList}
        \item $\lim |x_n| = |a|$ 
        \item $\lim (x_n + y_n) = a + b$
        \item $\lim (x_n - y_n) = a - b$ 
        \item $\lim (x_n \cdot y_n) = a \cdot b$ 
        \item $\forall n \in \N \to b \neq 0 \wedge y_n \neq 0$, то $\lim \frac{x_n}{y_n} = \frac{a}{b}$
    \end{MyList} 

    \begin{proof}
        \begin{MyList}
            \item $\forall \varepsilon > 0 \to \exists N : \forall n \geqslant N \to |x_n - a| < \varepsilon$. Заметим, что \[||x_n| - |a|| < |x_n - a| < \varepsilon\]
            \item $|(x_n + y_n) - (a + b)| \leqslant |x_n - a| + |y_n - b| < \varepsilon$ 
            \item Вместо $y_n$ рассмотрим $-y_n$
            \item $\forall \varepsilon > 0 \to \exists N : \forall n \geqslant N \to \begin{cases}
                |x_n - a| < \frac{\varepsilon}{M + |a|} \\
                |y_n - b| < \frac{\varepsilon}{M + |a|}
            \end{cases}, M : \forall n \to |y_n| < M$  \\
            \begin{equation*}
                |x_n \cdot y_n - ab| = |x_n y_n - ab - a \cdot y_n + a \cdot y_n| = |y_n (x_n - a) + a(y_n - b)| \leqslant |y_n||x_n - a| + |a||y_n - b| < \varepsilon
            \end{equation*}
            \item Достаточно доказать, что $\lim \frac{1}{y_n} = \frac{1}{b}$. 
            \[\left|\frac{1}{y_n} - \frac{1}{b}\right| = \frac{|y_n - b}{|y_n||b|} \leqslant \frac{|y_n - b|}{|\frac{b}{2}||b|} < \frac{\frac{b^2 \varepsilon}{2}}{|\frac{b}{2}||b|} = \varepsilon\]
        \end{MyList}
    \end{proof}
\end{Thm}

\begin{Prop}
    $x_n$ - бесконечно малая, $y_n$ - ограниченная. Тогда $\lim x_n \cdot y_n = 0$
    \begin{proof}
        $|x_n y_n| < |x_n| \cdot M < \varepsilon \cdot M, M : \forall n \in \N \to |y_n| < M$ 
    \end{proof} 
\end{Prop}

\begin{Prop}
    $\forall n \to x_n \neq 0$. Тогда $x_n$ - бесконечно большая $\Leftrightarrow \frac{1}{x_n}$ - бесконечно малая
    
    \begin{proof}
        $|x_n| > E \Leftrightarrow \frac{1}{|x_n|} < \frac{1}{E}$ 
    \end{proof}
\end{Prop}

\Pagebreak
\Subsection{Арифметические действия с бесконечностями}
\begin{MyList}
    \item $\lim x_n = +\infty, y_n$ ограничено снизу. Тогда $\lim (x_n + y_n) = +\infty$
    \item $\lim x_n = -\infty, y_n$ ограничено сверху. Тогда $\lim (x_n + y_n) = -\infty$
    \item $\lim x_n = +\infty, y_n \geqslant c > 0$. Тогда $\lim (x_n \cdot y_n) = +\infty$
    \item $\lim x_n = +\infty, y_n \leqslant c < 0$. Тогда $\lim (x_n \cdot y_n) = -\infty$
    \item $\lim x_n = a \neq 0, \lim y_n = 0$. Тогда $\lim \left(\frac{x_n}{y_n}\right) = \infty$
    \item $\lim x_n = a \in \R, \lim y_n = \infty$. Тогда $\lim \left( \frac{x_n}{y_n} \right) = 0$
    \item $\lim x_n = \infty, \lim y_n = b \in \R \wedge y_n \neq 0$. Тогда $\lim \left( \frac{x_n}{y_n} \right) = \infty$
\end{MyList}

\begin{Rem}
    $\lim x_n = a \in \overline{\R}, \lim y_n = b \in \overline{\R}$, то $\lim (x_n * y_n) = a * b$  
\end{Rem}

Запрещенные операции (неопределённости):
\begin{MyList}
    \item $\pm \infty + (\mp \infty)$ 
    \item $\pm \infty - (\pm \infty)$
    \item $0 \cdot \infty$ 
    \item $\frac{0}{0}$ 
    \item $\frac{\infty}{\infty}$  
\end{MyList}
\Pagebreak
\Subsection{Неравенство Бернулли}
\begin{Thm}[Неравенство Бернулли]
    Пусть $x > -1, n \in \N$. Тогда $(1 + x)^n \geqslant 1 + nx$
    
    \begin{proof}
        База $n = 1 : (1 + x)^1 \geqslant 1 + 1 \cdot x$ \\
        Индукционный переход $n \to n + 1$
        \[(1 + x)(1 + x)^n \geqslant (1 + x)(1 + nx) = 1 + n(x + 1) + nx^2 \geqslant 1 + (n + 1)x\] 
    \end{proof}
\end{Thm}


\Section{Пределы последовательностей}{}{Илья Дудников}

\Subsection{Число $e$}

Пусть $x_n = \left(1 + \frac{1}{n}\right)^n, y_n = \left(1 + \frac{1}{n}\right)^{n + 1}$.
Тогда 

\begin{align*}
    \frac{y_{n - 1}}{y_n} &= \frac{\left(1 + \frac{1}{n - 1}\right)^n}{\left(1 + \frac{1}{n}\right)^{n + 1}} = \frac{\left(\frac{n}{n - 1}\right)^n}{\left(\frac{n + 1}{n}\right)^{n + 1}} = \frac{\frac{n^n}{(n - 1)^n}}{\frac{(n + 1)^{n + 1}}{n^{n + 1}}} = \frac{n^n}{(n - 1)^n} \cdot \frac{n^{n + 1}}{(n + 1)^{n + 1}} = \frac{n^{2n}}{(n^2 - 1)^n} \cdot \frac{n}{n + 1} \\
    &= \frac{n^{2n + 2}}{(n^2 - 1)^{n + 1}} \cdot \frac{n - 1}{n} = \left(\frac{n^2}{n^2 - 1}\right)^{n + 1} \cdot \frac{n - 1}{n} = \left(1 + \frac{1}{n^2 - 1}\right)^{n + 1} \cdot \frac{n - 1}{n} \geqslant \\
    &\geqslant \left(1 + \frac{n + 1}{n^2 - 1}\right) \cdot \frac{n - 1}{n} = \left(1 + \frac{1}{n - 1}\right) \cdot \frac{n - 1}{n} = \frac{n}{n - 1} \cdot \frac{n - 1}{n} = 1 \Rightarrow \frac{y_{n - 1}}{y_n} \geqslant 1 \Rightarrow \\
    &\Rightarrow y_{n - 1} \geqslant y_n \Rightarrow y_n \text{ убывающая } \Rightarrow \exists \lim y_n \Rightarrow \exists \lim x_n \text{, т.к. } x_n = \frac{y_n}{1 + \frac{1}{n}} \Rightarrow \lim x_n = \lim y_n \\
\end{align*}

\begin{Def}
    $e = \lim_{n \to \infty}{\left(1 + \frac{1}{n}\right)^n}$ 
\end{Def}

\begin{Thm}
    $x_n > 0 \wedge \lim \frac{x_{n + 1}}{x_n} < 1$. Тогда $\exists \lim x_n = 0$  
\end{Thm}

\begin{proof}
    Пусть $q = \lim \frac{x_{n + 1}}{x_n}, q < 1$. $\exists N : \forall n \geqslant N \to \frac{x_{n + 1}}{x_n} < \frac{1 + q}{2}$. Тогда
    \[0 < x_n = \frac{x_n}{x_{n - 1}} \cdot \frac{x_{n - 1}}{x_{n - 2}} \cdot \frac{x_{n - 2}}{x_{n - 3}} \cdot ... \cdot \frac{x_{N + 1}}{x_N} \cdot x_N \leqslant x_N \cdot \left(\frac{1 + q}{2}\right)^{n - N} \to 0\]
\end{proof}

\begin{Cons}
    $a > 1, k \in \N, \lim \frac{n^k}{a^n} = 0$
    \[\frac{x_{n + 1}}{x_n} = \frac{(n + 1)^k}{a^{n + 1}} \cdot \frac{a^n}{n^k} = \left(\frac{n + 1}{n}\right)^k \cdot \frac{1}{a} \to \frac{1}{a} < 1\] 
\end{Cons}

\begin{Cons}
    $\lim \frac{a^n}{n!} = 0$ 
    \[\frac{x_{n + 1}}{x_n} = \frac{a^{n + 1}}{(n + 1)!} \cdot \frac{n!}{a^n} = \frac{a}{n + 1} \to 0 < 1\]
\end{Cons}

\begin{Cons}
    $\lim \frac{n!}{n^n} = 0$ 
    \[ \frac{x_{n + 1}}{x_n} = \frac{(n + 1)!}{(n + 1)^{n + 1}} \cdot \frac{n^n}{n!} = \left(\frac{n}{n + 1}\right)^n = \frac{1}{\left(\frac{n + 1}{n}\right)^n} = \frac{1}{\left(1 + \frac{1}{n}\right)^n} \to \frac{1}{e} < 1 \]
\end{Cons}

\Pagebreak
\Subsection{Теорема Штольца}

\begin{Thm}[Теорема Штольца]
        $y_1 < y_2 < y_3 < ...$. $\lim y_n = + \infty \wedge \lim \frac{x_n - x_{n - 1}}{y_n - y_{n - 1}} = l \in \overline{\R}$  
        Тогда $\exists \lim \frac{x_n}{y_n} = l$
    \begin{proof}
        \begin{MyList}
            \item $l = 0$. $\varepsilon_k = \frac{x_k - x_{k - 1}}{y_k - y_{k - 1}}$. $\forall \varepsilon > 0 \ \exists m : \forall k \geqslant m \to |\varepsilon_k| < \varepsilon$ \\
            $x_k - x_{k - 1} = \varepsilon_k (y_k - y_{k - 1})$
            \[x_n - x_m = (x_n - x_{n - 1}) + (x_{n - 1} - x_{n - 2}) + ... + (x_{m + 1} - x_m) = \sum_{k = m + 1}^{n}(x_k - x_{k - 1}) = \sum_{k=m + 1}^{n} \varepsilon_k (y_k - y_{k - 1})\]  
            \[|x_n - x_m| = \sum_{k=m + 1}^{n} |\varepsilon_k|(y_k - y_{k - 1}) \leqslant \varepsilon \sum_{k=m + 1}^{n} (y_k - y_{k - 1}) = \varepsilon (y_n - y_m) \leqslant \varepsilon \cdot y_n\]  

            Тогда $|x_n| \leqslant |x_m| + \varepsilon y_n$ 
            \[0 \leqslant \left|\frac{x_n}{y_n}\right| \leqslant \left|\frac{x_m}{y_n}\right| + \varepsilon \Rightarrow \lim \frac{x_n}{y_n} = 0\]   
        
            \item $l \neq 0, l \in \R$. Рассмотрим $\widetilde{x}_n = x_n - l \cdot y_n$. Тогда
            \[ \frac{\widetilde{x}_n - \widetilde{x}_{n - 1}}{y_n - y_{n - 1}} = \frac{x_n - l \cdot y_n - x_{n - 1} + l \cdot y_{n - 1}}{y_n - y_{n - 1}} = \frac{x_n - x_{n - 1}}{y_n - y_{n - 1}} - l \to 0 \]
            Тогда по п. 1 $\frac{\widetilde{x}_n}{y_n} \to 0$. $ \frac{x_n}{y_n} = \frac{\widetilde{x}_n + l \cdot y_n}{y_n} = \frac{x_n}{y_n} + l \to l $  

            \item $l = + \infty$. $ \frac{x_n - x_{n - 1}}{y_n - y_{n - 1}} \to +\infty$. Начиная с некоторого номера $> 1$ \\
            $$x_n - x_{n - 1} > y_n - y_{n - 1} \Leftrightarrow x_n - x_m > y_n - y_m \to +\infty$$
            Тогда $x_n$ возрастает и стремится к $+\infty$
            \[ \frac{y_n - y_{n - 1}}{x_n - x_{n - 1}} \to 0 \Rightarrow \frac{y_n}{x_n} \to 0 \Rightarrow \frac{x_n}{y_n} \to +\infty\]
            \item $l \to -\infty$. Следует рассмотреть $\{-x_n\}$ 
        \end{MyList}
    \end{proof}
\end{Thm}

\begin{Thm}
    $y_1 > y_2 > ... > 0 \wedge \lim y_n = \lim x_n = 0$. Если $\lim \frac{x_n - x_{n - 1}}{y_n - y_{n - 1}} = l \in \overline{\R}$, тогда $\exists \lim \frac{x_n}{y_n} = l$   
\end{Thm}

\begin{proof}
    Докажем для $l = 0$.
    \[\varepsilon_k = \frac{x_k - x_{k - 1}}{y_k - y_{k - 1}}, \forall \varepsilon > 0 \ \exists N : \forall k \geqslant N \to |\varepsilon_k| < \varepsilon\]

    Пусть $n > m \geqslant N$
    \[|x_n - x_m| = \sum_{k=n + 1}^{m} |\varepsilon_k| \cdot |y_{k - 1} - y_k| \leqslant \varepsilon \sum_{k=n + 1}^{m} (y_k - y_{k - 1}) = \varepsilon (y_m - y_n)\] 
    \[|x_n - x_m| \leqslant \varepsilon(y_m - y_n) \EQ |x_m| \leqslant \varepsilon \cdot y_m \SO \frac{|x_m|}{y_m} < \varepsilon\]
    Доказали, что $\forall \varepsilon > 0 \ \exists N : \forall m \geqslant N  \to \left|\frac{x_m}{y_m}\right| < \varepsilon$
    
    Для $l \neq 0$ доказывается аналогично предыдущей теореме.
\end{proof}

\Pagebreak
\Subsection{Подпоследовательности}
\begin{Def}
    Пусть дана последовательность $\left\{x_n\right\}_{n = 1}^{\infty}$. Подпоследовательностью этой последовательности называется $\left\{x_{n_k}\right\}_{k = 1}^\infty : n_1 < n_2 < n_3 < ...$  
\end{Def}

\begin{Thm}[О стягивающихся отрезках]
    $[a_1, b_1] \supset [a_2, b_2] \supset [a_3, b_3] \supset ..., \lim (b_n - a_n) = 0$. Тогда $\bigcap_{n = 1}^\infty [a_n, b_n]$ состоит из одной точки.
    Если эта точка $c$, то $\lim a_n = \lim b_n = c$ 
\end{Thm}

\begin{proof}
    $\bigcap_{n = 1}^\infty [a_n, b_n] \neq \varnothing$ (по лемме о вложенных отрезках). Пусть $c < d$ принадлежит этому пересечению.
    \[0 < d - c \leqslant b_n - a_n \to 0 \SO 0 \leqslant c - a_n \leqslant 0 \SO \text{ точка единственна}\] 
    \[0 \leqslant c - a_n \leqslant b_n - a_n \to \SO 0 \leqslant c - a_n \leqslant 0 \SO a_n \to c\]
    \[0 \leqslant b_n - c \leqslant b_n - a_n \SO b_n \to c\]
\end{proof}

\begin{Thm}[Теорема Больцано-Вейерштрасса]
    Из всякой ограниченной последовательности можно извлечь сходящуюся подпоследовательность.
\end{Thm}

\begin{proof}
    Возьмем $a_1 \leqslant b_1$ так, чтобы вся последовательность лежала между ними. 
    $x_{n_1} \in [a_1, b_1]$. Поделим отрезок пополам и возьмем ту половину, в которой лежит бесконечное число членов последовательности. Обозначим её $[a_2, b_2]$. 
    Теперь возьмем $x_{n_2} \in [a_2, b_2]$ и $n_2 > n_1$. $[a_3, b_3]$ - ту половину $[a_2, b_2]$, в которой бесконечное число членов последовательности и т.д.
    
    \[[a_1, b_1] \supset [a_2, b_2] \supset [a_3, b_3] \supset ... \text{ и длина } [a_k, b_k] = \frac{b_1 - a_1}{2^k} \to 0\]

    \[\bigcap_{n = 1}^\infty [a_n, b_n] = \left\{c\right\}, \lim a_n = \lim b_n = c\]

    $n_1 < n_2 < n_3 < ...$. $\left\{x_{n_k}\right\}_{k = 1}^{\infty}$ -- подпоследовательность $\left\{x_n\right\}$ и $$a_k \leqslant x_{n_k} \leqslant b_k \SO \lim_{k \to \infty} x_{n_k} = c$$    
\end{proof}

\begin{Thm}
    \begin{MyList}
        \item Если последовательность неограничена сверху, то из неё можно выделить подпоследовательность, стремящуюся к $+\infty$ 
        \item Если неограничена снизу, то можно выделить подпоследовательность, стремящуюся к $-\infty$ 
    \end{MyList}
\end{Thm}

\begin{proof}
    \begin{align*}
        &\exists n_1 : x_{n_1} > 1 \\
        &\exists n_2 : x_{n_2} > 2 \wedge n_2 > n_1 \\
        &\exists n_k : x_{n_k} > k \wedge n_k > n_{k - 1}
    \end{align*} 
\end{proof}

\Pagebreak
\begin{Cons}
    Из любой последовательсти можно выбрать подпоследовательность имеющую предел (конечный или бесконечный).
\end{Cons}

\begin{Def}
    Частичные пределы последовательности $\left\{x_{n_k}\right\}_{n = 1}^\infty$ -- пределы её подпоследовательностей.
    
    \begin{Rem}
        $\lim x_n = a, \left\{x_{n_k}\right\}$ - подпоследовательность $\SO x_{n_k} \to a$ 
    \end{Rem}
\end{Def}

\begin{Def}
    Последовательность $\left\{x_n\right\}$ -- фундаментальная, если 
    \[\forall \varepsilon > 0 \ \exists N : \forall m, n \geqslant N \to |x_m - x_n| < \varepsilon\]
\end{Def}

Свойства:
\begin{MyList}
    \item Фундаментальная последовательность ограничена
    \item Сходящаяся последовательность фундмаентальна
    \begin{proof}
        $a = \lim x_n$
        \[\forall \varepsilon > 0 \ \exists N : \forall n \geqslant N \to |x_n - a| < \frac{\varepsilon}{2}\]
        \[|x_n - x_m| = |x_n - a + a - x_m| \leqslant |x_n - a| + |x_m - a| < \varepsilon\]  
    \end{proof}
    \item Если у фундаментальной последовательности есть сходящаяся подпоследовательность, то эта последовательность сходится.
    \begin{proof}
        $\lim_{k \to \infty} x_{n_k} = a, \forall \varepsilon > 0 \ \exists K : \forall k \geqslant K \to |x_{n_k} - a| < \frac{\varepsilon}{2}$ 
        \[\forall \varepsilon > 0 \ \exists N : \forall m, n \geqslant N \to |x_n - x_m| < \frac{\varepsilon}{2}\]
        $M = \max \left\{n_K, N\right\}$. Тогда $\forall n \geqslant M \to |x_n - a| \leqslant |x_n - x_{n_k}| + |x_{n_k} - a| < \varepsilon$ 
    \end{proof}
\end{MyList}

\begin{Thm}[Критерий Коши сходимости последовательности]
    Последовательность сходится $\EQ$ она фундаментальна
\end{Thm}

\begin{proof}
    "$\SO$". Свойство 2. \\
    "$\Leftarrow$". Фундаментальа $\SO$ ограничена (свойство 1) $\SO \exists$ сходящаяся подпоследовательность (теорема Больцано-Вейерштрасса) $\SO$ сходится  
\end{proof}

\begin{Def}
    $\left\{x_n\right\}$ -- ограничена сверху. \\
    $\overline{\lim_{n \to \infty}} x_n = \limsup_{n \to \infty} x_k = \lim_{n \to \infty} \sup_{k \geqslant n} x_k$  -- верхний предел. \\
    $\underline{\lim_{n \to \infty}} x_n = \liminf_{n \to \infty} x_k = \lim_{n \to \infty} \inf_{k \geqslant n} x_k$  -- нижний предел. 
\end{Def} 

\Pagebreak
\begin{Thm}
    Пусть $y_n = \inf_{k \geqslant n}x_k, z_n = \sup_{k \geqslant n} x_k$. Тогда
    \[\exists \underline{\lim} x_n, \overline{\lim} x_n \wedge \underline{\lim} x_n \leqslant \overline{\lim} x_n\] 
\end{Thm}

\begin{proof}
    \begin{MyList}
        \item Если неограничена сверху, то $\overline{\lim x_n} = +\infty$
        \item Пусть $\left\{x_n\right\}$ -- ограчичена. $\forall n \to x_n \leqslant M$
        \[z_1 \geqslant z_2 \geqslant z_3 \geqslant ... \wedge z_n \leqslant M \SO \exists \lim z_n\]
        \item Аналогично для $y_n$
        \item $\forall n \to y_n \leqslant z_n \SO \underline{\lim}x_n \leqslant \overline{\lim}x_n$  
    \end{MyList}
\end{proof}

\begin{Thm}
    \begin{MyList}
        \item $\overline{\lim} x_n$ -- наибольший частичный предел $\{x_n\}$  
        \item $\underline{\lim} x_n$ -- наименьший частичный предел $\{x_n\}$ 
        \item $\exists \lim x_n$ в $\overline{\R} \EQ \overline{\lim}x_n = \underline{\lim}x_n$  
    \end{MyList}
\end{Thm}

\begin{proof}
    \begin{MyList}
        \item $\{x_n\}$ -- ограниченная последовательность, $b = \overline{\lim} x_n$. Построим $\{x_{n_k}\} : x_{n_k} \xrightarrow[k \to \infty]{} b$
        $z_1 = \sup \{x_1, x_2, ...\} > b - \frac{1}{2} \SO \exists x_{n_1} > b - \frac{1}{2}$ \\
        $z_{n_1} = \sup \{x_{n_1 + 1}, x_{n_1 + 2}, ...\} > b - \frac{1}{3} \SO \exists x_{n_2} > b - \frac{1}{3}, n_2 > n_1$  
        
        \[b - \frac{1}{k} < x_{n_k} \leqslant z_{n_k} \SO x_{n_k} \to b\]
        Рассмотрим $x_{m_k} \to c$. Тогда $x_{m_k} \leqslant z_{m_k} \SO c \leqslant b$
        
        Если $\{x_n\}$ неограничена 
        \begin{MyItemize}
            \item сверху: $\exists \overline{\lim}x_n = +\infty \SO \exists \{x_{n_k}\} : x_{n_k} \to +\infty$
            \item снизу: а) $\exists \overline{\lim} x_n = b \in \R$ -- аналогично п.1. б) $\exists \overline{\lim}x_n = -\infty \SO x_n \to -\infty$   
        \end{MyItemize} 

        \item Аналогично
        \item "$\SO$". $\exists \lim x_n = l \SO \forall \{x_{n_k}\} \to \lim x_{n_k} = l \SO \overline{\lim} x_n = \underline{\lim} x_n = l$
        
        "$\Leftarrow$". $y_n \leqslant x_n \leqslant z_n \SO \underline{\lim}x_n \leqslant \lim x_n \leqslant \overline{\lim} x_n \SO \exists \lim x_n = \overline{\lim} x_n = \underline{\lim} x_n$ 
    \end{MyList}
\end{proof}

\begin{Thm}[характеристические свойства $\overline{\lim}x_n$ и $\underline{\lim} x_n$]
  \[a = \underline{\lim} x_n \EQ \begin{cases}
    \forall \varepsilon > 0 \ \exists N : \forall n \geqslant N \to x_n > a - \varepsilon \\
    \forall \varepsilon > 0, \forall N \ \exists n \geqslant N : x_n < a + \varepsilon
  \end{cases}\] 

  \[b = \overline{\lim} x_n \EQ \begin{cases}
    \forall \varepsilon > 0 \ \exists N : \forall n \geqslant N \to x_n < a + \varepsilon \\
    \forall \varepsilon > 0, \forall N \ \exists n \geqslant N : x_n > a - \varepsilon
  \end{cases}\]
\end{Thm}

\Section{Ряды}{}{Илья Дудников}

\Subsection{Ряды}

\begin{Def}
    Дана $\{a_n\}_{n = 1}^\infty$. Рассмотрим $S_n = a_1 + a_2 + ... + a_n = \sum_{k=1}^{n} a_k$.
    
    Тогда ряд -- $\sum_{k=1}^{\infty} a_k = a_1 + a_2 + ...$  \\
    Если $S_n \to S \in \R$, то $S$ называют суммой ряда $\sum_{k=1}^{\infty} a_k$ (ряд сходится к $S$). \\
    Если $S_n$ не имеет предела в $\R$, то ряд называют расходящимся.  
\end{Def}

\begin{Example}
    $\frac{1}{2} + \frac{1}{4} + \frac{1}{8} + \frac{1}{16} + ...$, т.е.
    \[S_n = \sum_{k=1}^{\infty} \frac{1}{2^k} = 1 - \frac{1}{2^n} \to 1\]   
\end{Example}

\begin{Example}
    $1 + \frac{1}{2} + \frac{1}{3} + \frac{1}{4} + ...$. Пусть
    \[\sum_{k=1}^{\infty} \frac{1}{k} = S\]
    Тогда, если $S \in \R$, то $\frac{S}{2} = \frac{1}{2} + \frac{1}{4} + \frac{1}{6} + ...$. Значит
    \[\frac{S}{2} = S - \frac{S}{2} = 1 + \frac{1}{3} + \frac{1}{5} + \frac{1}{7} > \frac{S}{2} \SO \sum_{k=1}^{\infty} \frac{1}{k} = +\infty\]
    Данный ряд называется \textbf{гармоническим}  
\end{Example}

\begin{Rem}
    $S_n$ частичные суммы ряда
\end{Rem}

\begin{Example} \label{one-minus-one}
    $1 - 1 + 1 - 1 + 1 - 1 + ... = \sum_{k=1}^{\infty} (-1)^{k + 1}$
    \[S_1 = 1, S_2 = 0, S_3 = 1, S_4 = 0 \SO S_n = \frac{1 + (-1)^{n + 1}}{2} \SO \text{ не существует } \lim S_n\] 
\end{Example}

\begin{Thm}[Необходимое условие сходимости ряда]
    Если $\sum_{k=1}^{\infty} a_k$ сходится, то $\lim_{k \to \infty} a_k = 0$  
\end{Thm}

\begin{proof}
    \[\exists \lim S_n = S, a_n = S_n - S_{n - 1} \SO a_n \to 0\]
\end{proof}

\begin{Example}
    Гармонический ряд.

    $H_n = 1 + \frac{1}{2} + \frac{1}{3} + ...$
    \[\frac{1}{n + 1} + \frac{1}{n + 2} + \frac{1}{n + 3} + ... + \frac{1}{2n} \geqslant n \cdot \frac{1}{2n} = \frac{1}{2}\]
    Тогда $H_{2^n} \geqslant \frac{1}{2}(n + 1) \to +\infty \SO H_n \to +\infty$ (т.к. $H_n$ возрастает)  
\end{Example}

\begin{Example}
    $\frac{1}{1 \cdot 2} + \frac{1}{2 \cdot 3} + ...$
    \[S_n = \sum_{k=1}^{\infty} \frac{1}{k(k + 1)} = \sum_{k=1}^{\infty} \left(\frac{1}{k} - \frac{1}{k + 1}\right) = 1 - \frac{1}{n}\] 
\end{Example}

\Pagebreak
\Subsection{Свойства рядов}

\begin{MyList}
    \item Ряд не может иметь двух различных сумм.
    \item Если ряд сходится к $S$, то к $S$ сходится и ряд, полученный из данного любой расстановкой скобок.
    \[a_1 + a_2 + a_3 + a_4 + a_5 + a_6 \SO (a_1 + a_2) + a_3 + (a_4 + a_5 + a_6)\]
    Пусть $b_1 = a_1 + a_2, b_2 = a_3, b_3 = a_4 + a_5 + a_6$ и т.д.
    Заметим, что $\{S_n^b\}$ является подпоследовательность $S_n^a$.  

    \begin{Rem}
        "раскрывать скобки" вообще говоря нельзя.
        \[\sum_{k=1}^{\infty} (1 - 1) = (1 - 1) + (1 - 1) + ... = 0\]
        Но если раскрыть скобки, то получится пример \ref{one-minus-one}    
    \end{Rem}

    \item Добавление или отбрасывание конечного числа слагаемых не влияет на сходимость (но может повлиять на сумму)
\end{MyList}

\Section{Функции}{}{Илья Дудников}
\Subsection{Свойства пределов функций}

\begin{Thm}[Единственность предела функции]
    Пусть $D \subset \R, a$ -- предельная точка $D$, $f: D \to R$. Если $A$ и $B \in \overline{\R}$ и $f(x) \xrightarrow[x \to a]{} A, f(x) \xrightarrow[x \to a]{} B \SO A = B$   
\end{Thm}

\begin{proof}
    Возьмем $\{x_n\} : x_n \in D, x_n \neq a, x_n \to a$. По Гейне $f(x_n) \to A \wedge f(x_n) \to B$. Но $\{x_n\}$ имеет единственный предел $\SO A = B$.    
\end{proof}

\begin{Rem}
    Беззнаковая бесконечность: $A = +\infty, B = -\infty \SO f(x) \xrightarrow[x \to a]{} \infty$ 
\end{Rem}

\begin{Thm}[Локальная ограниченность функции, имеющей предел]
    $D \subset \R, a$ -- предельная точка $D$, $f: D \to \R, A \in \R, f(x) \xrightarrow[x \to a]{} A$. Тогда $\exists V(a) : f(x)$ ограничена в $D \cap V(a)$  
\end{Thm}

\begin{proof}
    Пусть $\varepsilon = 1$. $\exists \dot{V}(a) : |f(x) - A| < 1 \ \forall x \in \dot{V}(a) \cap D$. Тогда $|f(x)| < |A| + 1$.

    Если $a \in D$, то $|f(x)| < \max \{|A| + 1, f(a)\}$  
\end{proof}

\begin{Thm}[Стабилизация знака функции, имеющей предел]
    $D \subset \R, a$ -- предельная точка~$D, f: D \to \R$. Пусть $\lim_{x \to a} f(x) = B \in \overline{\R} \setminus \{0\}$.
    Тогда $\exists V(a)$ такая, что знаки $f(x)$ и $B$ совпадают на $\dot{V}(a) \cap D$     
\end{Thm}

\begin{proof}
    Пусть $B > 0$. Докажем от противного, т.е.
    \[\forall n \ \exists x_n \in \dot{V}_{\frac{1}{n}} (a) \cap D \wedge f(x_n) \leqslant 0\]
    Тогда $x_n \to a, x_n \neq a \SO f(x_n) \to B$, но $f(x_n) \leqslant 0 \SO B \leqslant 0$.  
\end{proof}

\begin{Thm}[Арифметические действия над функциями, имеющими предел]
    $D \subset \R$, \\ $a$ -- предельная точка $D$, $f, g : D \to \R, f \xrightarrow[x \to a]{} A, g \xrightarrow[x \to a]{} B$. Тогда 
    \begin{MyList}
        \item $f(x) + g(x) \to A + B$ 
        \item $f(x) \cdot g(x) \to A \cdot B$ 
        \item $f(x) - g(x) \to A - B$ 
        \item $|f(x)| \to |A|$ 
        \item Если $B \neq 0$, то $ \frac{f(x)}{g(x)} \to \frac{A}{B}$ 
    \end{MyList}
\end{Thm}

\begin{proof}
    Рассмотрим $\{x_n\} : x_n \to a, x_n \neq a, x_n \in D$. Тогда $f(x_n) \to A, g(x_n) \to B$. Достаточно применить теорему об арифметических действиях с пределами последовательностей.  
\end{proof}

\begin{Rem}
    Пункт 5) т.к. $B \neq 0$, то $\exists V(a) : \sign (g(x)) = \sign B$ в $V(a)$. Поэтому излишне требовать $g(x) \neq 0$   
\end{Rem}

\Pagebreak
\begin{Thm}[Предел композиции функций]
    $f : D \to \R, g : E \to \R, f(D) \subset E$
    \begin{MyList}
        \item $f(x) \xrightarrow[x \to a]{} A \in \overline{\R}$
        \item $A$ -- предельная точка множества $E$ и $g(x) \xrightarrow[x \to A]{} B \in \overline{R}$
        \item $\exists V(a) : f(x) \neq A \ \forall x \in \dot{V}(a) \cap D$   
    \end{MyList} 
    Тогда $(g \circ f)(x) \xrightarrow[x \to a]{} B$ 
\end{Thm}

\begin{proof}
    Возьмем $\{x_n\} : x_n \in D, x_n \to a, x_n \neq a$. \\ Обозначим $y_n = f(x_n) \SO y_n \in E, y_n \to A$.
    По 3) начиная с некоторого номера $x_n \in V(a)$, а значит $y_n \neq A$. Тогда $g(y_n) \to B$, т.е. $g(f(x_n)) \xrightarrow[n \to \infty]{} B$.
    Значит $(g \circ f)(x) \xrightarrow[x \to a]{} B$ 
\end{proof}

\begin{Thm}[Предельный переход в неравенстве]
    $D \subset \R, a $ -- предельная точка $D$. $f,~g~:~D~\to~\R$.
    \[f(x) \xrightarrow[x \to a]{}A \in \overline{\R}, g(x) \xrightarrow[x \to a]{} B \in \overline{\R}, f(x) \leqslant g(x) \ \forall x \in D \setminus \{a\}\] 
    Тогда $A \leqslant B$ 
\end{Thm}

\begin{proof}
    \[\{x_n\} : x_n \in D, x_n \to a, x_n \neq a \SO f(x_n) \to A, g(x_n) \to B = A \leqslant B\]
\end{proof}

\begin{Thm}[о сжатой функции]
    $D \subset \R, a$ -- предельная точка $D$, $f, h, g : D \to \R$ и
    \[f(x) \leqslant g(x) \leqslant h(x), \forall x \in D \setminus \{a\} \ f(x) \xrightarrow[x \to a]{} A, h(x) \xrightarrow[x \to a]{} A, A \in \R\]
    Тогда $g(x) \xrightarrow[x \to a]{} A$  
\end{Thm}

\begin{proof}
    $\{x_n\} : x_n \in D, x_n \to a, x_n \neq a \SO f(x_n) \to A, h(x_n) \to A$
    \[f(x_n) \leqslant g(x_n) \leqslant h(x_n) \SO A \leqslant \lim_{n \to \infty} g(x_n) \leqslant A \SO \exists \lim_{n \to \infty} g(x_n) = A \SO g(x) \to A\] 
\end{proof}

\begin{Rem}
    $f(x) \leqslant g(x) \ \forall x \in D \setminus \{a\}, f(x) \xrightarrow[x \to a]{} +\infty \SO g(x) \xrightarrow[x \to a]{} +\infty$ 
\end{Rem}

\Pagebreak
\begin{Def}
    $f : D \to \R, a$ -- предельная точка $D_1 \subset D$. Тогда $\lim_{x \to a} f|_{D_1} (x)$ -- предел $f$ в точке $a$ по множеству $D_1$. 
\end{Def}

\begin{Def}
    $f: D \to \R, D_1 = D \cap (-\infty, a), a$ -- предельная точка $D_1$. Предел $f$ в точке $a$ по множеству $D_1$ называется левосторонним пределом в точке $a$.

    Обозначение: 
    \[\lim_{x \to a-} f(x), \lim_{x \to a - 0} f(x)\]
\end{Def}

\begin{Def}
    $f: D \to \R, D_1 = D \cap (a, +\infty), a$ -- предельная точка $D_1$. Правосторонний предел -- предел $f$ в точке $a$ по множеству $D_1$
    
    Обозначение:
    \[\lim_{x \to a+} f(x), \lim_{x \to a + 0} f(x)\]
\end{Def}

\begin{Def}
    Левосторонний предел на разных "языках".
    \begin{MyItemize}
        \item $\forall \varepsilon \ \exists \delta > 0 : \forall x \in D, 0 < a - x < \delta \to |f(x) - A| < \varepsilon$
        \item $\forall V(A) \ \exists \delta > 0 : \forall x \in D, 0 < a - x < \delta \to f(x) \in V(A)$
        \item $\forall \{x_n\} : x_n \in D, x_n \to a, x_n < a \ f(x_n) \to A$  
    \end{MyItemize}
\end{Def}

\begin{Rem}
    $f: D \to \R, a \in \R$ -- предельная точка для $D_1 = D \cap (-\infty, a), D_2 = D \cap (a, +\infty)$
    Тогда 
    \[\exists \lim_{x \to a} f(x) \EQ \exists \lim_{x \to a-} f(x), \exists \lim_{x \to a+} f(x) \wedge \lim_{x \to a-} f(x) = \lim_{x \to a+} f(x)\]  
\end{Rem}

\begin{proof}
    "$\SO$". Очевидно. \\
    "$\Leftarrow$". Возьмем $\delta_1$ из определения левостороннего предела, $\delta_2$ из определения правостороннего предела.
    $\delta = \min \{\delta_1, \delta_2\}$. Тогда 
    \[\forall \varepsilon > 0 \ \exists \delta : \forall x \in D : |x - a| < \delta \to |f(x) - A| < \varepsilon\]  
\end{proof}

\begin{Thm}[Предел монотонной функции]
    $D \in \R, f : D \to \R, a \in (-\infty, +\infty]$ \\
    $D_1 = D \cap (-\infty, a), a$ -- предельная точка $D$.
    \begin{MyList}
        \item Если $f$ возрастает и ограничена сверху на $D_1$, то $\exists \lim_{x \to a-} f(x) \in \R$
        \item Если $f$ убывает и ограничена снизу на $D_1$, то $\exists \lim_{x \to a-} f(x) \in \R$  
    \end{MyList} 
\end{Thm}

\begin{proof}
    \begin{MyList}
        \item Пусть $A = \sup_{x \in D_1} f(x)$. Тогда $A \in \R$, т.к. $f$ ограничена сверху. Докажем, что $\lim_{x \to a-} f(x) = A$
        \[\forall \varepsilon > 0 \ \exists x_0 \in D_1 : f(x_0) > A - \varepsilon\]
        Тогда $\forall x \in D_1 : x > x_0$
        \[A - \varepsilon < f(x_0) \leqslant f(x) \leqslant A < A + \varepsilon\]
        Пусть $\delta = a - x_0$. Тогда $|f(x) - A| < \varepsilon \ \forall x : 0 < a - x < \delta$
        
        Если $a = +\infty \SO \Delta = \max \{x_0, 1\}$ 
    \end{MyList}
\end{proof}

\begin{Rem}
    $f$ возрастает и не ограничена сверху $\SO \lim_{x \to a-} f(x) = +\infty$ 
\end{Rem}

\begin{Thm}[Критерий Больцано-Коши для функций]
    $D \subset \R$. Тогда существование конечного $\lim_{x \to a} f(x)$ равносильно утверждению:
    
    \[\forall \varepsilon > 0 \ \exists V(a) : \forall x_1, x_2 \in \dot{V}(a) \cap D \to |f(x_1) - f(x_2)| < \varepsilon\]
\end{Thm}

\begin{proof}
    "$\SO$". $\exists \lim_{x \to a} f(x) = A \in \R$. Возьмем $\varepsilon > 0$
    Тогда $\exists V(a) : |f(x) - A| < \frac{\varepsilon}{2}$. Если $x_1, x_2 \in D \cap \dot{V}(a)$, то
    \[|f(x_1) - A| + |f(x_2) - A| < \varepsilon\]
    С другой стороный $|f(x_1) - f(x_2)| < |f(x_1) - A| + |f(x_2) - A| < \varepsilon$ \\
    "$\Leftarrow$". $\{x_n\} : x_n \in D, x_n \neq a, x_n \to a$ и докажем, что $\exists \lim f(x_n) \in \R$. \\
    Пусть $\varepsilon > 0$.
    \[\exists N : \forall n \geqslant N \to x_n \in \dot{V}(a)\]
    \[\forall n, l \geqslant N \to |f(x_n) = f(x_l)| < \varepsilon \SO \{f(x_n)\} \text { -- фундаментальна}\]
    Значит $\{f(x_n)\}$ сходится. 
\end{proof}

\Subsection{Непрерывные функции}

\begin{Def}
    $D \subset \R, a \in D$. Функция $f$ называется непрерывной в точке $a$, если выполнено одно из следующих условий:

    \begin{MyList}
        \item Предел $f$ в точке $a$ существует и равен $f(a)$ (только если $a$ -- предельная точка).
        \item $\forall \varepsilon > 0 \ \exists \delta > 0 : \forall x \in D : |x - a| < \delta \to |f(x) - f(a)| < \varepsilon$ 
        \item $\forall V (f(a)) \ \exists V(a) : f(V(a) \cap D) \subset V(f(a))$
        \item $\forall \{x_n\} : x_n \to a, x_n \in D \ f(x_n) \to f(a)$
        \item Бесконечно малому приращению аргумента соответствует бесконечно малое приращение функции (если $a$ -- предельная точка)
        \[\Delta x = x - a, \Delta f = f(x) - f(a) \SO \Delta f \xrightarrow[\Delta x \to 0]{} 0\]
    \end{MyList}
\end{Def}

\begin{Rem}
    Если $a$ -- изолированная точка $D$, то 
    \[f(V(a) \cap D) = \{f(a)\} \subset V(f(a))\]
    Т.е. любая $f$ непрерывна в точке $a$ 
\end{Rem}

\begin{Def}
    $D \subset \R, a \in D, f : D \to \R$. \\
    $a$ называется точкой разрыва $f$, если $f$ не непрерывна в точке $a$  
\end{Def}

\begin{Def}
    $D_1 = D \cap (-\infty, a], D_2 = D \cap [a, +\infty)$. \\
    Если сужение $f|_{D_1}$ непрерывно в точке $a$, то $f$ непрерывна в точке $a$ \textbf{слева}. \\
    Если сужение $f|_{D_2}$ непрерывно в точке $a$, то $f$ непрерывна в точке $a$ \textbf{справа} 
\end{Def}

\begin{Def}
    Если $\exists \lim_{x \to a+} f(x), \lim_{x \to a-} f(x), f(a)$ -- конечные, но не все равны, то $a$ -- точка разрыва I рода. \\  
\end{Def}

\begin{Def}
    Если хотя бы один предел не существует или бесконечен -- II рода.
\end{Def}

\begin{Def}
    Если в точке $a$ разрыв, но мы можем доопределить или переопределить $f$ в точке $a$ до непрерывности, то $a$ -- точка устранимого разрыва.
\end{Def}

\Section{Пределы функций}{}{Илья Дудников}

\Subsection{$\varepsilon$-окрестности}

\begin{Def}
    $\varepsilon$-окрестность точки $a - V_\varepsilon (a) : (a - \varepsilon, a + \varepsilon)$ \\
    проколотая $\varepsilon$-окрестность $a - \dot{V}_\varepsilon : (a - \varepsilon, a) \cup (a + \varepsilon)$   
\end{Def}

\begin{Def}
    $D \subset \R, a \in \R$. Точка $a$ называется точкой сгущения $D$, если в любой окрестности $a$ найдется точка из $D$, отличная от $a$
    \[\forall \dot{V}(a) \ \exists x \in D : x \in \dot{V}(a) \wedge x \neq a\]
\end{Def}

\begin{Example}
    $D = [1, 2)$. Точки сгущения: $[1, 2]$ 
\end{Example}

\begin{Rem}
    Точка сгущения может принадлежать множеству, а может и не принадлежать.
\end{Rem}

\begin{Rem}
    Если $a$ -- точка сгущения, тогда в $\forall \dot{V}(a)$ бесконечно много точек из $D$.
\end{Rem}

\begin{Rem}
    Точки сгущения называют предельными точками множества. \\
    $a$ -- точка сгущения $\EQ \exists \{x_n\} : x_n \in D, x_n \neq a, x_n \to a$ 
\end{Rem}

\begin{proof}
    "$\SO$". $\varepsilon = \frac{1}{k} \SO |x_k - a| < \frac{1}{k} \SO 0 \leqslant \lim |x_k - a| < 0 \SO \exists \lim x_k = 0$ \\
    "$\Leftarrow$". В $\forall V(a)$ лежит бесконечно много точек $\{x_n\}, x_n \neq a \SO a$ -- точка сгущения. 
\end{proof}

\begin{Def}
    $a \in D$, но $a$ -- не предельная точка. Тогда $a$ называется изолированной точкой множества $D$
\end{Def}

\begin{Rem}
    $+\infty$ может быть предельной точкой множества
    \[\dot{V}(+\infty) = (E, +\infty)\] 
\end{Rem}

\Subsection{Предел функции}

\begin{Def}
    $f: D \to \R, D \subset \R, a \in \R$ -- предельная точка $D$.

    Число $A \in \R$ называется пределом $f$ в точке $a$
    \[\lim_{x \to a} f(x) = A \text{ или } f(x) \xrightarrow[x \to a]{} A\]
    если выполняется одно из следующих условий:

    \begin{MyList}
        \item $\forall \varepsilon > 0 \ \exists \delta > 0 : \forall x \in D \setminus \{a\} : |x - a| < \delta \to |f(x) - A| < \varepsilon$ (Определение по Коши, определение на языке $\delta, \varepsilon$ )
        \item $\forall V(A) \ \exists V(a) : f(\dot{V}(a) \cap D) \subset V(A)$ (Определение на языке окрестностей)
        \item $\forall \{x_n\} : x_n \in D, x_n \neq a, x_n \to a \SO f(x_n) \to A$ (Определение по Гейне, на языке последовательностей)
    \end{MyList}
\end{Def}

\Pagebreak
\begin{Thm}[Эквивалентность определения по Коши и по Гейне]
    Определения 1) и 3) эквивалентны.
\end{Thm}

\begin{proof}
    1) $\SO$ 3). Рассмотрим какую-то $\{x_n\} : x_n \neq a, x_n \in D, x_n \to a$ (она существует по доказанному).
    Нужно доказать, что $f(x_n) \to A$.  \\
    Пусть $x_n \to a$, то 
    \[\forall \delta > 0 \ \exists N : \forall n \geqslant N \to |x_n - a| < \delta \SO |f(x_n) - A| < \varepsilon\]

    3) $\SO$ 1). Пусть это не так, т.е. 1) не выполнено
    \[\exists \varepsilon > 0 : \forall \delta > 0 \ \exists x \in D, x \neq a, |x - a| < \delta: |f(x) - A| \geqslant \varepsilon\]
    Возьмем последовательность $\delta_n = \frac{1}{n}$.
    \[|x_n - a| < \frac{1}{n} \SO x_n \to a \SO f(x_n) \to A, \text{ но } |f(x_n) - A| \geqslant \varepsilon\] 
\end{proof}

\begin{Rem}
    в $\overline{\R}$ 
    \begin{MyList}
        \item \[\lim_{x \to 5} f(x) = \infty \EQ \forall E \ \exists \delta > 0 : \forall x \in D \setminus \{5\}, |x - 5| < \delta \to f(x) > E\]
        \item \[\lim_{x \to -\infty} f(x) = 2 \EQ \forall E > 0 \ \exists \Delta : \forall x \in D, x < \Delta \to |f(x) - 2| < E\]
    \end{MyList}
\end{Rem}

\begin{Rem}
    В определении по Гейне есть "$\forall \{x_n\}$". Если $x_n$ и $y_n$ подходят под условия, то $\lim f(x_n) = \lim f(y_n)$ 
\end{Rem}

\begin{proof}
    Возьмем $z_n: z_1 = x_1, z_2 = y_1, z_3 = x_2, z_4 = y_2$ и т.д. $\{z_n\}$ подходит под определение $\SO \exists \lim f(z_n)$ 
\end{proof}

\begin{Rem}
    В определении предела функции не участвует значения функции в точке $a$.
\end{Rem}

\begin{Rem}
    Последовательность -- частный случай функции.
\end{Rem}

\Section{Непрерывность}{}{Илья Дудников}

\begin{Def}
    Функция называется непрерывной на множестве $D$, если она непрерывна в каждой точке $D$. \\
    $C(D)$ -- множество функций, непрерывных на $D$. $\langle a, b \rangle$ -- промежуток (неважно, включаются концы или нет).  
\end{Def}

\begin{Thm}[об арифметических действиях над непрерывными функциями]
    $f, g : D \to \R, D \in \R$ -- непрерывны в точке $x_0 \in D$. Тогда $f + g, f - g, |f|, f \cdot g$ также непрерывны в точке $x_0$.
    Если $g(x_0) \neq 0$, то $\frac{f}{g}$ тоже непрерывна в точке $x_0$.      
\end{Thm}

\begin{proof}
    \begin{MyItemize}
        \item $x_0$ -- изолированная точка $D$ -- очевидно.
        \item $x_0$ -- предельная точка. 
        $f(x) \xrightarrow[x \to x_0]{} f(x_0)$ и $g(x) \xrightarrow[x \to x_0]{} g(x_0)$. Тогда $f(x) + g(x) \xrightarrow[x \to x_0]{} f(x_0) + g(x_0)$.
        Далее по теореме об арифметических действиях с пределами функций, имеющих предел.   
    \end{MyItemize}
\end{proof}

\begin{Rem}
    Если $f$ непрерывна в точке $x_0 \in D$ и $f(x_0) \neq 0$, то найдется $V(x_0)$, что знак $f$ в $V(x_0) \cap D$ совпадает со знаком $f(x_0)$ 
\end{Rem}

\begin{Thm}[О непрерывности композиции]
    $f: D \to \R, g: E \to R, f(D) \subset E$. Пусть $f$ непрерывна в точке $x_0 \in D$ и $g$ непрерывна в точке $f(x_0)$. Тогда $g \circ f$ непрерывна в точке $x_0$.  
\end{Thm}

\begin{proof}
    Пусть $x_n \in D, x_n \to x_0$. Обозначим $y_n = f(x_n), y_0 = f(x_0)$. Т.к. $f$ непрерывна в точке $x_0$, то $y_n \to y_0$. Тогда $g(y_n) \to g(y_0)$, т.к. $g$ непрерывна в точке $y_0$.
    \[g(y_n) = g(f(x_n)) \to g(f(x_0))\]    
\end{proof}

\begin{Thm}[Первая теорема Вейерштрасса]
    Непрерывная на отрезке функция ограничена.
\end{Thm}

\begin{proof}
    $f \in C[a, b]$. Пусть $f$ не ограничена на $[a, b]$, т.е.
    \[\forall n \in \N \exists x_n \in [a, b] : |f(x_n)| > n\]  
    $x_n$ -- ограничена $\SO \exists \{f(x_{n_k})\}_{k = 1}^\infty : x_{n_k} \to c \in [a, b]$. Т.к. $f$ непрерывна, то $f(x_{n_k}) \to f(c) \SO \{f(x_{n_k})\}$ ограничена, т.к. сходится, но
    \[|f(x_{n_k})| > n_k \geqslant k \ \forall k \in \N\]   
    Получили противоречие
\end{proof}

\begin{Rem}
    Если возьмем интервал $(a, b)$, то теорема не выполняется. 
\end{Rem}

\begin{Thm}[Вторая теорема Вейерштрасса]
    Непрерывная на отрезке функция принимает наибольшее и наименьшее значение.
\end{Thm}

\begin{proof}
    $M = \sup_{x \in [a, b]} f(x_0)$. По первой теореме Вейерштрасса $f$ ограничена на $[a, b] \SO M \in \R$. Пусть $f$ не достигает $M$. Тогда $f(x) < M$ на $[a, b]$. 
    Рассмотрим $\varphi(x) = \frac{1}{M - f(x)}$ -- непрерывна на $[a, b]$. Значит она ограничена на $[a, b]$. $\exists m : \varphi(x) \leqslant m \ \forall x \in [a, b]$
    \[\frac{1}{M - f(x)} \leqslant m \EQ \frac{1}{m} \leqslant M - f(x) \EQ f(x) \leqslant M - \frac{1}{m}\]
    Значит $M$ -- не супремум -- противоречие       
\end{proof}

\begin{Thm}[Больцано-Коши о промежуточном значении]
    $f$ -- непрерывна на $[a, b]$. Тогда $\forall C$, лежащего между $f(a)$ и $f(b) \ \exists c \in (a, b) : f(c) = C$   
\end{Thm}

\begin{proof}
    \begin{MyItemize}
        \item Пусть $f(a)$ и $f(b)$ -- разных знаков. Тогда докажем, что $\exists c \in (a, b) : f(c) = 0$. Пусть $f(a) < 0 < f(b)$.
        Рассмотрим точку $ \frac{a + b}{2}$. Если $f( \frac{a + b}{2}) = 0$, то теорема доказана.
        Если $f(\frac{a + b}{2}) > 0$, то будем далее рассматривать отрезок $[a, \frac{a + b}{2}]$, иначе будем рассматривать отрезок $[ \frac{a + b}{2}, b]$. \\
        Получим $[a_1, b_1] : f(a_1) < 0 < f(b_1)$ и т.д. $[a_n, b_n]$ -- стягивающиеся отрезки $\SO \exists ! c \in \bigcap_{n = 1}^\infty [a_n, b_n], a_n, b_n \to c$
        \[f(a_n) < 0 < f(b_n) \EQ f(c) \leqslant 0 \leqslant f(c) \SO f(c) = 0\]         
        \item Рассмотрим $\varphi(x) = f(x) - C, \varphi \in C[a, b]$, $\varphi(a)$ и $\varphi(b)$ разных знаков.
        Тогда $\exists c \in (a, b) : \varphi(c) = 0 \SO f(c) = C$ 
    \end{MyItemize}
\end{proof}

\begin{Cons}
    Если непрерывная на отрезке функция принимает какие-то два значения, то она принимает и все значения между ними.
\end{Cons}

\begin{Thm}[О сохранении промежутка]
    Множество значений непрерывной на промежутке функции есть промежуток. 
\end{Thm}

\begin{proof}
    Пусть $f \in C \langle a, b\rangle$
    \[m = \inf_{x \in \langle a, b\rangle} f(x), M = \sup_{x \in \langle a, b\rangle} f(x)\]
    $m, M \in \overline{\R}, E = f(\langle a, b\rangle)$. Возьмем $x_1, x_2 \in \langle a, b\rangle$. $f$ принимает все значения между $f(x_1)$ и $f(x_2)$.
    Если $E$ не промежуток, то $\exists y \in E : f(x) \neq y \ \forall x \in \langle a, b\rangle$, но $\exists y_1 < y < y_2 : \exists x_1 : f(x_1) = y_1, \exists x_2 : f(x_2) = y_2$       
\end{proof}

\begin{Thm}[О разрывах и непрерывности монотонной функции]
    $f : \langle a, b\rangle \to \R$, мононтонна. Тогда
    \begin{MyList}
        \item $f$ не может иметь разрывов II рода
        \item $f$ -- непрерывная $\EQ$ её множество значения -- промежуток  
    \end{MyList} 
\end{Thm}

\begin{proof}
    \begin{MyList}
        \item Пусть $f$ возрастает. $x \in (a, b\rangle, x_1 \in \langle a, x_0)$. Тогда $f(x_1) \leqslant f(x) \leqslant f(x_0) \ \forall x \in (x_1, x_0) \SO$
        $f$ возрастает и ограничена сверху на $(x_1, x_0) \SO \exists$ конечный $f(x_0-)$. Кроме того, по используя предельный переход:
        \[f(x_1) \leqslant f(x_0-) \leqslant f(x_0)\]
        Повторим для $f(x_0+) \SO$ нет разрывов II рода.
        
        \item "$\SO$". Доказано \\
        "$\Leftarrow$". $f(\langle a, b\rangle)$ -- промежуток. Докажем непрерывность слева в точке $x_0 \in (a, b\rangle$. Пусть $f(x_0-) < f(x_0)$.
        Возьмем $y \in (f(x_0 -), f(x_0))$. Тогда если $a < x_1 < x_0$, то $y \in [f(x_1), f(x_0)]$. Значит $y$ -- значение функции.
        С другой стороны $\forall x \in \langle a, x_0) \to f(x) \leqslant f(x_0-) < y, \forall x \in [x_0, b\rangle \to f(x) \geqslant f(x_0) > y \SO f$ не принимает значение $y$ -- противоречие.
        Аналогично для $f(x_0+)$     
    \end{MyList}
\end{proof}

\begin{Thm}[Существование и непрерывность обратной функции]
    $f \in C\langle a, b\rangle, f$ строго мононтонна
    \[m = \inf_{x \in \langle a, b\rangle} f(x), M = \sup_{x \in \langle a, b\rangle} f(x)\]
    Тогда
    \begin{MyList}
        \item $f$ обратима, $f^{-1} : \langle m, M\rangle \to \langle a, b\rangle$ -- биекция.
        \item $f^{-1}$ строго монотонна (одноименно с $f$)
        \item $f^{-1}$  непрерывна на $\langle m, M\rangle$  
    \end{MyList}
\end{Thm}

\begin{proof}
    Пусть $f$ возрастает.
    \begin{MyList}
        \item $x_1, x_2 \in \langle a, b\rangle, x_1 < x_2$. Тогда $f(x_1) < f(x_2) \SO f$ обратима. \\
        $f(\langle a, b\rangle) = \langle m, M\rangle$. $f^{-1} : \langle m, M\rangle \to \langle a, b\rangle$. Если $y_1 \neq y_2 \in \langle m, M\rangle \SO f^{-1}(y_1) \neq f^{-1}(y_2)$       

        \item $y_1 < y_2 \in \langle m, M\rangle \SO y_1 = f(x_1), y_2 = f(x_2), x_1, x_2 \in \langle a, b\rangle$. $x_1 = f^{-1}(y_1), x_2 = f^{-1}(y_2), x_1 < x_2$ из-за возрастания $f$
        \item $f^{-1}$ строго возрастает на $\langle m, M\rangle$, множество значений функции $f^{-1}$ -- промежуток $\SO f^{-1}$ непрерывна по предыдущей теореме.    
    \end{MyList}
\end{proof}


\Section{Элементарные функции}{}{Илья Дудников}

\Subsection{Постоянная}

$f(x) = c, x \mapsto c$, непрерывна на $\R$

\Subsection{Степенная функция}

$e_\alpha (x) = x^\alpha$

При $\alpha = 1 \ e_1 (x) = x$ -- непрерывна на $\R$

При $\alpha = n \in \N$ 
\[e_\alpha (x) = x^n\]
Следовательно $e_n (x)$ непрерывна на $\R$ как произведение непрерывных.

При $\alpha = -n, n \in \N$
\[x^{-n} = \frac{1}{x^n}, \t x \in \R \setminus \{0\}\] 
Непрерывна на $\R \setminus \{0\}$ как частное непрерывных.

При $\alpha = 0$ полагаем $x^0 = 1$ при всех $x \neq 0$. Можно доопределить до непрерывности ( $0^0 = 1$ )

Если $n$ нечётно, то $e_n$ строго возрастает на $\R, \sup_{x \in \R} e_n(x) = +\infty, \inf_{x \in \R} e_n(x) = -\infty$. По теореме о сохранении промежутка $e_n(\R) = \R$.

Если $n$ четно, то функция $e_n$ строго возрастает на $\R_+, \sup_{x \in \R_+} e_n(x) = +\infty, \min_{x \in \R_+} e_n(x) = 0, e_n(\R_+) = \R_+$.
По теореме о существовании и непрерывности обратной функции существует и непрерывна функция

\[e_{\frac{1}{n}} = \begin{cases}
    e_n^{-1}, n \ \not\vdots \ 2 \\
    \left(e_n|_{R_+}\right)^{-1}, n \ \vdots \ 2 
\end{cases}\]
Это $\sqrt[n]{x}$, строго возрастает и непрерывна на $\R_+$

Теперь определим $x^\alpha$ при рациональном $\alpha = r = \frac{p}{q}, p \in \Z, q \in \N, \frac{p}{q}$ несократима.
\[x^r = (x^p)^{\frac{1}{q}} (e_r = e_{\frac{1}{q}} \circ e_p)\] 
Таким образом, $x^r$ определено следующим образом.
\begin{align*}
    &x > 0, r \text{ любое}, \\
    &x = 0, r \geqslant 0, \\ 
    &x < 0, q \ \not\vdots 2
\end{align*}
$e_r$ непрерывна на своей области определения, строго возрастает на $[0, +\infty)$ при $r > 0$, строго убывает на $(0, +\infty)$ при $r < 0$

\Pagebreak
\Subsection{Показательная функция}

$0^x = 0 \ \forall x > 0$

Пусть $a > 0$. Пока что $a^x$ определена только для $x \in \mathbb{Q}$. Обозначим эту функцию $a^x |_\mathbb{Q}$. Её свойства:
\begin{MyList}
    \item $r < s \SO a^r < a^s, a > 1$ и $a^r > a^s, 0 < a < 1$
    \item $a^{r + s} = a^r a^s$
    \item $(a^r)^s = a^{rs}$
    \item $(ab)^r = a^r b^r$   
\end{MyList}  

\begin{Def}
    Пусть $a > 0, x \in \R$ Положим
    \[a^x = \lim_{r \to x} a^r |_\mathbb{Q}\] 
\end{Def}

\begin{Lm}
    Пусть $a > 0, \{r_n\}$ -- последовательность рациональных чисел, $r_n \to 0$. Тогда $a^{r_n} \to 1$.   
\end{Lm}

\begin{proof}
    При $a = 1$ лемма очевидно, т.к. $a^{r_n} = 1 \ \forall n$.
    
    Пусть $a > 1$. Докажем лемму в частном случае $r_n = \frac{1}{n}$. Поскольку $a^{\frac{1}{n}} > 1$, имеем $a^{\frac{1}{n}} = 1 + \alpha_n, \alpha_n > 0$. Тогда по неравенству Бернулии
    \[a = (1 + \alpha_n)^n \geqslant 1 + n \alpha_n\]
    Откуда $0 < \alpha_n < \frac{a - 1}{n} \SO \alpha_n \to 0 \SO a^{\frac{1}{n}} \to 1$.

    Далее, по доказанному 
    \[a^{-\frac{1}{n}} = \frac{1}{a^{\frac{1}{n}}} \to \frac{1}{1} = 1\]
    Пусть теперь $\{r_n\}$ -- произвольная последовательность из условия леммы. Возьмем $\varepsilon > 0$. $\exists N_0:$
    \[1 - \varepsilon < a^{-\frac{1}{N_0}} < a^{\frac{1}{N_0}} < 1 + \varepsilon\]
    Поскольку $r_n \to 0$, найдется такой номер $N$, что $\forall n > N \to -\frac{1}{N_0} < r_n < \frac{1}{N_0}$. В силу строгой монотонности показательной функции рационального аргумента
    \[1 - \varepsilon < a^{-\frac{1}{N_0}} < a^{r_n} < a^{\frac{1}{N_0}} < 1 + \varepsilon\]
    Значит $a^{r_n} \to 1$
    
    Если $0 < a < 1$, то $\frac{1}{a} > 1$, и по доказанному
    \[a^{r_n} = \frac{1}{\left(\frac{1}{a}\right)^{r_n}} \to 1\]  
\end{proof}

\begin{Lm}
    Пусть $a > 0, x \in \R, \{r_n\}$ -- последовательность рациональных чисел, $r_n \to x$. Тогда существует конечный предел последовательности $\{a^{r_n}\}$
\end{Lm}

\begin{proof}
    При $a = 1$ лемма очевидна. 
    
    Пусть $a > 1$. Возьмем какую-либо возрастающую последовательность $\{s_n\}$ рациональных чисел, стремящуюся к $x$. Например
    \[s_n = \frac{[10^n x]}{10^n}\]
    Тогда $x - \frac{1}{10^n} < s_n \leqslant x \SO s_n \to x$. Докажем, что последовательность $\{s_n\}$ возрастает. Пусть $A = 10^n x$.
    Тогда $s_n \leqslant s_{n + 1} \EQ 10[A] \leqslant [10A]$, но $10[A]$ -- целое число, не превосходящее $10A$.
    
    $\{a^{s_n}\}$ возрастает и ограничена сверху числом $a^{[x] + 1}$. Значит $\{a^{s_n}\}$ сходится к некоторому пределу $L$. Но тогда    
    \[a^{r_n} = a^{r_n - s_n}a^{s_n} \to L\]
    Потому что $a^{r_n - s_n} \to 1$ по предыдущей лемме.
    
    Если $0 < a < 1$, то $\frac{1}{a} > 1$ и по доказанному $\left(\frac{1}{a}\right)^{r_n} \to L, L > 0$. Тогда
    \[a^{r_n} = \frac{1}{\left(\frac{1}{a}\right)^{r_n}} \to \frac{1}{L}\]   
\end{proof}

\Subsubsection{Свойства показательной функции}

\begin{MyList}
    \item $a^x$ строго возрастает на $\R$ при $a > 1$ и строго убывает на $\R$ при $a \in (0, 1)$
    \begin{proof}
        $a > 1$. Пусть $x < y$. Докажем, что $a^x < a^y$. Возьмем два числа $\overline{r}, \overline{\overline{r}} \in \Q$ между $x$ и $y$.
        Возьмем $\{\overline{r}_n\}_{n = 1}^\infty, \{\overline{\overline{r}}\}_{n = 1}^\infty :$ последовательности из $\Q : \overline{r}_n \to x, \overline{\overline{r}}_n \to y$.
        
        По доказанному $a^{\overline{r}_n} < a^{\overline{r}} < a^{\overline{\overline{r}}} < a^{\overline{\overline{r}}_n}$
        \[\SO a^x \leqslant a^{\overline{r}} < a^{\overline{\overline{r}}} \leqslant a^y \SO a^x < a^y\]
        
        $a \in (0, 1)$. Рассмотрим $b = \frac{1}{a} > 1$.  
    \end{proof}
    \item $a^{x + y} = a^x \cdot a^y$
    \begin{proof}
        $\{\overline{r}_n\}, \{\overline{\overline{r}}_n\}$ как в $1)$
        \[a^{\overline{r}_n + \overline{\overline{r}}_n} = a^{\overline{r}_n} \cdot a^{\overline{\overline{r}}_n} \SO a^{x + y} = a^x \cdot a^y\]
    \end{proof}
    \item $a^{-x} = a^0 \cdot a^{-x} = \frac{1}{a^x}$ 
    \item $a^x$ непрерывна на $\R$ 
    \begin{proof}
        $a > 1, \{x_n\}: x_n \to 0$. Докажем непрерывность в нуле.
        \[\forall \varepsilon > 0 \ \exists N : \forall n \geqslant N \to |x_n| < \varepsilon \SO -\frac{1}{n_0} < x_n < \frac{1}{n_0}, n_0 \in \N\]
        Тогда $1 - \varepsilon < a^{-\frac{1}{n_0}} < a^{x_n} < a^{\frac{1}{n_0}} < 1 + \varepsilon \ (a^{\frac{1}{n}} \to 1 \SO \exists n_0 : |a^{\frac{1}{n}} - 1| < \varepsilon)$
        \[\forall \varepsilon > 0 \ \exists N' : \forall n \geqslant N \to |a^{x_n} - 1| < \varepsilon \EQ a^{x_n} \to 1\]

        Докажем непрерывность в точке $x_0 \neq 0$. \\
        Рассмотрим $a^{x_0 + x_n} - a^{x_0} = a^{x_0} (a^{x_n} - 1) \to 0$  
    \end{proof}
    
    \item $(ab)^x = a^x b^x$
    \begin{proof}
        $\{r_n\}$ из $\Q$, $r_n \to x$. Тогда
        \[(ab)^{r_n} = a^{r_n} \cdot b^{r_n} \SO (ab)^x = a^x b^x\] 
    \end{proof}

    \item $(a^x)^y = a^{xy}$ 
    \begin{proof}
        $x_n \to x, y_n \to y, \{x_n\}, \{y_n\}$ из $\Q$. Тогда по непрерывности показательной и степенной функций
        \[(a^{x_n})^{y_m} = a^{x_n \cdot y_m} \xRightarrow[n \to \infty]{} (a^x)^{y_m} = a^{x \cdot y_m} \xRightarrow[m \to \infty]{} (a^x)^y = a^{x \cdot y}\] 
    \end{proof}

    \item $a^x$ -- биекция из $\R$ на $(0, +\infty)$  
    \begin{proof}
        $a > 1$. Тогда $a^x$ строго возрастает на $\R$.
        \[a^n = (1 + \alpha)^n \geqslant 1 + n \alpha \to +\infty \SO \lim_{x \to +\infty} a^x = +\infty\]
        \[\lim_{x \to -\infty} a^x = 0\]
    \end{proof}
\end{MyList}

\Subsection{Логарифм}

\begin{Def}
    Т.к. $a^x : \R \to (0, +\infty)$ -- биекция, то $\exists f^{-1} : (0, +\infty) \to \R$.
    \[\log_a x : (0, +\infty) \to \R\]
    Из теоремы об обратной функции $\log_a x$ монотонна и непрерывна.   
\end{Def}

\Subsubsection{Свойства логарифма}

\begin{MyList}
    \item $\log_a x + \log_a y = \log_a (xy), a \in (0, 1) \cup (1, +\infty), x, y > 0$
    \begin{proof}
        \[a^{\log_a x + \log_a y} = a^{\log_a x} \cdot a^{\log_a y} = x \cdot y = a^{\log_a (xy)}\]
    \end{proof}

    \item $\log_a x^b = b\log_a x, a \in (0, 1) \cup (1, +\infty), x > 0, b \in \R$
    \begin{proof}
        \[a^{b \log_a x} = (a^{log_a x})^b = x^b = a^{log_a x^b}\]
    \end{proof}

    \item $\log_a x = \frac{\log_b x}{\log_b a}, a, b \in (0, 1) \cup (1, +\infty), x > 0$
    \begin{proof}
        \[b^{log_a x \cdot \log_b a} = (b^{\log_b a})^{\log_a x} = a^{\log_a x} = x = b^{\log_b x}\] 
    \end{proof}
\end{MyList}

\begin{Def}
    $\ln x$ -- натуральный логарифм ( $\log_e x$ )
\end{Def}

Вернемся к степенной функции:

\begin{Def}
    $x > 0, \alpha \in \R \ x^\alpha = e^{\alpha \cdot \ln x}$. $0^\alpha = 0$. Покажем непрерывность справа в точке $0$.
    \[x_n \to 0, x_n > 0\]
    Пусть $y_n = \ln x_n \xrightarrow[n \to \infty]{} -\infty$. Значит $x_n^\alpha = e^{\alpha \ln x_n} \xrightarrow[n \to \infty]{} 0$ 
    \begin{align*}
        &x^\alpha : [0, +\infty) \to [0, +\infty), \alpha > 0 \text{ -- биекция} \\
        &x^\alpha : (0, +\infty) \to (0, +\infty), \alpha < 0 \text{ -- биекция}
    \end{align*} 
\end{Def}

\Subsection{Тригонометрические функции}

\begin{Prop}
    $x \in (0, \frac{\pi}{2})$. Тогда $\sin x < x < \tg x$.
\end{Prop}

\begin{proof}

    Нужно доказать: \(BC < \stackrel{\frown}{AB} < AD\) \\
    \begin{figure*}[h]
        \centering
        \input{images/trig.pdf_tex}
    \end{figure*}
    $\triangle OBA \subset \downslice OAB \subset \triangle OAD \EQ S_{\triangle OBA} < S_{\downslice OAB} < S_{\triangle OAD}$
    \begin{align*}
        S_{\triangle OBA} &= \frac{1}{2} |OA| \cdot |BC| = \frac{\sin x}{2} \\
        S_{\downslice OAB} &= \frac{1}{2} \cdot x \cdot |OA|^2 = \frac{x}{2} \\
        S_{\triangle OAD} &= |OA| \cdot |AD| \cdot \frac{1}{2} = 1 \cdot \tg x \cdot \frac{1}{2} = \frac{\tg x}{2} \\
    \end{align*} 
    Отсюда 
    \[ \frac{\sin x}{2} < \frac{x}{2} < \frac{\tg x}{2} \EQ \sin x < x < \tg x\]
\end{proof}

\begin{Cons}
    $|\sin x| \leqslant |x| \ \forall x \in \R$ (причем равенство достигается только в $0$ ) \\
    При $x \in (0, \frac{\pi}{2})$ доказано.
    \begin{align*}
        x &\geqslant \frac{\pi}{2} : |\sin x| \leqslant 1 < \frac{\pi}{2} \leqslant x \\
        x &\leqslant -\frac{\pi}{2} : |\sin x| = |\sin(-x)| < |-x| = x
    \end{align*} 
\end{Cons}

Свойства:
\begin{MyList}
    \item $\sin x$ -- непрерывная на $\R$ функция.
    \[\lim_{x \to x_0} \sin x = \sin x_0 \]
    \begin{proof}
        \[|\sin x - \sin x_0| = \left|2 \cdot \sin \frac{x - x_0}{2} \cdot \cos \frac{x + x_0}{2}\right| \leqslant 2 \left|\sin \frac{x - x_0}{2}\right| \leqslant 2 \left| \frac{x - x_0}{2}\right| \leqslant |x - x_0| \to 0\]
    \end{proof}

    \item $\cos x = \sin \left(\frac{\pi}{2} - x\right)$ -- непрерывна. 
    \item $\tg x = \frac{\sin x}{\cos x}, x \in \R \setminus \{\frac{\pi}{2} + \pi k, k \in \Z\}$
    \item $\ctg x = \frac{\cos x}{\sin x}, x \in \R \setminus \{\pi k, k \in \Z\}$ -- непрерывны на области определения.
\end{MyList}

\Subsubsection{Обратные тригонометрические функции}

$\sin x : \R \to [-1, 1]$ не обратимая.

$\sin x |_{x \in \left[-\frac{\pi}{2}, \frac{\pi}{2}\right]} : \left[-\frac{\pi}{2}, \frac{\pi}{2}\right] \to [-1, 1]$ -- биекция

\begin{Def}
    $\arcsin x = \left(\sin x |_{x \in \left[-\frac{\pi}{2}, \frac{\pi}{2}\right]}\right)^{-1}$. Монотонно возрастает и непрерывна 
\end{Def}

\begin{Def}
    $\arccos x = \left(\cos x |_{x \in [0, \pi]}\right)^{-1}$. Убывает, непрерывна
\end{Def}

\begin{Def}
    $\arctg x = \left(\tg x |_{x \in \left(-\frac{\pi}{2}, \frac{\pi}{2}\right)}\right)^{-1}$. Непрерывна, строго возрастает.
\end{Def}

\begin{Def}
    $\arcctg x = \left(\ctg x |_{x \in (0, \pi)}\right)^{-1}$ 
\end{Def}

\begin{Rem}
    Для обратимости строго монотонной функции непрерывность не нужна. 
\end{Rem}


\Section{Замечательные пределы}{}{Илья Дудников}

\begin{Def}
    Первый замечательный предел:
    \[\lim_{x \to 0} \frac{\sin x}{x} = 1\]    
\end{Def}

\begin{proof}
    $\sin x < x < \tg x$ на $\left(0, \frac{\pi}{2}\right)$, $\cos x < \frac{\sin x}{x} < 1$ \\
    $\cos x, \frac{\sin x}{x}, 1$ -- четные функции, значит верно и для $x \in \left(-\frac{\pi}{2}, 0\right)$. Перейдем к пределу при $x \to 0$
    \[1 \leqslant \lim_{x \to 0} \frac{\sin x}{x} \leqslant 1 \SO \exists \lim_{x \to 0} \frac{\sin x}{x} = 1\]
\end{proof}

\begin{Cons}
    $\lim_{x \to 0} \frac{1 - \cos x}{x^2} = \frac{1}{2}$ 
\end{Cons}

\begin{proof}
    $\cos 2\alpha = 1 - 2\sin^2 \alpha \EQ \sin^2 \alpha = \frac{1 - \cos 2\alpha}{2}$ 
    \[ \frac{1 - \cos x}{x^2} = \frac{2 \sin^2 \frac{x}{2}}{x^2} = \frac{1}{2} \cdot \frac{\sin^2 \frac{x}{2}}{\left(\frac{x}{2}\right)^2} = \frac{1}{2} \left( \frac{\sin \frac{x}{2}}{\frac{x}{2}}\right)^2 \to \frac{1}{2}\]
\end{proof}

\begin{Cons}
    $\lim_{x \to 0} \frac{\tg x}{x} = 1$ 
\end{Cons}

\begin{proof}
    \[\frac{\tg x}{x} = \frac{\sin x}{x} \cdot \frac{1}{\cos x} \xrightarrow[x \to 0]{} 1\]
\end{proof}

\begin{Cons}
    $\lim_{x \to 0} \frac{\arcsin x}{x} = 1$ 
\end{Cons}

\begin{proof}
    $\frac{\sin x}{x} = \frac{y}{\arcsin y}$, $y = \sin x$ в окрестности $x \in (-\varepsilon, \varepsilon) \SO \arcsin y = x$. 
    $\arcsin x$ непрерывна в нуле, в $0$ равен $0$. $ \frac{\sin x}{x}$ непрерывна в $(-\varepsilon, \varepsilon) \setminus \{0\}$  

    \[g(x) = \begin{cases}
        \frac{\sin x}{x}, x \neq 0 \\
        1, x = 0
    \end{cases} \text { -- непрерывна на } \R\]
    $\SO$ по теореме о непрерывности композиции $\frac{y}{\arcsin y} \xrightarrow[y \to 0]{} 1$  
\end{proof}

\begin{Cons}
    $\lim_{x \to 0} \frac{\arctg x}{x} = 1$ 
\end{Cons}

\begin{proof}
    Аналогично предыдущему следствию.
\end{proof}

\begin{Def}
    Второй замечательный предел
    \begin{align*}
        &\lim_{x \to +\infty} \left(1 + \frac{1}{x}\right)^x = e \\
        &\lim_{x \to -\infty} \left(1 + \frac{1}{x}\right)^x = e \\
        &\lim_{x \to 0}(1 + x)^{\frac{1}{x}} = e
    \end{align*}
\end{Def}

\begin{proof}
    $f(x) = \left(1 + \frac{1}{x}\right)^x$ задана на $\R \setminus [-1, 0]$. Пусть $x_n \to +\infty$. Нужно доказать, что $f(x_n) \to e$.
    
    \begin{MyList}
        \item Рассмотрим $\{x_n\}$ из $\N$. $f(x_n) \to e$ как подпоследовательность.
        \item $\{x_n\}$ из $\R$. Начиная с некоторого номера $x_n \geqslant 1$.
        \[\left(1 + \frac{1}{[x_n] + 1}\right)^{[x_n]} \leqslant \left(1 + \frac{1}{x_n}\right)^{x_n} \leqslant \left(1 + \frac{1}{[x_n]}\right)^{[x_n] + 1}\]
        Очевидно, $[x_n] \leqslant x_n \leqslant [x_n] + 1$. Тогда
        \[\frac{1}{1 + \frac{1}{[x_n] + 1}} \cdot f([x_n] + 1) \leqslant f(x_n) \leqslant f([x_n]) \cdot \left(1 + \frac{1}{[x_n]}\right)\] 
        $\{[x_n]\}_{n = 1}^\infty$ -- последовательность из $\N$. Выполним предельный переход в неравенстве.
        \[e \leqslant \lim_{n \to \infty} f(x_n) \leqslant e \SO \exists \lim_{n \to \infty} f(x_n) = e\]  
    \end{MyList}
\end{proof}

\begin{Def}
    Третий замечательный предел (обычно не нумеруется).
    \[\lim_{x \to 0} \frac{\log_a (1 + x)}{x} = \frac{1}{\ln a}, a > 0, a \neq 1\]
\end{Def}

\begin{proof}
    $\log_a (1 + x) = \frac{\ln (1 + x)}{\ln a}$ 
    \[\lim_{x \to 0} \frac{\ln (1 + x)}{x} = \lim_{x \to 0} \ln (1 + x)^{\frac{1}{x}} = \ln e = 1\]
\end{proof}

\begin{Def}
    Четвертый замечательный предел (обычно не нумеруется)
    \[\lim_{x \to 0} \frac{(1 + x)^\alpha - 1}{x} = \alpha, \alpha \in \R\]
\end{Def}

\begin{proof}
    $\alpha = 0$ тривиально. \\
    $\alpha \neq 0$. $x_n \to 0, x_n \neq 0, |x_n| < 1 \ \forall n$. Обозначим 
    \[y_n = (1 + x_n)^\alpha - 1 \xrightarrow[n \to \infty]{} 0, y_n \neq 0 \SO \alpha \ln(1 + x_n) = \ln(1 + y_n)\]   
    Тогда
    \[ \frac{(1 + x_n)^\alpha - 1}{x_n} = \frac{y_n}{x_n} = \frac{y_n}{\ln (1 + y_n)} \cdot \frac{\alpha \ln(1 + x_n)}{x_n} \to \alpha\]
\end{proof}

\begin{Def}
    Пятый замечательный предел (обычно не нумеруется)
    $\lim_{x \to 0} \frac{a^x - 1}{x} = \ln a, a > 0$ 
\end{Def}

\begin{proof}
    $a = 1$ тривиально. \\
    $a \neq 1$. $x_n \to 0, x_n \neq 0$
    \[y_n = a^{x_n} - 1 \to 0, y_n \neq 0, \ln (1 + y_n) = x_n \cdot \ln a\]  
    \[ \frac{a^{x_n} - 1}{x_n} = \frac{y_n}{x_n} = \frac{y_n}{\ln(1 + y_n)} \cdot \frac{x_n \ln a}{x_n} \xrightarrow[n \to \infty]{} \ln a\]
\end{proof}

\Section{Сравнение функций}{}{Илья Дудников}

\begin{Def}
    $f, g : D \to \R, D \subset \R, x_0$ -- предельная точка $D$ и $\exists \PHI : D \to \R : f(x) = \PHI (x) \cdot g(x) $ в $\dot{V}(x_0) \cap D$.
    
    \begin{MyList}
        \item Если $\PHI(x)$ ограничена на $\dot{V}(x_0) \cap D$, то говорят, что $f$ ограничена по сравнению с $g$ при $x \to x_0$
        \[f(x) = O(g(x)), x \to x_0\]
        \item Если $\PHI(x) \xrightarrow[x \to x_0]{} 0$, то говорят, что $f$ бесконечно малая по сравнению с $g$ при $x \to x_0$
        \[f(x) = o(g(x)), x \to x_0\]
        \item Если $\PHI(x) \xrightarrow[x \to x_0]{} 1$, то говорят, что $f$ и $g$ асимптотически равны.
        \[f(x) \thicksim g(x), x \to x_0\]
    \end{MyList}
\end{Def}

\begin{Rem}
    \begin{MyList}
        \item $ \frac{f(x)}{g(x)}$ ограничена в $\dot{V} \cap D$
        \item $\lim_{x \to x_0} \frac{f(x)}{g(x)} = 0$
        \item $\lim_{x \to x_0} \frac{f(x)}{g(x)} = 1$    
    \end{MyList}
\end{Rem}

\begin{Rem}
    Все пункты при $x \to x_0$ 
    \begin{MyList}
        \item $f \thicksim g, g \thicksim h \SO f \thicksim h$ 
        \item $f \thicksim f$ 
        \item $f \thicksim g, f = g + o(g), f = g + o(f)$ -- равносильные утверждения.
        \item Если $f = o(g)$, то $f = O(g)$ 
    \end{MyList}
\end{Rem}

\begin{Cons}
    При $x \to 0$
    \[
    \begin{array}{cc}
    \sin x = x + o(x) & \ln(1 + x) = x + o(x)  \\ 
    \tg x = x + o(x)  & \arcsin x = x + o(x) \\ 
    \cos x = 1 - \frac{x^2}{2} + o(x^2)  & a^x = 1 + x\ln a + o(x)  \\ 
    (1 + x)^\alpha = 1 + \alpha x + o(x)  &
    \end{array}
    \]
\end{Cons}

\begin{Thm}[О замене на эквивалентные]
    $f, \widetilde{f}, g, \widetilde{g} : D \to \R, x_0$ -- предельная точка $D$, $f \thicksim \widetilde{f}, g \thicksim \widetilde{g}$ при $x \to x_0$. 
    Тогда
    \begin{MyList}
        \item $\lim_{x \to x_0} f(x) \cdot g(x) = \lim_{x \to x_0} \widetilde{f}(x) \cdot \widetilde{g}(x)$
        \item $\lim_{x \to x_0} \frac{f(x)}{g(x)} = \lim_{x \to x_0} \frac{\widetilde{f}(x)}{\widetilde{g}(x)}$ (если $x_0$ -- предельная точка области определения $\frac{f}{g}$ )  
    \end{MyList}
\end{Thm}

\begin{proof}
    $\exists u(x_0) \ \exists \PHI : \PHI(x) \xrightarrow[x \to x_0]{} 1$. $f = \PHI \cdot \widetilde{f}$ в $u(x_0) \cap D$ \\
    $\exists v(x_0) \ \exists \psi : \psi(x) \xrightarrow[x \to x_0]{} 1$. $g = \psi \cdot \widetilde{g}$ в $v(x_0) \cap D$ \\
    $w(x_0) = u(x_0) \cap v(x_0)$. Тогда $f \cdot g = (\PHI \cdot \psi) \cdot \widetilde{f} \cdot \widetilde{g}$ в $w(x_0)$.
    Пусть $\lim_{x \to x_0} g \cdot f = A \in \R \EQ$ т.к. $\PHI \cdot \psi \xrightarrow[x \to x_0]{} 1$, то $\lim_{x \to x_0} \widetilde{f} \cdot \widetilde{g} = A$.
    Если $\lim_{x \to x_0} g \cdot f$ не существует, то $\lim_{x \to x_0} \widetilde{f}\widetilde{g}$ не существует.  \\
    Для частного доказательство аналогично.    
\end{proof}

\begin{Rem}
    Заменять на эквивалентные можно \textbf{только} в произведении и частном.
\end{Rem}

\begin{Def}
    Пусть $f \thicksim g, f \thicksim h, x \to x_0$. Если $f - h = o(f - g)$, то говорят, что асимптотическое равенство $f \thicksim h$ точнее,
    чем $f \thicksim g$.
\end{Def}

Пусть $f: D \to \R, D \subset \R, x_0$ -- предельная точка $D$.
Пусть задана система функций $\{g_k\}_{k = 0}^N : \forall k \in [0, N - 1] \cap \Z_+ \to g_{k + 1}(x) = o(g_k(x)), x \to x_0$

\[f(x) = \sum_{k=0}^{N} c_k \cdot g_k(x) + o(g_N(x))\] 

Многочлены получаются, если $g_k(x) = (x - x_0)^k$.

Если $f(x) \thicksim C \cdot (x - x_0)^k (C \neq 0)$, то $C \cdot (x - x_0)^k$ -- главная степенная часть.

\begin{Thm}[О единственности асимптотического разложения]
    $D \in \R, x_0$ -- предельная точка $D$, $n \in \Z_+; f, g_k : D \to \R, g_{k + 1}(x) = o(g_k(x)), x \to x_0 \ \forall k = 0, ..., n - 1$
    и $\forall V(x_0) \ \exists $ точка в $\dot{V}(x_0) : $ в ней $g_n$ не ноль. Тогда если существует асимптотическое разложение $f$ по системе функций $\{g_k\}$,
    то оно единственно.   
\end{Thm}

\begin{proof}
    Пусть не единственное. Тогда $\exists c_k, d_k, k = 0, ..., n : \exists i \ c_i \neq d_i$.
    
    $f(x) = \sum_{k=0}^{n} c_k \cdot g_k(x) + o(g_n(x))$ и $f(x) = \sum_{k=0}^{n} d_k \cdot g_k(x) + o(g_n(x))$ при $x \to x_0$.
    
    Т.к. $g_{k + 1}(x) = o(g_k(x))$, то $g_{k + 1}(x) = o(g_l(x)) \ \forall l \leqslant k$ при $x \to x_0$.

    Обозначим $E_k = \{x : g_k(x) \neq 0\}, k = 0, ..., n$. Если $g_k = 0$ на $V(x_0)$, то $g_{k + 1} = 0$ на $V(x_0), g_n = 0$ на $V(x_0)$. 
    \[g_{k + 1} = o(g_k) \EQ \exists \PHI : g_{k + 1} = \PHI \cdot g_k\]

    Если $x_0$ -- предельная точка $E_{k_0}$, то она предельная точка всех $E_k$.
    Пусть $m$ -- наименьший номер $: c_m \neq d_m$. Тогда 
    \[f(x) = \sum_{k=0}^{m} c_k g_k(x) + o(g_m(x)), f(x) = \sum_{k=0}^{m} d_k g_k(x) + o(g_m(x))\]
    Вычтем: $0 = (c_m - d_m) g_m(x) + o(g_m(x))$. Поделим на $g_m(x)$ 
    \[0 = (c_m - d_m) + \frac{o(g_m(x))}{g_m(x)} \xrightarrow[x \to x_0]{} c_m - d_m \SO c_m = d_m\]
\end{proof}

\begin{Def}
    $x_0 \in \R, f$ задана хотя бы на $\langle a, x_0)$ или $(x_0, b\rangle$ и действует в $\R$. 
    Тогда прямая $x = x_0$ называется вертикальной асимптотой функции $f$, если 
    \[\lim_{x \to x_0+}f(x) = \pm \infty \vee \lim_{x \to x_0-} f(x) = \pm \infty\] 
\end{Def}

\begin{Def}
    $\langle a, +\infty) \subset D \subset \R, f : D \to \R, \alpha, \beta \in \R$. 
    Прямая $y = \alpha x + \beta$ -- наклонная асимптота $f$ при $x \to +\infty$, если $f(x) = \alpha x + \beta + o(1)$ при $x \to +\infty$.
\end{Def}

\begin{Def}
    При $x \to -\infty$ аналогично.
\end{Def}

\begin{Thm}[Уравнение наклонной асимптоты]
    $\langle a, +\infty) \subset D \subset \R, f : D \to \R$. $\alpha, \beta \in \R$. \\
    Прямая $y = \alpha x + \beta$ является асимптотой $f$ при $x \to +\infty \EQ \alpha = \lim_{x \to +\infty} \frac{f(x)}{x}, \beta = \lim_{x \to +\infty} (f(x) - \alpha x)$    
\end{Thm}

\begin{proof}
    "$\SO$". По определению $f(x) = \alpha x + \beta + \PHI(x), \PHI \xrightarrow[x \to +\infty]{} 0$.
    Тогда $ \frac{f(x)}{x} = \alpha + \frac{\beta}{x} + \frac{\PHI(x)}{x}$
    \[\lim_{x \to +\infty} \frac{f(x)}{x} = \alpha\]

    $f(x) - \alpha x = \beta + \PHI(x)$ 
    \[\lim_{x \to +\infty} (f(x) - \alpha x) = \beta\]

    "$\Leftarrow$". Проделаем те же рассуждения "в обратную сторону".
\end{proof}

\Section{Дифференциальное исчисление}{}{Илья Дудников}

Пусть $f: E \to \R, E \subset \R, a$ -- предельная точка $E, n \in \Z_+$. Хотим найти многочлен степени не выше $n$ $(P(x) = \sum_{k=0}^{n} c_k (x - x_0)^k)$
\begin{equation}
f(a) = P(a), f(x) = P(x) + o((x - a)^n), x \to a
\end{equation}

\begin{Rem}
    Если такой многочлен существует, то он единственный.
\end{Rem}

\begin{proof}
    Пусть $\exists P(x), Q(x)$, удовлетворяющие условию (1). Тогда 
    \[0 = P(x) - Q(x) + o((x - a)^n)\]
    Если $P(x) \neq Q(x)$, то $P(x) - Q(x) = \sum_{k=0}^{n} r_k (x - a)^k = r(x)$
    \[\SO r(x) = o((x - a)^n), x \to a\]
    $r(x) = r_m(x - a)^m + ... + r_n(x - a)^n, m \leqslant n, r_m \neq 0$
    \[\SO \frac{r(x)}{(x - a)^m} = o((x - a)^{n - m}) \SO r_m = 0\] 
\end{proof}

\begin{Def}
    Многочлен, удовлетворяющий условию (1) называется многочленом Тейлора функции $f$ в точке $a$ порядка $n$ $T_{a, n} f$ 
\end{Def}

\begin{Def}
    Функция $f$ называется дифференцируемой в точке $a$ ( $\langle A, B\rangle \to \R, a \in (A, B)$ ),
    если $\exists k \in \R : $ 
    \[f(x) = f(a) + k(x - a) + o(x - a), x \to a\]
\end{Def}

\begin{Def}
    $f : \langle A, B\rangle \to \R, a \in (A, B)$, если $\exists \lim_{x \to a} \frac{f(x) - f(a)}{x - a} = K \in \R$, то $K$ называется производной 
    функции $f$ в точке $a$. (Обозначение $f'(a), \frac{df}{dx}(a), D f(a)$ ) \\
    $\Delta_a f = f(x) - f(a)$ -- приращение функции $f$ в точке $a$. \\
    $x - a = \Delta_a x$.
    \[f'(a) = \lim_{\Delta_a x \to 0} \frac{\Delta_a f}{\Delta_a x}\]
\end{Def}

\begin{Thm}
    $f : \langle A, B\rangle \to \R, a \in (A, B)$. Тогда равносильны три утверждения:
    \begin{MyList}
        \item $f$ дифференцируема в точке $a$
        \item $\lim_{x \to a} \frac{f(x) - f(a)}{x - a}$ существует и равен $k$
        \item $\exists F(x) : F: \langle A, B\rangle \to \R, F$ непрерывна в точке $a$, $F(a) = k$ и $f(x) - f(a) = F(x)(x - a), x \in \langle A, B\rangle$ 
    \end{MyList}
\end{Thm}

\begin{proof}
    $1) \SO 2)$. $\exists k : f(x) - f(a) = k(x - a) + o(x - a), x \to a$
    \[ \frac{f(x) - f(a)}{x - a} = k + \frac{o(x - a)}{x - a} \to k\]
    $2) \SO 3)$. 
    \[F(x) = \begin{cases}
        \frac{f(x) - f(a)}{x - a}, x \neq a \\
        k, x = a
    \end{cases}\]
    из 2) следует непрерывность $F$ в точке $a$ 

    $3) \SO 1)$. По 3) $\exists F : $
    \[f(x) - f(a) = F(x)(x - a) \EQ f(x) = f(a) + F(x)(x - a) = f(a) + k(x - a) + (F(x) - k) \cdot (x - a)\] 
    $F(x) \xrightarrow[x \to a]{} F(a) = k \SO (F(x) - k)(x - a) = o((x - a))$  
\end{proof}

\Subsection{Связь с физикой}

\[\lim_{\Delta t \to 0} \frac{\Delta S}{\Delta t} \text{ -- мгновенная скорость}\]

\Subsection{Связь с геометрией}

Рассмотрим функции: $l_k(x) = f(a) + k(x - a)$, графики -- прямые, проходящие через точку $(a; f(a))$
\[f(x) - l_k(x) = f(x) - f(a) - k(x - a)\]
Если $f(x)$ дифференцируема в точке $a$
\[f(x) = f(a) + f'(a)(x - a) + o(x - a) \EQ f(x) - l_k(x) = (x - a) \cdot (f'(a) - k) + o(x - a)\]
При $k = f'(a)$ разность есть $o(x - a)$.
\[y = f(a) + f'(a)(x - a)\]
касательная в точке $a$ к функции $f$. $\tg \alpha = f'(a)$.

\Subsection{Бесконечные производные}

\[\lim_{x \to a} \frac{f(x) - f(a)}{x - a} = +\infty \SO f'(a) = +\infty\]
В таком случае $f$ не является дифференцируемой в точке $a$.

Односторонняя производная:
\[\exists \lim_{x \to a\pm} \frac{f(x) - f(a)}{x - a}\]

\begin{Rem}
    Если $f$ дифференцируема в точке $a$, то $f$ непрерывна в точке $a$.
    \begin{proof}
        \[f(x) - f(a) = f'(a)(x - a) + o(x - a), x \to a \SO f(x) \xrightarrow[x \to a]{} f(a)\]
    \end{proof}
    Обратное не выполняется. Например, $f(x) = |x|$ 
\end{Rem}

\Pagebreak
\Subsection{Правила дифференцирования}

\begin{Thm}[Производная композиции]
    $f: \langle A, B\rangle \to \langle C, D\rangle, g: \langle C, D\rangle \to \R, a \in \langle A, B\rangle$.
    Если $f$ дифференцируема в точке $a$, $g$ дифференцируема в точке $f(a)$, то $g \circ f$ дифференцируема в точке $a$ и 
    \[(g \circ f)'(a) = g'(f(a)) \cdot f'(a)\] 
\end{Thm}

\begin{proof}
    $\exists F : \langle A, B\rangle \to \R, F(a) = f'(a)$ и $f(x) - f(a) = F(x)(x - a), x \in \langle A, B\rangle$, $F$ непрерывна в точке $a$ \\
    $\exists G : \langle C, D\rangle \to \R, G(f(a)) = g'(f(a))$ и $g(y) - g(f(a)) = G(y)(y - f(a)), y \in \langle C, D\rangle$, $G$ непрерывна в точке $f(a)$ 
    Подставим $y = f(x)$
    \[g(f(x)) - g(f(a)) = G(f(x))(f(x) - f(a)) = G(f(x))F(x)(x - a) = H(x)(x - a)\]
    $H(x)$ -- непрерывна в точке $x = a$, $H : \langle A, B\rangle \to \R$.
    Тогда $(g \circ f)'(a) = H(a) = G(f(a)) \cdot F(a) = g'(f(a)) \cdot f'(a)$.
\end{proof}

\begin{Rem}
    Это "правило цепочки".
    \[(g \circ h \circ f)'(a) = g'(h \circ f(a)) \cdot h'(f(a)) \circ f'(a)\]
\end{Rem}

\begin{Thm}[Арифметические операции]
    $f, g: \langle A, B\rangle \to \R, a \in \langle A, B\rangle, f, g$ -- дифференцируемы в точке $a$. Тогда

    \begin{MyList}
        \item $\forall \alpha, \beta \in \R$, то $\alpha f + \beta g$ -- дифференцируемая в точке $a$ функция и 
        \[(\alpha f + \beta g)'(a) = \alpha f'(a) + \beta g'(a)\]
        \item $f \cdot g$ -- дифференцируема в точке $a$ и
        \[(f \cdot g)'(a) = f'(a) \cdot g(a) + f(a) \cdot g'(a)\]
        \item если $g(a) \neq 0$, то $\frac{f}{g}$ -- дифференцируема в точке $a$ и
        \[\left(\frac{f}{g}\right)'(a) = \frac{f'(a) \cdot g(a) - f(a) \cdot g'(a)}{g^2(a)}\]
    \end{MyList}
\end{Thm}

\begin{proof}
    \begin{MyList}
        \item $(\alpha f + \beta g)'(a) = \lim_{x \to a} \frac{(\alpha f(x) + \beta g(x)) - (\alpha f (a) + \beta g(a))}{x - a} =$ 
        \[= \lim_{x \to a} \left(\alpha \cdot \frac{f(x) - f(a)}{x - a} + \beta \cdot \frac{g(x) - g(a)}{x - a}\right) = \alpha \lim_{x \to a} \frac{f(x) - f(a)}{x - a} + \beta \lim_{x \to a} \frac{g(x) - g(a)}{x - a} = \alpha f'(a) + \beta g'(a)\]
        
        \item Докажем частный случай $g = f$, т.е. докажем $(f^2)'(a) = 2f'(a)f(a)$. 
        Возьмем $h(t) = t^2$, тогда $f^2(x) = (h \circ f)(x)$. Тогда по предыдущей теореме
        \[(f^2)'(a) = h'(f(a)) \cdot f'(a) = 2 \cdot f(a) \cdot f'(a)\]
        Вернемся к общей формуле:
        \begin{align*}
        f \cdot g &= \frac{1}{4}((f + g)^2 - (f - g)^2) \SO (f \cdot g)'(a) = \frac{1}{4}((f + g)^2 - (f - g)^2)'(a) = \\
        &= \frac{1}{4} (2 \cdot (f(a) + g(a)) \cdot (f'(a) + g'(a)) - 2(f(a) - g(a)) \cdot (f'(a) - g'(a))) = \\
        &= \frac{1}{2}(2 f(a) \cdot g'(a) + 2f'(a) \cdot g(a)) = f(a)g'(a) + f'(a)g(a)       
        \end{align*} 
        \begin{Ex}
            Получить эту формулу непосредственно из определения производной
        \end{Ex}

        \item $\left(\frac{1}{g}\right)'(a) = - \frac{g'(a)}{g^2(a)}$. Возьмем $h(t) = \frac{1}{t} \SO \frac{1}{g(x)} = (h \circ g)(x)$ 
        \[\left(\frac{1}{g}\right)'(a) = h'(g(a)) \cdot g'(a) = -\frac{1}{g^2(a)} \cdot g'(a)\]
        Теперь $f \cdot \frac{1}{g}$.
        \[\left(\frac{f}{g}\right)'(a) = \left(f \cdot \frac{1}{g}\right)'(a) = f'(a) \cdot \frac{1}{g(a)} + f(a) \cdot - \frac{g'(a)}{g^2(a)} = \frac{f'(a)g(a) - f(a)g'(a)}{g^2(a)}\]
    \end{MyList}
\end{proof}

\begin{Cons}
    \[(f \cdot (h \cdot g))'(a) = f'(a) \cdot (h \cdot g)(a) + f(a) \cdot (h \cdot g)'(a) = f'(a) \cdot h(a) \cdot g(a) + f(a)(h'(a)g(a) + h(a)g'(a)) = \] 
    \[= f'(a) h(a)g(a) + f(a)h'(a) + g(a) + f(a)h(a)g'(a)\]
\end{Cons}

\begin{Thm}[Дифференцирование обратной функции]
    $f$ -- строго монотонная непрерывная функция на $\langle A, B\rangle$, $a \in \langle A, B\rangle, f$ -- дифференцируема в точке $a$ и $f'(a) \neq 0$. 
    Тогда $f^{-1} $ -- дифференцируема в точке $f(a)$ и $(f^{-1})'(f(a)) = \frac{1}{f'(a)}$.
    
    \begin{Rem}
        Геометрический смысл. Рисунок: \TODO 
    \end{Rem}
\end{Thm}

\begin{proof}
    $g(x) = f^{-1}(x), f(a) = b$. $f: \langle A, B\rangle \xrightarrow{\text{на}} \langle C, D\rangle, g : \langle C, D\rangle \xrightarrow{\text{на}} \langle A, B\rangle$ -- непрерывны.
    
    $f$ -- дифференцируема, тогда $\exists F(x) : \langle A, B\rangle$ непрерывная в точке $a$ 
    \[F(a) = f'(a), f(x) - f(a) = F(x)(x - a)\]
    $f$ строго монотонна $\SO \forall x \neq a \ f(x) \neq f(a) \SO F(x) \neq 0$ если $x \neq a$ и по условию $f'(a) = F(a) \neq 0$,
    т.е. $F(x) \neq 0 \ \forall x \in \langle A, B\rangle$ 

    \[x = g(y) \ (y = f(x))\]
    Тогда $y - b = f(x) - f(a) = F(x)(x - a) = F(g(y))(g(y) - g(b)) \SO g(y) - g(b) = \frac{1}{F(g(y))} (y - b) = H(y)(y - b)$ \\
    $H$ определена на $\langle C, D\rangle$, непрерывна в точке $b = f(a) \SO g'(b) = H(b) = \frac{1}{F(g(b))} = \frac{1}{F(a)} = \frac{1}{f'(a)}$ 
\end{proof}

\Subsection{Формулы для вычисления производных}

$f'(a), a \in E$ $a \mapsto f'(a)$

\begin{MyList}
    \item $f(x) \equiv 1, a \in \R$
    \[f'(a) = \lim_{x \to a} \frac{f(x) - f(a)}{x - a} = \lim_{x \to a} \frac{1 - 1}{x - a} = 0\]
    
    \item $f(x) = b^x, b > 0, a \in \R$
    \[f'(x) = \lim_{x \to a} \frac{b^x - b^a}{x - a} = \lim_{x \to a} b^a \cdot \frac{b^{x - a} - 1}{x - a} = b^a \cdot \ln a\]
    В частности, $(e^x)' = e^x$

    \item $f(x) = \log_b x, b > 0, b \neq 1, a \in (0, +\infty)$
    \[\lim_{x \to a} \frac{\log_b x - \log_b a}{x - a} = \lim_{x \to a} \frac{\log_b \frac{x}{a}}{x - a}\]
    \[\frac{x}{a} \xrightarrow[x \to a]{} 1 \SO \log_b \frac{x}{a} = \frac{\ln \frac{x}{a}}{\ln b} = \frac{\ln \left(1 + \left(\frac{x}{a} - 1\right)\right)}{\ln b} \thicksim \frac{\frac{x}{a} - 1}{\ln b}\]
    \[\SO \lim_{x \to a} \frac{\log_b \frac{x}{a}}{x - a} = \lim_{x \to a} \frac{\frac{x}{a} - 1}{(x - a)\ln b} = \lim_{x \to a} \frac{x - a}{a(x - a)\ln b} = \frac{1}{a\ln b}\]
    Значит
    \[(\log_b x)' = \frac{1}{x\ln b}\]
    В частности, $(\ln x)' = \frac{1}{x}$ 

    \item $f(x) = x^\alpha, \alpha \neq 0$
    \begin{center}
        \fbox{
            \begin{minipage}{25em}
                \begin{MyItemize}
                    \item $\alpha \in \N, x \in \R$
                    \item $\alpha \in \Z \setminus \N, x \in \R \setminus \{0\}$ 
                    \item $\alpha \in \Q, \alpha = \frac{m}{2n + 1}, n \in \N, x \in \R (\alpha > 0), x \in \R \setminus \{0\}(\alpha < 0)$ 
                    \item $\alpha = \frac{m}{2n}, \alpha \in \R \setminus \Q, x \in [0, +\infty) (\alpha > 0), x \in (0, +\infty) (\alpha < 0)$
                \end{MyItemize}
            \end{minipage}
        }
    \end{center}
    
    \[\lim_{x \to a} \frac{x^\alpha - a^\alpha}{x - a} = \lim_{x \to a} a^\alpha \cdot \frac{\left(\frac{x}{a}\right)^\alpha - 1}{x - a} = \lim_{x \to a} a^\alpha \frac{\left(1 + \left(\frac{x}{a} - 1\right)\right)^\alpha - 1}{x - a} = \lim_{x \to a} a^\alpha \cdot \frac{\alpha \cdot \left(\frac{x}{a} - 1\right)}{x - a} = \]
    \[= \lim_{x \to a} a^\alpha \cdot \frac{\alpha(x - a)}{a(x - a)} = \alpha \cdot a^{\alpha - 1}\]

    \[f'(0) = \lim_{x \to 0} \frac{x^\alpha}{x} = \lim_{x \to 0} x^{\alpha - 1} = \begin{cases}
        0, \alpha > 1 \\
        1, \alpha = 1 \\
        \infty, \alpha < 1
    \end{cases}\]
    Выводы: $(x^\alpha)' = \alpha \cdot x^{\alpha - 1}$ (с точностью до области определения функции).

    \item $f(x) = \sin x, a \in \R$ 
    \[f'(a) = \lim_{x \to a} \frac{\sin x - \sin a}{x - a} = \lim_{x \to a} \frac{2 \sin \frac{x - a}{2} \cos \frac{x + a}{2}}{x - a} = \lim_{x \to a}2 \frac{ \frac{x - a}{2} \cdot \cos a}{x - a} = \cos a\]
    
    \item $f(x) = \cos x$
    \[(\cos x)' = \left(\sin \left(\frac{\pi}{2} - x\right)\right)' = \cos \left(\frac{\pi}{2} - x\right) \cdot \left(\frac{\pi}{2} - x\right)' = -\sin x\]

    \item $f(x) = \tg x, a \neq \frac{\pi}{2} + \pi k, k \in \Z$ 
    \[(\tg x)' = \left( \frac{\sin x}{\cos x}\right)' = \frac{(\sin x)' \cos x - \sin x (\cos x)'}{\cos^2 x} = \frac{\cos^2 x + \sin^2x}{\cos^2 x} = \frac{1}{\cos^2 x}\]

    \item $f(x) = \ctg x, a \neq \pi k, k \in \Z$ 
    \[(\ctg x)' = \left(\tg \left(\frac{\pi}{2} - x\right)\right)' = \frac{1}{\cos \left(\frac{\pi}{2} - x\right)} \cdot \left(\frac{\pi}{2} - x\right)' = -\frac{1}{\sin^2 x}\]

    \item $f(x) = \arcsin x, x \in [-1, 1]$. Пусть $g(y) = \sin y \SO b = \arcsin a, g'(b) = \cos b > 0$, т.к. $b \in \left[-\frac{\pi}{2}, \frac{\pi}{2}\right]$ 
    \[f'(a) = \frac{1}{g'(b)} = \frac{1}{\cos b} = \frac{1}{\sqrt{1 - \sin^2 b}} = \frac{1}{\sqrt{1 - a^2}}\]
    \[(\arcsin x)' = \frac{1}{\sqrt{1 - x^2}}, x \in (-1, 1)\]

    \item $(\arccos x)' = \left(\frac{\pi}{2} - \arcsin x\right) = -\frac{1}{\sqrt{1 - x^2}}, x \in (-1, 1)$
    \item $f(x) = \arctg x, g(y) = \tg y, b = \arctg a, b \in \left(-\frac{\pi}{2}, \frac{\pi}{2}\right)$
    \[f'(a) = \frac{1}{g'(b)} = \cos^2 b = \frac{1}{\tg^2 b + 1} = \frac{1}{a^2 + 1}\]
    \[(\arctg x)' = \frac{1}{1 + x^2}\]

    \item $(\arcctg x)' = \left(\frac{\pi}{2} - \arctg x\right)' = -\frac{1}{x^2 + 1}$ 
\end{MyList} 

\Subsection{Теоремы о средних}

\begin{Thm}[Теорема Ферма]
    $a \in (A, B), f : \langle A, B\rangle \to \R$ -- дифференцируема в точке $a$. 
    Если $f(a) = \max_{a \in \langle A, B\rangle} f$ или $f(a) = \min_{a \in \langle A, B\rangle} f$, то $f'(a) = 0$.

    Геометрический смысл: 
    \begin{figure*}[h]
        \centering
        \def\svgwidth{0.5\columnwidth} 
        \input{images/fermat.pdf_tex}
        \caption{Горизонтальная касательная}
    \end{figure*}
\end{Thm}

\begin{proof}
    $f(a) = \max_{\langle A, B\rangle} f \SO f(x) - f(a) \leqslant 0 \ \forall x \in \langle A, B\rangle$. 
    Если $x > a$, то $ \frac{f(x) - f(a)}{x - a} \leqslant 0$ 
    \[f_+'(a) = \lim_{x \to a+} \frac{f(x) - f(a)}{x - a} \leqslant 0\] 
    Если $x < a$, то $ \frac{f(x) - f(a)}{x - a} \geqslant 0$ 
    \[f_-'(a) = \lim_{x \to a-} \frac{f(x) - f(a)}{x - a} \geqslant 0\]
    $f$ дифференцируема в точке $a \SO f_-'(a) = f_+'(a) = f'(a) \SO f'(a) = 0$ 
\end{proof}

\Pagebreak
\begin{Thm}[Теорема Ролля]
    $f: [a, b] \to \R$. Если
    \begin{MyList}
        \item $f$ дифференцируема на $(a, b)$ (т.е. дифференцируема в каждой точке).
        \item непрерывна на $[a, b]$ 
        \item $f(a) = f(b)$
    \end{MyList}
    Тогда $\exists c \in (a, b) : f'(c) = 0$
    \begin{figure*}[h]
        \centering
        \def\svgwidth{0.5\columnwidth} 
        \input{images/rolle.pdf_tex}
        \caption{Теорема Ролля}
    \end{figure*}
\end{Thm}

\begin{proof}
    $f$ непрерывна на $[a, b] \SO f$ достигает наибольшего и наименьшего значения.
    Если $a, b$ -- те точки, в которых достигается наибольшее и наименьшее значение, то $f$ постоянная на $[a, b] \SO f'(x) = 0 \ \forall x \in (a, b)$.

    Если хотя бы в одной из точек $a$ и $b$ не достигает наибольшего или наименьшего значения, тогда одно из них достигается на $(a, b)$. Тогда по теореме Ферма в этой точке производная равна нулю.
\end{proof}

\begin{Rem}
    Все три условия существенны.
\end{Rem}

\begin{Thm}[Теорема Лагранжа или формула конечных приращений.]
    $f : [a, b] \to \R$. $f$ непрерывна на $[a, b], f$ -- дифференцируема на $(a, b)$.
    Тогда $\exists c \in (a, b) : f(b) - f(a) = f'(c)(b - a)$ 

    Геометрический смысл: $ \frac{f(b) - f(a)}{b - a} = f'(c)$.
    
    \begin{figure*}[h]
        \centering
        \def\svgwidth{0.4\columnwidth} 
        \input{images/lagrange.pdf_tex}
        \caption{Теорема Лагранжа}
    \end{figure*}
\end{Thm}

\begin{proof}
    $g(x) = f(x) - kx$ -- непрерывна на $[a, b]$, дифференцируема на $(a, b)$. Хотим подобрать $k : g(a) = g(b)$
    \[f(a) - ka = f(b) - kb \SO k = \frac{f(b) - f(a)}{b - a}\]
    Тогда $g(x) = f(x) - x \cdot \frac{f(b) - f(a)}{b - a}$ подходит под условия теоремы Ролля.
    Тогда $\exists c \in (a, b) : g'(c) = 0$.
    \[g'(x) = f'(x) - \frac{f(b) - f(a)}{b - a} \SO \exists c \in (a, b) : f'(c) = \frac{f(b) - f(a)}{b - a}\]  
\end{proof}

\begin{Thm}[Теорема Коши]
    $f, g : [a, b] \to \R, f, g$ -- непрерывны на $[a, b]$, дифференцируемы на $(a, b)$, $\forall x \in (a, b) \ g'(x) \neq 0$.
    Тогда $\exists c \in (a, b) : \frac{f(b) - f(a)}{g(b) - g(a)} = \frac{f'(c)}{g'(c)}$    
\end{Thm}

\begin{proof}
    $h(x) = f(x) - kg(x)$ -- непрерывна на $[a, b]$, дифференцируема на $(a, b)$. 
    Подберем $k : h(a) = h(b)$ 
    \[f(a) - kg(a) = f(b) - kg(b) \SO k = \frac{f(b) - f(a)}{g(b) - g(a)}\]
    Тогда $h(x) = f(x) - g(x) \cdot \frac{f(b) - f(a)}{g(b) - g(a)} \SO \exists c \in (a, b) : h'(c) = 0$
    \[f'(c) - g'(c) \cdot \frac{f(b) - f(a)}{g(b) - g(a)} = 0\] 
\end{proof}

\begin{Rem}
    \begin{MyList}
        \item Точка $c$ может быть не единственной.
        \item Теорема Лагранжа -- частный случай теоремы Коши, теорема Ролля -- частный случай теоремы Лагранжа.
        \item Теорему Лагранжа можно записать в следующем виде:
        \[ \frac{f(b) - f(a)}{b - a} = f'(a + \Theta \cdot (b - a)), \Theta \in (0, 1)\]
    \end{MyList}
\end{Rem}

\begin{Cons}[Оценка конечных приращений]
    $f$ непрерывна на $[a, b]$, дифференцируема на $(a, b)$. 
    Если $\exists m, M \in \R : m \leqslant f'(x) \leqslant M \ \forall x \in (a, b)$, тогда 
    \[m(b - a) \leqslant f(b) - f(a) \leqslant M(b - a)\]
    В частности, если $\exists M \in \R : |f'(x)| \leqslant M \ \forall x \in (a, b)$, то
    \[|f(b) - f(a)| \leqslant M \cdot (b - a)\] 
\end{Cons}

\begin{proof}
    По теореме Лагранжа $\exists c \in (a, b) : f(b) - f(a) = f'(c)(b - a)$
    \[m(b - a) \leqslant f(b) - f(a) = f'(c)(b - a) \leqslant M(b - a)\] 
\end{proof}

\begin{Cons}
    Если $\forall x \in (a, b) \ f'(x) \geqslant 0$, то $f$ нестрого монотонно возрастает.
\end{Cons}

\begin{proof}
    $x_1 < x_2 \in (a, b)$. $\exists c \in (x_1, x_2)$ :
    \[f(x_2) - f(x_1) = f'(c)(x_2 - x_1) \SO f(x_2) \geqslant f(x_1)\]
\end{proof}

\begin{Cons}
    Если $\forall x \in (a, b) \ f'(x) > 0$, то $f$ строго возрастает.
\end{Cons}

\begin{Cons}
    Если $\forall x \in (a, b) \ f'(x) \leqslant 0$, то $f$ нестрого монотонно убывает.
\end{Cons}

\begin{Cons}
    Если $\forall x \in (a, b) \ f'(x) < 0$, то $f$ строго монотонно убывает.
\end{Cons}

\begin{Rem}
    Если $f$ дифференцируема на $(a, b)$ и $f$ строго монотонно убывает $\SO f'(x) < 0 \ \forall x \in (a, b)$ -- вообще говоря, неверно.

    $f(x) = -x^3, f'(x) = -3x^2 \leqslant 0$ и равенство достигается при $x = 0$.
\end{Rem}

\begin{Thm}[Теорема Дарбу]
    $f : [a, b] \to \R$ -- дифференцируема на $[a, b]$.
    Пусть $C$ лежит строго между $f'(a)$ и $f'(b)$. 
    Тогда $\exists c \in (a, b) : f'(c) = C$ 
\end{Thm}

\begin{proof}
    \begin{MyList}
        \item $C = 0$. Для определенности $f'(a) < 0 < f'(b)$. $f$ непрерывна на $[a, b] \SO f$ достигает наибольшего и наименьшего значения на $[a, b]$.
        При таких знаках производной наименьшее значение достигается на $(a, b) \SO$ в такой точке минимума $f'(c) = 0$ (по теореме Ферма).

        \item $C \neq 0$. Рассмотрим $h(x) = f(x) - Cx$.
        \[h'(a) = f'(a) - C, h'(b) = f'(b) - C \SO h'(a) \text{ и } h'(b) \text{ -- разных знаков}\]
        Тогда по предыдущему пункту $\exists c \in (a, b) : h'(c) = 0 \SO f'(c) = C$ 
    \end{MyList}
\end{proof}

\begin{Thm}[Правило Лопиталя]
    Пусть $-\infty \leqslant a < b \leqslant +\infty$.
    $f, g : (a, b) \to \R,$ дифференцируемы на $(a, b), g'(x) \neq 0 \ \forall x \in (a, b)$ 
    и $\lim_{x \to b-} f(x) = \lim_{x \to b-} g(x) = 0$.

    Если $\lim_{x \to b-} \frac{f'(x)}{g'(x)} = l \in \overline{\R}$, то $\exists \lim_{x \to b-} \frac{f(x)}{g(x)} = l$
\end{Thm}

\begin{proof}
    $g$ дифференцируема на $(a, b) \SO g$ непрерывна на $(a, b)$. Кроме того, $g'(x)~\neq~0 \ \forall x \in (a, b) \SO g$ строго монотонна на $(a, b)$
    $\SO$ знакопостоянна на $(a, b)$ (и ни в какой точке не равна нулю).

    По Гейне: $\forall \{x_n\} : x_n \to b, x_n \in (a, b) \ g(x_n) \to 0$. 
    Возьмем строго возрастающую последовательность $\{x_n\} : x_n \in (a, b), x_n \to b$. Тогда
    $\{g(x_n)\}$ строго монотонна. Тогда по теореме Штольца 
    \[\lim_{n \to \infty} \frac{f(x_n)}{g(x_n)} = \lim_{n \to \infty} \frac{f(x_n) - f(x_{n - 1})}{g(x_n) - g(x_{n - 1})}\]
    если предел справа существует.

    По теореме Коши:
    \[\exists c_n \in (x_{n - 1}, x_n) : \frac{f(x_n) - f(x_{n - 1})}{g(x_n) - g(x_{n - 1})} = \frac{f'(c_n)}{g'(c_n)}\]
    \[\lim_{n \to \infty} \frac{f'(c_n)}{g'(c_n)} = \lim_{x \to b-} \frac{f'(x)}{g'(x)} \text{ -- существует и равен } l\]
    т.к. $x_n \to b \SO c_n \to b$.
    
\end{proof}

\begin{Thm}[правило Лопиталя для бесконечностей]
    Условия те же, но $\lim_{x \to b-} f(x) = \lim_{x \to b-} g(x) = +\infty$.
    Тогда если $\lim_{x \to b-} \frac{f'(x)}{g'(x)} = l \in \overline{\R}$, то $\exists \lim_{x \to b-} \frac{f(x)}{g(x)} = l$. 
\end{Thm}

\begin{Rem}
    Обратить правило Лопиталя нельзя.
    $f(x) = x + \sin x, g(x) = x$ 
    \[\lim_{x \to +\infty} \frac{x + \sin x}{x} = \lim_{x \to +\infty} \left(1 + \frac{\sin x}{x}\right) = 1\]
    Но
    \[ \frac{f'(x)}{g'(x)} = \frac{1 + \cos x}{1} \SO \not\exists \lim_{x \to +\infty} \frac{f'(x)}{g'(x)}\]
\end{Rem}

\Subsection{Производные высших порядков}

$f$ дифференцируема на $E$. $x \mapsto f'(x)$ область определения $E$.  

Если $f'(x)$ дифференцируема на $E_1$, то $f$ дифференцируема на $E_1$ дважды.

\begin{Def}
    Второй производной функции $f$ в точке $a$ называется $(f')'(a) = f''(a)$,
    третья производная -- $f'''(a) = (f'')'(a)$. $n$-ая производная -- $(f^{(n - 1)})' = f^{(n)}$ 
\end{Def}

\begin{Example}
    $(x^3)''' = (3x^2)'' = (6x)' = 6$ 
\end{Example}

\begin{Example}
    $(\sin x)^{(14)} = (\cos x)^{(13)} = (-\sin x)^{(12)} = (-\cos x)^{(11)} = (\sin x)^{(10)} = (\sin x)'' = -\sin x$ 
\end{Example}

\begin{Thm}[Арифметические действия с производными высших порядков]
    $f$ и $g$ $n$ раз дифференцируемы в точке $a$. Тогда 
    
    \begin{MyList}
        \item $\forall \alpha, \beta \in \R \ \alpha f + \beta g$ -- $n$ раз дифференцируема и 
        $$(\alpha f + \beta g)^{(n)}(a) = \alpha \cdot f^{(n)}(a) + \beta \cdot g^{(n)}(a)$$
        \item $f \cdot g$ -- $n$ раз дифференцируема в точке $a$ и 
        \[(fg)^{(n)}(a) = \sum_{k=0}^{n} C_n^k \cdot f^{(k)}(a) \cdot g^{(n - k)}(a)\]
    \end{MyList}
\end{Thm}

\begin{proof}
    \begin{MyList}
        \item База при $n = 1$ -- верно. \\
        Индукционный переход: пусть верно для $l$. Тогда для $l + 1$:
        \[(\alpha f + \beta g)^{(l + 1)} = \left((\alpha f + \beta g)^{(l)}\right)' = \left(\alpha f^{(l)} + \beta g^{(l)}\right)' = \alpha f^{(l + 1)} + \beta g^{(l + 1)}\]

        \item База при $n = 1$: $(fg)' = fg' + f'g$. \\
        Индукционный переход: пусть верно для $l$

        \begin{align*}
            (fg)^{(l + 1)} &= ((fg)^{(l)})' = \left(\sum_{k=0}^{l} C_l^k f^{(k)} g^{(l - k)}\right)' = \sum_{k=0}^{l} C_l^k \left(f^{(k)} g^{(l - k)}\right)' = \\
            &= \sum_{k=0}^{l} C_l^k \left(f^{(k + 1)} g^{(l - k)} + f^{(k)} g^{(l + 1 - k)}\right) = \sum_{k=0}^{l} C_l^k f^{(k + 1)}g^{(l - k)} + \sum_{k=0}^{l} C_l^k f^{(k)} g^{(l + 1 - k)} = \\
            &= \sum_{j=1}^{l + 1} C_l^{j - 1} f^{(j)} g^{(l + 1 - j)} + \sum_{k=0}^{l} C_l^k f^{(k)} g^{(l + 1 - k)} = \\
            &= C_l^l f^{(l + 1)} g + \sum_{j=1}^{l} \left(C_l^{j - 1} + C_l^j\right)f^{(j)} g^{(l + 1 - j)} + C_l^0 f g^{(l + 1)} \\
            &= \sum_{k=0}^{l + 1} C_{l + 1}^k f^{(k)} g^{(l + 1 - k)}
        \end{align*}

    \end{MyList}
\end{proof}

\begin{Prop}
    $\left(f(\alpha x + \beta)\right)^{(n)} = \alpha^n + f^{(n)} (\alpha x + \beta)$ 
\end{Prop}

\begin{Def}
    $f$ дифференцируема на $E$ и $f'$ непрерывна на $E$. Тогда $f$ называется непрерывно дифференцируемой. \\
    $f \in C^1 (E)$ -- непрерывно дифференцируемые функции. \\
    $f \in C^2 (E)$ -- дважды непрерывно дифференцируемые функции. \\
    $f \in C^n (E)$ -- $n$ раз непрерывно дифференцируемые функции. \\
    $f \in C^\infty (E)$ -- бесконечно непрерывно дифференцируемые функции.
\end{Def}

\begin{Example}
    $f(x) = |x| \in C(\R)$, но $f(x) \notin C^1 (\R)$  
\end{Example}

\begin{Example}
    $f(x) = x^2 \in C^{\infty} (\R)$ 
\end{Example}

\begin{Example}
    $f(x) = x^{\frac{4}{3}}$ на $\R$. $f'(x) = \frac{4}{3}x^{\frac{1}{3}}$ на $\R$. \\
    $f''(x) = \frac{4}{3} \cdot \frac{1}{3} \cdot x^{-\frac{2}{3}}$ на $\R$ разрывна в нуле.
    Тогда $f(x) \in C^1 (\R)$,но $f(x) \notin C^2(\R)$    
\end{Example}

\begin{Ex}
    $g(x) = \begin{cases}
        x^2, x \in \Q \\
        0, x \notin \Q
    \end{cases}$. Дифференцируема ли $g(x)$ в нуле?
\end{Ex}

\Subsection{Формула Тейлора}

$P(x)$ многочлен степени не выше $n$, $a \in \R$ 
\[P(x) = \sum_{k=0}^{n} c_k (x - a)^k, c_i \in \R\]
$c_0 = P(a)$

\begin{Thm}[Формула Тейлора для многочлена]
    Пусть $n \in \Z_+, P$ -- многочлен степени не выше $n$. Тогда $\forall a, x \in \R$ 
    \[P(x) = \sum_{k=0}^{n} \frac{P^{(k)}(a)}{k!}(x - a)^k\] 
\end{Thm}

\begin{proof}
    Проверим, что $\left.\left((x - a)^k\right)^{(m)} \right|_{x = a} = \begin{cases}
        0, k \neq m \\
        k!, k = m
    \end{cases}$. \\
    $k > m$ 
    \[\left((x - a)^k\right)^{(m)} = \left(k(x - a)^{k - 1}\right)^{(m - 1)} = k...(k - m + 1)(x - a)^{k - m} = 0\] 
    $k < m \SO \left((x - a)^k\right)^{(m)} = 0$ \\
    $k = m$
    \[\left((x - a)^k\right)^{(k)} = k!\]
    $P(x) = \sum_{k=0}^{n} c_k(x - a)^k \SO P^{(m)}(a) = c_m \cdot m! \SO c_m = \frac{P^{(m)}(a)}{m!}$ 
\end{proof}

\Pagebreak
\begin{Lm}
    Пусть $E \subset \R, a \in E, g : E \to \R, n \in \N$. Предположим, что $g$ дифференцируема в точке $a$ $n$ раз и $g(a) = g'(a) = g''(a) = g^{(n)}(a) = 0$.
    Тогда $g(x) = o((x - a)^n), x \to a$.   
\end{Lm}

\begin{proof}
    База: $k = 1. g(a) = g'(a) = 0$  
    \[g(x) = g(a) + g'(a)(x - a) + o(x - a), x \to a \SO g(x) = o(x - a), x \to a\]
    Индукционный переход: Пусть при $n = k$ выполняется. При $n = k + 1 \ g \ k + 1$ раз дифференцируема в точке $a$ и $g(a) = g'(a) = ... = g^{(k + 1)}(a) = 0$
    \[g'(a) = \left(g'\right)'(a) = ... = \left(g'\right)^{(k)}(a) = 0 \SO g'(x) = o((x - a)^k), x \to a\]
    $|g'(x)| \leqslant \varepsilon |x - a|^k, |x - a| < \delta$. По формуле конечных приращений 
    \[\exists \Theta : g(x) - g(a) = g'(a + \Theta(x - a)) \cdot (x - a)\] 
    $|a + \Theta(x - a) - a| = \Theta(x - a) < \delta$. Тогда $|g'(a + \Theta(x - a))| \leqslant \varepsilon \cdot |x - a|^k$
    \[|g(x)| = |g'(a + \Theta(x - a)) \cdot (x - a)| \leqslant \varepsilon |x - a|^k \cdot |x - a| = \varepsilon |x - a|^{k + 1}\] 
    $\SO g(x) = o \left((x - a)^{n + 1}\right)$ 
\end{proof}

\begin{Thm}[Формула Тейлора]
    $E \subset \R, a \in E, f : E \to \R, n \in \N$. Пусть $f \ n$ раз дифференцируема в точке $a$. Тогда 
    \[f(x) = \sum_{k=0}^{n} \frac{f^{(k)}(a)}{k!}(x - a)^k + \underbrace{o((x - a)^n)}_\text{остаток в форме Пеано}, x \to a\]
\end{Thm}

\begin{proof}
    Положим $P(x) = \sum_{k=0}^{n} \frac{f^{(k)}(a)}{k!} \cdot (x - a)^k$.
    По формуле Тейлора для многочлена:
    \[P(x) = \sum_{k=0}^{n} \frac{P^{(k)}(a)}{k!}(x - a)^k \SO f^{(k)}(a) = P^{(k)}(a) \ \forall k = 0, ..., n\]
    Возьмем $g = f - P$.
    \begin{align*}
        g(a) &= f(a) - P(a) = 0 \\
        g'(a) &= f'(a) - P'(a) = 0 \\
        g^{(n)}(a) &= f^{(n)}(a) - P^{(n)}(a) = 0
    \end{align*}
    $\SO$ По лемме $g(x) = o((x - a)^n), x \to a \SO f(x) = P(x) + o((x - a)^n), x \to a$, т.е.
    \[f(x) = \sum_{k=0}^{n} \frac{f^{(k)}(a)}{k!}(x - a)^k + o((x - a)^n)\]
\end{proof}

\begin{Thm}[Формула Тейлора-Лагранжа]
    $a, x \in \R, a \neq x$. Обозначим $\Delta_{a, x}$ -- отрезок $[a, x]$ или $[x, a]$, $\widetilde{\Delta}_{a, x}$ -- интервал с концами $a$ и $x$.
    $n \in \Z_+, f \ n + 1$ раз дифференцируема в на $\langle A, B\rangle, a, x \in \langle A, B\rangle$.
    Тогда $\exists c \in \widetilde{\Delta}_{a, x}$, для которой 
    \[f(x) = \sum_{k=0}^{n} \frac{f^{(k)}(a)}{k!}(x - a)^k + \underbrace{\frac{f^{(n + 1)}(c)}{(n + 1)!}(x - a)^{n + 1}}_\text{остаток в форме Лагранжа}\] 
\end{Thm}

\begin{Rem}
    Точка $c$ зависит от $x$, поэтому, вообще говоря, не многочлен. \\
    Можно взять $c = a + \Theta(x - a), \Theta(0, 1)$ 
\end{Rem}

\begin{proof}
    $t \in \Delta_{a, x}$. Пусть $\PHI(t) = (x - t)^{n + 1}, F(t) = f(x) - f(t) - \sum_{k=1}^{n} \frac{f^{(k)}(t)}{k!}(x - t)^k$.
    Тогда $t \in \widetilde{\Delta}_{a, x}$.
    \begin{align*}
        F'(t) &= 0 - f'(t) - \sum_{k=1}^{n} \left( \frac{f^{(k + 1)}(t)}{k!}(x - t)^k - \frac{f^{(k)}(t)}{k!} \cdot k \cdot (x - t)^{k - 1)}\right) \\
        &= -f'(t) + \sum_{k=1}^{n} \frac{f^{(k)}(t)}{(k - 1)!}(x - t)^{k - 1} - \sum_{k=1}^{n} \frac{f^{(k + 1)}(t)}{k!}(x - t)^k \\
        &= -f'(t) + \sum_{m=0}^{n - 1} \frac{f^{(m + 1)}(t)}{m!}(x - t)^m - \sum_{k=1}^{n} \frac{f^{(k + 1)}(t)}{k!}(x - t)^k \\
        &= -f'(t) + f'(t) - \frac{f^{(n + 1)}(t)}{(n)!}(x - t)^{n} = - \frac{f^{(n + 1)}(t)}{n!}(x - t)^n
    \end{align*}
    По теореме Коши $\exists c \in \widetilde{\Delta}_{a, x} : $ 
    \[ \frac{F(a)}{(x - a)^{n + 1}} = \frac{F(a) - F(x)}{\PHI(a) - \PHI(x)} = \frac{F'(c)}{\PHI'(c)} = \frac{\frac{-f^{(n + 1)}(c)}{n!}(x - c)^n}{-(n + 1)(x - c)^n} = \frac{f^{(n + 1)}(c)}{(n + 1)!}\]
    $\SO F(a) = \frac{f^{(n + 1)}(c)}{(n + 1)!}(x - a)^{n + 1}$ 
\end{proof}

\begin{Rem}
    Формула Тейлора-Пеано $\Leftarrow$ формула Тейлора-Лагранжа.
    $f$ -- $n$ раз дифференцируема в точке $a$, $f^{(n)}$ -- непрерывна в точке $a$.
    \[f(x) = \sum_{k=0}^{n - 1} \frac{f^{(k)}(a)}{k!}(x - a)^k + \frac{f^{(n)}(c(x))}{n!}(x - a)^n = \sum_{k=0}^{n} \frac{f^{(k)}(a)}{k!}(x - a)^k + \frac{f^{(n)}(c(x))}{n!}(x - a)^n - \frac{f^{(n)}(a)}{n!}(x - a)^n\]
    Рассмотрим: 
    \[ \frac{f^{(n)}\left(c(x)) - f^{(n)}(a)\right)}{n!}(x - a)^n, c(x) \in \widetilde{\Delta}_{a, x} \SO |c(x) - a| < |x - a|\]
    Значит, если $x \to a$, то $c(x) \to a$. $f^{(n)}(c(x)) \xrightarrow[x \to a]{} f^{(n)}(a)$ (по непрерывности).
    Тогда $\frac{f^{(n)}(c(x)) - f^{(n)}(a))}{n!}(x - a)^n = o((x - a)^n), x \to a$ 
\end{Rem}

\begin{Rem}
    Обозначения: $T_{a, n} f$ -- многочлен Тейлора фукнции $f$ в точке $a$ порядка $n$. \\
    Остаток: $R_{a, n} f(x) \left(f(x) - T_{a, n} f\right)$ 
\end{Rem}

\Subsection{Формулы Тейлора-Маклорена}

Пусть $n \in \Z_+$. Выведем формулы при $x \to 0$.

\begin{MyList}
    \item $e^x = \sum_{k=0}^{n} \frac{x^k}{k!} + o(x^n) = 1 + \frac{x}{1!} + \frac{x^2}{2!} + ... + \frac{x^n}{n!} + o(x^n)$ 
    \begin{proof}
        $f(x) = e^x, f^{(x)} = e^x \ \forall x \in \N, x \in \R \SO f^{(k)}(0) = 1$
        \[T_{0, n}f(x) = \sum_{k=0}^{n} \frac{x^k}{k!}\] 
    \end{proof}

    \item $\sin x = \sum_{k=0}^{n} (-1)^k \frac{x^{2k + 1}}{(2k + 1)!} + o(x^{2n + 2}) = x - \frac{x^3}{3!} + \frac{x^5}{5!} - ... + (-1)^n \frac{x^{2n + 1}}{(2n + 1)!} + o(x^{2n + 2})$ 
    \begin{proof}
        $f(x) = \sin x$ \\
        $f^{(m)}(x) = \sin \left(x + \frac{\pi m}{2}\right) \ \forall m \in \Z_+$
        \begin{proof}
            $m = 0$ верно (база). \\
            $f^{(m + 1)}(x) = \cos \left(x + \frac{\pi m}{2}\right) = \sin \left(x + \frac{\pi m}{2} + \frac{\pi}{2}\right) = \sin \left(x + \frac{\pi (m + 1)}{2}\right)$  
        \end{proof} 
        \[f^{(m)}(0) = \sin \frac{\pi m}{2} = \begin{cases}
            0, m \ \vdots \ 2 \\
            (-1)^{ \frac{m - 1}{2}}, \text{иначе}
        \end{cases}\]

        \[f(x) = 0 + x + - \frac{x^3}{3!} + 0 + \frac{x^5}{5!} + (*)\] 
        $(*) : $ если $n$ -- нечетное, то $(*) = (-1)^n \frac{x^n}{n!} + o(x^n)$.
        Заметим, что $T_{0, 2k + 1}f = T_{0, 2k + 2} f = \sum_{k=0}^{n} (-1)^k \frac{x^{2k + 1}}{(2k + 1)!}$.
        Если $n$ -- четное, то последнее слагаемое в $T_{0, n}f$ равно 0  
    \end{proof}

    \item $\cos x = 1 - \frac{x^2}{2!} + \frac{x^4}{4!} - ... + (-1)^n \frac{x^{2n}}{(2n)!} + o(x^{2n + 1}) = \sum_{k=0}^{n} (-1)^k \frac{x^{2k}}{(2k)!} + o(x^{2n + 1})$ 
    \item $\ln (1 + x) = x - \frac{x^2}{2} + \frac{x^3}{3} - \frac{x^4}{4} + ... + (-1)^{n + 1} \frac{x^n}{n} + o(x^n) = \sum_{k=1}^{n} (-1)^{k + 1} \frac{x^k}{k} + o(x^n)$ 
    \begin{proof}
        $f(x) = \ln(1 + x), f'(x) = \frac{1}{1 + x}, f'' = -\frac{1}{(1 + x)^2}, f'''(x) = \frac{2}{(1 + x)^3}$
        \[\SO f^{(m)}(x) = (-1)^{m + 1} \frac{(m - 1)!}{(1 + x)^m}, x > -1\]
        \[\SO f^{(m)}(0) = (-1)^{m + 1} \cdot (m - 1)!\]
        Рассмотрим $m$-тое слагаемое:
        \[ \frac{f^{(m)}(0)}{m!}(x - 0)^m = \frac{(-1)^{m + 1}(m - 1)!}{m!}x^m = \frac{(-1)^{m + 1}}{m}x^m\] 
    \end{proof}

    \begin{Conj}
        У четных функций только четные степени, у нечетных функций -- только нечетные.
        У функций общего вида -- и те, и другие.
    \end{Conj}

    \begin{Ex}
        Объяснить это.
    \end{Ex}

    \item $(1 + x)^\alpha = \sum_{k=0}^{n} C_\alpha^k x^k + o(x^n)$. $\alpha \in \R, k \in \Z_+$. $C_\alpha^k = \frac{\alpha (\alpha - 1)(\alpha - 2) ... (\alpha - k + 1)}{k!}$ \\
    \begin{Rem}
        При $\alpha \in \N \ f(x) = (1 + x)^\alpha = T_{0, \alpha} f(x)$ 
    \end{Rem}

    \begin{proof}
        $f(x) = (1 + x)^\alpha$ 
        \begin{gather*}
            f^{(m)}(x) = \alpha (\alpha - 1) (\alpha - 2) ... (\alpha - m + 1)(1 + x)^{\alpha - m} \\
            f^{(m)}(0) = \alpha (\alpha - 1)(\alpha - 2) ... (\alpha - m + 1) \\
            f(0) = 1
        \end{gather*}
        $T_{0, n} f(x) = \sum_{k=0}^{n} \frac{\alpha (\alpha - 1) ... (\alpha - k + 1)}{k!}x^k$ 
    \end{proof}
\end{MyList}

\begin{Prop}
    $\forall n \in \N \ \exists c \in (0, 1) : $
    \[e^1 = 1 + \frac{1}{1!} + \frac{1}{2!} + ... + \frac{1}{n!} + \frac{e^c}{(n + 1)!}\] 
\end{Prop}

\begin{Cons}
    $e - \left(1 + 1 + \frac{1}{2!} + ... + \frac{1}{n!}\right) < \frac{3}{(n + 1)!}$ 
\end{Cons}

\begin{Cons}
    $e$ -- иррациональное число.
\end{Cons}

\begin{proof}
    Пусть $e \in \Q, e \in (2, 3), e = \frac{m}{n}, m, n \in \N$.
    \[\frac{m}{n} = e = 1 + 1 + \frac{1}{2!} + \frac{1}{3!} + ... + \frac{1}{n!} + \frac{e^c}{(n + 1)!}, c \in (0, 1) = \]
    \[= \underbrace{m (n - 1)!}_{\in \Z} = \underbrace{n! + n! + \frac{n!}{2!} + ... + 1}_{\in \Z} + \frac{e^c}{n + 1} \SO \frac{e^c}{n + 1} \in Z\]
    но $0 < e^c < e < 3, n + 1 \geqslant 3$  !?
\end{proof}


\begin{Prop}[Критерий постоянства]
    Пусть $f$ непрерывна на $\langle A, B\rangle$ и дифференцируема на $(A, B)$. Тогда равносильны следующие утверждения:
    \begin{MyList}
        \item $f$ постоянна на $\langle A, B\rangle$ 
        \item $f' = 0 \ \forall x \in (A, B)$
    \end{MyList}
\end{Prop}

\begin{proof}
    $1) \SO 2)$ очевидно. \\
    $2) \SO 1)$. $f'(x) = 0 \ \forall x \in (A, B) \SO f'(x) \geqslant 0, f'(x) \leqslant 0 \SO$ $f$ нестрого убывает и нестрого возрастает $\SO f$ -- постоянна. 
\end{proof}

\begin{Example}
    $\arccos x + \arcsin x = \frac{\pi}{2}$. 
\end{Example}

\begin{proof}
    $f(x) = \arccos x + \arcsin x$.
    $f'(x) = -\frac{1}{\sqrt{1 - x^2}} + \frac{1}{\sqrt{1 + x^2}} = 0 \ \forall x \in (-1, 1) \SO f$ -- постоянна.
    \[f(0) = \frac{\pi}{2} + 0 = \frac{\pi}{2} \SO f(x) = \frac{\pi}{2} \ \forall x \in [-1, 1]\]
\end{proof}

\begin{Prop}
    $f, g$ непрерывны на $[A, B\rangle$ и дифференцируемы на $(A, B)$. 
    Если $f(A) = g(A), f'(x) > g'(x) \ \forall x \in (A, B)$. Тогда
    \[f(x) > g(x) \ \forall x \in (A, B\rangle\]
\end{Prop}

\begin{proof}
    $h = f - g$ непрерывна на $[A, B\rangle$, дифференцируема на $(A, B), h'(x) > 0 \ \forall x \in (A, B), h(A) = 0$
    $\SO h$ строго возрастает $\SO h(x) > 0 \ \forall x \in (A, B\rangle$   
\end{proof}

\begin{Example}
    $\cos x > 1 - \frac{x^2}{2} \ \forall x > 0$.
\end{Example}

\begin{proof}
    $\cos 0 = 1 - \frac{0^2}{2}, (\cos x)' = -\sin x, (1 - \frac{x^2}{2})' = -x$.
    $\sin x < x \ \forall x > 0 \EQ -\sin x > -x \SO \cos x > 1 - \frac{x^2}{2}$   
\end{proof}

\begin{Def}
    $E \subset \R, f : E \to \R, a \in E$.
    \begin{MyList}
        \item Пусть $\exists \delta > 0 : \forall x \in (a - \delta, a + \delta) \cap E \to f(x) \geqslant f(a)$. Тогда $a$ -- точка (локального) минимума $f$.
        Если выполнено $f(x) \leqslant f(a)$, то $a$ точка (локального) максимума $f$.
        \item Если $\forall x \in (a - \delta, a + \delta) \setminus \{a\}$неравенства строгие, то $a$ -- точка строгого минимума или максимума. 
        \item Такие точки $a$ называются \textbf{точками (локального) экстремума}.
    \end{MyList}
\end{Def}

\begin{Thm}[Необходимое условие экстремума]
    $f : \langle A, B\rangle \to \R, a \in (A, B), f $ -- дифференцируема в точке $a$.
    Если $a$ является точкой экстремума $f$, то $f'(a) = 0$.  
\end{Thm}

\begin{proof}
    Пусть $a$ -- точка минимума. $\exists \delta > 0 : [a - \delta, a + \delta] \subset (A, B), f(x) \geqslant f(a) \ \forall x \in [a - \delta, a + \delta]$.
    Рассмотрим сужение $f|_{[a - \delta, a + \delta]}$. По теореме Ферма $f'(a) = 0$.  
\end{proof}

\begin{Def}
    Точки, в которых $f' = 0$ называются стационарными.
\end{Def}

\begin{Def}
    Пусть $a \in (A, B)$. Будем называть $a$ критической точкой (точкой, подозрительной на экстремум), если $f'(a) = 0$ или $f$ не дифференцируема в точке $a$.
\end{Def}

План исследования на наибольшее и наименьшее значение на отрезке:

\begin{MyList}
    \item Найти множество всех критических точек -- $C$.
    \item Посчитать значения $f$ в каждой точке из $C$ и на концах отрезка.
    \item Выбрать наибольшее и наименьшее.
\end{MyList}

\begin{Thm}[Достаточное условие экстремума в терминах первой производной]
    $f : \langle A, B\rangle \to \R, a \in (A, B), \delta : (a - \delta, a + \delta) \subset \langle A, B\rangle$. 
    Пусть $f$ непрерывна в точке $a$ и дифференцируема на $(a -\delta, a) \cup (a, a + \delta)$
    \begin{MyList}
        \item Если $f'(x) < 0$ при $x \in (a -\delta, a), f'(x) > 0$ при $x \in (a, a + \delta)$, то $a$ -- точка строгого минимума.
        \item Если $f'(x) > 0$ при $x \in (a -\delta, a), f'(x) < 0$ при $x \in (a, a + \delta)$, то $a$ -- точка строгого максимума.     
    \end{MyList}  
\end{Thm}

\begin{proof}
    Докажем первое утверждение. \\
    $f$ строго убывает на $(a - \delta, a] \SO f(x) > f(a) \ \forall x \in (a - \delta, a)$ \\
    $f$ строго возрастает на $[a, a + \delta) \SO f(x) > f(a) \ \forall x \in (a, a + \delta) \SO a$ -- точка строгого локального минимума.  
\end{proof}

\begin{Rem}
    Если $f'$ не меняет знак на $(a - \delta, a) \cup (a, a + \delta)$, то $f$ не имеет экстремума в точке $a$.  
\end{Rem}

\begin{proof}
    $f$ монотонна на $(a - \delta, a + \delta)$ 
\end{proof}

\begin{Rem}
    Верно ли, что если $f$ дифференцируема на $(A, B)$ и в точке $a \in (A, B)$ $f$ имеет строгий локальный минимум, то $\exists \delta : f'(x) < 0 \ \forall x \in (a - \delta, a)$ и $f'(x) > 0 \ \forall x \in (a, a + \delta)$  
    Спойлер: нет.
\end{Rem}

\begin{Example}
    \[f(x) = \begin{cases}
        x^2 \left(\sin\frac{1}{x} + 2\right), x \neq 0 \\
        0, x = 0
    \end{cases}\]
    $f$ дифференцируема на $\R$. $f(0) = 0$ и $f(x) > 0 \ \forall x \neq 0 \SO 0$ точка строгого минимума.
    \[f'(x) = 2x\left(\sin \frac{1}{x} + 2\right) + x^2 \cos \frac{1}{x} \cdot -\frac{1}{x^2} = 2x\left(\sin \frac{1}{x} + 2\right) - \cos \frac{1}{x}\]
    При $x \to 0+ : \ 2x \left(\sin \frac{1}{x} + 2\right) \to 0$. А $\cos \frac{1}{x}$ может принимать все значения он $[-1, 1]$ при $x \in (0, \delta) \ \forall \delta > 0$ 
\end{Example}

\begin{Thm}[Достаточное условие экстремума в терминах второй производной]
    $f : \langle A, B\rangle \to \R, a \in (A, B)$. Пусть $f$ дважды дифференцируема в точке $a$ и $f'(a) = 0$.
    \begin{MyList}
        \item Если $f''(a) > 0$, то $a$ -- точка строгого минимума.
        \item Если $f''(a) < 0$, то $a$ -- точка строгого максимума. 
    \end{MyList}
\end{Thm}

\begin{proof}
    Докажем первое утверждение. Применим к $f$ формулу Тейлора с остатком в форме Пеано:
    \[f(x) = f(a) + f'(a) \cdot (x - a) + f''(a) \cdot \frac{(x - a)^2}{2} + o((x - a)^2)\]
    \[f(x) - f(a) = f''(a) \cdot \frac{(x - a)^2}{2} + o((x - a)^2) = (x-a)^2 \cdot \frac{f''(a)}{2}(1 + o(1))\]
    \[f(x) - f(a) = (x - a)^2 \frac{f''(a)}{2}(1 + o(1))\]
    т.к. $1 + o(1) \xrightarrow[x \to a]{} 1$, то $\exists \delta > 0 : \forall x \in (a - \delta, a + \delta) \to (1 + o(1)) > 0$.
    Тогда $\forall x \in (a - \delta, a + \delta) \setminus \{a\} \ f(x) - f(a) > 0$  
\end{proof}

\begin{Rem}
    Если $f''(a) = 0$, то эта теорема не дает ответа на вопрос об экстремуме.
\end{Rem}

\begin{Thm}[О связи экстремума со старшими производными]
    $f: \langle A, B\rangle, a \in (A, B), n \in \N$. Пусть $f$ $n$ раз дифференцируема в точке $a$, причем $f'(a) = f''(a) = ... = f^{(n - 1)}(a) = 0$.
    Тогда
    \begin{MyList}
        \item Если $n$ -- нечетно, то $f$ не имеет экстремума в точке $a$.
        \item Если $n$ -- четно и $f^{(n)}(a) > 0$, то $a$ -- точка строгого минимума.
        \item Если $n$ -- четно и $f^{(n)}(a) < 0$, то $a$ -- точка строгого максимума. 
    \end{MyList}
\end{Thm}

\Subsection{Выпуклость}

\begin{Def}
    $f: \langle A, B\rangle \to \R$ 
    \begin{MyList}
        \item Пусть $\forall a, b \in \langle A, B\rangle$ и $\lambda \in (0, 1)$ справедливо неравенство
        \[f(\lambda a + (1 - \lambda) b) \leqslant \lambda f(a) + (1 - \lambda) f(b)\]
        Тогда $f$ называется выпуклой на $\langle A, B\rangle$ 

        \item Если знак в неравенстве строгий, то $f$ строго выпукла.
        \item Если знак ``$\geqslant$'' ,то $f$ называется вогнутой на $\langle A, B\rangle$.
        \item Если знак ``$>$'' ,то $f$ называется строго вогнутой на $\langle A, B\rangle$ . 
    \end{MyList}
\end{Def}

\begin{Rem}
    Не умаляя общности, $a < b$. 
\end{Rem}

\begin{Rem}
    Если $a = b$, то знак "$=$" 
\end{Rem}

\begin{Rem}
    Иногда называются выпуклая вниз и выпуклая вверх.
\end{Rem}

\begin{Rem}
    $x = \lambda a + (1 - \lambda) b$.
    При $\lambda \in (0, 1)$ точка $x$ пробегает $(a, b)$.
    \[\lambda = \frac{b - x}{b - a}, 1 - \lambda = \frac{x - a}{b - a}\]
    То есть определение можно переписать так:
    \[f(x) \leqslant \underbrace{\frac{b - x}{b - a}f(a) + \frac{x - a}{b - a} \cdot f(b)}_{\text{хорда, проходящая через }(a, f(a)), (b, f(b))}\]
    т.е. график $f$ лежит не выше, чем любая хорда.
\end{Rem}

\begin{Rem}
    Пусть $f$ и $g$ -- выпуклые на $\langle A, B\rangle$
    \begin{MyList}
        \item $f + g$ тоже выпуклая на $\langle A, B\rangle$ 
        \item $\forall \alpha > 0 \ \alpha \cdot f$ выпуклая
        \item $\forall \alpha < 0 \ \alpha \cdot f$ вогнутая.
    \end{MyList}
\end{Rem}

\begin{Lm}[Лемма о трех хордах]
    $f : \langle A, B\rangle \to \R$. Тогда равносильны следующие утверждения:

    \begin{MyList}
        \item $f$ строго выпукла на $\langle A, B\rangle$ 
        \item $\forall a, b, c \in \langle A, B\rangle : a < c < b$ выполняется
        \[ \frac{f(c) - f(a)}{c - a} < \frac{f(b) - f(c)}{b - c}\] 

        \item $\forall a, b, c \in \langle A, B\rangle : a < c < b$ выполнены неравенства:
        \[ \frac{f(c) - f(a)}{c - a} < \frac{f(b) - f(a)}{b - a} < \frac{f(b) - f(c)}{b - c}\]
    \end{MyList}
\end{Lm}

\begin{Rem}
    Рисунок \TODO
\end{Rem}

\begin{proof}
    $3) \SO 2)$ очевидно. \\
    $1) \SO 3)$. $f$ строго выпукла. Положим $\lambda = \frac{b - c}{b - a} \in (0, 1)$. Тогда $c = \lambda a + (1 - \lambda) b$.
    \[f(c) < \lambda f(a) + (1 -\lambda)f(b)\]
    Перепишем это неравенство в двух разных формах:
    \[f(c) - f(b) < \lambda(f(a) - f(b)) \EQ f(b) - f(c) > \frac{b - c}{b - a}(f(b) - f(a)) \EQ \frac{f(b) - f(c)}{b - c} > \frac{f(b) - f(c)}{b - a}\] 
    \[f(c) - f(a) < (1 - \lambda)(f(b) - f(a)) \EQ f(c) - f(a) < \frac{c - a}{b - a}(f(b) - f(a)) \EQ \frac{f(c) - f(a)}{c - a} < \frac{f(b) - f(a)}{b - a}\] 
    $2) \SO 1)$. Пусть $a, b \in \langle A, B\rangle$. Обозначим $c = \lambda a + (1 - \lambda)b, \lambda \in (0, 1)$.
    Тогда $\lambda = \frac{b - c}{b - a}$ и $1 - \lambda = \frac{c - a}{b - a}$ 
    \[\SO \frac{c - a}{1 - \lambda} = \frac{b - c}{\lambda}\]
    Тогда 
    \[\frac{f(c) - f(a)}{c - a} < \frac{f(b) - f(c)}{b - c} \EQ \frac{f(c) - f(a)}{1 - \lambda} < \frac{f(b) - f(c)}{\lambda} \EQ \lambda(f(c) - f(a)) < (1 - \lambda)(f(b) - f(c))\]  
    \[\EQ \lambda f(c) - \lambda f(a) < f(b) - f(c) - \lambda f(b) + \lambda f(c) \EQ f(c) < (1 - \lambda) f(b) + \lambda f(a)\]
\end{proof}

\begin{Cons}
    $f: \langle A, B\rangle \to \R, a \in \langle A, B\rangle$ и
    \[F(x) = \frac{f(x) - f(a)}{x - a}\]
    Тогда 
    \begin{MyList}
        \item Если $f$ выпукла на $\langle A, B\rangle$, то $F$ возрастает на $\langle A, B\rangle \setminus \{a\}$.
        \item Если $f$ строго выпукла на $\langle A, B\rangle$, то $F$ строго возрастает на $\langle A, B\rangle \setminus \{a\}$.
    \end{MyList}
\end{Cons}

\begin{proof}
    Докажем 2. Пусть $x < y \in \langle A, B\rangle \setminus \{a\}$. Докажем, что $F(x) < F(y)$.
    \begin{MyItemize}
        \item $a < x < y \SO$ по лемме о трех хордах $\SO \frac{f(x) - f(a)}{x - a} < \frac{f(y) - f(a)}{y - a} \SO F(x) < F(y)$.
        \item $x < y < a \SO \frac{f(a) - f(x)}{a - x} < \frac{f(a) - f(y)}{a - y} \SO F(x) < F(y)$.
        \item $x < a < y \SO \frac{f(a) - f(x)}{a - x} < \frac{f(y) - f(a)}{y - a} \SO F(x) < F(y)$.
    \end{MyItemize}
\end{proof}

\begin{Example}
    $f(x) = x^2$ строго выпукла на $\R$.
\end{Example}

\begin{proof}
    Пусть $a < c < b$. Тогда 
    \[ \frac{f(c) - f(a)}{c - a} - \frac{f(b) - f(c)}{b - c} = \frac{c^2 - a^2}{c - a} - \frac{b^2 - c^2}{b - c} = (c + a) - (b + c) = a - b < 0\]
    $\SO f(x)$ 
\end{proof} строго выпукла.

\begin{Thm}[Об односторонних производных]
    $f : \langle A, B\rangle \to \R, f$ -- выпукла на $\langle A, B\rangle$. Тогда
    \begin{MyList}
        \item $\forall a < B \ \exists f_+'(a) \in [-\infty, +\infty), \forall a > A \ \exists f_-'(a) \in (-\infty, +\infty]$
        \item Если $a \in (A, B)$, то $f_+'(a)$ и $f_-'(a)$ конечны и $f_-'(a) \leqslant f_+'(a)$.
    \end{MyList}
\end{Thm}

\begin{proof}
    Рассмотрим $F(x) = \frac{f(x) - f(a)}{x - a}, F : \langle A, B\rangle \setminus \{a\} \to \R$.
    \begin{MyList}
        \item Пусть $a < B$. Тогда $F$ возрастает на $(a, B\rangle$. Тогда 
        \[\exists \lim_{x \to a+} F(x) \in [-\infty, +\infty) \text{(по т. о пределе монотонной функции)}\]
        \[\lim_{x \to a+} \frac{f(x) - f(a)}{x - a} = f_+'(a)\]
        Для $f_-'(a)$ аналогично.

        \item Пусть $a \in (A, B)$. Возьмем $x < a < y, x, y \in \langle A, B\rangle$. Тогда по следствию 
        \[F(x) < F(y) \xRightarrow[x, y \to a]{} f_-'(a) \leqslant f_+'(a)\]
        $f_-'(a) \in (-\infty, +\infty], f_+'(a) \in [-\infty, +\infty) \SO$ обе конечны. 
    \end{MyList}
\end{proof}

\begin{Cons}
    Если $f$ выпукла на $\langle A, B\rangle$, то она непрерывна на $(A, B)$.
\end{Cons}

\begin{proof}
    Пусть $a \in (A, B)$. Т.к. $\exists f_+'(a)$, то $f$ непрерывна в точке $a$ справа,
    т.к. $\exists f_-'(a)$, то $f$ непрерывна на т. $a$ слева $\SO f$ непрерывна в точке $a$. 
\end{proof}

\begin{Example}
    Рисунок \TODO.
    \[f(x) = \begin{cases}
        -\sqrt{1 - x^2}, x \in (-1, 1] \\
        1, x = -1
    \end{cases}\]  
\end{Example}

\begin{Ex}
    $f_+'(-1) = ?, f_-'(1) = ?$.
\end{Ex}

\begin{Thm}[Критерий выпуклости в терминах касательных]
    $f : \langle A, B\rangle \to \R, f$ дифференцируема на $\langle A, B\rangle$. Тогда 
    \begin{MyList}
        \item Функция $f$ выпукла на $\langle A, B\rangle \EQ \forall a, x \in \langle A, B\rangle \to f(x) \geqslant T_{a, 1}f(x)$ 
        \item Функция $f$ строго выпукла $\EQ \forall a \neq x \in \langle A, B\rangle \to f(x) > T_{a, 1} f(x)$.
    \end{MyList}
\end{Thm}

\begin{Ex}
    Доказать.
\end{Ex}

\begin{Thm}[Выпуклость и асимптоты]
    $f : \langle A, +\infty) \to \R$ и имеет асимптоту $y = kx + b$.
    Тогда
    \begin{MyList}
        \item Если $f$ выпукла на $\langle A, +\infty)$, то $f(x) \geqslant kx + b \ \forall x \geqslant A$
        \item Если $f$ строго выпукла на $\langle A, +\infty)$, то $f(x) > kx + b \ \forall x > A$.
    \end{MyList}
\end{Thm}

\begin{Ex}
    Доказать.
\end{Ex}

\begin{Thm}[Критерий выпуклости в терминах первой производной]
    $f : \langle A, B\rangle \to \R$, непрерывна на $\langle A, B\rangle$ и дифференцируема на $(A, B)$. Тогда
    \begin{MyList}
        \item $f$ выпукла на $\langle A, B\rangle \EQ f'$ возрастает на $(A, B)$.
        \item $f$ строго выпукла на $\langle A, B\rangle \EQ f'$ строго возрастает на $(A, B)$.    
    \end{MyList} 
\end{Thm}

\begin{proof}
    Докажем 2. ``$\SO$'' 
    Возьмем $x < y \in (A, B)$. Покажем, что $f'(x) < \frac{f(y) - f(x)}{y - x} < f'(y)$.
    \[F(t) = \frac{f(t) - f(x)}{t - x}, t \neq x\]
    Т.к. $f$ строго выпукла, то $F$ строго возрастает на $(x, y]$
    \[\SO f'(x) = \lim_{t \to x+} F(t) = \inf_{t \in (x, y]} F(t) < F(y) = \frac{f(y) - f(x)}{y - x}\]
    Аналогично: $f'(y) > F(y)$. \\
    ``$\Leftarrow$''. Достаточно показать, что $\forall a < c < b \in \langle A, B\rangle$ верно
    \[ \frac{f(c) - f(a)}{c - a} < \frac{f(b) - f(c)}{b - c}\]
    По теореме Лагранжа $\exists \alpha \in (a, c) : \frac{f(c) - f(a)}{c - a} = f'(\alpha), \exists \beta \in (c, b) : \frac{f(b) - f(c)}{b - a} = f'(\beta)$
    $\SO \alpha < \beta$, т.к. $f'$ строго возрастает, то $f'(\alpha) < f'(\beta)$ 
\end{proof}

\begin{Thm}[Критерий выпуклости в терминах второй производной]
    $f : \langle A, B\rangle \to \R, $ непрерывна на $\langle A, B\rangle$ и дважды дифференцируема на $(A, B)$.
    Тогда
    \begin{MyList}
        \item $f$ выпукла на $\langle A, B\rangle \EQ f''(x) \geqslant 0 \ \forall x \in (A, B)$.
        \item Если $f''(x) > 0 \ \forall x \in (A, B)$, то $f$ строго выпукла.
    \end{MyList}

    \begin{Rem}
        Обратное к 2 не всегда выполнено: $f(x) = x^4$ -- строго выпукла, но $f''(x) = 12x^2, f''(0) = 0$ 
    \end{Rem}
\end{Thm}

\begin{proof}
    \begin{MyList}
        \item По предыдущей теореме выпуклость $\EQ$ возрастанию $f' \EQ f''(x) \geqslant 0$.
        
        \item $f''(x) > 0 \SO f'(x)$ строго возрастает $\SO f$ строго выпукла. 
    \end{MyList}
\end{proof}

\begin{Example}
    $f(x) = \sin x$ на $\left[0, \frac{\pi}{2}\right]$. $f'(x) = \cos x, f''(x) = -\sin x \leqslant 0$ на $\left[0, \frac{\pi}{2}\right]$
    $\SO f$ строго вогнута на $\left(0, \frac{\pi}{2}\right)$ ( $f''(x) = 0$ только при $x = 0$) $\SO \sin x > \frac{2}{\pi} x$ на $\left(0, \frac{\pi}{2}\right)$.  
\end{Example}

\begin{Def}
    $f : \langle A, B\rangle \to \R, a \in (A, B)$. Предположим, что выполнены следующие условия:
    \begin{MyList}
        \item $\exists \delta > 0 : (a - \delta, a + \delta) \subset (A, B), f$ имеет разный характер выпуклости на $(a - \delta, a]$ и $[a, a + \delta)$.
        \item $f$ непрерывна в точке $a$.
        \item $f'(a) \in \overline{\R}$
    \end{MyList}
    Тогда $a$ называется точкой перегиба функции $f$.
\end{Def}

\begin{Thm}[Необходимое условие перегиба]
    $f : \langle A, B\rangle \to \R, a \in (A, B)$. Пусть $f$ дважды дифференцируема в точке $a$. 
    Если $a$ является точкой перегиба, то $f''(a) = 0$.    
\end{Thm}

\begin{proof}
    Пусть $f$ вогнута слева от $a$, выпукла справа от $a$. Возьмем $\delta > 0 : f$ дифференцируема на $(a - \delta, a + \delta)$.
    Тогда $f'$ убывает на $(a - \delta, a]$ и $f'$ возрастает на $[a, a + \delta)$. Тогда $a$ -- точка минимума $f' \SO f''(a) = 0$.
\end{proof}

\begin{Thm}[Достаточное условие перегиба]
    $f : \langle A, B\rangle \to \R, a \in (A, B), f$ непрерывна в точке $a$ и $f'(a) \in \overline{\R}$.
    Пусть $\exists \delta > 0 : f$ дважды дифференцируема на $(a - \delta, a) \cup (a, a + \delta)$ и выполнено одно из следующих условий:
    \begin{MyList}
        \item $f'' > 0$ на $(a - \delta, a)$ и $f'' < 0$ на $(a, a + \delta)$ 
        \item $f'' < 0$ на $(a - \delta, a)$ и $f'' > 0$ на $(a, a + \delta)$   
    \end{MyList}     
    Тогда $a$ -- точка перегиба.
\end{Thm}

\begin{proof}
    Пусть выполнено 1. Тогда $f$ выпукла на $(a - \delta, a)$, вогнута на $(a, a + \delta)$
    $\SO a$ -- точка перегиба.  
\end{proof}

План исследования функции:
\begin{MyList}
    \item Область определения функции (и множество значений функции).
    \item Нули функции, знакопостоянство.
    \item Четность/нечетность.
    \item Периодичность.
    \item Разрывы функции.
    \item Монотонность (экстремумы).
    \item Выпуклость (перегибы).
    \item Асимптоты. 
\end{MyList}

\begin{Prop}[Обобщение неравенства Бернулли]
    $\alpha > 1$. Тогда $(1 + x)^\alpha > 1 + \alpha x \ \forall x > -1, x \neq 0$.
\end{Prop}

\begin{proof}
    $f(x) = (1 + x)^\alpha, f''(x) = \alpha(\alpha - 1)(1 + x)^{\alpha - 2} > 0 \ \forall x > -1$
    $\SO f$ строго выпукла. $1 + \alpha x = T_{0, 1} f(x)$. По теореме о касательных $f$ лежит над своей касательной в $x = 0$,
    т.е. $f(x) > T_{0, 1} f(x)$ (при $x = 0$ равен).
\end{proof}

\Subsection{Классические неравенства}

$p, q > 1$ и $\frac{1}{p} + \frac{1}{q} = 1$ -- сопряженные показатели.

\begin{Thm}[Неравенство Юнга]
    Пусть $x, y \geqslant 0, p, q$ -- сопряженные показатели. Тогда
    \[xy \leqslant \frac{x^p}{p} + \frac{y^q}{q}\]
    Причем равенство достигается тогда и только тогда, когда $x^p = y^q$.
\end{Thm}

\begin{proof}
    Если $x = 0$ или $y = 0$, то очевидно. Будем считать, что $x > 0$ и $y > 0$. Возьмем $f(t) = \ln t$.
    $f$ строго вогнутая на $(0, +\infty)$. Подставим в определение точки $x^p$ и $y^q$.
    \[\forall \lambda \in (0, 1) \ f(\lambda x^p + (1 - \lambda)y^q) > \lambda f(x^p) + (1 - \lambda)f(y^q)\]
    Причем равенство возможно лишь когда $x^p = y^q$. Возьмем $\lambda = \frac{1}{p}$, тогда $1 - \lambda = \frac{1}{q}$.\
    \[\ln \left(\frac{1}{p} x^p + \frac{1}{q}y^q\right) > \frac{1}{p} \ln x^p + \frac{1}{q} \ln y^q = \ln(xy)\]
    Т.к. $\ln x$ строго возрастает на $(0, +\infty)$, то $ \frac{x^p}{p} + \frac{y^q}{q} > xy$.
\end{proof}

Векторы в $\R^n$. $x = (x_1, ..., x_n), y = (y_1, ..., y_n)$, $x, y \in \R^n$ 

\begin{MyList}
    \item $x + y = (x_1 + y_1, ..., x_n + y_n)$ 
    \item $\lambda x = (\lambda x_1, ..., \lambda x_n)$ 
    \item Скалярное произведение: $x \cdot y = x_1 y_1 + ... + x_n y_n$.
    \item Длина вектора: $|x| = \sqrt{x_1^2 + ... + x_n^2}$ 
    \item $p$-норма вектора: $||x||_p = \sqrt[p]{|x_1|^p + ... + |x_n|^p}$ (обощение понятия длины).
    \item $x$ и $y$ коллинеарны, если либо один из них нулевой, либо $\exists \lambda \in \R \setminus \{0\} : x = \lambda y$.
    \item $x$ и $y$ сонаправлены, если либо один из них нулевой, либо $\exists \lambda > 0 : x = \lambda y$.  
\end{MyList}

\begin{Thm}[Неравенство Минковского для неотрицательных чисел]
    $n \in \N, p \geqslant 1$. Предположим, что векторы $x$ и $y \in \R^n$ имеют неотрицательные координаты. Тогда
    \[ \left(\sum_{k=1}^{n} (x_k + y_k)^p \right)^{\frac{1}{p}} \leqslant \left( \sum_{k=1}^{n} x_k^p\right)^{\frac{1}{p}} + \left(\sum_{k = 1}^n y_k^p\right)^{\frac{1}{p}}\]
    Равенство $\EQ$ либо $p = 1$, либо $x$ и $y$ сонаправлены.
\end{Thm}

\begin{proof}
    Если $p = 1$, то очевидно. \\
    Пусть $p > 1$ и $x$ и $y$ ненулевые.
    \[X = ||x||_p, Y = ||y||_p, X, Y > 0\]
    Рассмотрим $f(x) = x^p$ -- строго выпуклая на $[0, +\infty)$.
    $f''(x) = p(p - 1)x^{p - 2} > 0$ при $x > 0$ и $f$ непрерывна в точке $0$.
    Запишем определение выпуклости для $ \frac{x_k}{X}$ и $ \frac{y_k}{Y}$.
    \[\forall \lambda \in (0, 1) \ f\left(\lambda \frac{x_k}{X} + (1 - \lambda)\frac{y_k}{Y}\right) \leqslant \lambda f\left(\frac{x_k}{X}\right) + (1 - \lambda) f\left(\frac{y_k}{Y}\right)\]
    Равенство тогда и только тогда, когда $ \frac{x_k}{X} = \frac{y_k}{Y}$. Возьмем $\lambda = \frac{X}{X + Y}, 1 - \lambda = \frac{Y}{X + Y}$
    \[f \left( \frac{x_k + y_k}{X + Y}\right) \leqslant \frac{X}{X + Y} f \left( \frac{x_k}{X}\right) + \frac{Y}{X + Y} f \left( \frac{y_k}{Y}\right)\]
    \[ \frac{(x_k + y_k)^p}{(X + Y)^p} \leqslant \frac{X}{X + Y} \cdot \frac{x_k^p}{X^p} + \frac{Y}{X + Y} \cdot \frac{y_k^p}{Y^p} \ \forall k = 1, ..., n\]
    Сложим все неравенства:
    \[ \frac{\sum_{k = 1}^n (x_k + y_k)^p}{(X + Y)^p} \leqslant \frac{X}{X + Y} \cdot \frac{\sum_{k=1}^n x_k^p}{X^p} + \frac{Y}{X + Y} \cdot \frac{\sum_{k=1}^n y_k^p}{Y^p} =\]
    Заметим, что $\sum_{k=1}^{n} \frac{x_k^p}{X^p} = \frac{1}{||x||_p^p} \sum_{k=1}^{n} x_k^p = 1$ 
    \[= \frac{X}{X + Y} + \frac{Y}{X + Y} = 1 \SO ||x + y||_p \leqslant X + Y = ||x||_p + ||y||_p\]
    Равенство $\EQ$ равенство $\forall k$, т.е. $ \frac{x_k}{X} = \frac{y_k}{Y} \SO x_k = \frac{X}{Y} y_k \SO$ $x$ и $y$ сонаправлены. 
\end{proof}

\begin{Cons}[Неравенство Минковского в $\R^n$ ]
    Пусть $n \in \N, p \geqslant 1, x = (x_1, ..., x_n), y = (y_1, ..., y_n) \in \R^n$.
    Тогда
    \[\left(\sum_{k=1}^{n} |x_k + y_k|^p\right)^{\frac{1}{p}} \leqslant \left(\sum_{k = 1}^n |x_k|^p\right)^{\frac{1}{p}} + \left(\sum_{k=1}^{n} |y_k|^p\right)^{\frac{1}{p}}\]
    С случае $p > 1$ равенство тогда и только тогда, когда $x$ и $y$ сонаправлены.
\end{Cons}

\begin{proof}
    \begin{MyItemize}
        \item $x$ или $y$ нулевой -- очевидно.
        \item $p = 1$ -- очевидно (неравенство треугольника). 
        \item $p > 1, x$ и $y$ ненулевые.
        Применим теорему к векторам с координатами $|x_k|$ и $|y_k|$ 
        \[\left(\sum_{k = 1}^n |x_k + y_k|^p\right)^{\frac{1}{p}} \leqslant \left(\sum_{k = 1}^n \left(|x_k| + |y_k|\right)^p\right)^{\frac{1}{p}} \leqslant \left(\sum_{k = 1}^n |x_k|^p\right)^{\frac{1}{p}} + \left(\sum_{k = 1}^n |y_k|^p\right)^{\frac{1}{p}}\]
        Первый переход превращается в равенство, если $x_k$ и $y_k$ одного знака.
    \end{MyItemize}
\end{proof}

\begin{Ex}
    Когда равенство при $p = 1$?
\end{Ex}

\begin{Rem}
    Можно записать в виде \[||x + y||_p \leqslant ||x||_p + ||y||_p\]
    При $p = 2$ неравенство треугольника с вершинами в точках $0, x, x + y$.
    Равенство, если эти три точки лежат на одной прямой.
    Т.е. неравенство Минковского обобщает неравенство треугольника.
\end{Rem}

\begin{Thm}[Неравенство Гёльдера для неотрицательных чисел]
    $p$ и $q$ -- сопряженные показатели, $n \in \N$. $x, y \in \R^n$ -- с неотрицательными координатами.
    Тогда 
    \[\sum_{k = 1}^n x_k y_k \leqslant \left(\sum_{k = 1}^n x_k^p\right)^{\frac{1}{p}} \cdot \left(\sum_{k = 1}^n y_k^q\right)^{\frac{1}{q}}\]
    Равенство тогда и только тогда, когда $x_k^p ||y||_q^q = y_k^q ||x||_p^p \ \forall k = 1, ..., n$.
\end{Thm}

\begin{proof}
    $a_k = \frac{x_k}{||x||_p}, b_k = \frac{y_k}{||y||_q} \ \forall k = 1, ..., n$.
    \[\sum_{k = 1}^n a_k^p = \frac{1}{||x||_p^p} \sum_{k = 1}^n x_k^p = 1, \sum_{k = 1}^n b_k^q = 1\]
    По неравенству Юнга $a_k b_k \leqslant \frac{a_k^{p}}{p} + \frac{b_k^q}{q} \ \forall k = 1, ..., n$.
    \[\sum_{k = 1}^n a_kb_k \leqslant \frac{1}{p}\sum_{k = 1}^n a_k^p + \frac{1}{q} \sum_{k = 1}^n b_k^q = \frac{1}{p} + \frac{1}{q} = 1\]  
    \[\sum_{k = 1}^n x_k y_k \leqslant ||x||_p \cdot ||y||_q\]
    Равенство $\EQ$ равенство для любого $k : a_k^p = b_k^q$ -- из неравенства Юнга. 
\end{proof}

\begin{Rem}
    Равенство $x_k^p ||y||_q^q = y_k^q ||x||_p^p \ \forall k = 1, ..., n$ означает, что 
    векторы $(x_1^p, x_2^p, ..., x_n^p)$ и $(y_1^q, y_2^q, ..., y_n^q)$ коллинеарны.  
    \[x_k^{p} = y_k^{q} \cdot \frac{||x||_p^p}{||y||_q^q}\]
\end{Rem} 

\begin{Cons}[Неравенство Гёльдера в $\R^n$ ]
    $n \in \N, p, q$ -- сопряженные показатели, $x, y \in \R^n$. Тогда
    \[\left|\sum_{k = 1}^n x_ky_k\right| \leqslant \left(\sum_{k = 1}^n |x_k|^p\right)^{\frac{1}{p}} \cdot \left(\sum_{k = 1}|y_k|^q\right)^{\frac{1}{q}}\]
\end{Cons}

\begin{proof}
    По неравенству Гёльдера для $(|x_1|, |x_2|, ..., |x_n|), (|y_1|, ..., |y_n|)$ 
    \[\left|\sum_{k = 1}^n x_k y_k\right| \leqslant \sum_{k = 1}^n |x_k||y_k| \leqslant \left(\sum_{k = 1}^n |x_k|^p\right)^{\frac{1}{p}} \cdot \left(\sum_{k = 1}|y_k|^q\right)^{\frac{1}{q}}\]
\end{proof}

\begin{Thm}[Неравенство Коши в $\R^n$ ]
    $n \in N, x, y \in \R^n$. Тогда 
    \[\left|\sum_{k = 1}^n x_ky_k\right| \leqslant \sqrt{\sum_{k = 1}^n x_k^2} \cdot \sqrt{\sum_{k = 1}^n y_k^2}\] 
    Равенство $\EQ$ $x$ и $y$ коллинеарны.
\end{Thm}

\begin{proof}
    \begin{MyItemize}
        \item Если $x$ или $y$ нулевой -- очевидно.
        \item Пусть $x$ и $y$ -- ненулевые.
        \[\left|\sum_{k = 1}^n x_k y_k\right| \leqslant \sum_{k = 1}^n |x_k| |y_k|\]
        Равенство $\EQ$ все $(x_k y_k)$ имеют один знак.
        По неравенству Гёльдера при $p = q = 2$:
        \[\sum_{k = 1}^n |x_k||y_k| \leqslant \sqrt{\sum_{k = 1}^n x_k^2} \cdot \sqrt{\sum_{k = 1}^n y_k^2}\]
        Равенство $\EQ$ $x_k^2 = \frac{||x||_2}{||y||_2} y_k^2$, т.е. $x$ и $y$ коллинеарны.
    \end{MyItemize}
\end{proof}

\begin{Rem}
    В $\R^2 \ x \cdot y = |x| \cdot |y| \cdot \cos \alpha \EQ |x \cdot y| \leqslant |x| \cdot |y|$ 
\end{Rem}

\begin{Thm}[Неравенство Йенсена]
    $f : \langle A, B\rangle \to \R, n \in \N$ \\ $x_1, x_2, ..., x_n \in \langle A, B\rangle, \lambda_1, \lambda_2, ..., \lambda_n$ -- положительные числа: 
    $\sum_{k = 1}^n \lambda_k = 1$. Тогда
    \begin{MyList}
        \item Если $f$ выпукла на $\langle A, B\rangle$, то
        \[f(\lambda_1 x_1 + \lambda_2 x_2 + ... + \lambda_n x_n) \leqslant \lambda_1f(x_1) + \lambda_2 f(x_2) + ... + \lambda_n f(x_n)\]

        \item Если $f$ строго выпукла, то 
        \[f(\lambda_1 x_1 + \lambda_2 x_2 + ... + \lambda_n x_n) < \lambda_1f(x_1) + \lambda_2 f(x_2) + ... + \lambda_n f(x_n)\]
    \end{MyList} 
    Если $x_1 = x_2 = ... = x_n$, то равенство.
\end{Thm}

\begin{proof}
    \begin{MyItemize}
        \item Если все $x_k$ совпадают -- очевидно.
        \item Пусть не все $x_k$ совпадают
        \[c = \lambda_1 x_1 + ... + \lambda_n x_n \in (A, B)\]
        По индукции: $n = 1$ -- очевидно $f(x_1) = f(x_1)$.
        Пусть $\lambda_1 + \lambda_2 + ... + \lambda_n = 1 - \lambda_{n + 1}$ и 
        \[a = \frac{\sum_{k = 1}^n \lambda_k x_k}{1 - \lambda_{n + 1}}\]
        По предположению индукции:
        \[f(a) \leqslant \frac{\lambda_1}{1 - \lambda_{n + 1}} f(x_1) + \frac{\lambda_2}{1 - \lambda_{n + 1}}f(x_2) + ... + \frac{\lambda_n}{1 - \lambda_{n + 1}} f(x_n)\]
        Тогда
        \[f((1 - \lambda_{n + 1}) \cdot a + \lambda_{n + 1} \cdot x_{n + 1}) \leqslant (1 - \lambda_{n + 1}) f(a) + \lambda_{n + 1} \cdot f(x_{n + 1}) \leqslant\]
        \[\leqslant \lambda_1 f(x_1) + \lambda_2 f(x_2) + ... + \lambda_{n + 1} f(x_{n + 1})\] 
    \end{MyItemize}
\end{proof}

\begin{Rem}
    Для вогнутых поменять знак.
\end{Rem}

\begin{Rem}
    $n = 2, \lambda_1 + \lambda_2 = 1, \lambda_2 = 1 - \lambda_1 $ -- определение выпуклости. 
\end{Rem}

\begin{Rem}
    Точка $c = \lambda_1 x_1 + ... + \lambda_n x_n \in \langle A, B\rangle $ -- упражнение.
    Если $x_1 = x_2 = ... = x_n$, то $c = x_i$.\\
    Если не все равны, то $c \in (A, B)$. \\
    Если $x_1 \leqslant x_2 \leqslant ... \leqslant x_n$ и не все равны, то $x_1 < c < x_n$ -- упражнение.
\end{Rem}

\Subsection{Дифференциалы}

$d_a f(h) : x \mapsto f'(a) \cdot h$ -- линейная часть приращения.

\begin{MyList}
    \item Композиции : $d_a (g \circ f) = d_{f(a)} g \circ d_a f$ 
    \begin{proof}
        $d_a (g \circ f) = (g \circ f)'(a) \cdot h = g'(f(a)) \cdot f'(a) h = d_{f(a)} g(d_a f(h)) = d_{f(a)} g \circ d_a f(h)$ 
    \end{proof}

    \item Обратная функция $d_{f(a)} f^{-1} = (d_a f)^{-1}$ -- упражнение.
    \item Старшие дифференциалы:
    \[d_a^n f(h) = d_a(d_a^{n - 1} f(h))\]
    \[d_a^2 f(h) = d_a (d_a f(h)) = d_a (f'(a) h) = f''(a) h^2\]
    \[d_a^n f(h) = f^{(n)}(a) h^n\]

    \item Формула Тейлора.
    \[f(x) = \sum_{k=0}^{n}  \frac{d_a^{k} f(x - a)}{k!} + o((x - a)^n)\]

    \item $d_a (f \cdot g) = (f'(a) \cdot g(a) + f(a) \cdot g'(a)) h = g(a) \cdot d_a f(h) + f(a) d_a g(h)$ 
\end{MyList}

$f$ непрерывна на $\langle A, B\rangle$ 
\[\forall x_0 \in \langle A, B\rangle \ \forall \varepsilon > 0 \ \exists \delta > 0 : \forall x \ |x - x_0| < \delta, x \in \langle A, B\rangle \SO |f(x) - f(x_0)| < \varepsilon\]    

\[\forall \varepsilon > 0 \ \exists \delta > 0 : \forall x_0 \in \langle A, B\rangle \ \forall x : |x - x_0| < \delta, x \in \langle A, B\rangle \SO |f(x) - f(x_0)| < \varepsilon\]

\begin{Def}
    $f : D \to \R$ называется равномерно непрерывной на $D$, если 
    \[\forall \varepsilon > 0 \ \exists \delta > 0 : \forall x_1, x_2 \in D : |x_1 - x_2| < \delta \to |f(x_1) - f(x_2)| < \varepsilon\]
\end{Def}

\begin{Thm}[Теорема Кантора]
    Непрерывная на отрезке функция равномерна непрерывна.
\end{Thm}

\begin{proof}
    $f \in C[a, b]$. Пусть $f$ не равномерна непрерывна. Тогда
    \[\exists \varepsilon > 0 : \forall \delta > 0 \ \exists x_1, x_2 \in [a, b] : |x_1 - x_2| < \delta \ |f(x_1) - f(x_2)| \geqslant \varepsilon\]
    Будем рассматривать $\delta = \frac{1}{n}, n \in \N$ 
    \[|\overline{x}_n - \overline{\overline{x}}_n| < \frac{1}{n}\]
    По принципу выбора Больцано-Вейерштрасса
    \[\exists \{\overline{x}_{n_k}\}_{k = 1}^\infty : \overline{x}_{n_k} \to c \in [a, b]\]
    Тогда $\overline{\overline{x}}_{n_k} \to c$. Тогда $f(\overline{x}_{n_k}) \to f(c)$, т.к. $f$ непрерывна, $f(\overline{\overline{x}}_{n_k}) \to f(c)$.
    \[|f(\overline{x}_{n_k}) - f(\overline{\overline{x}}_{n_k})| \geqslant \varepsilon*\]  
\end{proof}

\begin{Thm}
    $f$ дифференцируема на $\langle A, B\rangle$ и $\exists M > 0 \ |f'(x)| \leqslant M \ \forall x \in \langle A, B\rangle$.
    Тогда $f$ равномерное непрерывна. 
\end{Thm}

\begin{proof}
    Пусть $\varepsilon > 0, \delta = \frac{\varepsilon}{M}$. Если $x, y \in \langle A, B\rangle$ и $|x - y| < \delta$, то
    \[\exists c \in \langle A, B\rangle : f(x) - f(y) = f'(c) (x - y) \SO |f(x) - f(y)| \leqslant M \cdot |x - y| \leqslant M \cdot \delta = \varepsilon\]
\end{proof}

\begin{Def}
    $f : \langle A, B\rangle \to \R, F : \langle A, B\rangle \to \R$ называется первообразной функцией $f$, если $F$ дифференцируема на $\langle A, B\rangle, F'(x) = f(x) \ \forall x \in \langle A, B\rangle$.
\end{Def}

\begin{Thm}
    Пусть $f, F, G : \langle A, B\rangle \to \R, F$ -- первообразная $f$. Тогда
    $G$ -- первообразная $f \EQ \exists c \in \R : F(x) + c = G(x)$.
\end{Thm}

\begin{proof}
    $H(x) = F(x) - G(x) \EQ H'(x) = F'(x) - G'(x) = f(x) - f(x) = 0 \EQ H'(x) = 0 \SO H(x) \equiv \text{const}$ \\
    $(F(x) + c)' = (G(x))' \EQ f(x) = F'(x) = G'(x) \SO G$ -- первообразная.  
\end{proof}

\begin{Def}
    $f : \langle A, B\rangle \to \R, F$ -- первообразная $f$. Множество функций 
    $\{F(x) + c, c \in \R\}$ называется неопределенным интегралом $f$.
    \[\int f(x) dx = F(x) + c, c \in \R \] 
\end{Def}

\end{document}