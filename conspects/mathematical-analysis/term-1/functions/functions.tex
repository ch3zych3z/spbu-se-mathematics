\documentclass[12pt]{article}

% Автор стиля: Сергей Копелиович
% Автор конспекта: Илья Дудников

\usepackage{cmap}
\usepackage[T2A]{fontenc}
\usepackage[utf8]{inputenc}
\usepackage[russian]{babel}
\usepackage{graphicx}
\usepackage{amsthm,amsmath,amssymb}
\usepackage{listings}
\usepackage{color}
\usepackage{xcolor}
\usepackage{array}
\usepackage{mathrsfs}
\usepackage{epigraph}

\usepackage[russian,colorlinks=true,urlcolor=red,linkcolor=blue]{hyperref}
\usepackage{enumerate}
\usepackage{datetime}
\usepackage{fancyhdr}
\usepackage{lastpage}
\usepackage{verbatim}
\usepackage{tikz}
\usetikzlibrary{arrows,decorations.markings,decorations.pathmorphing}
\usepackage{pgfplots}

\usepackage{ifthen}
\usepackage{mathtools}

%\usepackage{tabls}
%\usepackage{tabularx}
%\usepackage{xifthen}
%\listfiles

\usepackage{import}
\usepackage{xifthen}
\usepackage{pdfpages}
\usepackage{transparent}

\def\NAME{Лекция, 10/13/21}

\sloppy
\voffset=-20mm
\textheight=235mm
\hoffset=-22mm
\textwidth=180mm
\headsep=12pt
\footskip=20pt

\parskip=0em
\parindent=0em

\setlength\epigraphwidth{.8\textwidth}

\newlength{\tmplen}
\newlength{\tmpwidth}
\newcounter{listcounter}

% Список с маленькими отступами
\newenvironment{MyList}[1][4pt]{
  \begin{enumerate}[1.]
  \setlength{\parskip}{0pt}
  \setlength{\itemsep}{#1}
}{       
  \end{enumerate}
}
% Вложенный список с маленькими отступами
\newenvironment{InnerMyList}[1][0pt]{
  \vspace*{-0.5em}
  \begin{enumerate}[(a)]
  \setlength{\parskip}{-0pt}
  \setlength{\itemsep}{#1}
}{       
  \end{enumerate}
  \vspace*{-0.5em}
}
% Список с маленькими отступами
\newenvironment{MyItemize}[1][4pt]{
  \begin{itemize}
  \setlength{\parskip}{0pt}
  \setlength{\itemsep}{#1}
}{       
  \end{itemize}
}

% Основные математические символы
\def\TODO{{\color{red}\bf TODO}}
\def\N{\mathbb{N}}       %
\def\R{\mathbb{R}}       %
\def\F2{\mathbb{F}_2}    %
\def\Z{\mathbb{Z}}       %
\def\INF{\t{+}\infty}    % +inf
\def\EPS{\varepsilon}    %
\def\EMPTY{\varnothing}  %
\def\PHI{\varphi}        %
\def\SO{\Rightarrow}     % =>
\def\EQ{\Leftrightarrow} % <=>
\def\t{\texttt}          % mono font
\def\c#1{{\rm\sc{#1}}}   % font for classes NP, SAT, etc
\def\O{\mathcal{O}}      %
\def\NO{\t{\#}}          % #
\def\XOR{\text{ {\raisebox{-2pt}{\ensuremath{\Hat{}}}} }}
\renewcommand{\le}{\leqslant}
\renewcommand{\ge}{\geqslant}
\newcommand{\q}[1]{\langle #1 \rangle}               % <x>
\newcommand\URL[1]{{\footnotesize{\url{#1}}}}        %
% \newcommand{\sfrac}[2]{{\scriptscriptstyle\frac{#1}{#2}}}  % Очень маленькая дробь
% \newcommand{\mfrac}[2]{{\scriptstyle\frac{#1}{#2}}}    % Небольшая дробь
\newcommand{\sfrac}[2]{{\scriptstyle\frac{#1}{#2}}}  % Очень маленькая дробь
\newcommand{\mfrac}[2]{{\textstyle\frac{#1}{#2}}}    % Небольшая дробь

\newcommand{\fix}[1]{{\color{fixcolor}{#1}}} % \underline
\def\bonus{\t{\red{(*)}}}
\def\ifbonus#1{\ifthenelse{\equal{#1}{}}{}{\bonus}}
\def\smallsquare{$\scalebox{0.5}{$\square$}$}

\newlength{\myItemLength}
\setlength{\myItemLength}{0.3em}
\def\ItemSymbol{\smallsquare}
\def\Item{\vspace*{\myItemLength}\ItemSymbol \ \ }

\newcommand{\LET}{%
  % [line width=0.6pt]
  \begin{tikzpicture}%
  \draw(0.8ex,0) -- (0.8ex,1.6ex);%
  \draw(0,1.6ex) -- (0.8ex,1.6ex);%
  \end{tikzpicture}%
  \hspace*{0.1em}%
}

% Отступы
\def\makeparindent{\hspace*{\parindent}\unskip}
\def\up{\vspace*{-0.5em}}%{\vspace*{-\baselineskip}}
\def\down{\vspace*{0.5em}}
\def\LINE{\vspace*{-1em}\noindent \underline{\hbox to 1\textwidth{{ } \hfil{ } \hfil{ } }}}
\def\BOX#1{\mbox{\fbox{\bf{#1}}}}
\def\Pagebreak{\pagebreak\vspace*{-1.5em}}

% Мелкий заголовок
\newcommand{\THEE}[1]{
  \vspace*{0.5em}
  \noindent{\bf \underline{#1}}%\hspace{0.5em}
  \vspace*{0.2em}
}
% Другой тип мелкого заголовка
\newcommand{\THE}[1]{
  \vspace*{0.5em} $\bullet$
  \noindent{\bf #1}%\hspace{0.5em}
  \vspace*{0.2em}
}

\newenvironment{MyTabbing}{
  \t\bgroup
  \vspace*{-\baselineskip}
  \begin{tabbing}
    aaaa\=aaaa\=aaaa\=aaaa\=aaaa\=aaaa\kill
}{
  \end{tabbing}
  \t\egroup
}

% Код с правильными отступами
\lstnewenvironment{code}{
  \lstset{}
%  \vspace*{-0.2em}
}%
{
%  \vspace*{-0.2em}
}
\lstnewenvironment{codep}{
  \lstset{language=python}
}%
{
}

% Формулы с правильными отступами
\newenvironment{smallformula}{
 
  \vspace*{-0.8em}
}{
  \vspace*{-1.2em}
  
}
\newenvironment{formula}{
 
  \vspace*{-0.4em}
}{
  \vspace*{-0.6em}
  
}

% Большая квадратная скобка
\makeatletter
\newenvironment{sqcases}{%
  \matrix@check\sqcases\env@sqcases
}{%
  \endarray\right.%
}
\def\env@sqcases{%
  \let\@ifnextchar\new@ifnextchar
  \left\lbrack
  \def\arraystretch{1.2}%
  \array{@{}l@{\quad}l@{}}%
}
\makeatother

\DeclareMathOperator{\sort}{sort}

% Определяем основные секции: \begin{Lm}, \begin{Thm}, \begin{Def}, \begin{Rem}
\renewcommand{\qedsymbol}{$\blacksquare$}
\theoremstyle{definition} % жирный заголовок, плоский текст
\newtheorem{Thm}{\underline{Теорема}}[subsection] % нумерация будет "<номер subsection>.<номер теоремы>"
\newtheorem{Lm}[Thm]{\underline{Lm}} % Нумерация такая же, как и у теорем
\newtheorem{Ex}[Thm]{Упражнение} % Нумерация такая же, как и у теорем
\newtheorem{Example}[Thm]{Пример} % Нумерация такая же, как и у теорем
\newtheorem{Solution}[Thm]{Решение}
\newtheorem{Code}[Thm]{Код} % Нумерация такая же, как и у теорем
\theoremstyle{plain} % жирный заголовок, курсивный текст
\newtheorem{Def}[Thm]{Def.} % Нумерация такая же, как и у теорем
\theoremstyle{remark} % курсивный заголовок, плоский текст
\newtheorem{Cons}[Thm]{Следствие} % Нумерация такая же, как и у теорем
\newtheorem{Conj}[Thm]{Гипотеза} % Нумерация такая же, как и у теорем
\newtheorem{Prop}[Thm]{Утверждение} % Нумерация такая же, как и у теорем
\newtheorem{Rem}[Thm]{Замечание} % Нумерация такая же, как и у теорем
\newtheorem{Remark}[Thm]{Замечание} % Нумерация такая же, как и у теорем
\newtheorem{Algo}[Thm]{Алгоритм} % Нумерация такая же, как и у теорем

% Определяем ЗАГОЛОВКИ
\def\SectionName{Матанализ}
\def\AuthorName{Илья Дудников}

\newlength{\sectionvskip}
\setlength{\sectionvskip}{0.5em}
\newcommand{\Section}[4][]{
  % Заголовок
  \pagebreak
%  \ifthenelse{\isempty{#1}}{
    \refstepcounter{section}
%  }{}
  \vspace{0.5em}
%  \ifthenelse{\isempty{#1}}{
%    \addtocontents{toc}{\protect\addvspace{-5pt}}%
    \addcontentsline{toc}{section}{\arabic{section}. #2}
%  }{}
  


  % Запомнили название и автора главы
  \gdef\SectionName{#2}
  \gdef\AuthorName{#4}

  % Заголовок страницы
  \lhead{Конспект, \NAME}
  \chead{}
  \rhead{\SectionName}
  \renewcommand{\headrulewidth}{0.4pt}
    
  \lfoot{}
  \cfoot{\thepage\t{/}\pageref*{LastPage}}
  \rfoot{Автор: \AuthorName}
  \renewcommand{\footrulewidth}{0.4pt}
}

\newcommand{\Subsection}[2][]{
  \refstepcounter{subsection}
  \vspace*{1em}
  \ifthenelse{\equal{#1}{}}
    {\addcontentsline{toc}{subsection}{\arabic{section}.\arabic{subsection}. #2}}
    {\addcontentsline{toc}{subsection}{\arabic{section}.\arabic{subsection}. \bonus\,#2}}
  {\color{blue}\bf\large \arabic{section}.\arabic{subsection}. \ifbonus{#1}\,{#2}} 
  \vspace*{0.5em}
  \makeparindent
}
\newcommand{\Subsubsection}[2][]{
  \refstepcounter{subsubsection}
  \vspace*{1em}
  \ifthenelse{\equal{#1}{}}
    {\addcontentsline{toc}{subsubsection}{\arabic{section}.\arabic{subsection}.\arabic{subsubsection}. #2}}
    {\addcontentsline{toc}{subsubsection}{\arabic{section}.\arabic{subsection}.\arabic{subsubsection}. \bonus\,#2}}
  {\color{blue}\bf\large \arabic{section}.\arabic{subsection}.\arabic{subsubsection}. \ifbonus{#1}\,#2}
  \vspace*{0.5em}
  \makeparindent
}

\newcommand{\Header}{
  \pagestyle{empty}
  \renewcommand{\dateseparator}{--}
  \begin{center}
    {\Large\bf 
    Матанализ 1 семестр ПИ,\\
    \vspace{0.3em}
     \NAME}\\
    \vspace{0.7em}
    {Собрано {\today} в {\currenttime}}
  \end{center}

  \LINE
  \vspace{0em}

  \renewcommand{\baselinestretch}{0.98}\normalsize
  \tableofcontents
  \renewcommand{\baselinestretch}{1.0}\normalsize
  \pagebreak
}

\newcommand{\BeginConspect}{
  \pagestyle{fancy}
  \setcounter{page}{1}
}

\DeclareMathOperator{\sign}{sign}

\definecolor{mygray}{rgb}{0.7,0.7,0.7}
\definecolor{ltgray}{rgb}{0.9,0.9,0.9}
\definecolor{fixcolor}{rgb}{0.7,0,0}
\definecolor{red2}{rgb}{0.7,0,0}
\definecolor{dkred}{rgb}{0.4,0,0}
\definecolor{dkblue}{rgb}{0,0,0.6}
\definecolor{dkgreen}{rgb}{0,0.6,0}
\definecolor{brown}{rgb}{0.5,0.5,0}

\newcommand{\green}[1]{{\color{green}{#1}}}
\newcommand{\black}[1]{{\color{black}{#1}}}
\newcommand{\red}[1]{{\color{red}{#1}}}
\newcommand{\dkred}[1]{{\color{dkred}{#1}}}
\newcommand{\blue}[1]{{\color{blue}{#1}}}
\newcommand{\dkgreen}[1]{{\color{dkgreen}{#1}}}

\newcommand{\Mod}[1]{\ (\mathrm{mod}\ #1)}

\begin{document}

\Header

\BeginConspect

\Section{Функции}{}{Илья Дудников}
\Subsection{Свойства пределов функций}

\begin{Thm}[Единственность предела функции]
    Пусть $D \subset \R, a$ -- предельная точка $D$, $f: D \to R$. Если $A$ и $B \in \overline{\R}$ и $f(x) \xrightarrow[x \to a]{} A, f(x) \xrightarrow[x \to a]{} B \SO A = B$   
\end{Thm}

\begin{proof}
    Возьмем $\{x_n\} : x_n \in D, x_n \neq a, x_n \to a$. По Гейне $f(x_n) \to A \wedge f(x_n) \to B$. Но $\{x_n\}$ имеет единственный предел $\SO A = B$.    
\end{proof}

\begin{Rem}
    Беззнаковая бесконечность: $A = +\infty, B = -\infty \SO f(x) \xrightarrow[x \to a]{} \infty$ 
\end{Rem}

\begin{Thm}[Локальная ограниченность функции, имеющей предел]
    $D \subset \R, a$ -- предельная точка $D$, $f: D \to \R, A \in \R, f(x) \xrightarrow[x \to a]{} A$. Тогда $\exists V(a) : f(x)$ ограничена в $D \cap V(a)$  
\end{Thm}

\begin{proof}
    Пусть $\varepsilon = 1$. $\exists \dot{V}(a) : |f(x) - A| < 1 \ \forall x \in \dot{V}(a) \cap D$. Тогда $|f(x)| < |A| + 1$.

    Если $a \in D$, то $|f(x)| < \max \{|A| + 1, f(a)\}$  
\end{proof}

\begin{Thm}[Стабилизация знака функции, имеющей предел]
    $D \subset \R, a$ -- предельная точка~$D, f: D \to \R$. Пусть $\lim_{x \to a} f(x) = B \in \overline{\R} \setminus \{0\}$.
    Тогда $\exists V(a)$ такая, что знаки $f(x)$ и $B$ совпадают на $\dot{V}(a) \cap D$     
\end{Thm}

\begin{proof}
    Пусть $B > 0$. Докажем от противного, т.е.
    \[\forall n \ \exists x_n \in \dot{V}_{\frac{1}{n}} (a) \cap D \wedge f(x_n) \leqslant 0\]
    Тогда $x_n \to a, x_n \neq a \SO f(x_n) \to B$, но $f(x_n) \leqslant 0 \SO B \leqslant 0$.  
\end{proof}

\begin{Thm}[Арифметические действия над функциями, имеющими предел]
    $D \subset \R$, \\ $a$ -- предельная точка $D$, $f, g : D \to \R, f \xrightarrow[x \to a]{} A, g \xrightarrow[x \to a]{} B$. Тогда 
    \begin{MyList}
        \item $f(x) + g(x) \to A + B$ 
        \item $f(x) \cdot g(x) \to A \cdot B$ 
        \item $f(x) - g(x) \to A - B$ 
        \item $|f(x)| \to |A|$ 
        \item Если $B \neq 0$, то $ \frac{f(x)}{g(x)} \to \frac{A}{B}$ 
    \end{MyList}
\end{Thm}

\begin{proof}
    Рассмотрим $\{x_n\} : x_n \to a, x_n \neq a, x_n \in D$. Тогда $f(x_n) \to A, g(x_n) \to B$. Достаточно применить теорему об арифметических действиях с пределами последовательностей.  
\end{proof}

\begin{Rem}
    Пункт 5) т.к. $B \neq 0$, то $\exists V(a) : \sign (g(x)) = \sign B$ в $V(a)$. Поэтому излишне требовать $g(x) \neq 0$   
\end{Rem}

\Pagebreak
\begin{Thm}[Предел композиции функций]
    $f : D \to \R, g : E \to \R, f(D) \subset E$
    \begin{MyList}
        \item $f(x) \xrightarrow[x \to a]{} A \in \overline{\R}$
        \item $A$ -- предельная точка множества $E$ и $g(x) \xrightarrow[x \to A]{} B \in \overline{R}$
        \item $\exists V(a) : f(x) \neq A \ \forall x \in \dot{V}(a) \cap D$   
    \end{MyList} 
    Тогда $(g \circ f)(x) \xrightarrow[x \to a]{} B$ 
\end{Thm}

\begin{proof}
    Возьмем $\{x_n\} : x_n \in D, x_n \to a, x_n \neq a$. \\ Обозначим $y_n = f(x_n) \SO y_n \in E, y_n \to A$.
    По 3) начиная с некоторого номера $x_n \in V(a)$, а значит $y_n \neq A$. Тогда $g(y_n) \to B$, т.е. $g(f(x_n)) \xrightarrow[n \to \infty]{} B$.
    Значит $(g \circ f)(x) \xrightarrow[x \to a]{} B$ 
\end{proof}

\begin{Thm}[Предельный переход в неравенстве]
    $D \subset \R, a $ -- предельная точка $D$. $f,~g~:~D~\to~\R$.
    \[f(x) \xrightarrow[x \to a]{}A \in \overline{\R}, g(x) \xrightarrow[x \to a]{} B \in \overline{\R}, f(x) \leqslant g(x) \ \forall x \in D \setminus \{a\}\] 
    Тогда $A \leqslant B$ 
\end{Thm}

\begin{proof}
    \[\{x_n\} : x_n \in D, x_n \to a, x_n \neq a \SO f(x_n) \to A, g(x_n) \to B = A \leqslant B\]
\end{proof}

\begin{Thm}[о сжатой функции]
    $D \subset \R, a$ -- предельная точка $D$, $f, h, g : D \to \R$ и
    \[f(x) \leqslant g(x) \leqslant h(x), \forall x \in D \setminus \{a\} \ f(x) \xrightarrow[x \to a]{} A, h(x) \xrightarrow[x \to a]{} A, A \in \R\]
    Тогда $g(x) \xrightarrow[x \to a]{} A$  
\end{Thm}

\begin{proof}
    $\{x_n\} : x_n \in D, x_n \to a, x_n \neq a \SO f(x_n) \to A, h(x_n) \to A$
    \[f(x_n) \leqslant g(x_n) \leqslant h(x_n) \SO A \leqslant \lim_{n \to \infty} g(x_n) \leqslant A \SO \exists \lim_{n \to \infty} g(x_n) = A \SO g(x) \to A\] 
\end{proof}

\begin{Rem}
    $f(x) \leqslant g(x) \ \forall x \in D \setminus \{a\}, f(x) \xrightarrow[x \to a]{} +\infty \SO g(x) \xrightarrow[x \to a]{} +\infty$ 
\end{Rem}

\begin{Def}
    $f : D \to \R, a$ -- предельная точка $D_1 \subset D$. Тогда $\lim_{x \to a} f|_{D_1} (x)$ -- предел $f$ в точке $a$ по множеству $D_1$. 
\end{Def}

\begin{Def}
    $f: D \to \R, D_1 = D \cap (-\infty, a), a$ -- предельная точка $D_1$. Предел $f$ в точке $a$ по множеству $D_1$ называется левосторонним пределом в точке $a$.

    Обозначение: 
    \[\lim_{x \to a-} f(x), \lim_{x \to a - 0} f(x)\]
\end{Def}

\begin{Def}
    $f: D \to \R, D_1 = D \cap (a, +\infty), a$ -- предельная точка $D_1$. Правосторонний предел -- предел $f$ в точке $a$ по множеству $D_1$
    
    Обозначение:
    \[\lim_{x \to a+} f(x), \lim_{x \to a + 0} f(x)\]
\end{Def}

\begin{Def}
    Левосторонний предел на разных "языках".
    \begin{MyItemize}
        \item $\forall \varepsilon \ \exists \delta > 0 : \forall x \in D, 0 < a - x < \delta \to |f(x) - A| < \varepsilon$
        \item $\forall V(A) \ \exists \delta > 0 : \forall x \in D, 0 < a - x < \delta \to f(x) \in V(A)$
        \item $\forall \{x_n\} : x_n \in D, x_n \to a, x_n < a \ f(x_n) \to A$  
    \end{MyItemize}
\end{Def}

\begin{Rem}
    $f: D \to \R, a \in \R$ -- предельная точка для $D_1 = D \cap (-\infty, a), D_2 = D \cap (a, +\infty)$
    Тогда 
    \[\exists \lim_{x \to a} f(x) \EQ \exists \lim_{x \to a-} f(x), \exists \lim_{x \to a+} f(x) \wedge \lim_{x \to a-} f(x) = \lim_{x \to a+} f(x)\]  
\end{Rem}

\begin{proof}
    "$\SO$". Очевидно. \\
    "$\Leftarrow$". Возьмем $\delta_1$ из определения левостороннего предела, $\delta_2$ из определения правостороннего предела.
    $\delta = \min \{\delta_1, \delta_2\}$. Тогда 
    \[\forall \varepsilon > 0 \ \exists \delta : \forall x \in D : |x - a| < \delta \to |f(x) - A| < \varepsilon\]  
\end{proof}

\begin{Thm}[Предел монотонной функции]
    $D \in \R, f : D \to \R, a \in (-\infty, +\infty]$ \\
    $D_1 = D \cap (-\infty, a), a$ -- предельная точка $D$.
    \begin{MyList}
        \item Если $f$ возрастает и ограничена сверху на $D_1$, то $\exists \lim_{x \to a-} f(x) \in \R$
        \item Если $f$ убывает и ограничена снизу на $D_1$, то $\exists \lim_{x \to a-} f(x) \in \R$  
    \end{MyList} 
\end{Thm}

\begin{proof}
    \begin{MyList}
        \item Пусть $A = \sup_{x \in D_1} f(x)$. Тогда $A \in \R$, т.к. $f$ ограничена сверху. Докажем, что $\lim_{x \to a-} f(x) = A$
        \[\forall \varepsilon > 0 \ \exists x_0 \in D_1 : f(x_0) > A - \varepsilon\]
        Тогда $\forall x \in D_1 : x > x_0$
        \[A - \varepsilon < f(x_0) \leqslant f(x) \leqslant A < A + \varepsilon\]
        Пусть $\delta = a - x_0$. Тогда $|f(x) - A| < \varepsilon \ \forall x : 0 < a - x < \delta$
        
        Если $a = +\infty \SO \Delta = \max \{x_0, 1\}$ 
    \end{MyList}
\end{proof}

\begin{Rem}
    $f$ возрастает и не ограничена сверху $\SO \lim_{x \to a-} f(x) = +\infty$ 
\end{Rem}

\begin{Thm}[Критерий Больцано-Коши для функций]
    $D \subset \R$. Тогда существование конечного $\lim_{x \to a} f(x)$ равносильно утверждению:
    
    \[\forall \varepsilon > 0 \ \exists V(a) : \forall x_1, x_2 \in \dot{V}(a) \cap D \to |f(x_1) - f(x_2)| < \varepsilon\]
\end{Thm}

\begin{proof}
    "$\SO$". $\exists \lim_{x \to a} f(x) = A \in \R$. Возьмем $\varepsilon > 0$
    Тогда $\exists V(a) : |f(x) - A| < \frac{\varepsilon}{2}$. Если $x_1, x_2 \in D \cap \dot{V}(a)$, то
    \[|f(x_1) - A| + |f(x_2) - A| < \varepsilon\]
    С другой стороный $|f(x_1) - f(x_2)| < |f(x_1) - A| + |f(x_2) - A| < \varepsilon$ \\
    "$\Leftarrow$". $\{x_n\} : x_n \in D, x_n \neq a, x_n \to a$ и докажем, что $\exists \lim f(x_n) \in \R$. \\
    Пусть $\varepsilon > 0$.
    \[\exists N : \forall n \geqslant N \to x_n \in \dot{V}(a)\]
    \[\forall n, l \geqslant N \to |f(x_n) = f(x_l)| < \varepsilon \SO \{f(x_n)\} \text { -- фундаментальна}\]
    Значит $\{f(x_n)\}$ сходится. 
\end{proof}

\Subsection{Непрерывные функции}

\begin{Def}
    $D \subset \R, a \in D$. Функция $f$ называется непрерывной в точке $a$, если выполнено одно из следующих условий:

    \begin{MyList}
        \item Предел $f$ в точке $a$ существует и равен $f(a)$ (только если $a$ -- предельная точка).
        \item $\forall \varepsilon > 0 \ \exists \delta > 0 : \forall x \in D : |x - a| < \delta \to |f(x) - f(a)| < \varepsilon$ 
        \item $\forall V (f(a)) \ \exists V(a) : f(V(a) \cap D) \subset V(f(a))$
        \item $\forall \{x_n\} : x_n \to a, x_n \in D \ f(x_n) \to f(a)$
        \item Бесконечно малому приращению аргумента соответствует бесконечно малое приращение функции (если $a$ -- предельная точка)
        \[\Delta x = x - a, \Delta f = f(x) - f(a) \SO \Delta f \xrightarrow[\Delta x \to 0]{} 0\]
    \end{MyList}
\end{Def}

\begin{Rem}
    Если $a$ -- изолированная точка $D$, то 
    \[f(V(a) \cap D) = \{f(a)\} \subset V(f(a))\]
    Т.е. любая $f$ непрерывна в точке $a$ 
\end{Rem}

\begin{Def}
    $D \subset \R, a \in D, f : D \to \R$. \\
    $a$ называется точкой разрыва $f$, если $f$ не непрерывна в точке $a$  
\end{Def}

\begin{Def}
    $D_1 = D \cap (-\infty, a], D_2 = D \cap [a, +\infty)$. \\
    Если сужение $f|_{D_1}$ непрерывно в точке $a$, то $f$ непрерывна в точке $a$ \textbf{слева}. \\
    Если сужение $f|_{D_2}$ непрерывно в точке $a$, то $f$ непрерывна в точке $a$ \textbf{справа} 
\end{Def}

\begin{Def}
    Если $\exists \lim_{x \to a+} f(x), \lim_{x \to a-} f(x), f(a)$ -- конечные, но не все равны, то $a$ -- точка разрыва I рода. \\  
\end{Def}

\begin{Def}
    Если хотя бы один предел не существует или бесконечен -- II рода.
\end{Def}

\begin{Def}
    Если в точке $a$ разрыв, но мы можем доопределить или переопределить $f$ в точке $a$ до непрерывности, то $a$ -- точка устранимого разрыва.
\end{Def}

\end{document}