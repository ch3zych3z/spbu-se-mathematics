\documentclass[12pt]{article}

% Автор стиля: Сергей Копелиович
% Автор конспекта: Илья Дудников

\usepackage{cmap}
\usepackage[T2A]{fontenc}
\usepackage[utf8]{inputenc}
\usepackage[russian]{babel}
\usepackage{graphicx}
\usepackage{amsthm,amsmath,amssymb}
\usepackage{listings}
\usepackage{color}
\usepackage{xcolor}
\usepackage{array}
\usepackage{mathrsfs}
\usepackage{epigraph}

\usepackage[russian,colorlinks=true,urlcolor=red,linkcolor=blue]{hyperref}
\usepackage{enumerate}
\usepackage{datetime}
\usepackage{fancyhdr}
\usepackage{lastpage}
\usepackage{verbatim}
\usepackage{tikz}
\usetikzlibrary{arrows,decorations.markings,decorations.pathmorphing}
\usepackage{pgfplots}

\usepackage{ifthen}
\usepackage{mathtools}

%\usepackage{tabls}
%\usepackage{tabularx}
%\usepackage{xifthen}
%\listfiles

\usepackage{import}
\usepackage{MnSymbol}
\usepackage{xifthen}
\usepackage{pdfpages}
\usepackage{transparent}

\def\NAME{Лекция, 11/03/21}

\sloppy
\voffset=-20mm
\textheight=235mm
\hoffset=-22mm
\textwidth=180mm
\headsep=12pt
\footskip=20pt

\parskip=0em
\parindent=0em

\setlength\epigraphwidth{.8\textwidth}

\newlength{\tmplen}
\newlength{\tmpwidth}
\newcounter{listcounter}

% Список с маленькими отступами
\newenvironment{MyList}[1][4pt]{
  \begin{enumerate}[1.]
  \setlength{\parskip}{0pt}
  \setlength{\itemsep}{#1}
}{       
  \end{enumerate}
}
% Вложенный список с маленькими отступами
\newenvironment{InnerMyList}[1][0pt]{
  \vspace*{-0.5em}
  \begin{enumerate}[(a)]
  \setlength{\parskip}{-0pt}
  \setlength{\itemsep}{#1}
}{       
  \end{enumerate}
  \vspace*{-0.5em}
}
% Список с маленькими отступами
\newenvironment{MyItemize}[1][4pt]{
  \begin{itemize}
  \setlength{\parskip}{0pt}
  \setlength{\itemsep}{#1}
}{       
  \end{itemize}
}

% Основные математические символы
\def\TODO{{\color{red}\bf TODO}}
\def\Q{\mathbb{Q}}
\def\N{\mathbb{N}}       %
\def\R{\mathbb{R}}       %
\def\F2{\mathbb{F}_2}    %
\def\Z{\mathbb{Z}}       %
\def\INF{\t{+}\infty}    % +inf
\def\EPS{\varepsilon}    %
\def\EMPTY{\varnothing}  %
\def\PHI{\varphi}        %
\def\SO{\Rightarrow}     % =>
\def\EQ{\Leftrightarrow} % <=>
\def\t{\texttt}          % mono font
\def\c#1{{\rm\sc{#1}}}   % font for classes NP, SAT, etc
\def\O{\mathcal{O}}      %
\def\NO{\t{\#}}          % #
\def\XOR{\text{ {\raisebox{-2pt}{\ensuremath{\Hat{}}}} }}
\renewcommand{\le}{\leqslant}
\renewcommand{\ge}{\geqslant}
\newcommand{\q}[1]{\langle #1 \rangle}               % <x>
\newcommand\URL[1]{{\footnotesize{\url{#1}}}}        %
% \newcommand{\sfrac}[2]{{\scriptscriptstyle\frac{#1}{#2}}}  % Очень маленькая дробь
% \newcommand{\mfrac}[2]{{\scriptstyle\frac{#1}{#2}}}    % Небольшая дробь
\newcommand{\sfrac}[2]{{\scriptstyle\frac{#1}{#2}}}  % Очень маленькая дробь
\newcommand{\mfrac}[2]{{\textstyle\frac{#1}{#2}}}    % Небольшая дробь

\newcommand{\fix}[1]{{\color{fixcolor}{#1}}} % \underline
\def\bonus{\t{\red{(*)}}}
\def\ifbonus#1{\ifthenelse{\equal{#1}{}}{}{\bonus}}
\def\smallsquare{$\scalebox{0.5}{$\square$}$}

\newlength{\myItemLength}
\setlength{\myItemLength}{0.3em}
\def\ItemSymbol{\smallsquare}
\def\Item{\vspace*{\myItemLength}\ItemSymbol \ \ }

\newcommand{\LET}{%
  % [line width=0.6pt]
  \begin{tikzpicture}%
  \draw(0.8ex,0) -- (0.8ex,1.6ex);%
  \draw(0,1.6ex) -- (0.8ex,1.6ex);%
  \end{tikzpicture}%
  \hspace*{0.1em}%
}

% Отступы
\def\makeparindent{\hspace*{\parindent}\unskip}
\def\up{\vspace*{-0.5em}}%{\vspace*{-\baselineskip}}
\def\down{\vspace*{0.5em}}
\def\LINE{\vspace*{-1em}\noindent \underline{\hbox to 1\textwidth{{ } \hfil{ } \hfil{ } }}}
\def\BOX#1{\mbox{\fbox{\bf{#1}}}}
\def\Pagebreak{\pagebreak\vspace*{-1.5em}}

% Мелкий заголовок
\newcommand{\THEE}[1]{
  \vspace*{0.5em}
  \noindent{\bf \underline{#1}}%\hspace{0.5em}
  \vspace*{0.2em}
}
% Другой тип мелкого заголовка
\newcommand{\THE}[1]{
  \vspace*{0.5em} $\bullet$
  \noindent{\bf #1}%\hspace{0.5em}
  \vspace*{0.2em}
}

\newenvironment{MyTabbing}{
  \t\bgroup
  \vspace*{-\baselineskip}
  \begin{tabbing}
    aaaa\=aaaa\=aaaa\=aaaa\=aaaa\=aaaa\kill
}{
  \end{tabbing}
  \t\egroup
}

% Код с правильными отступами
\lstnewenvironment{code}{
  \lstset{}
%  \vspace*{-0.2em}
}%
{
%  \vspace*{-0.2em}
}
\lstnewenvironment{codep}{
  \lstset{language=python}
}%
{
}

% Формулы с правильными отступами
\newenvironment{smallformula}{
 
  \vspace*{-0.8em}
}{
  \vspace*{-1.2em}
  
}
\newenvironment{formula}{
 
  \vspace*{-0.4em}
}{
  \vspace*{-0.6em}
  
}

% Большая квадратная скобка
\makeatletter
\newenvironment{sqcases}{%
  \matrix@check\sqcases\env@sqcases
}{%
  \endarray\right.%
}
\def\env@sqcases{%
  \let\@ifnextchar\new@ifnextchar
  \left\lbrack
  \def\arraystretch{1.2}%
  \array{@{}l@{\quad}l@{}}%
}
\makeatother

\DeclareMathOperator{\sort}{sort}

% Определяем основные секции: \begin{Lm}, \begin{Thm}, \begin{Def}, \begin{Rem}
\renewcommand{\qedsymbol}{$\blacksquare$}
\theoremstyle{definition} % жирный заголовок, плоский текст
\newtheorem{Thm}{\underline{Теорема}}[subsection] % нумерация будет "<номер subsection>.<номер теоремы>"
\newtheorem{Lm}[Thm]{\underline{Lm}} % Нумерация такая же, как и у теорем
\newtheorem{Ex}[Thm]{Упражнение} % Нумерация такая же, как и у теорем
\newtheorem{Example}[Thm]{Пример} % Нумерация такая же, как и у теорем
\newtheorem{Solution}[Thm]{Решение}
\newtheorem{Code}[Thm]{Код} % Нумерация такая же, как и у теорем
\theoremstyle{plain} % жирный заголовок, курсивный текст
\newtheorem{Def}[Thm]{Def.} % Нумерация такая же, как и у теорем
\theoremstyle{remark} % курсивный заголовок, плоский текст
\newtheorem{Cons}[Thm]{Следствие} % Нумерация такая же, как и у теорем
\newtheorem{Conj}[Thm]{Гипотеза} % Нумерация такая же, как и у теорем
\newtheorem{Prop}[Thm]{Утверждение} % Нумерация такая же, как и у теорем
\newtheorem{Rem}[Thm]{Замечание} % Нумерация такая же, как и у теорем
\newtheorem{Remark}[Thm]{Замечание} % Нумерация такая же, как и у теорем
\newtheorem{Algo}[Thm]{Алгоритм} % Нумерация такая же, как и у теорем

% Определяем ЗАГОЛОВКИ
\def\SectionName{Матанализ}
\def\AuthorName{Илья Дудников}

\newlength{\sectionvskip}
\setlength{\sectionvskip}{0.5em}
\newcommand{\Section}[4][]{
  % Заголовок
  \pagebreak
%  \ifthenelse{\isempty{#1}}{
    \refstepcounter{section}
%  }{}
  \vspace{0.5em}
%  \ifthenelse{\isempty{#1}}{
%    \addtocontents{toc}{\protect\addvspace{-5pt}}%
    \addcontentsline{toc}{section}{\arabic{section}. #2}
%  }{}
  


  % Запомнили название и автора главы
  \gdef\SectionName{#2}
  \gdef\AuthorName{#4}

  % Заголовок страницы
  \lhead{Конспект, \NAME}
  \chead{}
  \rhead{\SectionName}
  \renewcommand{\headrulewidth}{0.4pt}
    
  \lfoot{}
  \cfoot{\thepage\t{/}\pageref*{LastPage}}
  \rfoot{Автор: \AuthorName}
  \renewcommand{\footrulewidth}{0.4pt}
}

\newcommand{\Subsection}[2][]{
  \refstepcounter{subsection}
  \vspace*{1em}
  \ifthenelse{\equal{#1}{}}
    {\addcontentsline{toc}{subsection}{\arabic{section}.\arabic{subsection}. #2}}
    {\addcontentsline{toc}{subsection}{\arabic{section}.\arabic{subsection}. \bonus\,#2}}
  {\color{blue}\bf\large \arabic{section}.\arabic{subsection}. \ifbonus{#1}\,{#2}} 
  \vspace*{0.5em}
  \makeparindent
}
\newcommand{\Subsubsection}[2][]{
  \refstepcounter{subsubsection}
  \vspace*{1em}
  \ifthenelse{\equal{#1}{}}
    {\addcontentsline{toc}{subsubsection}{\arabic{section}.\arabic{subsection}.\arabic{subsubsection}. #2}}
    {\addcontentsline{toc}{subsubsection}{\arabic{section}.\arabic{subsection}.\arabic{subsubsection}. \bonus\,#2}}
  {\color{blue}\bf\large \arabic{section}.\arabic{subsection}.\arabic{subsubsection}. \ifbonus{#1}\,#2}
  \vspace*{0.5em}
  \makeparindent
}

\newcommand{\Header}{
  \pagestyle{empty}
  \renewcommand{\dateseparator}{--}
  \begin{center}
    {\Large\bf 
    Матанализ 1 семестр ПИ,\\
    \vspace{0.3em}
     \NAME}\\
    \vspace{0.7em}
    {Собрано {\today} в {\currenttime}}
  \end{center}

  \LINE
  \vspace{0em}

  \renewcommand{\baselinestretch}{0.98}\normalsize
  \tableofcontents
  \renewcommand{\baselinestretch}{1.0}\normalsize
  \pagebreak
}

\newcommand{\BeginConspect}{
  \pagestyle{fancy}
  \setcounter{page}{1}
}

\definecolor{mygray}{rgb}{0.7,0.7,0.7}
\definecolor{ltgray}{rgb}{0.9,0.9,0.9}
\definecolor{fixcolor}{rgb}{0.7,0,0}
\definecolor{red2}{rgb}{0.7,0,0}
\definecolor{dkred}{rgb}{0.4,0,0}
\definecolor{dkblue}{rgb}{0,0,0.6}
\definecolor{dkgreen}{rgb}{0,0.6,0}
\definecolor{brown}{rgb}{0.5,0.5,0}

\newcommand{\green}[1]{{\color{green}{#1}}}
\newcommand{\black}[1]{{\color{black}{#1}}}
\newcommand{\red}[1]{{\color{red}{#1}}}
\newcommand{\dkred}[1]{{\color{dkred}{#1}}}
\newcommand{\blue}[1]{{\color{blue}{#1}}}
\newcommand{\dkgreen}[1]{{\color{dkgreen}{#1}}}

\DeclareMathOperator{\lcm}{lcm}

\newcommand{\Mod}[1]{\ (\mathrm{mod}\ #1)}

\begin{document}

\Header

\BeginConspect

\Subsection{О-символика}

\begin{Def}
    Пусть $f \thicksim g, f \thicksim h, x \to x_0$. Если $f - h = o(f - g)$, то говорят асимптотическое равенство $f \thicksim h$ точнее,
    чем $f \thicksim g$.
\end{Def}

Пусть $f: D \to \R, D \subset \R, x_0$ -- предельная точка $D$.
Пусть задана система функций $\{g_k\}_{k = 0}^N : \forall k \in [0, N - 1] \cap \Z_+ \to g_{k + 1}(x) = o(g_k(x)), x \to x_0$

\[f(x) = \sum_{k=0}^{N} c_k \cdot g_k(x) + o(g_N(x))\] 

Многочлены получаются, если $g_k(x) = (x - x_0)^k$.

Если $f(x) \thicksim C \cdot (x - x_0)^k (C \neq 0)$, то $C \cdot (x - x_0)^k$ -- главная степенная часть.

\begin{Thm}[О единственности асимптотического разложения]
    $D \in \R, x_0$ -- предельная точка $D$, $n \in \Z_+; f, g_k : D \to \R, g_{k + 1}(x) = o(g_k(x)), x \to x_0 \ \forall k = 0, ..., n - 1$
    и $\forall V(x_0) \ \exists $ точка в $\dot{V}(x_0) : $ в нкй $g_n$ не ноль. Тогда если существует асимптотическое разложение $f$ по системе функций $\{g_k\}$,
    то оно единственно.   
\end{Thm}

\begin{proof}
    Пусть не единственное. Тогда $\exists c_k, d_k, k = 0, ..., n : \exists i \ c_i \neq d_i$.
    
    $f(x) = \sum_{k=0}^{n} c_k \cdot g_k(x) + o(g_n(x))$ и $f(x) = \sum_{k=0}^{n} d_k \cdot g_k(x) + o(g_n(x))$ при $x \to x_0$.
    
    Т.к. $g_{k + 1}(x) = o(g_k(x))$, то $g_{k + 1}(x) = o(g_l(x)) \ \forall l \leqslant k$ при $x \to x_0$.

    Обозначим $E_k = \{x : g_k(x) \neq 0, k = 0, ..., n\}$. Если $g_k = 0$ на $V(x_0)$, то $g_{k + 1} = 0$ на $V(x_0), g_n = 0$ на $V(x_0)$. 
    \[g_{k + 1} = o(g_k) \EQ \exists \PHI : g_{k + 1} = \PHI \cdot g_k\]

    Если $x_0$ -- предельная точка $E_{k_0}$, то она предельная точка всех $E_k$.
    Пусть $m$ -- наименьший номер $: c_m \neq d_m$. Тогда 
    \[f(x) = \sum_{k=0}^{m} c_k g_k(x) = o(g_m(x)), f(x) = \sum_{k=0}^{m} d_k g_k(x) + o(g_m(x))\]
    Вычтем: $0 = (c_m - d_m) g_m(x) + o(g_m(x))$. Поделим на $g_m(x)$ 
    \[0 = (c_m - d_m) + \frac{o(g_m(x))}{g_m(x)} \xrightarrow[x \to x_0]{} c_m - d_m \SO c_m = d_m\]
\end{proof}

\begin{Def}
    $x_0 \in \R, f$ задана хотя бы на $\langle a, x_0)$ или $(x_0, b\rangle$ и действует в $\R$. 
    Тогда прямая $x = x_0$ называется вертикальной асимптотой функции $f$, если 
    \[\lim_{x \to x_0+}f(x) = \pm \infty \vee \lim_{x \to x_0-} f(x) = \pm \infty\] 
\end{Def}

\begin{Def}
    $\langle a, +\infty) \subset D \subset \R, f : D \to \R, \alpha, \beta \in \R$. 
    Прямая $y = \alpha x + \beta$ -- наклонная асимптота $f$ при $x \to +\infty$, если $f(x) = \alpha x + \beta + o(1)$ при $x \to +\infty$.
\end{Def}

\begin{Def}
    При $x \to -\infty$ аналогично.
\end{Def}

\begin{Thm}[Уравнение наклонной асимптоты]
    $\langle a, +\infty) \subset D \subset \R, f : D \to \R$. $\alpha, \beta \in \R$. \\
    Прямая $y = \alpha x + \beta$ является асимптотой $f$ при $x \to +\infty \EQ \alpha = \lim_{x \to +\infty} \frac{f(x)}{x}, \beta = \lim_{x \to +\infty} (f(x) - \alpha x)$    
\end{Thm}

\begin{proof}
    "$\SO$". По определению $f(x) = \alpha x + \beta + \PHI(x), \PHI \xrightarrow[x \to +\infty]{} 0$.
    Тогда $ \frac{f(x)}{x} = \alpha + \frac{\beta}{x} + \frac{\PHI(x)}{x}$
    \[\lim_{x \to +\infty} \frac{f(x)}{x} = \alpha\]

    $f(x) - \alpha x = \beta + \PHI(x)$ 
    \[\lim_{x \to +\infty} (f(x) - \alpha x) = \beta\]

    "$\Leftarrow$". Проделаем те же рассуждения "в обратную сторону".
\end{proof}

\Section{Дифференциальное исчисление}{}{Илья Дудников}

Пусть $f: E \to \R, E \subset \R, a$ -- предельная точка $E, n \in \Z_+$. Хотим найти многочлен степени не выше $n$ $(P(x) = \sum_{k=0}^{n} c_k (x - x_0)^k)$
\begin{equation}
f(a) = P(a), f(x) = P(x) + o((x - a)^n), x \to a
\end{equation}

\begin{Rem}
    Если такой многочлен существует, то он единственный.
\end{Rem}

\begin{proof}
    Пусть $\exists P(x), Q(x)$, удовлетворяющие условию (1). Тогда 
    \[0 = P(x) - Q(x) + o((x - a)^n)\]
    Если $P(x) \neq Q(x)$, то $P(x) - Q(x) = \sum_{k=0}^{n} r_k (x - a)^k = r(x)$
    \[\SO r(x) = o((x - a)^n), x \to a\]
    $r(x) = r_m(x - a)^m + ... + r_n(x - a)^n, m \leqslant n, r_m \neq 0$
    \[\SO \frac{r(x)}{(x - a)^m} = o((x - a)^{n - m}) \SO r_m \neq 0 = 0\] 
\end{proof}

\begin{Def}
    Многочлен, удовлетворяющий условию (1) называется многочленом Тейлора функции $f$ в точке $a$ порядка $n$ $T_{a, n} f$ 
\end{Def}

\begin{Def}
    Функция $f$ называется дифференцируемой в точке $a$ ( $\langle A, B\rangle \to \R, a \in (A, B)$ ),
    если $\exists k \in \R : $ 
    \[f(x) = f(a) + k(x - a) + o(x - a), x \to a\]
\end{Def}

\begin{Def}
    $f : \langle A, B\rangle \to \R, a \in (A, B)$, если $\exists \lim_{x \to a} \frac{f(x) - f(a)}{x - a} = K \in \R$, то $K$ называется производной 
    функции $f$ в точке $a$. (Обозначение $f'(a), \frac{df}{dx}(a), D f(a)$ ) \\
    $\Delta_a f = f(x) - f(a)$ -- приращение функции $f$ в точке $a$. \\
    $x - a = \Delta_a x$.
    \[f'(a) = \lim_{\Delta_a x \to 0} \frac{\Delta_a f}{\Delta_a x}\]
\end{Def}

\begin{Thm}
    $f : \langle A, B\rangle \to \R, a \in (A, B)$. Тогда равносильны три утверждения:
    \begin{MyList}
        \item $f$ дифференцируема в точке $a$
        \item $\lim_{x \to a} \frac{f(x) - f(a)}{x - a}$ существует и равен $k$
        \item $\exists F(x) : F: \langle A, B\rangle \to \R, F$ непрерывна в точке $a$, $F(a) = k$ и $f(x) - f(a) = F(x)(x - a), x \in \langle A, B\rangle$ 
    \end{MyList}
\end{Thm}

\begin{proof}
    $1) \SO 2)$. $\exists k : f(x) - f(a) = k(x - a) + o(x - a), x \to a$
    \[ \frac{f(x) - f(a)}{x - a} = k + \frac{o(x - a)}{x - a} \to k\]
    $2) \SO 3)$. 
    \[F(x) = \begin{cases}
        \frac{f(x) - f(a)}{x - a}, x \neq a \\
        k, x = a
    \end{cases}\]
    из 2) следует непрерывность $F$ в точке $a$ 

    $3) \SO 1)$. По 3) $\exists F : $
    \[f(x) - f(a) = F(x)(x - a) = \EQ f(X) = f(a) + F(x)(x - a) = f(a) + k(x - a) + (F(x) - k) \cdot (x - a)\] 
    $F(x) \xrightarrow[x \to a]{} F(a) = k \SO (F(x) - k)(x - a) = o((x - a))$  
\end{proof}

\Subsection{Связь с физикой}

\[\lim_{\Delta t \to 0} \frac{\Delta S}{\Delta t} \text{ -- мгновенная скорость}\]

\Subsection{Связь с геометрией}

Рассмотрим функции: $l_k(x) = f(a) + k(x - a)$, графики -- прямые, проходящие через точку $(a; f(a))$
\[f(x) - l_k(x) = f(x) - f(a) - k(x - a)\]
Если $f(x)$ дифференцируема в точке $a$
\[f(x) = f(a) + f'(a)(x - a) + o(x - a) \EQ f(x) - l_k(x) = (x - a) \cdot (f'(a) - k) + o(x - a)\]
При $k = f'(a)$ разность есть $o(x - a)$.
\[y = f(a) + f'(a)(x - a)\]
касательная в точке $a$ к функции $f$. $\tg \alpha = f'(a)$.

\Subsection{Бесконечные производные}

\[\lim_{x \to a} \frac{f(x) - f(a)}{x - a} = +\infty \SO f'(a) = +\infty\]
В таком случае $f$ не является дифференцируемой в точке $a$.

Односторонняя производная:
\[\exists \lim_{x \to a\pm} \frac{f(x) - f(a)}{x - a}\]

\end{document}