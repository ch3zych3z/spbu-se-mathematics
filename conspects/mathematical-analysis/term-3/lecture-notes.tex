\input{preamble.tex}

\begin{document}
    \Header

    \BeginConspect

    \Section{Метрические пространства}{}{Илья Дудников}

    Пусть $X$ -- некоторое множество. Зададим функцию $\rho : X \times X \to \R$.
    \begin{Def}[Метрика]
        $\rho$ называется метрикой, если выполняются следующие три свойства:
        \begin{MyList}
            \item $\rho (x, y) \geqslant 0 \ \forall x, y \in X$
            \item[] $\rho(x, y) = 0 \EQ x = y$
            \item $\forall x, y \in X \ \rho (x, y) = \rho (y, x)$ 
            \item $\forall x, y, z \in X \ \rho (x, y) \leqslant \rho(x, z) + \rho(z, y)$ -- неравенство треугольника. 
        \end{MyList}
    \end{Def} 

    \begin{Def}[Метрическое пространство]
        Пара $(X, \rho)$ называется \textit{метрическим пространством}.
    \end{Def}

    \begin{Example}
        Метрика на $\R^2$: $\rho_2 (x, y) = \sqrt{(x_1 - y_1)^2 + (x_2 - y_2)^2}$ 
    \end{Example}

    \begin{Example}
        Метрика на $\R^d$: $\rho_2 (x, y) = \sqrt{\sum_{i=1}^{d} (x_i - y_i)^2}$ 
    \end{Example}

    \begin{Example}[Дискретная метрика]
        Пусть $X$ -- некоторое множество. Зададим 
        \[\rho (x, y) = \begin{cases}
            0, &x = y \\
            1, &x\neq y
        \end{cases}\]
        Действительно, все свойства выполняются, поэтому $\rho$ -- метрика.
    \end{Example}

    \begin{Example}[Манхэттенская метрика]
        В $\R^2$:
        \[\rho_1(x, y) = |x_1 - y_1| + |x_2 - y_2|\]

        \begin{figure}[H]
            \centering
            \def\svgwidth{.3\columnwidth}
            \input{img/Manhattan_distance.pdf_tex}
        \end{figure}
    \end{Example}

    \begin{Example}
        $\rho_\infty(x, y) = \max \{|x_1 - y_1|, |x_2 - y_2|\}$ 
    \end{Example}

    \begin{Example}
        Рассмотрим $C[a, b]$. Тогда $\rho (f, g) = \int_{a}^{b} |f - g|$ -- метрика.
    \end{Example}

    \begin{notation}[Открытый шар]
        $B_r(a) = \left\{x \in X, \ \rho (x, a) < r\right\}$
    \end{notation}

    \begin{notation}[Замкнутый шар]
        $\overline{B}_r(a) = \left\{x \in X, \ \rho(x, a) \leqslant r\right\}$
    \end{notation}

    \begin{notation}[Сфера]
        $S_r(a) = \{x \in X, \ \rho(x, a) = r\}$
    \end{notation}

    \begin{Example}
        В дискретной метрике при $r < 1$ замкнутый шар $\overline{B}_r(a)$ включает только одну точку -- $a$, 
        а при $r \geqslant 1$ -- всё множество $X$.
    \end{Example}

    \begin{Example}
        Замкнутый шар в манхэттенской метрике: 
        \begin{figure}[H]
            \centering
            \def\svgwidth{.3\columnwidth}
            \input{img/manhattan_closed_ball.pdf_tex}
        \end{figure}
    \end{Example}

    \begin{Example}
        Замкнутый шар в $\rho_\infty$: 
        \begin{figure}[H]
            \centering
            \def\svgwidth{.3\columnwidth}
            \input{img/rho_infty_closed_ball.pdf_tex}
        \end{figure}
    \end{Example}

    \begin{Prop}
        Пусть $B_{r_1}(a)$ и $B_{r_2}(a)$ -- шары. Тогда 
        \[B_{r_1}(a) \cap B_{r_2}(a) = B_{\min (r_1, r_2)} (a)\]
    \end{Prop}

    \begin{proof}
        Возьмем $x \in B_{r_1} (a) \cap B_{r_2} (a)$. Тогда $\rho(x, a) < r_1$ и $\rho(x, a) < r_2$, значит $\rho(x, a) < \min (r_1, r_2)$.  
    \end{proof}

    \begin{Prop}
        $\forall a \neq b \ \exists r : \overline{B}_r(a) \cap \overline{B}_r(b) = \varnothing$.
    \end{Prop}

    \begin{proof}
        Возьмем $r = \frac{\rho(a, b)}{3}$. Предположим, что пересечение непусто, т.е. $\exists x  : x \in \overline{B}_r(a)$ и $x \in \overline{B}_r(b)$.
        Тогда 
        \[\rho(a, b) \leqslant \rho(a, x) + \rho(x, b) \leqslant \frac{\rho(a, b)}{3} + \frac{\rho(a, b)}{3}\]   
    \end{proof}

    \begin{notation}
        $V_x$ -- окрестность точки $x$ (шар).
    \end{notation}

    \begin{notation}
        $\dot{V}_x$ -- проколотая окрестность $x$ (шар, не содержащий точку $x$).
    \end{notation}

    \begin{Def}[Внутренняя точка множества] 
        Пусть $A \subset X$. Точка $a$ называется \textit{внутренней} точкой $A$, если $\exists V_a \subset A$.
    \end{Def}

    \begin{Def}[Внешняя точка множества]
        Пусть $A \subset X$. Тогда точка $b$ называется \textit{внешней} точкой $A$, если $b$ -- внутренняя точка $X \setminus A$. 
    \end{Def}

    \begin{Def}[Граничная точка множества]
        Пусть $A \subset X$. Тогда $c$ является \textit{граничной} точкой множества $A$, если она не является ни внутренней, ни внешней.
        Иначе, точка $c$ назывется граничной, если
        \[\forall V_c \ \exists x, y \in V_c : x \in A \wedge y \in X \setminus A\]        
    \end{Def}

    \begin{Def}[Открытое множество]
        Множество $A \subset X$ называется \textit{открытым}, если любая его точка -- внутренняя.
    \end{Def}

    \begin{Thm}[Об открытых множествах]
        \begin{MyList}
            \item $\varnothing$ и $X$ -- открытые множества
            \item Объединение любого числа открытых множеств -- открытое множество
            \item Пересечение конечного числа открытых множеств -- открытое множество
            \item Открытый шар -- это открытое множество  
        \end{MyList}
    \end{Thm}

    \begin{proof}
        \begin{MyList}
            \item Очевидно
            \item Пусть $B = \bigcup_{\alpha \in I} A_\alpha$. Возьмем $x \in B$. Тогда $\exists \beta \in I : x \in A_\beta$. Т.к. $A_\beta$ -- открытое множество, то 
            $x$ принадлежит $A_\beta$ с какой-то своей окрестностью, а значит она принадлежит и всему объединению с этой окрестностью.

            \item Пусть $B = \bigcap_{i = 1}^n A_i$. Возьем $x \in B$. Тогда $x \in A_i \ \forall i$. Точка $x$ принадлежит всем $A_i$ с какой-то круговой окрестностью $r_i$. Тогда она принадлежит пересечению с круговой окрестностью $\min r_i$.    
            \item Рассмотрим $B_R(a) = \{x \in X, \ \rho(x, a) < R\}$. Пусть точка $x \in B_R(a), \ \rho(x, a) < R$. Положим $r = R - \rho(x, a)$. Возьмем $y$ из окрестности $x$ радиуса $r$. Тогда в силу неравенства треугольника
            \[\rho(a, y) \leqslant \rho(a, x) + \rho(x, y) < \rho(a, x) + R - \rho(x, a) = R\]
            \begin{figure}[H]
                \centering
                \def\svgwidth{.3\columnwidth}
                \input{img/open_ball_theorem.pdf_tex}
            \end{figure}

            А значит $\forall y \in V_x(r) \ y \in B_R(a)$, т.е. любая точка $x \in B_R(a)$ принадлежит шару $B_R(a)$ с какой-то своей окрестностью.
        \end{MyList}
    \end{proof}

    \begin{Rem}
        Конечность в пункте 3 существенна: рассмотрим $\bigcap_{n = 1}^\infty \left(-\frac{1}{n}; 1\right) = [0, 1)$. 
    \end{Rem}

    \begin{notation}
        $\Int A$ -- множество всех внутренних точек множества $A$. 
    \end{notation}

    \begin{Thm}[Свойства]
        \begin{MyList}
            \item $\Int A \subset A$
            \item $\Int A = \bigcup$ всех открытых множеств, которые содержатся в $A$
            \item $\Int A$ -- открытое множество
            \item $A$ -- открытое $\EQ A = \Int A$
            \item $A \subset B \SO \Int A \subset \Int B$
            \item $\Int (A \cap B) = \Int A \cap \Int B$
            \item $\Int (\Int A) = \Int A$   
        \end{MyList}
    \end{Thm}

    \begin{Def}[Замкнутое множество]
        Множество $A \subset X$ называется \textit{замкнутным}, если $X \setminus A$ -- открыто.
    \end{Def}

    \begin{Thm}[О замкнутых множествах]
        \begin{MyList}
            \item $\varnothing, \ X$ -- замкнутые множества.
            \item Пересечение любого числа замкнутых множеств -- замкнутое множество
            \item Конечное объединение замкнутых множеств -- замкнутое множество
            \item Замкнутый шар -- это замкнутое множество.
        \end{MyList}
    \end{Thm}

    \begin{proof}
        \begin{MyList}
            \item Очевидно
            \item Пусть $B = \bigcap_{\alpha \in I} A_\alpha$. Тогда
            \[X \setminus B = X \setminus \left(\bigcap_{\alpha \in I} A_\alpha\right) = \bigcup_{\alpha \in I} \left(X \setminus A_\alpha\right)\]
            Поскольку $\forall \alpha \in I \ A_\alpha$ -- замкнутое, т.е. $X \setminus A_\alpha$ -- открытое, то $X \setminus B$ -- открытое (по теореме об открытых множествах), значит $B$ -- замкнутое множество.
            \item Доказывается аналогично предыдущему пункту.
            \item Рассмотрим $\overline{B}_R(a)$. Возьмем $x \in X \setminus \overline{B}_R(a)$ и $y$ из окрестности $x$, т.е. $y \in B_r(x)$, где $r = \rho(a, x) - R$.
            По неравенству треугольника:
            \[\rho(a, x) \leqslant \rho(a, y) + \rho(y, x)\]
            Поэтому
            \[\rho (a, y) \geqslant r + R - \rho(y, x) > R\]
            \begin{figure}[H]
                \centering
                \def\svgwidth{.3\columnwidth}
                \input{img/closed_ball_theorem.pdf_tex}
            \end{figure}
        \end{MyList}
    \end{proof}

    \begin{Rem}
        Конечность в пункте 3 существенна: $\bigcup \left[\frac{1}{n}; 1\right] = (0; 1]$ -- незамкнутое множество.
    \end{Rem}

    Пусть $E \subset \R^n, \ F \subset E$.

    \begin{Def}
        Точка $a \in F$ называется внутренней для $F$ в $E$, если $\exists V_a^E \subset F$.
    \end{Def}

    \begin{Def}
        $F$ называется открытым в $E$, если все его точки внутренние в $E$.
    \end{Def}

    \begin{Rem}
        Если множество $E$ открыто, то ничего не изменилось. Если же множество $E$ не открыто, то мы получаем новые определения. 
    \end{Rem}

    \begin{Example}
        Множество $(0, 1) \cap \Q$ открыто в $\Q$.
    \end{Example}

    \begin{Example}
        $(1, 2]$ открыто в $(0, 2]$.  
    \end{Example}

    \begin{Rem}
        Множество $F$ открыто в $E \EQ \exists G$ -- открытое в $\R^n : F = E \cap G$. 
    \end{Rem}

    \begin{Example}
        $(0, 1]$ -- замкнуто в $(0, 2]$.
    \end{Example}

    \begin{notation}[Замыкание множества]
        $\Cl A$ -- пересечение всех замкнутых множеств, которые содержат $A$. 
    \end{notation}

    \begin{Example}
        $\Cl (0, 1) = [0, 1]$. 
    \end{Example}

    \begin{Thm}[Свойства]
        \begin{MyList}
            \item $A \subset \Cl A$
            \item $\Cl A$ -- замкнутое множество
            \item $A$ -- замкнуто $\EQ A = \Cl A$.
            \item $A \subset B \SO \Cl A \subset \Cl B$
            \item $\Cl (A \cup B) = \Cl A \cup \Cl B$
            \item $\Cl (\Cl A) = \Cl A$      
        \end{MyList}
    \end{Thm}

    \begin{Thm}
        $x \in \Cl A \EQ \forall r > 0 \ B_r(x) \cap A \neq \varnothing$. 
    \end{Thm}

    \begin{proof}
        Докажем, что $x \notin \Cl A \EQ \exists r > 0 \ B_r(x) \cap A = \varnothing$.
        \[x \in \left(X \setminus \Cl A\right) \EQ x \in \Int (X \setminus A)\]
    \end{proof}

    Пусть $A'$ -- множество предельных точек $A$. Тогда
    \begin{Thm}[Свойства]
        \begin{MyList}
            \item $\Cl A = A \cup A'$ 
            \item $A \subset B \SO A' \subset B'$ 
            \item $A$ -- замкнуто $\EQ A' \subset A$
        \end{MyList}
    \end{Thm}

    \begin{proof}
        \begin{MyList}
            \item[3.] $A$ -- замкнуто $\EQ \Cl A = A, \ \Cl A = A \cup A'$.
        \end{MyList}
    \end{proof}

    \begin{Thm}
        $x \in A' \EQ \forall B_r(x)$ содержит бесконечно много точек из $A$.
    \end{Thm}

    \Subsection{Нормированные пространства}

    Пусть $X$ -- векторное пространство над $\R$.
    \begin{Def}[Норма]
        Функция $||\cdot|| : X \to \R$ называется \textit{нормой}, если выпоняются следующие свойства:
        \begin{MyList}
            \item $||x|| \geqslant 0$
            \item[] $||x|| = 0 \EQ x = 0$
            \item $||\lambda x|| = |\lambda| \cdot ||x||, \ \lambda \in \R$
            \item $||x + y|| \leqslant ||x|| + ||y||$.  
        \end{MyList}
    \end{Def}

    \begin{Example}
        $||x||_1 = |x_1| + |x_2|$
    \end{Example}

    \begin{Example}
        $||x||_2 = \sqrt{x_1^2 + x_2^2}$ 
    \end{Example}

    \begin{Example}
        $||x||_\infty = \max_{i \in \{1, ...\}} |x_i|$
    \end{Example}

    \begin{Example}
        $C[a, b]$, $||f|| = \max_{x \in [a, b)} |f|$ 
    \end{Example}

    \begin{Def}[Скалярное произведение]
        $\langle \cdot, \cdot \rangle : X \times X \to \R$ -- скалярное произведение, если выполняются:
        \begin{MyList}
            \item $\langle x, x\rangle \geqslant 0$ 
            \item[] $\langle x, x\rangle = 0 \EQ = x = 0$
            \item $\langle x + y, z\rangle = \langle x, z\rangle + \langle y, z\rangle$
            \item $\langle x, y\rangle = \langle y, x\rangle$
            \item $\langle \lambda x, y\rangle = \lambda \langle x, y\rangle$. 
        \end{MyList} 
    \end{Def}

    \begin{Ex}
        Вспомнить неравенство Коши-Буняковского
    \end{Ex}

    \begin{Prop}
        $||x|| = \sqrt{\langle x, x\rangle}$
    \end{Prop}

    \begin{proof}
        1-2 очевидно.
        \begin{MyList}
            \item[3.]
            \begin{align*}
                \langle x + y, x + y \rangle \leqslant \langle x, x\rangle + \langle y, y\rangle + 2\sqrt{\langle x, x\rangle \cdot \langle y, y\rangle}
            \end{align*}
            С другой стороны:
            \[\langle x + y, x + y \rangle = \langle x, x\rangle + \langle y, y\rangle + 2 \langle x, y\rangle\]
        \end{MyList}
    \end{proof}

    \begin{Def}[Полное пространство]
        Пространство называется \textit{полным}, если в нем любая фундаментальная последовательность сходится.
    \end{Def}

    \begin{Ex}
        Доказать, что $\R^2, \R^3, \R^n$ -- полные пространства.
    \end{Ex}

    \begin{notation}
        $\overline{\R}^d = \R^d \cup \{\infty\}$ 
    \end{notation}

    \begin{Rem}
        Под $V_\infty$ будем понимать $\{x : ||x|| > \delta\}$. 
    \end{Rem}    
    
    \begin{Thm}[Сходимость и покоординатная сходимость] \label{thm:convergence}
        $x^i \in \R^d$ ($x^i = (x_1^i, x_2^i, ..., x_d^i)$). Рассмотрим последовательность $\{x^i\}_{i = 1}^\infty$. Тогда 
        равносильны утверждения: 

        \begin{MyList}
            \item $\{x^i\}_{i = 1}^\infty$ сходится
            \item $\{x^i\}_{i = 1}^\infty$ сходится покоординатно.  
        \end{MyList}
    \end{Thm}

    \begin{proof}
        \begin{MyList}
            \item[] 1 $\SO$ 2.
            \[\forall \varepsilon > 0 \ \exists N : \forall n > N \ ||x^n - a|| < \varepsilon\]

            \[|x_k^n - a_k| \leqslant ||x^n - a||\]

            \item[] 2 $\SO$ 1. 
            \[\sqrt{(x_1^n - a_1)^2 + (x_2^n - a_2)^2 + ... + (x_d^n - a_d)^2}\]
        \end{MyList}
    \end{proof}

    \begin{Rem}
        Если $\lim_{k \to \infty} x^k = \infty$, то координатные последовательности $\{x^k\}$ могут и не иметь предела. Пусть, например, последовательность в $\R^2$ определяется формулой 
        \[x^k = \left(k \cos \frac{\pi k}{2}, k \sin \frac{\pi k}{2}\right)\]
        Тогда 
        \[||x^k|| \sqrt{k^2 \cos^2 \frac{\pi k}{2} + k^2 \sin^2 \frac{\pi k}{2}} = k \to +\infty\]
        То есть $x^k \to \infty$. Тем не менее, последовательности $x_1^k = k \cos \frac{\pi k}{2}$ и $x_2^k = k \sin \frac{\pi k}{2}$ предела не имеют. 
    \end{Rem}

    \begin{Thm}[Арифметические действия и пределы]
        Пусть $\{x^k\}, \ \{y^k\} \subset \R^n$, $\lim x^k = a, \ \lim y^k = b$. Тогда
        \begin{MyList}
            \item $\lim(x^k + y^k) = a + b$
            \item Пусть $\{\lambda_n\}$ -- последовательность из $\R$, $\lim \lambda_k = \lambda$. Тогда
            \[\lim \lambda_k x_k = \lambda a\] 
            \item $\lim ||x^k|| = ||a||$
            \item $\lim \langle x^k, y^k\rangle = \langle a, b\rangle$
        \end{MyList} 
    \end{Thm}

    \begin{proof}
        \begin{MyList}
            \item По теореме \ref{thm:convergence} для любого $i = 1, ..., n$
            \[\lim_{k \to \infty} x_i^k = a_i, \quad \lim_{k \to \infty} y_i^k = b_i\]
            Тогда $\lim_{k \to \infty} (x_i^k + y_i^k) = a_i + b_i$. Применяя теорему \ref{thm:convergence} еще раз, получаем, что
            $\lim_{k \to \infty} (x^k + y^k) = a + b$.
            \item[4.] Заметим, что 
            \begin{align*}
                \frac{1}{4} (||a + b|| - ||a - b||) &= \frac{1}{4}(\langle a + b, a + b\rangle - \langle a - b, a - b\rangle) = \\
                &= \frac{1}{4} (4 \langle a, b\rangle) = \langle a, b\rangle
            \end{align*}
            Применяя пункты 1 и 3, получаем нужное утверждение.
        \end{MyList}
    \end{proof}

    \begin{Def}[Ограниченное множество]
        Множество $E$ называется ограниченным в $\R^n$, если $\exists c : E \subset V_0(c)$,
        т.е. $\forall x \in E \ ||x|| < c$.
    \end{Def}

    \begin{Rem}
        Ограниченность множества в $\R^n$ равносильна следующему условию:
        \[\sup_{x \in E} ||x|| < +\infty\]
    \end{Rem}

    \begin{Def}
        Проекцией $E \subset \R^n$ будем называть $E_i = \{x_i : x \in E\}$. 
    \end{Def}

    \begin{Rem}
        Ограниченность множества $E$ равносильна ограниченности всех проекций.
    \end{Rem}

    \begin{Thm}[Принцип выбора Больцано-Вейерштрасса]
        Пусть $\{x^k\}$ -- последовательность в $\R^n$. Тогда
        \begin{MyList}
            \item Если $\{x^k\}$ ограничена, то из неё можно выделить сходящуюся подпоследовательность.
            \item Если $\{x^k\}$ не ограничена, то из неё можно выделить подпоследовательность, стремящуюся к $\infty$.
        \end{MyList}
    \end{Thm}

    \begin{proof}
        \begin{MyList}
            \item $\{x_1^k\}$ -- ограниченная последовательность в $\R \SO \exists x_1^{r_k}$ -- сходящаяся подпоследовательность (по принципу выбора Больцано-Коши для числовых последовательностей).
            Рассмотрим теперь $\{x_2^{r_k}\}$ -- ограничена в $\R \SO \exists \{x_2^{s_k}\}$ -- сходящаяся подпоследовательность, где $\{s_k\}$ -- подпоследовательность $\{r_k\}$.
            После $n$-ного шага мы построили последовательность $\{x_n^{l_k}\} \SO \{x^{l_k}\}$ сходится.
        
            \item Можем построить $\{x^{r_k}\} : ||x^{r_k}|| > k$. Будем выбирать $r_1 < r_2 < r_3 < ...$. 
        \end{MyList}
    \end{proof}
    
    \begin{Thm}[Критерий Больцано-Коши]
        Пусть $\{x^k\}$ -- последовательность из $\R^n$. Тогда равносильны следующие условия:
        \begin{MyList}
            \item $\{x^k\}$ -- сходится
            \item $\forall \varepsilon > 0 \ \exists N : \forall m, n > N \ ||x^m - x^n|| < \varepsilon$.
        \end{MyList}
    \end{Thm}

    \begin{proof}
        Заметим, что 
        \[|x_i| \leqslant ||x|| \leqslant \sqrt{n} \cdot \max_{i \in [1, n]} |x_i|\]
        \begin{MyList}
            \item[] $1 \SO 2$. Поскольку $\{x^k\}$ сходится, то 
            \[\forall i = 1..n \ \forall \varepsilon > 0 \ \exists N : \forall n, m > N \ |x_i^m - x_i^n| < \frac{\varepsilon}{\sqrt{n}}\]
            Тогда
            \[||x^n - x^m|| < \sqrt{n} \cdot \frac{\varepsilon}{\sqrt{n}}\]

            \item[] $2 \SO 1$.
            \[|x_i^m - x_i^n| \leqslant ||x_i^m - x_i^n|| \leqslant \varepsilon\]
        \end{MyList}
    \end{proof}

    \begin{Def}[Покрытие]
        $\Omega$ -- семейство множеств из $\R^n$. $\Omega$ называется \textit{покрытием}  множества $E \subset \R^n$, если $E \subset \bigcup_{A \in \Omega} A$. 
    \end{Def}

    \begin{Def}[Открытое покрытие]
        Если все множества из $\Omega$ открытые, то $\Omega$ называется \textit{открытым} покрытием.
    \end{Def}

    \begin{Def}
        Пусть $\widetilde{\Omega}$ -- подсемейство $\Omega$, которое также покрывает $E$. Тогда $\widetilde{\Omega}$ называется \textit{подпокрытием} $\Omega$. 
    \end{Def}

    \begin{Def}
        Множество $E$ называется \textit{компактным}, если из любого его открытого покрытия можно выбрать конечное подпокрытие.
    \end{Def}

    \begin{Example}
        $(0, 1)$ -- не компакт. $\Omega = \{\left(\frac{1}{n}, 1\right), n \in \N\}$ -- покрытие $(0, 1)$. Из него нельзя выбрать конечное подпокрытие $(0, 1)$.
    \end{Example}

    \begin{Def}
        Пусть $a, b \in \R^n : a_1 \leqslant b_1, a_2 \leqslant b_2, ..., a_n \leqslant b_n$. 
        Тогда \textit{замкнутым параллелепипедом} будем называть следующее множество:
        \[[a; b] = [a_1, b_1] \times [a_2, b_2] \times ... \times [a_n, b_n]\]  

        \textit{Открытым параллелепипедом} называется множество:
        \[(a; b) = (a_1, b_1) \times (a_2, b_2) \times ... \times (a_n, b_n)\]
    \end{Def}

    \begin{Ex}
        Открытый параллелепипед -- открытое множество, замкнутый параллелепипед -- замкнутое множество.
    \end{Ex}

    \begin{Def}[Диаметр множества]
        $\displaystyle \diam E = \sup_{x, y \in E} ||x - y||$ 
    \end{Def}

    \begin{Thm}[О стягивающихся параллелепипедах]
        Рассмотрим параллелепипеды $P_k \subset \R^n$ -- замкнутые, $P_1 \supset P_2 \supset P_3 \supset ..., \ \diam P_k \xrightarrow[k \to \infty]{}0$.
        Тогда существует только одна точка, принадлежащая всем параллелепипедам.
    \end{Thm}

    \begin{Thm}
        Замкнутый куб в $\R^n$ является компактом.
    \end{Thm}

    \begin{Lm}
        $E \subset \R^n$. $E$ замкнуто $\EQ \forall$ сходящаяся последовательность в $E$ имеет пределом точку из $E$. 
    \end{Lm}

    \begin{proof}
        $\SO$. Пусть $\{x^k\}_{k = 1}^\infty$ -- последовательность в $E$, $a \in \R^n$, $\lim_{k \to \infty} x_k = a$. Покажем, что $a \in E$.
        Если это не так, то $a \in \R^n \setminus E$, и в силу открытости $\R^n \setminus E$ найдется окрестность $V_a$ точки $a$, лежащая в $\R^n \setminus E$. 
        По определению предела при всех достаточно больших $k \in \N$ справедливо включение $x^k \in V_a \subset \R^n \setminus E$. С другой стороны, $x^k \in E \ \forall k \in \N$, и мы получаем противоречие. 

        $\Leftarrow$. $E' \subset E \SO E$ -- замкнуто.
    \end{proof}

    \begin{Lm}
        Замкнутое подмножество компакта -- компакт.
    \end{Lm}

    \begin{proof}
        Пусть $F$ -- замнутое подмножество компакта $E$, $\Omega$ -- открытое покрытие $F$. Покажем, что из $\Omega$ можно выбрать конечное подпокрытие.
        Добавляя к $\Omega$ множество $\R^n \setminus F$, мы получим открытое покрытие компакта $E$. Выберем из этого покрытия конечное подсемейство $\widetilde{\Omega}$, которое также покрывает $E$.
        Если множество $\R^n \setminus F$ входит в $\widetilde{\Omega}$, удалим его оттуда. Мы получим конечное подпокрытие $\Omega$ множества $F$. 
    \end{proof}

    \begin{Thm}
        Пусть $E \subset \R^n$. Тогда равносильны следующие условия:
        \begin{MyList}
            \item $E$ -- компакт
            \item $E$ ограничено и замкнуто
            \item Из любой последовательности в $E$ можно выбрать подпоследовательность, сходящуюся к точке из $E$.
        \end{MyList}
    \end{Thm}

    \begin{proof}
        \begin{MyList}
            \item[] $3 \SO 2$. Пусть $\{x^k\}$ -- последовательность в $E$, сходящаяся к некоторой точке $a \in \R^n$. 
            Из условия 3) вытекает, что у $\{x^k\}$ есть подпоследовательность, предел которой лежит в $E$. Но любая подпоследовательность $\{x^k\}$ сходится к $a$, откуда $a \in E \SO E$ замкнуто.

            Докажем теперь ограниченность. Если $E$ не ограничено, то по любому $k \in \N$ найдется $x^k \in E$, для которого $||x^k|| \geqslant k$. 
            Но тогда $x^k \to \infty$ при $k \to \infty$ $\SO$ все подпоследовательности $\{x^k\}$ также стремятся к бесконечности. Получили противоречие с условием 3). 

            \item[] $2 \SO 1$. $E$ ограничено $\SO \exists c : [-c, c]^n \supset E$. Тогда $E$ -- замкнутое подмножество компакта $\SO E$ -- компакт.
            \item[] $1 \SO 3$. Пусть $a$ -- предел последовательности из $E$, но $a \notin E$. Положим $\Omega = \{\R^n \setminus \overline{B}_\varepsilon(a)\}$.
            Тогда $\bigcup_{A \in \Omega} A = \R^n \setminus \{a\} \SO \Omega$ -- покрытие $E$. 
            Пусть $\widetilde{\Omega}$ -- произвольное конечное подсемейство $\Omega$. Тогда, для некоторого $m \in \N$ и положительных чисел
            $\varepsilon_1, ..., \varepsilon_m$
            \[\widetilde{\Omega} = \left\{\R^n \setminus \overline{B}_{\varepsilon_k}(a)\right\}_{k = 1}^m\]   
            Множество $B(a) = \displaystyle{\bigcap_{k = 1}^m}B_{\varepsilon_k}(a)$ является окрестность точки $a$. Заметим, что
            \[\bigcup_{A \in \widetilde{\Omega}}A = \R^n \setminus \bigcap_{k = 1}^m \overline{B}_{\varepsilon_k}(a) \subset \R^n \setminus B(a)\]
            Таким образом, множество $\widetilde{\Omega}$ не покрывает $E$, что противоречит компактности $E$. 
        \end{MyList}
    \end{proof}

    \Subsection{Отображения}
    
    $f : E \subset \R^n \to \R^m$.

    \begin{Example}
        $m = 1$ -- функция нескольких переменных. $f(x_1, x_2, ..., x_n)$.
    \end{Example}

    \begin{Example}
        $n = 1$ -- вектор-функция. $(f_1(x), f_2(x), ..., f_m(x)) = f(x)$.
    \end{Example}

    \begin{Def}[Ограниченное отображение]
        Отображение $f$ называется \textit{ограниченным}, если
        \[\sup_{x \in E} ||f(x)|| < +\infty\] 
    \end{Def}

    \begin{Rem}
        Ограниченность $f$ равносильна ограниченности координатных функций.
    \end{Rem}

    \begin{Def}[Предел по Коши]
        $f : E \subset \R^n \to \R^m$, $a$ -- предельная точка $E$, ($a \in \overline{R}^n$).
        \[\lim_{x \to a}f(x) = A \EQ \forall \varepsilon > 0 \ \exists \delta > 0 : \forall x \in E \cap \dot{B}_\delta(a) \ f(x) \in B_\varepsilon(A)\]
        Иначе, пусть $a \in \R^n, A \in \R^m$. Тогда $A$ -- предел, если 
        \[\forall \varepsilon > 0 \ \exists \Delta > 0 : \forall x \in E 0 < ||x - a|| < \delta \ ||f(x) - A|| < \varepsilon\] 
    \end{Def}

    \begin{Def}[Предел по Гейне]
        $f : E \subset \R^n \to \R^m$, $a$ -- предельная точка $E$. 
        Тогда $\lim_{x \to a} f(x) = A$, если
        \[\forall \{x^n\} \ x^k \to a, \ x^k \neq a, \ x^k \in E \ \lim f(x_k) = A\]
    \end{Def}

    \begin{Thm}[Эквивалентность определений предела]
        Пусть $f : E \subset \R^n \to \R^m$, точка $a \in \overline{\R^n}$ является предельной для $E$, $A \in \overline{\R^m}$. Тогда равносильны утверждения:
        \begin{MyList}
            \item $\displaystyle{\lim_{x \to a}} f(x) = A$ в смысле Коши
            \item $\displaystyle{\lim_{x \to a}} f(x) = A$ в смысле Гейне 
        \end{MyList}
    \end{Thm}

    \begin{proof}
        $1) \SO 2)$. Пусть $A$ -- предел $f$ в смысле Коши. Возьмем последовательность $\{x^k\}_{k = 1}^\infty$ в $E \setminus \{a\}$, стремящуюся к $a$.
        В силу 1) по любому $\varepsilon > 0$ можно подобрать $\delta > 0$, для которого
        \[f(x) \in V_A(\varepsilon) \quad \forall x \in \dot{V}_a(\delta) \cap E\]
        Поскольку $x^k \to a$, существует такое $N \in \N$, что при всех $k > N$ справедливо включение $x^k \in V_a(\delta)$. Кроме того, $x^k \in E \setminus \{a\}$, откуда $x^k \in \dot{V}_a(\delta) \cap E$.
        Поэтому 
        \[f(x^k) \in V_A(\varepsilon) \quad \forall k > N\]
        Таким образом, $f(x^k) \to A$ при $k \to \infty$, то есть $\lim_{x \to a} f(x) = A$ и в смысле Гейне.
        
        $2) \SO 1)$. Пусть $A$ -- предел $f$ по Гейне. Докажем, что предел $f$ в смысле Коши также существует и равен $A$.
        Действительно, если это не так, то
        \[\exists \varepsilon : \forall \delta > 0 \ \exists x \in \dot{V}_a(\delta) \cap E : f(x) \notin V_A(\varepsilon)\]
        Положим 
        \[F = \{x \in E \setminus \{a\}: f(x) \notin V_A(\varepsilon)\}\]
        Таким образом, $\dot{V}_a(\delta) \cap F = \varnothing$ при любом $\delta > 0$, то есть $a$ является предельной точкой $F$
        $\SO$ найдется последовательность $\{x^k\}_{k = 1}^\infty$ в $F$, стремящаяся к $a$. Тогда $\lim_{k \to \infty} f(x^k) = A$, что невозможно, т.к. $f(x^k) \notin V_A(\varepsilon)$ при всех $k \in \N$. 
    \end{proof}

    \begin{Thm}[Единственность предела отображения]
        Пусть $f : \R^n \to \R^m$, $A, B \in \overline{\R^m}$, $\lim_{x \to a} f(x) = A, \ \lim_{x \to a} f(x) = B$. Тогда $A = B$.  
    \end{Thm}

    \begin{Thm}
        Пусть $f : \R^n \to \R^m$, $A \in \R^m$. Тогда равносильны следующие условия:
        \begin{MyList}
            \item $\lim_{x \to a} f(x) = A$ 
            \item $\lim_{x \to a} f_i(x) = A_i$
        \end{MyList}
    \end{Thm}

    \begin{Example}
        Пусть
        \[f(x) = \begin{cases}
            \frac{xy}{x^2 + y^2}, &x^2 + y^2 \neq 0 \\
            0, &x^2 + y^2 = 0
        \end{cases}\]
        Положим
        \[z^k = \left(\frac{1}{k}, 0\right), \quad w^k = \left(\frac{1}{k}, \frac{1}{k}\right), \quad (k \in \N)\]
        Тогда $z^k \to (0, 0)$ и $w^k \to (0, 0)$ при $k \to \infty$. Кроме того,
        \[f(z^k) \to 0, \quad f(w^k) = \frac{1 / k^2}{2 / k^2} \to \frac{1}{2}\]
        Если бы предел $f$ в точке $(0, 0)$ существовал, то он был бы равен одновременно $0$ и $\frac{1}{2}$, что невозможно.  
    \end{Example}

    \begin{Example}
        Пусть
        \[\lim_{(x, y) \to (0, 0)} \frac{x^2 - y^2}{x^2 + y^2}\]
        Заметим, что 
        \[\lim_{x \to 0} \lim_{y \to 0} f(x, y) = 1, \quad \lim_{y \to 0}\lim_{x \to 0} f(x, y) = -1\]
        $\left(\frac{1}{k}, 0\right), \left(0, \frac{1}{k}\right)$ -- по Гейне предела не существует.
    \end{Example}

    \begin{Thm}[Арифметические действия с пределами]
        Пусть $E \subset \R^n$, $f, g : E \to \R^m$, $a \in \overline{\R^n}$,
        $\lim_{x \to a} f(x) = A$, $\lim_{x \to a} g(x) = B$, $A, B \in \R^m$ . Тогда
        \begin{MyList}
            \item $\lim_{x \to a} f(x) + g(x) = A + B$
            \item Если $\lambda \in \R$, то $\lim_{x \to a} \lambda f(x) = \lambda A$  
            \item[2'.] Если $\lambda(x) : E \to \R$, $\lim_{x \to a} \lambda(x) = \alpha \in \R$, то 
            \[\lim_{x \to a} \lambda(x) f(x) = \alpha A\]
            \item $\langle f, g\rangle \xrightarrow[x \to a]{} \langle A, B\rangle$
            \item Если $m = 1$, $B \neq 0$, то $\lim_{x \to a} \frac{f(x)}{g(x)} = \frac{A}{B}$ 
        \end{MyList}
    \end{Thm}

    \begin{Thm}[Предел композиции]
        Пусть $E \subset \R^n$, $F \subset \R^m$, $f : E \to F$, $g : F \to \R^s$, 
        $\lim_{x \to a} f(x) = b, \lim_{x \to b} g(x) = c$. Тогда
        \[\lim_{x \to a} g(f(x)) = c\] 
    \end{Thm}

    \begin{Thm}[Критерий Больцано-Коши]
        Пусть $E \subset \R^n$, $a \in \overline{\R^n}$ -- предельная точка $E$, $f : E \to \R^m$. 
        Тогда равносильны следующие условия:
        \begin{MyList}
            \item $f$ имеет пределом в точке $a$ точку в $\R^m$ 
            \item $\forall \varepsilon > 0 \ \exists V_a : \forall x_1, x_2 \in V_a \cap E \ ||f(x_1) - f(x_2)|| < \varepsilon$
        \end{MyList}
    \end{Thm}

    \Subsection{Непрерывность отображений}

    \begin{Def}[Непрерывные отображения]
        $E \subset \R^n$, $f : E \to \R^m$, $a \in \R^n$. Отображение $f$ называется \textit{непрерывным} в точке $a$, если 
        \[\forall \varepsilon > 0 \ \exists \delta : \forall x \in E \cap V_\delta(a) \ f(x) \in V_\varepsilon (f(a))\]
    \end{Def}

    \begin{Rem}
        Если точка $a$ -- изолированная точка $E$, то отображение в $a$ всегда непрерывно.
    \end{Rem}

    \begin{Rem}[Непрерывность на языке неравенств]
        \[\forall \varepsilon > 0 \ \exists \delta > 0 : \forall x \ ||x - a|| < \delta \ ||f(x) - f(a)|| < \varepsilon\]
    \end{Rem}

    \begin{Rem}
        $f$ непрерывна $\EQ$ все её координатные функции непрерывны.
    \end{Rem}

    \begin{Thm}[Непрерывность композиции]
        Пусть $E \subset \R^n$, $F \subset \R^m$, $f : E \to F$, $g : F \to \R^s$, 
        Если $f$ непрерывно в точке $a$ и $g$ непрерывно в точке $f(a)$, то $g \circ f$ непрерывно в точке $a$.
    \end{Thm}

    \begin{proof}
        Пусть $\{x^k\}_{k = 1}^\infty$ -- последовательность в $E$, стремящаяся к $a$. Тогда
        \[f(x^k) \to f(a), \quad g(f(x^k)) \to g(f(a))\]
        Таким образом, $g(f(x^k)) \to (g \circ f)(a)$ при $k \to \infty$, что и означает непрерывность $g \circ f$ в точке $a$. 
    \end{proof}

    \begin{Thm}
        Пусть $E \subset \R^n$, $f, g : E \to \R^m$ непрерывны в точке $a \in E$. Тогда 
        \begin{MyList}
            \item $f + g$ непрерывно в точке $a$
            \item Если функция $\lambda : E \to \R$ непрерывна в точке $a$, то $\lambda f$ непрерывно в точке $a$
            \item $f \cdot g$ непрерывно в точке $a$
            \item Если $m = 1$ и $g(a) \neq 0$, то функция $\frac{f}{g}$ непрерывна в точке $a$.
        \end{MyList}
    \end{Thm}
    
    \begin{Def}[Непрерывное на множестве отображение]
        Будем называть отображение $f : E \to \R^m$, $E \subset \R^n$ \textit{непрерыным на } $E$, если оно непрерывно в каждой точке $E$. 
    \end{Def}

    \begin{notation}
        $C(E \to \R^m)$, $C(E, \R^m)$, $C(\R^n, \R^m)$.
    \end{notation}

    \begin{Thm}[Непрерывный образ компакта]
        Пусть $E \subset \R^n$ -- компактное, $f$ непрерывно на $E$.
        Тогда образ множества $E$ -- компакт.
        Таким образом, непрерывный образ компакта -- компакт.
    \end{Thm}

    \begin{proof}
        Рассмотри $f(E) \subset \R^m$. Пусть $\{y_k\} \in f(E)$ .
        Тогда $\exists x_k \in E : f(x_k) = y_k$. Поскольку $E$ -- компакт, то $\exists \{x_{k_l}\} : \lim_{l \to \infty} x_{k_l} = a \in E$.
        Рассмотрим теперь последовательность $y_{k_l} = f(x_{k_l})$. При $l \to \infty$ она стремится к $f(a)$ (по непрерывности). $a \in E \SO f(a) \in f(E)$ 
    \end{proof}

    \begin{Thm}[Вейерштрасса]
        Пусть $E \subset \R^n$, $f : E \to \R^m$ -- непрерывна на $E$, $E$ -- компакт.
        Тогда 
        \begin{MyList}
            \item $f$ ограничена на $E$
            \item Если $m = 1$, то $f$ достигает своего наибольшего и наименьшего значения.
        \end{MyList}
    \end{Thm}

    \begin{proof}
        \begin{MyList}
            \item По предыдущей теореме.
            \item Положим
            \[M = \sup_{x \in E} f(x) \in \R\]
            Построим последовательность $\{y_k\} : M - \frac{1}{k} < y_k \leqslant M$, $y_k \in f(E)$.
            $y_k \to M \SO M \in E \SO M = \displaystyle \max_{x \in E} f(x) \in \R$.
        \end{MyList}
    \end{proof}

    \begin{Rem}
        Любой отрезок в $\R^n$ есть компакт.
        \[\Delta_{a, b} = \{a + t(b - a), \ t \in [0, 1], \ a, b \in \R^n\}\]
    \end{Rem}

    \begin{Thm}
        Пусть $E \subset \R^n$, $f : E \to \R^m$. Тогда равносильны следующие условия
        \begin{MyList}
            \item $f$ непрерывно на $E$
            \item $\forall $ открытого $G \subset \R^m \ f^{-1}(G)$ открыто в $E$.
        \end{MyList}
    \end{Thm}

    \Subsection{Линейные отображения}

    \begin{Def}
        Пусть $E \subset \R^n$, $T : E \to \R^m$. Будем называть отображение $f$ \textit{линейным}, если 
        $\forall x, y \in E, \ \forall \alpha, b \in \R$ 
        \[T(\alpha x + \beta y) = \alpha T(x) + \beta T(y)\]
    \end{Def} 

    \begin{notation}
        $T(x) = T_x$
    \end{notation}

    \begin{notation}
        $\mathbb{O}_x = 0 \ \forall x \in \R^m$
    \end{notation}

    \begin{notation}
        $x_k = \begin{pmatrix}
        k \\ 
        k \\ 
        \vdots \\ 
        k
        \end{pmatrix}$, $T(x_k) = k \cdot T(x_1)$  
    \end{notation}

    \begin{Rem}[Композиция линейных операторов]
        Если $S \in \mathcal{L} (\R^n, \R^m)$, $T \in \mathcal{L}(\R^m, \R^l)$ 
        Тогда $T(S(x)) \in \mathcal{L} (\R^n, \R^l)$. 
    \end{Rem}    

    \begin{Thm}
        Пусть $T \in \mathcal{L}(\R^n, \R^m)$. Тогда
        \begin{MyList} 
            \item $T$ непрерывен на $\R^n$
            \item $\exists C \geqslant 0 : ||T_x|| \leqslant C \cdot ||x|| \ \forall x \in \R^n$
            \item Пусть $E$ -- ограниченное подмножество $\R^n$. Тогда $T(E)$ ограничен в $\R^m$.
        \end{MyList}
    \end{Thm}

    \begin{proof}
        \begin{MyList} 
            \item[2.] Ранее было доказано, что сфера $S = \{x \in \R^n : ||x|| = 1\}$ компактна в $\R^n$. Положим
            \[C = \sup_{x \in S} ||T_x||\]
            Так как сам оператор $T$ непрерывен, то по теореме Вейерштрасса $C < +\infty$. При $x = 0$ требуемое неравенство очевидно. 
            Для $x \neq 0$ положим $z = \frac{x}{||x||}$. Тогда $z \in S$, и в силу линейности $T$ 
            \[||T_x|| = ||T(z||x||) = ||T_z|| \cdot ||x|| \leqslant C \cdot ||x||\]
            \item Пусть множество $E$ ограничено в $\R^n$. Тогда существует такое $\delta > 0$, что $E \subset V_\delta(0)$. По предыдущему пункту,
            \[||T_x|| \leqslant C ||x|| \leqslant C \delta\]    
            Поэтому $T(E) \subset V_{C \delta}(0)$, что и дает ограниченность $T(E)$.
        \end{MyList}
    \end{proof}

    \begin{Def}[Операторная норма]
        $||T|| = \displaystyle{\sup \frac{||T_x||}{||x||}, \ x \in \R^n \setminus \{0\}}$.
    \end{Def}

    \begin{Rem}
        Если нам удалось подобрать $C$, что выполняется неравенство из пункта 2 теоремы, то $||T|| \leqslant C$. 
        Если же для какого-то $x \in \R^n \setminus \{0\}$ верно неравенство $||T_x|| \geqslant C||x||$, то $||T|| \geqslant C$.
    \end{Rem}

    \begin{Rem}
        $||T|| = \displaystyle{\sup_{||x|| = 1} ||T_x|| = \sup_{||x|| \leqslant 1} ||T_x||}$
    \end{Rem}

    \begin{proof}
        Пусть $A = \displaystyle{\sup_{||x|| = 1} ||T_x||}$, $B = \displaystyle{\sup_{||x|| \leqslant 1} ||T_x||}$ 
        \begin{MyList} 
            \item $A \leqslant B $ -- очевидно.
            \item \[
                ||T|| = \sup_{x \neq 0} \frac{||T_x||}{||x||} = \sup_{x \neq 0} \left|\left|T\left(\frac{x}{||x||}\right)\right|\right| = A
            \]
            \item \[||T|| = \sup_{x \neq 0} \frac{||T_x||}{||x||} \geqslant \sup_{0 < ||x|| \leqslant 1} \frac{||T_x||}{||x||} \geqslant \sup_{||x|| \leqslant 1} ||T_x|| = B\]
        \end{MyList}
    \end{proof}

    \begin{Thm}
        Операторная норма -- это действительно норма.
    \end{Thm}

    \begin{proof}
        \begin{MyList} 
            \item Неравенство $||T|| \geqslant 0$ очевидно. Если $||T|| = 0$, то 
            \[||T_x|| \leqslant ||T|| \cdot ||x|| = 0 \quad \forall x \in \R^n\]
            откуда $T_x = 0$ для любых $x \in \R^n$, поэтому $T = \mathbb{O}$ 
            \item $||\lambda T|| = \displaystyle{\sup_{||x|| = 1} || \lambda T(x)||}$
            \item Возьмем $x : ||x|| = 1$. Тогда
            \[||T + S|| = \sup_{||x|| = 1} ||T_x + S_x|| \leqslant \sup_{||x|| = 1} (||T_x|| + ||S_x||)\leqslant \sup_{||x|| = 1} ||T_x|| + \sup_{||x|| = 1} ||S_x||\]
        \end{MyList}
    \end{proof}

    \begin{Thm}[Оценка нормы линейного оператора]
        $T \in \mathcal{L}(\R^n, \R^m)$, $A$ -- матрица оператора $T$.
        Тогда
        \[||T|| \leqslant \sqrt{\sum_{i = 1}^m \sum_{j = 1}^n a_{ij}^2}\]
    \end{Thm}

    \begin{proof}
        Хотим доказать, что $\forall x \in \R^n$ 
        \[||T_x|| \leqslant \sqrt{\sum_{i = 1}^m \sum_{j = 1}^n a_{ij}^2} \cdot ||x||\]
        \[||T_x||^2 = \sum_{i=1}^{m} (T_i(x))^2 = \sum_{i=1}^{m} \left(\sum_{j=1}^{n} a_{ij} \cdot x\right)^2 \leqslant \sum_{i=1}^{m} \left(\sum_{j=1}^{n} (a_{ij})^2 \cdot \sum_{j=1}^{n} x_j^2\right)\]
    \end{proof}

    \begin{Thm}
        Пусть $A$ -- матрица размером $n \times n$. Тогда равносильны следующие утверждения:
        \begin{MyList} 
            \item $\det A \neq 0$ 
            \item Неоднородная система
            \[A \cdot x = y, \quad y - n \times 1\]
            имеет единственное решение при каждом $y \in \R^n$
            \item Однородная система 
            \[A \cdot x = 0\]
            имеет только нулевое решение.
        \end{MyList}
    \end{Thm}
\end{document}