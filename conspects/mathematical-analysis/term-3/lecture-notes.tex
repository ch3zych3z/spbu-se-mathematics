\documentclass[12pt]{article}

% Автор: Илья Дудников
% Автор стиля: Сергей Копелиович

\usepackage{cmap}
\usepackage[T2A]{fontenc}
\usepackage[utf8]{inputenc}
\usepackage[russian]{babel}
\usepackage{graphicx}
\usepackage{amsthm,amsmath,amssymb}
\usepackage{listings}
\usepackage{color}
\usepackage{array}
\usepackage{epigraph}
\usepackage{multicol}
\usepackage{cancel}
\usepackage{float}
\usepackage{wrapfig}
\usepackage{caption}
\usepackage{subcaption}

\usepackage[usenames,dvipsnames]{xcolor}
\usepackage[russian,colorlinks=true,urlcolor=red,linkcolor=blue]{hyperref}
\usepackage{enumerate}
\usepackage{datetime}
\usepackage{fancyhdr}
\usepackage{lastpage}
\usepackage{verbatim}
\usepackage{tikz}
\usepackage{MnSymbol}
\usetikzlibrary{arrows,decorations.markings,decorations.pathmorphing}
\usepackage{pgfplots}
\usepackage{ifthen}
\usepackage{mathtools}
\usepackage{mathrsfs}

%\usepackage{tabls}
%\usepackage{tabularx}
%\usepackage{xifthen}
%\listfiles

\def\NAME{Лекции}
\def\SEASON{Конспект лекций по дискретной математике, ПИ, 2 семестр}

\sloppy
\voffset=-20mm
\textheight=235mm
\hoffset=-22mm
\textwidth=180mm
\headsep=12pt
\footskip=20pt

\parskip=0em
\parindent=0em

\setlength\epigraphwidth{.8\textwidth}

\newlength{\tmplen}
\newlength{\tmpwidth}
\newcounter{listcounter}

% Список с маленькими отступами
\newenvironment{MyList}[1][4pt]{
  \begin{enumerate}[1.]
  \setlength{\parskip}{0pt}
  \setlength{\itemsep}{#1}
}{       
  \end{enumerate}
}
% Вложенный список с маленькими отступами
\newenvironment{InnerMyList}[1][0pt]{
  \vspace*{-0.5em}
  \begin{enumerate}[(a)]
  \setlength{\parskip}{-0pt}
  \setlength{\itemsep}{#1}
}{       
  \end{enumerate}
  \vspace*{-0.5em}
}
% Список с маленькими отступами
\newenvironment{MyItemize}[1][4pt]{
  \begin{itemize}
  \setlength{\parskip}{0pt}
  \setlength{\itemsep}{#1}
}{       
  \end{itemize}
}

% Основные математические символы
\def\TODO{{\color{red}\bf TODO}}
\def\C{\mathbb{C}}       %
\def\Q{\mathbb{Q}}       %
\def\N{\mathbb{N}}       %
\def\R{\mathbb{R}}       %
\def\F2{\mathbb{F}_2}    %
\def\Z{\mathbb{Z}}       %
\def\INF{\t{+}\infty}    % +inf
\def\EPS{\varepsilon}    %
\def\EMPTY{\varnothing}  %
\def\PHI{\varphi}        %
\def\SO{\Rightarrow}     % =>
\def\EQ{\Leftrightarrow} % <=>
\def\t{\texttt}          % mono font
\def\c#1{{\rm\sc{#1}}}   % font for classes NP, SAT, etc
\def\O{\mathcal{O}}      %
\def\NO{\t{\#}}          % #
\def\XOR{\text{ {\raisebox{-2pt}{\ensuremath{\Hat{}}}} }}
\renewcommand{\le}{\leqslant}
\renewcommand{\ge}{\geqslant}
\newcommand{\q}[1]{\langle #1 \rangle}               % <x>
\newcommand\URL[1]{{\footnotesize{\url{#1}}}}        %
% \newcommand{\sfrac}[2]{{\scriptscriptstyle\frac{#1}{#2}}}  % Очень маленькая дробь
% \newcommand{\mfrac}[2]{{\scriptstyle\frac{#1}{#2}}}    % Небольшая дробь
\newcommand{\sfrac}[2]{{\scriptstyle\frac{#1}{#2}}}  % Очень маленькая дробь
\newcommand{\mfrac}[2]{{\textstyle\frac{#1}{#2}}}    % Небольшая дробь

\newcommand{\fix}[1]{{\color{fixcolor}{#1}}} % \underline
\def\bonus{\t{\red{(*)}}}
\def\ifbonus#1{\ifthenelse{\equal{#1}{}}{}{\bonus}}
\def\smallsquare{$\scalebox{0.5}{$\square$}$}

\newlength{\myItemLength}
\setlength{\myItemLength}{0.3em}
\def\ItemSymbol{\smallsquare}
\def\Item{\vspace*{\myItemLength}\ItemSymbol \ \ }

\newcommand{\LET}{%
  % [line width=0.6pt]
  \begin{tikzpicture}%
  \draw(0.8ex,0) -- (0.8ex,1.6ex);%
  \draw(0,1.6ex) -- (0.8ex,1.6ex);%
  \end{tikzpicture}%
  \hspace*{0.1em}%
}

% Отступы
\def\makeparindent{\hspace*{\parindent}\unskip}
\def\up{\vspace*{-0.5em}}%{\vspace*{-\baselineskip}}
\def\down{\vspace*{0.5em}}
\def\LINE{\vspace*{-1em}\noindent \underline{\hbox to 1\textwidth{{ } \hfil{ } \hfil{ } }}}
\def\BOX#1{\mbox{\fbox{\bf{#1}}}}
\def\Pagebreak{\pagebreak\vspace*{-1.5em}}

% Мелкий заголовок
\newcommand{\THEE}[1]{
  \vspace*{0.5em}
  \noindent{\bf \underline{#1}}%\hspace{0.5em}
  \vspace*{0.2em}
}
% Другой тип мелкого заголовка
\newcommand{\THE}[1]{
  \vspace*{0.5em} $\bullet$
  \noindent{\bf #1}%\hspace{0.5em}
  \vspace*{0.2em}
}

\newenvironment{MyTabbing}{
  \t\bgroup
  \vspace*{-\baselineskip}
  \begin{tabbing}
    aaaa\=aaaa\=aaaa\=aaaa\=aaaa\=aaaa\kill
}{
  \end{tabbing}
  \t\egroup
}

% Код с правильными отступами
\lstnewenvironment{code}{
  \lstset{}
%  \vspace*{-0.2em}
}%
{
%  \vspace*{-0.2em}
}
\lstnewenvironment{codep}{
  \lstset{language=python}
}%
{
}

% Формулы с правильными отступами
\newenvironment{smallformula}{
 
  \vspace*{-0.8em}
}{
  \vspace*{-1.2em}
  
}
\newenvironment{formula}{
 
  \vspace*{-0.4em}
}{
  \vspace*{-0.6em}
  
}

% Большая квадратная скобка
\makeatletter
\newenvironment{sqcases}{%
  \matrix@check\sqcases\env@sqcases
}{%
  \endarray\right.%
}
\def\env@sqcases{%
  \let\@ifnextchar\new@ifnextchar
  \left\lbrack
  \def\arraystretch{1.2}%
  \array{@{}l@{\quad}l@{}}%
}
\makeatother

% theorems
\makeatother
\usepackage{thmtools}
\usepackage[framemethod=TikZ]{mdframed}
\mdfsetup{skipabove=1em,skipbelow=0em}


\theoremstyle{definition}

\declaretheoremstyle[
    headfont=\bfseries\sffamily\color{ForestGreen!70!black}, bodyfont=\normalfont,
    mdframed={
        linewidth=2pt,
        rightline=false, topline=false, bottomline=false,
        linecolor=ForestGreen, backgroundcolor=ForestGreen!5,
    }
]{thmgreenbox}

\declaretheoremstyle[
    headfont=\bfseries\sffamily\color{NavyBlue!70!black}, bodyfont=\normalfont,
    mdframed={
        linewidth=2pt,
        rightline=false, topline=false, bottomline=false,
        linecolor=NavyBlue, backgroundcolor=NavyBlue!5,
    }
]{thmbluebox}

\declaretheoremstyle[
    headfont=\bfseries\sffamily\color{NavyBlue!70!black}, bodyfont=\normalfont,
    mdframed={
        linewidth=2pt,
        rightline=false, topline=false, bottomline=false,
        linecolor=NavyBlue
    }
]{thmblueline}

\declaretheoremstyle[
    headfont=\bfseries\sffamily\color{RawSienna!70!black}, bodyfont=\normalfont,
    mdframed={
        linewidth=2pt,
        rightline=false, topline=false, bottomline=false,
        linecolor=RawSienna, backgroundcolor=RawSienna!5,
    }
]{thmredbox}

\declaretheoremstyle[
    headfont=\bfseries\sffamily\color{RawSienna!70!black}, bodyfont=\normalfont,
    numbered=no,
    mdframed={
        linewidth=2pt,
        rightline=false, topline=false, bottomline=false,
        linecolor=RawSienna, backgroundcolor=RawSienna!1,
    },
    qed=\qedsymbol
]{thmproofbox}

\declaretheoremstyle[
    headfont=\bfseries\sffamily\color{NavyBlue!70!black}, bodyfont=\normalfont,
    numbered=no,
    mdframed={
        linewidth=2pt,
        rightline=false, topline=false, bottomline=false,
        linecolor=NavyBlue, backgroundcolor=NavyBlue!1,
    },
]{thmexplanationbox}

\declaretheorem[style=thmgreenbox, name=Определение]{Def}
\declaretheorem[style=thmbluebox, numbered=no, name=Пример]{Example}
\declaretheorem[style=thmredbox, name=Утверждение]{Prop}
\declaretheorem[style=thmredbox, name=Теорема]{Thm}
\declaretheorem[style=thmredbox, name=Алгоритм]{Algo}
\declaretheorem[style=thmredbox, name=Свойство]{Property}
\declaretheorem[style=thmredbox, name=Лемма]{Lm}
\declaretheorem[style=thmredbox, numbered=no, name=Следствие]{Cons}

\declaretheorem[style=thmproofbox, name=Доказательство]{replacementproof}
\renewenvironment{proof}[1][\proofname]{\vspace{-10pt}\begin{replacementproof}}{\end{replacementproof}}


\declaretheorem[style=thmexplanationbox, name=Доказательство]{tmpexplanation}
\newenvironment{explanation}[1][]{\vspace{-10pt}\begin{tmpexplanation}}{\end{tmpexplanation}}

\declaretheorem[style=thmblueline, numbered=no, name=Замечание]{Rem}
\declaretheorem[style=thmblueline, numbered=no, name=Note]{note}

\newtheorem*{notation}{Обозначение}
\newtheorem*{Ex}{Упражнение}

% Определяем основные секции: \begin{Lm}, \begin{Thm}, \begin{Def}, \begin{Rem}
% \renewcommand{\qedsymbol}{$\blacksquare$}
% \theoremstyle{definition} % жирный заголовок, плоский текст
% \newtheorem{Thm}{\underline{Теорема}}[subsection] % нумерация будет "<номер subsection>.<номер теоремы>"
% \newtheorem{Lm}[Thm]{\underline{Lm}} % Нумерация такая же, как и у теорем
% \newtheorem{Ex}[Thm]{Упражнение} % Нумерация такая же, как и у теорем
% \newtheorem{Example}[Thm]{Пример} % Нумерация такая же, как и у теорем
% \newtheorem{Code}[Thm]{Код} % Нумерация такая же, как и у теорем
% \theoremstyle{plain} % жирный заголовок, курсивный текст
% \newtheorem{Def}[Thm]{Def} % Нумерация такая же, как и у теорем
% \theoremstyle{remark} % курсивный заголовок, плоский текст
% \newtheorem{Cons}[Thm]{Следствие} % Нумерация такая же, как и у теорем
% \newtheorem{Conj}[Thm]{Гипотеза} % Нумерация такая же, как и у теорем
% \newtheorem{Prop}[Thm]{Утверждение} % Нумерация такая же, как и у теорем
% \newtheorem{Rem}[Thm]{Замечание} % Нумерация такая же, как и у теорем
% \newtheorem{Remark}[Thm]{Замечание} % Нумерация такая же, как и у теорем
% \newtheorem{Algo}[Thm]{Алгоритм} % Нумерация такая же, как и у теорем

% Определяем ЗАГОЛОВКИ
\def\SectionName{unknown}
\def\AuthorName{unknown}

\newlength{\sectionvskip}
\setlength{\sectionvskip}{0.5em}
\newcommand{\Section}[4][]{
  % Заголовок
  \pagebreak
%  \ifthenelse{\isempty{#1}}{
    \refstepcounter{section}
%  }{}
  \vspace{0.5em}
%  \ifthenelse{\isempty{#1}}{
%    \addtocontents{toc}{\protect\addvspace{-5pt}}%
    \addcontentsline{toc}{section}{\arabic{section}. #2}
%  }{}
  \begin{center}
    {\Large \bf Раздел \NO{\arabic{section}}: #2} \\ 
    \vspace{\sectionvskip}
    \ifthenelse{\equal{#3}{}}{}{{\large #3}\\}
  \end{center}

  \LINE

  % Запомнили название и автора главы
  \gdef\SectionName{#2}
  \gdef\AuthorName{#4}

  % Заголовок страницы
  \lhead{\SEASON}
  \chead{}
  \rhead{\SectionName}
  \renewcommand{\headrulewidth}{0.4pt}

  \lfoot{Глава \NO{\arabic{section}}.}
  \cfoot{\thepage\t{/}\pageref*{LastPage}}
  \rfoot{Автор: \AuthorName}
  \renewcommand{\footrulewidth}{0.4pt}
}

\newcommand{\Subsection}[2][]{
  \refstepcounter{subsection}
  \vspace*{1em}
  \ifthenelse{\equal{#1}{}}
    {\addcontentsline{toc}{subsection}{\arabic{section}.\arabic{subsection}. #2}}
    {\addcontentsline{toc}{subsection}{\arabic{section}.\arabic{subsection}. \bonus\,#2}}
  {\color{blue}\bf\large \arabic{section}.\arabic{subsection}. \ifbonus{#1}\,{#2}} 
  \vspace*{0.5em}
  \makeparindent
}
\newcommand{\Subsubsection}[2][]{
  \refstepcounter{subsubsection}
  \vspace*{1em}
  \ifthenelse{\equal{#1}{}}
    {\addcontentsline{toc}{subsubsection}{\arabic{section}.\arabic{subsection}.\arabic{subsubsection}. #2}}
    {\addcontentsline{toc}{subsubsection}{\arabic{section}.\arabic{subsection}.\arabic{subsubsection}. \bonus\,#2}}
  {\color{blue}\bf\large \arabic{section}.\arabic{subsection}.\arabic{subsubsection}. \ifbonus{#1}\,#2}
  \vspace*{0.5em}
  \makeparindent
}

\makeatletter
\newcommand*{\encircled}[1]{\relax\ifmmode\mathpalette\@encircled@math{#1}\else\@encircled{#1}\fi}
\newcommand*{\@encircled@math}[2]{\@encircled{$\m@th#1#2$}}
\newcommand*{\@encircled}[1]{%
  \tikz[baseline,anchor=base]{\node[draw,circle,outer sep=0pt,inner sep=.2ex] {#1};}}
\makeatother

\newcommand{\Header}{
  \pagestyle{empty}
  \renewcommand{\dateseparator}{--}
  \begin{center}
    {\Large\bf 
     Дискретная математика \\ 2 семестр ПИ,\\
    \vspace{0.3em}
    \NAME}\\
    \vspace{0.7em}
    {Собрано {\today} в {\currenttime}}
  \end{center}

  \LINE
  \vspace{0em}

  \renewcommand{\baselinestretch}{0.98}\normalsize
  \tableofcontents
  \renewcommand{\baselinestretch}{1.0}\normalsize
  \pagebreak
}

\newcommand{\BeginConspect}{
  \pagestyle{fancy}
  \setcounter{page}{1}
}

\definecolor{mygray}{rgb}{0.7,0.7,0.7}
\definecolor{ltgray}{rgb}{0.9,0.9,0.9}
\definecolor{fixcolor}{rgb}{0.7,0,0}
\definecolor{red2}{rgb}{0.7,0,0}
\definecolor{dkred}{rgb}{0.4,0,0}
\definecolor{dkblue}{rgb}{0,0,0.6}
\definecolor{dkgreen}{rgb}{0,0.6,0}
\definecolor{brown}{rgb}{0.5,0.5,0}

\newcommand{\green}[1]{{\color{green}{#1}}}
\newcommand{\black}[1]{{\color{black}{#1}}}
\newcommand{\red}[1]{{\color{red}{#1}}}
\newcommand{\dkred}[1]{{\color{dkred}{#1}}}
\newcommand{\blue}[1]{{\color{blue}{#1}}}
\newcommand{\dkgreen}[1]{{\color{dkgreen}{#1}}}

\newcommand{\Mod}[1]{\ (\mathrm{mod}\ #1)}

\DeclareMathOperator{\Real}{Re}
\DeclareMathOperator{\Imag}{Im}
\DeclareMathOperator{\lcm}{lcm}
\DeclareMathOperator{\sign}{sign}
\DeclareMathOperator{\Si}{Si}
\DeclareMathOperator{\const}{const}
\DeclareMathOperator{\Arg}{Arg}

\begin{document}
    \Header

    \BeginConspect

    \Section{Метрические пространства}{}{Илья Дудников}

    Пусть $X$ -- некоторое множество. Зададим функцию $\rho : X \times X \to \R$.
    \begin{Def}[Метрика]
        $\rho$ называется метрикой, если выполняются следующие три свойства:
        \begin{MyList}
            \item $\rho (x, y) \geqslant 0 \ \forall x, y \in X$
            \item[] $\rho(x, y) = 0 \EQ x = y$
            \item $\forall x, y \in X \ \rho (x, y) = \rho (y, x)$ 
            \item $\forall x, y, z \in X \ \rho (x, y) \leqslant \rho(x, z) + \rho(z, y)$ -- неравенство треугольника. 
        \end{MyList}
    \end{Def} 

    \begin{Def}[Метрическое пространство]
        Пара $(X, \rho)$ называется \textit{метрическим пространством}.
    \end{Def}

    \begin{Example}
        Метрика на $\R^2$: $\rho_2 (x, y) = \sqrt{(x_1 - y_1)^2 + (x_2 - y_2)^2}$ 
    \end{Example}

    \begin{Example}
        Метрика на $\R^d$: $\rho_2 (x, y) = \sqrt{\sum_{i=1}^{d} (x_i - y_i)^2}$ 
    \end{Example}

    \begin{Example}[Дискретная метрика]
        Пусть $X$ -- некоторое множество. Зададим 
        \[\rho (x, y) = \begin{cases}
            0, &x = y \\
            1, &x\neq y
        \end{cases}\]
        Действительно, все свойства выполняются, поэтому $\rho$ -- метрика.
    \end{Example}

    \begin{Example}[Манхэттенская метрика]
        В $\R^2$:
        \[\rho_1(x, y) = |x_1 - y_1| + |x_2 - y_2|\]

        \begin{figure}[H]
            \centering
            \def\svgwidth{.3\columnwidth}
            \input{img/Manhattan_distance.pdf_tex}
        \end{figure}
    \end{Example}

    \begin{Example}
        $\rho_\infty(x, y) = \max \{|x_1 - y_1|, |x_2 - y_2|\}$ 
    \end{Example}

    \begin{Example}
        Рассмотрим $C[a, b]$. Тогда $\rho (f, g) = \int_{a}^{b} |f - g|$ -- метрика.
    \end{Example}

    \begin{notation}[Открытый шар]
        $B_r(a) = \left\{x \in X, \ \rho (x, a) < r\right\}$
    \end{notation}

    \begin{notation}[Замкнутый шар]
        $\overline{B}_r(a) = \left\{x \in X, \ \rho(x, a) \leqslant r\right\}$
    \end{notation}

    \begin{notation}[Сфера]
        $S_r(a) = \{x \in X, \ \rho(x, a) = r\}$
    \end{notation}

    \begin{Example}
        В дискретной метрике при $r < 1$ замкнутый шар $\overline{B}_r(a)$ включает только одну точку -- $a$, 
        а при $r \geqslant 1$ -- всё множество $X$.
    \end{Example}

    \begin{Example}
        Замкнутый шар в манхэттенской метрике: 
        \begin{figure}[H]
            \centering
            \def\svgwidth{.3\columnwidth}
            \input{img/manhattan_closed_ball.pdf_tex}
        \end{figure}
    \end{Example}

    \begin{Example}
        Замкнутый шар в $\rho_\infty$: 
        \begin{figure}[H]
            \centering
            \def\svgwidth{.3\columnwidth}
            \input{img/rho_infty_closed_ball.pdf_tex}
        \end{figure}
    \end{Example}

    \begin{Prop}
        Пусть $B_{r_1}(a)$ и $B_{r_2}(a)$ -- шары. Тогда 
        \[B_{r_1}(a) \cap B_{r_2}(a) = B_{\min (r_1, r_2)} (a)\]
    \end{Prop}

    \begin{proof}
        Возьмем $x \in B_{r_1} (a) \cap B_{r_2} (a)$. Тогда $\rho(x, a) < r_1$ и $\rho(x, a) < r_2$, значит $\rho(x, a) < \min (r_1, r_2)$.  
    \end{proof}

    \begin{Prop}
        $\forall a \neq b \ \exists r : \overline{B}_r(a) \cap \overline{B}_r(b) = \varnothing$.
    \end{Prop}

    \begin{proof}
        Возьмем $r = \frac{\rho(a, b)}{3}$. Предположим, что пересечение непусто, т.е. $\exists x  : x \in \overline{B}_r(a)$ и $x \in \overline{B}_r(b)$.
        Тогда 
        \[\rho(a, b) \leqslant \rho(a, x) + \rho(x, b) \leqslant \frac{\rho(a, b)}{3} + \frac{\rho(a, b)}{3}\]   
    \end{proof}

    \begin{notation}
        $V_x$ -- окрестность точки $x$ (шар).
    \end{notation}

    \begin{notation}
        $\dot{V}_x$ -- проколотая окрестность $x$ (шар, не содержащий точку $x$).
    \end{notation}

    \begin{Def}[Внутренняя точка множества] 
        Пусть $A \subset X$. Точка $a$ называется \textit{внутренней} точкой $A$, если $\exists V_a \subset A$.
    \end{Def}

    \begin{Def}[Внешняя точка множества]
        Пусть $A \subset X$. Тогда точка $b$ называется \textit{внешней} точкой $A$, если $b$ -- внутренняя точка $X \setminus A$. 
    \end{Def}

    \begin{Def}[Граничная точка множества]
        Пусть $A \subset X$. Тогда $c$ является \textit{граничной} точкой множества $A$, если она не является ни внутренней, ни внешней.
        Иначе, точка $c$ назывется граничной, если
        \[\forall V_c \ \exists x, y \in V_c : x \in A \wedge y \in X \setminus A\]        
    \end{Def}

    \begin{Def}[Открытое множество]
        Множество $A \subset X$ называется \textit{открытым}, если любая его точка -- внутренняя.
    \end{Def}

    \begin{Thm}[Об открытых множествах]
        \begin{MyList}
            \item $\varnothing$ и $X$ -- открытые множества
            \item Объединение любого числа открытых множеств -- открытое множество
            \item Пересечение конечного числа открытых множеств -- открытое множество
            \item Открытый шар -- это открытое множество  
        \end{MyList}
    \end{Thm}

    \begin{proof}
        \begin{MyList}
            \item Очевидно
            \item Пусть $B = \bigcup_{\alpha \in I} A_\alpha$. Возьмем $x \in B$. Тогда $\exists \beta \in I : x \in A_\beta$. Т.к. $A_\beta$ -- открытое множество, то 
            $x$ принадлежит $A_\beta$ с какой-то своей окрестностью, а значит она принадлежит и всему объединению с этой окрестностью.

            \item Пусть $B = \bigcap_{i = 1}^n A_i$. Возьем $x \in B$. Тогда $x \in A_i \ \forall i$. Точка $x$ принадлежит всем $A_i$ с какой-то круговой окрестностью $r_i$. Тогда она принадлежит пересечению с круговой окрестностью $\min r_i$.    
            \item Рассмотрим $B_R(a) = \{x \in X, \ \rho(x, a) < R\}$. Пусть точка $x \in B_R(a), \ \rho(x, a) < R$. Положим $r = R - \rho(x, a)$. Возьмем $y$ из окрестности $x$ радиуса $r$. Тогда в силу неравенства треугольника
            \[\rho(a, y) \leqslant \rho(a, x) + \rho(x, y) < \rho(a, x) + R - \rho(x, a) = R\]
            \begin{figure}[H]
                \centering
                \def\svgwidth{.3\columnwidth}
                \input{img/open_ball_theorem.pdf_tex}
            \end{figure}

            А значит $\forall y \in V_x(r) \ y \in B_R(a)$, т.е. любая точка $x \in B_R(a)$ принадлежит шару $B_R(a)$ с какой-то своей окрестностью.
        \end{MyList}
    \end{proof}

    \begin{Rem}
        Конечность в пункте 3 существенна: рассмотрим $\bigcap_{n = 1}^\infty \left(-\frac{1}{n}; 1\right) = [0, 1)$. 
    \end{Rem}

    \begin{notation}
        $\Int A$ -- множество всех внутренних точек множества $A$. 
    \end{notation}

    \begin{Thm}[Свойства]
        \begin{MyList}
            \item $\Int A \subset A$
            \item $\Int A = \bigcup$ всех открытых множеств, которые содержатся в $A$
            \item $\Int A$ -- открытое множество
            \item $A$ -- открытое $\EQ A = \Int A$
            \item $A \subset B \SO \Int A \subset \Int B$
            \item $\Int (A \cap B) = \Int A \cap \Int B$
            \item $\Int (\Int A) = \Int A$   
        \end{MyList}
    \end{Thm}

    \begin{Def}[Замкнутое множество]
        Множество $A \subset X$ называется \textit{замкнутным}, если $X \setminus A$ -- открыто.
    \end{Def}

    \begin{Thm}[О замкнутых множествах]
        \begin{MyList}
            \item $\varnothing, \ X$ -- замкнутые множества.
            \item Пересечение любого числа замкнутых множеств -- замкнутое множество
            \item Конечное объединение замкнутых множеств -- замкнутое множество
            \item Замкнутый шар -- это замкнутое множество.
        \end{MyList}
    \end{Thm}

    \begin{proof}
        \begin{MyList}
            \item Очевидно
            \item Пусть $B = \bigcap_{\alpha \in I} A_\alpha$. Тогда
            \[X \setminus B = X \setminus \left(\bigcap_{\alpha \in I} A_\alpha\right) = \bigcup_{\alpha \in I} \left(X \setminus A_\alpha\right)\]
            Поскольку $\forall \alpha \in I \ A_\alpha$ -- замкнутое, т.е. $X \setminus A_\alpha$ -- открытое, то $X \setminus B$ -- открытое (по теореме об открытых множествах), значит $B$ -- замкнутое множество.
            \item Доказывается аналогично предыдущему пункту.
            \item Рассмотрим $\overline{B}_R(a)$. Возьмем $x \in X \setminus \overline{B}_R(a)$ и $y$ из окрестности $x$, т.е. $y \in B_r(x)$, где $r = \rho(a, x) - R$.
            По неравенству треугольника:
            \[\rho(a, x) \leqslant \rho(a, y) + \rho(y, x)\]
            Поэтому
            \[\rho (a, y) \geqslant r + R - \rho(y, x) > R\]
            \begin{figure}[H]
                \centering
                \def\svgwidth{.3\columnwidth}
                \input{img/closed_ball_theorem.pdf_tex}
            \end{figure}
        \end{MyList}
    \end{proof}

    \begin{Rem}
        Конечность в пункте 3 существенна: $\bigcup \left[\frac{1}{n}; 1\right] = (0; 1]$ -- незамкнутое множество.
    \end{Rem}

    Пусть $E \subset \R^n, \ F \subset E$.

    \begin{Def}
        Точка $a \in F$ называется внутренней для $F$ в $E$, если $\exists V_a^E \subset F$.
    \end{Def}

    \begin{Def}
        $F$ называется открытым в $E$, если все его точки внутренние в $E$.
    \end{Def}

    \begin{Rem}
        Если множество $E$ открыто, то ничего не изменилось. Если же множество $E$ не открыто, то мы получаем новые определения. 
    \end{Rem}

    \begin{Example}
        Множество $(0, 1) \cap \Q$ открыто в $\Q$.
    \end{Example}

    \begin{Example}
        $(1, 2]$ открыто в $(0, 2]$.  
    \end{Example}

    \begin{Rem}
        Множество $F$ открыто в $E \EQ \exists G$ -- открытое в $\R^n : F = E \cap G$. 
    \end{Rem}

    \begin{Example}
        $(0, 1]$ -- замкнуто в $(0, 2]$.
    \end{Example}

    \begin{notation}[Замыкание множества]
        $\Cl A$ -- пересечение всех замкнутых множеств, которые содержат $A$. 
    \end{notation}

    \begin{Example}
        $\Cl (0, 1) = [0, 1]$. 
    \end{Example}

    \begin{Thm}[Свойства]
        \begin{MyList}
            \item $A \subset \Cl A$
            \item $\Cl A$ -- замкнутое множество
            \item $A$ -- замкнуто $\EQ A = \Cl A$.
            \item $A \subset B \SO \Cl A \subset \Cl B$
            \item $\Cl (A \cup B) = \Cl A \cup \Cl B$
            \item $\Cl (\Cl A) = \Cl A$      
        \end{MyList}
    \end{Thm}

    \begin{Thm}
        $x \in \Cl A \EQ \forall r > 0 \ B_r(x) \cap A \neq \varnothing$. 
    \end{Thm}

    \begin{proof}
        Докажем, что $x \notin \Cl A \EQ \exists r > 0 \ B_r(x) \cap A = \varnothing$.
        \[x \in \left(X \setminus \Cl A\right) \EQ x \in \Int (X \setminus A)\]
    \end{proof}

    Пусть $A'$ -- множество предельных точек $A$. Тогда
    \begin{Thm}[Свойства]
        \begin{MyList}
            \item $\Cl A = A \cup A'$ 
            \item $A \subset B \SO A' \subset B'$ 
            \item $A$ -- замкнуто $\EQ A' \subset A$
        \end{MyList}
    \end{Thm}

    \begin{proof}
        \begin{MyList}
            \item[3.] $A$ -- замкнуто $\EQ \Cl A = A, \ \Cl A = A \cup A'$.
        \end{MyList}
    \end{proof}

    \begin{Thm}
        $x \in A' \EQ \forall B_r(x)$ содержит бесконечно много точек из $A$.
    \end{Thm}

    \Subsection{Нормированные пространства}

    Пусть $X$ -- векторное пространство над $\R$.
    \begin{Def}[Норма]
        Функция $||\cdot|| : X \to \R$ называется \textit{нормой}, если выпоняются следующие свойства:
        \begin{MyList}
            \item $||x|| \geqslant 0$
            \item[] $||x|| = 0 \EQ x = 0$
            \item $||\lambda x|| = |\lambda| \cdot ||x||, \ \lambda \in \R$
            \item $||x + y|| \leqslant ||x|| + ||y||$.  
        \end{MyList}
    \end{Def}

    \begin{Example}
        $||x||_1 = |x_1| + |x_2|$
    \end{Example}

    \begin{Example}
        $||x||_2 = \sqrt{x_1^2 + x_2^2}$ 
    \end{Example}

    \begin{Example}
        $||x||_\infty = \max_{i \in \{1, ...\}} |x_i|$
    \end{Example}

    \begin{Example}
        $C[a, b]$, $||f|| = \max_{x \in [a, b)} |f|$ 
    \end{Example}

    \begin{Def}[Скалярное произведение]
        $\langle \cdot, \cdot \rangle : X \times X \to \R$ -- скалярное произведение, если выполняются:
        \begin{MyList}
            \item $\langle x, x\rangle \geqslant 0$ 
            \item[] $\langle x, x\rangle = 0 \EQ = x = 0$
            \item $\langle x + y, z\rangle = \langle x, z\rangle + \langle y, z\rangle$
            \item $\langle x, y\rangle = \langle y, x\rangle$
            \item $\langle \lambda x, y\rangle = \lambda \langle x, y\rangle$. 
        \end{MyList} 
    \end{Def}

    \begin{Ex}
        Вспомнить неравенство Коши-Буняковского
    \end{Ex}

    \begin{Prop}
        $||x|| = \sqrt{\langle x, x\rangle}$
    \end{Prop}

    \begin{proof}
        1-2 очевидно.
        \begin{MyList}
            \item[3.]
            \begin{align*}
                \langle x + y, x + y \rangle \leqslant \langle x, x\rangle + \langle y, y\rangle + 2\sqrt{\langle x, x\rangle \cdot \langle y, y\rangle}
            \end{align*}
            С другой стороны:
            \[\langle x + y, x + y \rangle = \langle x, x\rangle + \langle y, y\rangle + 2 \langle x, y\rangle\]
        \end{MyList}
    \end{proof}

    \begin{Def}[Полное пространство]
        Пространство называется \textit{полным}, если в нем любая фундаментальная последовательность сходится.
    \end{Def}

    \begin{Ex}
        Доказать, что $\R^2, \R^3, \R^n$ -- полные пространства.
    \end{Ex}

    \begin{notation}
        $\overline{\R}^d = \R^d \cup \{\infty\}$ 
    \end{notation}

    \begin{Rem}
        Под $V_\infty$ будем понимать $\{x : ||x|| > \delta\}$. 
    \end{Rem}    
    
    \begin{Thm}[Сходимость и покоординатная сходимость] \label{thm:convergence}
        $x^i \in \R^d$ ($x^i = (x_1^i, x_2^i, ..., x_d^i)$). Рассмотрим последовательность $\{x^i\}_{i = 1}^\infty$. Тогда 
        равносильны утверждения: 

        \begin{MyList}
            \item $\{x^i\}_{i = 1}^\infty$ сходится
            \item $\{x^i\}_{i = 1}^\infty$ сходится покоординатно.  
        \end{MyList}
    \end{Thm}

    \begin{proof}
        \begin{MyList}
            \item[] 1 $\SO$ 2.
            \[\forall \varepsilon > 0 \ \exists N : \forall n > N \ ||x^n - a|| < \varepsilon\]

            \[|x_k^n - a_k| \leqslant ||x^n - a||\]

            \item[] 2 $\SO$ 1. 
            \[\sqrt{(x_1^n - a_1)^2 + (x_2^n - a_2)^2 + ... + (x_d^n - a_d)^2}\]
        \end{MyList}
    \end{proof}

    \begin{Rem}
        Если $\lim_{k \to \infty} x^k = \infty$, то координатные последовательности $\{x^k\}$ могут и не иметь предела. Пусть, например, последовательность в $\R^2$ определяется формулой 
        \[x^k = \left(k \cos \frac{\pi k}{2}, k \sin \frac{\pi k}{2}\right)\]
        Тогда 
        \[||x^k|| \sqrt{k^2 \cos^2 \frac{\pi k}{2} + k^2 \sin^2 \frac{\pi k}{2}} = k \to +\infty\]
        То есть $x^k \to \infty$. Тем не менее, последовательности $x_1^k = k \cos \frac{\pi k}{2}$ и $x_2^k = k \sin \frac{\pi k}{2}$ предела не имеют. 
    \end{Rem}

    \begin{Thm}[Арифметические действия и пределы]
        Пусть $\{x^k\}, \ \{y^k\} \subset \R^n$, $\lim x^k = a, \ \lim y^k = b$. Тогда
        \begin{MyList}
            \item $\lim(x^k + y^k) = a + b$
            \item Пусть $\{\lambda_n\}$ -- последовательность из $\R$, $\lim \lambda_k = \lambda$. Тогда
            \[\lim \lambda_k x_k = \lambda a\] 
            \item $\lim ||x^k|| = ||a||$
            \item $\lim \langle x^k, y^k\rangle = \langle a, b\rangle$
        \end{MyList} 
    \end{Thm}

    \begin{proof}
        \begin{MyList}
            \item По теореме \ref{thm:convergence} для любого $i = 1, ..., n$
            \[\lim_{k \to \infty} x_i^k = a_i, \quad \lim_{k \to \infty} y_i^k = b_i\]
            Тогда $\lim_{k \to \infty} (x_i^k + y_i^k) = a_i + b_i$. Применяя теорему \ref{thm:convergence} еще раз, получаем, что
            $\lim_{k \to \infty} (x^k + y^k) = a + b$.
            \item[4.] Заметим, что 
            \begin{align*}
                \frac{1}{4} (||a + b|| - ||a - b||) &= \frac{1}{4}(\langle a + b, a + b\rangle - \langle a - b, a - b\rangle) = \\
                &= \frac{1}{4} (4 \langle a, b\rangle) = \langle a, b\rangle
            \end{align*}
            Применяя пункты 1 и 3, получаем нужное утверждение.
        \end{MyList}
    \end{proof}

    \begin{Def}[Ограниченное множество]
        Множество $E$ называется ограниченным в $\R^n$, если $\exists c : E \subset V_0(c)$,
        т.е. $\forall x \in E \ ||x|| < c$.
    \end{Def}

    \begin{Rem}
        Ограниченность множества в $\R^n$ равносильна следующему условию:
        \[\sup_{x \in E} ||x|| < +\infty\]
    \end{Rem}

    \begin{Def}
        Проекцией $E \subset \R^n$ будем называть $E_i = \{x_i : x \in E\}$. 
    \end{Def}

    \begin{Rem}
        Ограниченность множества $E$ равносильна ограниченности всех проекций.
    \end{Rem}

    \begin{Thm}[Принцип выбора Больцано-Вейерштрасса]
        Пусть $\{x^k\}$ -- последовательность в $\R^n$. Тогда
        \begin{MyList}
            \item Если $\{x^k\}$ ограничена, то из неё можно выделить сходящуюся подпоследовательность.
            \item Если $\{x^k\}$ не ограничена, то из неё можно выделить подпоследовательность, стремящуюся к $\infty$.
        \end{MyList}
    \end{Thm}

    \begin{proof}
        \begin{MyList}
            \item $\{x_1^k\}$ -- ограниченная последовательность в $\R \SO \exists x_1^{r_k}$ -- сходящаяся подпоследовательность (по принципу выбора Больцано-Коши для числовых последовательностей).
            Рассмотрим теперь $\{x_2^{r_k}\}$ -- ограничена в $\R \SO \exists \{x_2^{s_k}\}$ -- сходящаяся подпоследовательность, где $\{s_k\}$ -- подпоследовательность $\{r_k\}$.
            После $n$-ного шага мы построили последовательность $\{x_n^{l_k}\} \SO \{x^{l_k}\}$ сходится.
        
            \item Можем построить $\{x^{r_k}\} : ||x^{r_k}|| > k$. Будем выбирать $r_1 < r_2 < r_3 < ...$. 
        \end{MyList}
    \end{proof}
    
    \begin{Thm}[Критерий Больцано-Коши]
        Пусть $\{x^k\}$ -- последовательность из $\R^n$. Тогда равносильны следующие условия:
        \begin{MyList}
            \item $\{x^k\}$ -- сходится
            \item $\forall \varepsilon > 0 \ \exists N : \forall m, n > N \ ||x^m - x^n|| < \varepsilon$.
        \end{MyList}
    \end{Thm}

    \begin{proof}
        Заметим, что 
        \[|x_i| \leqslant ||x|| \leqslant \sqrt{n} \cdot \max_{i \in [1, n]} |x_i|\]
        \begin{MyList}
            \item[] $1 \SO 2$. Поскольку $\{x^k\}$ сходится, то 
            \[\forall i = 1..n \ \forall \varepsilon > 0 \ \exists N : \forall n, m > N \ |x_i^m - x_i^n| < \frac{\varepsilon}{\sqrt{n}}\]
            Тогда
            \[||x^n - x^m|| < \sqrt{n} \cdot \frac{\varepsilon}{\sqrt{n}}\]

            \item[] $2 \SO 1$.
            \[|x_i^m - x_i^n| \leqslant ||x_i^m - x_i^n|| \leqslant \varepsilon\]
        \end{MyList}
    \end{proof}

    \begin{Def}[Покрытие]
        $\Omega$ -- семейство множеств из $\R^n$. $\Omega$ называется \textit{покрытием}  множества $E \subset \R^n$, если $E \subset \bigcup_{A \in \Omega} A$. 
    \end{Def}

    \begin{Def}[Открытое покрытие]
        Если все множества из $\Omega$ открытые, то $\Omega$ называется \textit{открытым} покрытием.
    \end{Def}

    \begin{Def}
        Пусть $\widetilde{\Omega}$ -- подсемейство $\Omega$, которое также покрывает $E$. Тогда $\widetilde{\Omega}$ называется \textit{подпокрытием} $\Omega$. 
    \end{Def}

    \begin{Def}
        Множество $E$ называется \textit{компактным}, если из любого его открытого покрытия можно выбрать конечное подпокрытие.
    \end{Def}

    \begin{Example}
        $(0, 1)$ -- не компакт. $\Omega = \{\left(\frac{1}{n}, 1\right), n \in \N\}$ -- покрытие $(0, 1)$. Из него нельзя выбрать конечное подпокрытие $(0, 1)$.
    \end{Example}

    \begin{Def}
        Пусть $a, b \in \R^n : a_1 \leqslant b_1, a_2 \leqslant b_2, ..., a_n \leqslant b_n$. 
        Тогда \textit{замкнутым параллелепипедом} будем называть следующее множество:
        \[[a; b] = [a_1, b_1] \times [a_2, b_2] \times ... \times [a_n, b_n]\]  

        \textit{Открытым параллелепипедом} называется множество:
        \[(a; b) = (a_1, b_1) \times (a_2, b_2) \times ... \times (a_n, b_n)\]
    \end{Def}

    \begin{Ex}
        Открытый параллелепипед -- открытое множество, замкнутый параллелепипед -- замкнутое множество.
    \end{Ex}

    \begin{Def}[Диаметр множества]
        $\displaystyle \diam E = \sup_{x, y \in E} ||x - y||$ 
    \end{Def}

    \begin{Thm}[О стягивающихся параллелепипедах]
        Рассмотрим параллелепипеды $P_k \subset \R^n$ -- замкнутые, $P_1 \supset P_2 \supset P_3 \supset ..., \ \diam P_k \xrightarrow[k \to \infty]{}0$.
        Тогда существует только одна точка, принадлежащая всем параллелепипедам.
    \end{Thm}

    \begin{Thm}
        Замкнутый куб в $\R^n$ является компактом.
    \end{Thm}

    \begin{Lm}
        $E \subset \R^n$. $E$ замкнуто $\EQ \forall$ сходящаяся последовательность в $E$ имеет пределом точку из $E$. 
    \end{Lm}

    \begin{proof}
        $\SO$. Пусть $\{x^k\}_{k = 1}^\infty$ -- последовательность в $E$, $a \in \R^n$, $\lim_{k \to \infty} x_k = a$. Покажем, что $a \in E$.
        Если это не так, то $a \in \R^n \setminus E$, и в силу открытости $\R^n \setminus E$ найдется окрестность $V_a$ точки $a$, лежащая в $\R^n \setminus E$. 
        По определению предела при всех достаточно больших $k \in \N$ справедливо включение $x^k \in V_a \subset \R^n \setminus E$. С другой стороны, $x^k \in E \ \forall k \in \N$, и мы получаем противоречие. 

        $\Leftarrow$. $E' \subset E \SO E$ -- замкнуто.
    \end{proof}

    \begin{Lm}
        Замкнутое подмножество компакта -- компакт.
    \end{Lm}

    \begin{proof}
        Пусть $F$ -- замнутое подмножество компакта $E$, $\Omega$ -- открытое покрытие $F$. Покажем, что из $\Omega$ можно выбрать конечное подпокрытие.
        Добавляя к $\Omega$ множество $\R^n \setminus F$, мы получим открытое покрытие компакта $E$. Выберем из этого покрытия конечное подсемейство $\widetilde{\Omega}$, которое также покрывает $E$.
        Если множество $\R^n \setminus F$ входит в $\widetilde{\Omega}$, удалим его оттуда. Мы получим конечное подпокрытие $\Omega$ множества $F$. 
    \end{proof}

    \begin{Thm}
        Пусть $E \subset \R^n$. Тогда равносильны следующие условия:
        \begin{MyList}
            \item $E$ -- компакт
            \item $E$ ограничено и замкнуто
            \item Из любой последовательности в $E$ можно выбрать подпоследовательность, сходящуюся к точке из $E$.
        \end{MyList}
    \end{Thm}

    \begin{proof}
        \begin{MyList}
            \item[] $3 \SO 2$. Пусть $\{x^k\}$ -- последовательность в $E$, сходящаяся к некоторой точке $a \in \R^n$. 
            Из условия 3) вытекает, что у $\{x^k\}$ есть подпоследовательность, предел которой лежит в $E$. Но любая подпоследовательность $\{x^k\}$ сходится к $a$, откуда $a \in E \SO E$ замкнуто.

            Докажем теперь ограниченность. Если $E$ не ограничено, то по любому $k \in \N$ найдется $x^k \in E$, для которого $||x^k|| \geqslant k$. 
            Но тогда $x^k \to \infty$ при $k \to \infty$ $\SO$ все подпоследовательности $\{x^k\}$ также стремятся к бесконечности. Получили противоречие с условием 3). 

            \item[] $2 \SO 1$. $E$ ограничено $\SO \exists c : [-c, c]^n \supset E$. Тогда $E$ -- замкнутое подмножество компакта $\SO E$ -- компакт.
            \item[] $1 \SO 3$. Пусть $a$ -- предел последовательности из $E$, но $a \notin E$. Положим $\Omega = \{\R^n \setminus \overline{B}_\varepsilon(a)\}$.
            Тогда $\bigcup_{A \in \Omega} A = \R^n \setminus \{a\} \SO \Omega$ -- покрытие $E$. 
            Пусть $\widetilde{\Omega}$ -- произвольное конечное подсемейство $\Omega$. Тогда, для некоторого $m \in \N$ и положительных чисел
            $\varepsilon_1, ..., \varepsilon_m$
            \[\widetilde{\Omega} = \left\{\R^n \setminus \overline{B}_{\varepsilon_k}(a)\right\}_{k = 1}^m\]   
            Множество $B(a) = \displaystyle{\bigcap_{k = 1}^m}B_{\varepsilon_k}(a)$ является окрестность точки $a$. Заметим, что
            \[\bigcup_{A \in \widetilde{\Omega}}A = \R^n \setminus \bigcap_{k = 1}^m \overline{B}_{\varepsilon_k}(a) \subset \R^n \setminus B(a)\]
            Таким образом, множество $\widetilde{\Omega}$ не покрывает $E$, что противоречит компактности $E$. 
        \end{MyList}
    \end{proof}

    \Subsection{Отображения}
    
    $f : E \subset \R^n \to \R^m$.

    \begin{Example}
        $m = 1$ -- функция нескольких переменных. $f(x_1, x_2, ..., x_n)$.
    \end{Example}

    \begin{Example}
        $n = 1$ -- вектор-функция. $(f_1(x), f_2(x), ..., f_m(x)) = f(x)$.
    \end{Example}

    \begin{Def}
        Отображение $f$ называется \textit{ограниченным}, если
        \[\sup_{x \in E} ||f(x)|| < +\infty\] 
    \end{Def}

    \begin{Rem}
        Ограниченность $f$ равносильна ограниченности координатных функций.
    \end{Rem}

    \begin{Def}[Предел по Коши]
        $f : E \subset \R^n \to \R^m$, $a$ -- предельная точка $E$, ($a \in \overline{R}^n$).
        \[\lim_{x \to a}f(x) = A \EQ \forall \varepsilon > 0 \ \exists \delta > 0 : \forall x \in E \cap \dot{B}_\delta(a) \ f(x) \in B_\varepsilon(A)\]
        Иначе, пусть $a \in \R^n, A \in \R^m$. Тогда $A$ -- предел, если 
        \[\forall \varepsilon > 0 \ \exists \Delta > 0 : \forall x \in E 0 < ||x - a|| < \delta \ ||f(x) - A|| < \varepsilon\] 
    \end{Def}

    \begin{Def}[Предел по Гейне]
        $f : E \subset \R^n \to \R^m$, $a$ -- предельная точка $E$. 
        Тогда $\lim_{x \to a} f(x) = A$, если
        \[\forall \{x^n\} \ x^k \to a, \ x^k \neq a, \ x^k \in E \ \lim f(x_k) = A\]
    \end{Def}

    \begin{Thm}[Эквивалентность определений предела]
        Пусть $f : E \subset \R^n \to \R^m$, точка $a \in \overline{\R^n}$ является предельной для $E$, $A \in \overline{\R^m}$. Тогда равносильны утверждения:
        \begin{MyList}
            \item $\displaystyle{\lim_{x \to a}} f(x) = A$ в смысле Коши
            \item $\displaystyle{\lim_{x \to a}} f(x) = A$ в смысле Гейне 
        \end{MyList}
    \end{Thm}

    \begin{proof}
        $1) \SO 2)$. Пусть $A$ -- предел $f$ в смысле Коши. Возьмем последовательность $\{x^k\}_{k = 1}^\infty$ в $E \setminus \{a\}$, стремящуюся к $a$.
        В силу 1) по любому $\varepsilon > 0$ можно подобрать $\delta > 0$, для которого
        \[f(x) \in V_A(\varepsilon) \quad \forall x \in \dot{V}_a(\delta) \cap E\]
        Поскольку $x^k \to a$, существует такое $N \in \N$, что при всех $k > N$ справедливо включение $x^k \in V_a(\delta)$. Кроме того, $x^k \in E \setminus \{a\}$, откуда $x^k \in \dot{V}_a(\delta) \cap E$.
        Поэтому 
        \[f(x^k) \in V_A(\varepsilon) \quad \forall k > N\]
        Таким образом, $f(x^k) \to A$ при $k \to \infty$, то есть $\lim_{x \to a} f(x) = A$ и в смысле Гейне.
        
        $2) \SO 1)$. Пусть $A$ -- предел $f$ по Гейне. Докажем, что предел $f$ в смысле Коши также существует и равен $A$.
        Действительно, если это не так, то
        \[\exists \varepsilon : \forall \delta > 0 \ \exists x \in \dot{V}_a(\delta) \cap E : f(x) \notin V_A(\varepsilon)\]
        Положим 
        \[F = \{x \in E \setminus \{a\}: f(x) \notin V_A(\varepsilon)\}\]
        Таким образом, $\dot{V}_a(\delta) \cap F = \varnothing$ при любом $\delta > 0$, то есть $a$ является предельной точкой $F$
        $\SO$ найдется последовательность $\{x^k\}_{k = 1}^\infty$ в $F$, стремящаяся к $a$. Тогда $\lim_{k \to \infty} f(x^k) = A$, что невозможно, т.к. $f(x^k) \notin V_A(\varepsilon)$ при всех $k \in \N$. 
         
    \end{proof}
\end{document}