\input{preamble.tex}

\begin{document}
	\Header
	\BeginConspect

	\Section{Системы линейных уравнений}{}{Илья Дудников}

	\[ (*)\begin{cases}
		a_{11} x_1 + a_{12} x_2 + ... + a_{1n} x_n = b_1 \\ 
		a_{21} x_1 + a_{22} x_2 + ... + a_{2n} x_n = b_2 \\ 
		\cdots\\
		a_{m1} x_1 + a_{m2} x_2 + ... + a_{mn} x_n = b_n
	\end{cases}\]
	$A = (a_{ij})$ -- матрица коэффициентов, $X = \begin{pmatrix}
	x_1 \\ 
	x_2 \\ 
	\vdots \\ 
	x_n
	\end{pmatrix}$, $B = \begin{pmatrix}
	b_1 \\ 
	b_2 \\ 
	\vdots \\ 
	b_n
	\end{pmatrix}$.  

	\begin{Def}
		Решение СЛУ $(*)$ называется $\alpha_1, ..., \alpha_n \in K : $ при $x_i = \alpha_i$ все уравнения становятся верными.
	\end{Def}

	\begin{Def}
		СЛУ $(*)$ совместна, если $\exists$ хотя бы одно решение. Иначе - несовместна.
	\end{Def}

	\Subsection{Ранг матрицы}

	$A - m \times n, A = (A_1, A_2, ..., A_m), A_i$ -- строки. \\
	$A = (A^1, A^2, ..., A^n), A^j$ -- столбцы.
	
	\begin{Def}
		Строчным (столбцовым) рангом матрицы $A$ называется максимальное число ЛНЗ строк (столбцов). \\
		Иначе, количество элементов в базисе $\langle A_1, ..., A_m\rangle (\langle A^1, ..., A^n\rangle)$. 
	\end{Def}

	\begin{Thm}
		Строчный и столбцовый ранги совпадают.
	\end{Thm}

	Обозначение: $\rank A$.

	\begin{Def}
		Минором матрицы $A - m \times n$ $k$-го порядка называется определитель, 
		составленный из элементов матрицы $A$, стоящих на $k$ выбранных строках и на $k$ выбранных столбцов.  
	\end{Def}

	\begin{Example}
		$\left(\begin{array}{cccc}
		1 & 4 & 8 & -3 \\ 
		2 & 5 & 9 & -4 \\ 
		3 & 6 & -2 & -5
		\end{array}\right)$. Если вы выберем вторую и третью строку, а также первый и последний столбец, то минор второго порядка:
		\[\left|\begin{array}{cc}
		2 & -4 \\ 
		3 & -5
		\end{array}\right|\]
	\end{Example}

	\begin{Thm}
		Ранг матрицы $A$ равен наибольшему порядку минора, отличного от нуля.
	\end{Thm}

	\begin{Thm}[Связь определителя с рангом матрицы]
		$A - n \times n$. Тогда $\rank A < n \EQ \det A = 0$.
	\end{Thm}

	\begin{proof}
		$\SO$. $\rank A < n \SO$ строки $A_1, ..., A_n$ ЛЗ, т.е.
		$\exists \alpha_1, ..., \alpha_n \in K : \alpha_1 A_1 + \alpha_2 A_2 + ... + \alpha_n A_n = 0$ ( $\alpha_i$ не все равны нулю). 
		Пусть $\alpha_1 \neq 0 \SO A_1 = - \frac{\alpha_2}{\alpha_1} A_2 - ... - \frac{\alpha_n}{\alpha_1} A_n$.
		Обнулим первую строку: прибавим к ней $A_2$, умноженную на $-\frac{\alpha_2}{\alpha_1}$, $A_3$, умноженную на $-\frac{\alpha_3}{\alpha_1}$ и т.д.
		Поскольку теперь первая строка целиком нулевая, то $\det A = 0$. \\

		$\Leftarrow$. Индукция $n = 1 \SO a_{11} = 0$. $n - 1 \to n$.
		\[\left|\begin{array}{cccc}
		a_{11} & a_{12} & \cdots & a_{1n} \\ 
		a_{21} & a_{22} & \cdots & a_{2n} \\ 
		\vdots & \vdots & \ddots & \vdots \\ 
		a_{n1} & a_{n2} & \cdots & a_{nn}
		\end{array}\right| = \]
		Можем считать, что $A^1 \neq 0, a_{11} \neq 0$. Домножим первую строку на $- \frac{a_{21}}{a_{11}}$ и прибавляем ко второй строке.
		Затем домножаем первую строку на $-\frac{a_{31}}{a_{11}}$ и прибавляем ко третьей строке и т.д.  
		\[= \left|\begin{array}{cccc}
		a_{11} & a_{12} & \cdots & a_{1n} \\ 
		0 & a_{22}' & \cdots & a_{2n}' \\ 
		\vdots & \vdots & \ddots & \vdots \\ 
		0 & a_{n2}' & \cdots & a_{nn}'
		\end{array}\right| = a_{11} \cdot \left|\begin{array}{ccc}
		a_{22}' & \cdots & a_{2n}' \\ 
		\vdots & \ddots & \vdots \\ 
		a_{n2}' & \cdots & a_{nn}'
		\end{array}\right|\]
		По предположению $A_2', ..., A_n'$ -- ЛЗ. $\begin{cases}
			A_2' = A_2 - \frac{a_{21}}{a_{11}} \cdot A_1 \\
			\cdots \\
			A_n' = A_n - \frac{a_{n1}}{a_{11}} \cdot A_1
		\end{cases}$. \\  
		$0 = \alpha_2 A_2' + ... + \alpha_n A_n' = (...) A_1 + \alpha_2 \cdot A_2 + ... + \alpha_n A_n \SO A_1, ..., A_n$ -- ЛЗ $\SO \rank A < n$.
	\end{proof}

	\begin{Def}
		Элементарными преобразованиями над строками (столбцами) называется
		\begin{MyList}
			\item Перестановка строк (столбцов).
			\item Умножение строки (столбца) на $\lambda \neq 0$.
			\item Прибавление к одной строке (столбцу) другой строки (столбца), умноженной на $\lambda \neq 0$.
		\end{MyList}
	\end{Def}

	\begin{Thm}
		При элементарных преобразованиях ранг матрицы не меняется.
	\end{Thm}

	\begin{proof}
		$1, 2$ -- очевидно.
		$(A_1, ..., A_i, ..., A_j, ..., A_n) \to (A_1, ..., A_i + \lambda A_j, ..., A_j, ..., A_n)$ 
	\end{proof}

	\begin{Def}
		Матрица называется трапецевидной, если у неё в $\forall$ ненулевой строке число нулей слева различно.
	\end{Def}

	\begin{Rem}
		$\rank$ трапецевидной матрицы равен числу ненулевых строк.
	\end{Rem}

	\begin{Thm}[О вычислении ранга]
		Любую матрицу с помощью элементарных преобразований можно привести к трапецевидной.
	\end{Thm}

	\Subsection{Структура решений СЛУ}

	\begin{Def}
		СЛУ (*) называется однородной, если все свободные члены равны нулю.
	\end{Def}

	\begin{Def}
		Нулевое решение однородной СЛУ называется тривиальным. Любое другое решение -- нетривиальным.
	\end{Def}

	\begin{Lm}
		Пусть $Y, Z$ -- решения $AX = 0 \SO \alpha Y + \beta Z$ -- тоже решение, $\alpha, \beta \in K$.
	\end{Lm}

	\begin{proof}
		\[AY = 0, AZ = 0 \SO A(\alpha Y + \beta Z) = \alpha AY + \beta AZ = 0\]
	\end{proof}

	\begin{Thm}[Структура решений однородной СЛУ]
		$AX = 0, A - m \times n, n$ -- число неизвестных, $r = \rank A \SO$
		$\exists n - r$ ЛНЗ решений $X_1, ..., X_{n - r} : \forall$ решение $X = \alpha_1 X_1 + ... + \alpha_{n - r} X_{n - r}$.  
	\end{Thm}

	\begin{proof}
		$A = (A^1, ..., A^n)$, $A^1, ..., A^r$ -- ЛНЗ столбцы $\SO $
		\[
		\begin{cases}
			A_{r + 1} = \beta_{r + 1 \ 1}A^1 + ... + \beta_{r + 1 \ n} A^{r} \\
			\cdots \\
			A^n = \beta_{n \ 1}A^1 + ... + \beta_{n \ r}A^r
		\end{cases}
		\]
		$AX = 0 \EQ x_1 A^1 + x_2 A^2 + ... + x_n A^n = 0$. \\
		$X_1 = \begin{pmatrix}
		\beta_{r + 1 \ 1} \\ 
		\vdots \\ 
		\beta_{r + 1 \ r} \\ 
		-1 \\ 
		0 \\ 
		\vdots \\ 
		0
		\end{pmatrix}, X_2 = \begin{pmatrix}
		\beta_{r + 2 \ 1} \\ 
		\vdots \\ 
		\beta_{r + 2 \ r} \\ 
		0 \\ 
		-1 \\
		0 \\ 
		\vdots \\ 
		0
		\end{pmatrix}, ..., X_{n + r} = \begin{pmatrix}
		\beta_{n \ 1} \\ 
		\vdots \\ 
		\beta_{n \ r} \\ 
		0 \\ 
		\vdots \\ 
		-1 \\ 
		\end{pmatrix}$ -- решения. \\
		Пусть $Z = \begin{pmatrix}
		x_1^* \\ 
		\vdots \\ 
		x_r^* \\ 
		\vdots \\ 
		x_n^*
		\end{pmatrix}$ -- решение. Рассмотрим $Y = Z + x_{r + 1}^* X_1 + x_{r + 2}^* X_2 + ... + x_n^* X_{n - r}$. 
		$Y = \begin{pmatrix}
		y_1 \\ 
		\vdots \\ 
		y_r \\ 
		0 \\ 
		\vdots \\
		0
		\end{pmatrix}$ -- решение~$\{y_1 A_1 + ... + y_r A_r = 0\}$. Но $A_1, ..., A_r$ -- ЛНЗ $\SO Y = \begin{pmatrix}
		0 \\ 
		\vdots \\ 
		0
		\end{pmatrix} \SO 0 = Z + x_{r + 1}^* X_1 + x_{r + 2}^* X_2 + ... + x_n^* X_{n - r}$.
	\end{proof}

	\begin{Def}
		$\forall n - r$ ЛНЗ решений однородной системы линейных уравнений называется
		\textbf{фундаментальной системой решений},
		решение вида $X = \alpha_1 X_1 + ... + \alpha_{n - r} X_{n - r}$ -- \textbf{общее решение}.
	\end{Def}

	\Subsection{Неоднородные СЛУ}

	$AX = B, A - m \times n, X = \begin{pmatrix}
	x_1 \\ 
	\vdots \\ 
	x_n
	\end{pmatrix}, B = \begin{pmatrix}
	b_1 \\ 
	\vdots \\ 
	b_n
	\end{pmatrix}$. \\
	$\overline{A} = \left(\begin{array}{c|c}
	A & B
	\end{array}\right)$ -- расширенная матрица $m \times (n + 1)$.

	\begin{Thm}[Кронекера-Капелли]
		$(*)$ -- совместна $\EQ \rank A = \rank \overline{A}$. 
	\end{Thm}

	\begin{proof}
		$\SO$. $AX = B$ -- совместна $\SO \exists$ решение $x_1 A^1 + ... + x_n A^n = B \SO$
		$B$ -- линейная комбинация $A^1, ..., A^n \SO \rank A = \rank \overline{A}$.

		$\Leftarrow$. $\rank A = \rank \overline{A} = r \SO \exists A^1, ..., A^r$ -- ЛНЗ $\SO A^1, ..., A^r, B$ -- ЛЗ $\SO B = \alpha_1 A^1 + ... + \alpha_r A^r$, не все $\alpha_i = 0 \SO (\alpha_1, ..., \alpha_r, 0, ..., 0)$ -- решение системы.   
	\end{proof}

	\begin{Thm}[О структуре решений неоднородной СЛУ]
		$AX = B, \rank A = r, n$ -- число неизвестных, система совместна. 
		$X_*$ -- какое-то решение СЛУ, $X_1, ..., X_{n - r}$ -- фундаментальные решения $AX = 0$.
		Тогда любое решение $(*)$ имеет вид $X = \alpha_1 X_1 + ... + \alpha_{n - r} X_{n - r} + X_*, \alpha_1, ..., \alpha_{n - r} \in K$. 
	\end{Thm}

	\begin{proof}
		$AX_* = B \SO AX = AX_* \SO A (X - X_*) = 0 \SO X - X* = \alpha_1 X_1 + ... + \alpha_{n - r} X_{n - r}$.
	\end{proof}

	\begin{Example}[Решение СЛУ методом Гаусса]
		\begin{gather*}
			\begin{cases}
				x_1 + x_2 + x_3 + x_4 = 4 \\
				x_1 + x_2 + 2x_3 + 2x_4 = 2 \\
				2x_1 + 2x_2 + 3x_3 + 3x_4 = 6
			\end{cases} \thicksim \left(\begin{array}{cccc|c}
			1 & 1 & 1 & 1 & 4 \\ 
			1 & 1 & 2 & 2 & 2 \\ 
			2 & 2 & 3 & 3 & 6
			\end{array}\right) \thicksim \\
			\thicksim \left(\begin{array}{cccc|c}
			1 & 1 & 1 & 1 & 4 \\ 
			0 & 0 & 1 & 1 & -2 \\ 
			0 & 0 & 1 & 1 & -2
			\end{array}\right) \thicksim \left(\begin{array}{cccc|c}
			1 & 1 & 1 & 1 & 4 \\ 
			0 & 0 & 1 & 1 & -2 \\ 
			\end{array}\right) \thicksim \left(\begin{array}{cccc|c}
			1 & 1 & 1 & 1 & 4 \\ 
			0 & 1 & 0 & 0 & \alpha \\ 
			0 & 0 & 1 & 1 & -2 \\ 
			0 & 0 & 0 & 1 & \beta
			\end{array}\right) \thicksim \left(\begin{array}{cccc|c}
			1 & 1 & 1 & 0 & 4 - \beta \\ 
			0 & 1 & 0 & 0 & \alpha \\ 
			0 & 0 & 1 & 0 & -2 - \beta \\ 
			0 & 0 & 0 & 1 & \beta
			\end{array}\right) \thicksim \\
			\thicksim \left(\begin{array}{cccc|c}
			1 & 1 & 0 & 0 & 6 \\ 
			0 & 1 & 0 & 0 & \alpha \\ 
			0 & 0 & 1 & 0 & -2 - \beta \\ 
			0 & 0 & 0 & 1 & \beta
			\end{array}\right) \thicksim \left(\begin{array}{cccc|c}
			1 & 0 & 0 & 0 & 6 - \alpha \\ 
			0 & 1 & 0 & 0 & \alpha \\ 
			0 & 0 & 1 & 0 & -2 - \beta \\ 
			0 & 0 & 0 & 1 & \beta
			\end{array}\right) \SO \begin{pmatrix}
				x_1 \\ 
				x_2 \\ 
				x_3 \\ 
				x_4
				\end{pmatrix} = \begin{pmatrix}
				6 \\ 
				0 \\ 
				-2 \\ 
				0
				\end{pmatrix} + \alpha \begin{pmatrix}
				-1 \\ 
				1 \\ 
				0 \\ 
				0
				\end{pmatrix} + \beta \begin{pmatrix}
				0 \\ 
				0 \\ 
				-1 \\ 
				1
				\end{pmatrix}
		\end{gather*}
	\end{Example}

\Section{Линейные отображения векторных пространств}{}{Илья Дудников}

\begin{Def}
	$V, W$ -- векторные пространства над $K$. Отображение $f : V \to W$ называется линейным,
	если:
	\begin{MyList}
		\item $f(x + y) = f(x) + f(y) \ \forall x, y \in V$
		\item $f(\alpha x) = \alpha f(x) \ \forall x \in V, \alpha \in K$ 
	\end{MyList}
\end{Def}

\begin{Rem}
	$1, 2 \thicksim f(\alpha x + \beta y) = \alpha f(x) + \beta f(y) \ \forall x, y \in V, \alpha, \beta \in K$. 
\end{Rem}

\begin{Def}
	$\Hom (V, W) = \{f : V \to W \text{ -- линейные}\}$
\end{Def}

\begin{Lm}
	$\Hom (V, W)$ -- векторное пространство над $K$.
\end{Lm}

\begin{proof}
	$f, g \in \Hom (V, W), (f + g)(x) = f(x) + g(x), (\alpha f)(x) = \alpha f(x) \SO f + g, \alpha f \in \Hom (V, W)$. 
\end{proof}

\begin{Def}
	$f \in \Hom (V, W), \ker f = \{x \in V : f(x) = 0\}$ -- ядро отображения $f$, $\Image f = \{f(x), x \in V\}$ -- образ $f$. 
\end{Def}

\begin{Lm}
	$\ker f \subset V, \Image f \subset W$ -- подпространства. 
\end{Lm}

\begin{proof}
	$x, y \in \ker f, f(x + y) = f(x) + f(y) = 0 + 0 = 0 \SO x + y \in \ker f$.
	Аналогично, $f(\alpha x) = \alpha f(x) = 0 \SO \alpha x \in \ker f \ \forall \alpha \in K \SO \ker f$ -- подпространство.  
\end{proof}

\begin{Ex}
	$\Image f$ -- подпространство. 
\end{Ex}

\begin{Thm}
	$f \in \Hom (V, W)$.
	\begin{MyList}
		\item $f$  -- инъективно $\EQ \ker f = \{0\}$.
		\item $f$ -- сюръективно $\EQ \Image f = W$. 
	\end{MyList} 
\end{Thm} 

\begin{proof}
	$\Leftarrow.$ $x_1 \neq x_2$, если $f(x_1) = f(x_2) \SO f(x_1 - x_2) = 0 \SO x_1 - x_2 \in \ker f \SO x_1~-~x_2~=~0~!? $  \\
	$\SO$. Пусть $x \in \ker f, x \neq 0 \SO f(x) = f(0) = 0 !?$.  
\end{proof}

\Subsection{Матрица линейного отображения}

$e_1, ..., e_n$ -- базис $V, e_1', ..., e_m'$ -- базис $W, f \in \Hom (V, W)$ \\
$x \in V, x = x_1 e_1 + ... + x_n e_n, x_i \in K, f(x) = x_1 f(e_1) + ... + x_n f(e_n) \EQ$ задать $f$ значит задать $f(e_i),~i~=~1, ..., n$. \\

% $f(e_1) = a_{11} e_1' + a_{21} e_2' + ... + a_{m1} e_m'$

\[\begin{cases}
	f(e_1) = a_{11} e_1' + a_{21} e_2' + ... + a_{m1} e_m' \\
	\cdots \\
	f(e_n) = a_{1n} e_1' + a_{2n} e_2' + ... + a_{mn} e_m'
\end{cases}\]

\begin{Def}
	Матрицей $f \in \Hom(V, W)$ в базисе $e_1, ..., e_n$ и $e_1', ..., e_m'$ назыается 
	\[A=\left(\begin{array}{cccc}
	a_{11} & a_{12} & \cdots & a_{1n} \\ 
	\vdots & \vdots & \ddots & \vdots \\ 
	a_{m1} & a_{m2} & \cdots & a_{mn}
	\end{array}\right) = \left(\begin{array}{cccc}
	f(e_1) & f(e_2) & \cdots & f(e_n)
	\end{array}\right)\]
\end{Def}

\begin{Thm}
	\begin{MyList}
		\item $\Hom(V, W)$ взаимно-однозначно соответствует $M(m, n, K)$.
		\item $e_1, ..., e_n$ -- базис $V$, $e_1', ..., e_m'$ -- базис $W$, $x \in V \to X = \begin{pmatrix}
		x_1 \\ 
		\vdots \\ 
		x_n
		\end{pmatrix}, f(x) \in W \to Y = \begin{pmatrix}
		y_1 \\ 
		\vdots \\ 
		y_m
		\end{pmatrix}, f \to A$
		$\SO AX = Y$.  
	\end{MyList}
\end{Thm}

\begin{proof}
	\begin{MyList}
		\item $f \to A$ отображение однозначно определяется $f(e_i) \SO A$ определена однозначно.
		С другой стороны, взяв произвольную матрицу $B$, можем построить по ней отображение $g$.

		\item $f \to A = (a_{ij}), 1 \leqslant i \leqslant n, 1 \leqslant j \leqslant m$.
		\begin{gather*}
			f(x) = f(x_1 e_1 + ... x_n e_n) = x_1 f(e_1) + ... + x_n f(e_n) = \\
			= x_1 (a_{11} e_1' + a_{21} e_2' + ... + a_{m1} e_m') + ... + x_n (a_{1n} e_1' + a_{2n} e_2' + ... + a_{mn} e_m') = \\
			= \underbrace{(a_{11} x_1 + a_{12} x_2 + ... + a_{1n} x_n)}_{y_1} e_1' + ... + \underbrace{(a_{m1} x_1 + a_{m2} x_2 + ... + a_{mn} x_n)}_{y_m} e_m' \SO Y = AX
		\end{gather*}
	\end{MyList}
\end{proof}

\begin{Cons}
	\begin{MyList}
		\item $\dim \Hom(V, W) = \dim V \cdot \dim W$
		\item $\alpha, \beta \in K, f, g \in \Hom(V, W), f \to A, g \to B$ в фиксированных базисах $\SO \alpha f + \beta g \to \alpha A + \beta B$.
		\item $f : V \to W, g : W \to U \SO g \circ f : V \to U, g \circ f(x) = g(f(x))$.
		Фиксируем базисы, $f \to A, g \to B \SO g \circ f \to BA$  
	\end{MyList}
\end{Cons}

\end{document}