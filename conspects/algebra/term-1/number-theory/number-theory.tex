\documentclass[12pt]{article}

% Автор стиля: Сергей Копелиович
% Автор конспекта: Илья Дудников

\usepackage{cmap}
\usepackage[T2A]{fontenc}
\usepackage[utf8]{inputenc}
\usepackage[russian]{babel}
\usepackage{graphicx}
\usepackage{amsthm,amsmath,amssymb}
\usepackage{listings}
\usepackage{color}
\usepackage{xcolor}
\usepackage{array}
\usepackage{epigraph}

\usepackage[russian,colorlinks=true,urlcolor=red,linkcolor=blue]{hyperref}
\usepackage{enumerate}
\usepackage{datetime}
\usepackage{fancyhdr}
\usepackage{lastpage}
\usepackage{verbatim}
\usepackage{tikz}
\usetikzlibrary{arrows,decorations.markings,decorations.pathmorphing}
\usepackage{pgfplots}

\usepackage{ifthen}
\usepackage{mathtools}

%\usepackage{tabls}
%\usepackage{tabularx}
%\usepackage{xifthen}
%\listfiles

\def\NAME{Лекция, 09/17/21}

\sloppy
\voffset=-20mm
\textheight=235mm
\hoffset=-22mm
\textwidth=180mm
\headsep=12pt
\footskip=20pt

\parskip=0em
\parindent=0em

\setlength\epigraphwidth{.8\textwidth}

\newlength{\tmplen}
\newlength{\tmpwidth}
\newcounter{listcounter}

% Список с маленькими отступами
\newenvironment{MyList}[1][4pt]{
  \begin{enumerate}[1.]
  \setlength{\parskip}{0pt}
  \setlength{\itemsep}{#1}
}{       
  \end{enumerate}
}
% Вложенный список с маленькими отступами
\newenvironment{InnerMyList}[1][0pt]{
  \vspace*{-0.5em}
  \begin{enumerate}[(a)]
  \setlength{\parskip}{-0pt}
  \setlength{\itemsep}{#1}
}{       
  \end{enumerate}
  \vspace*{-0.5em}
}
% Список с маленькими отступами
\newenvironment{MyItemize}[1][4pt]{
  \begin{itemize}
  \setlength{\parskip}{0pt}
  \setlength{\itemsep}{#1}
}{       
  \end{itemize}
}

% Основные математические символы
\def\TODO{{\color{red}\bf TODO}}
\def\N{\mathbb{N}}       %
\def\R{\mathbb{R}}       %
\def\F2{\mathbb{F}_2}    %
\def\Z{\mathbb{Z}}       %
\def\INF{\t{+}\infty}    % +inf
\def\EPS{\varepsilon}    %
\def\EMPTY{\varnothing}  %
\def\PHI{\varphi}        %
\def\SO{\Rightarrow}     % =>
\def\EQ{\Leftrightarrow} % <=>
\def\t{\texttt}          % mono font
\def\c#1{{\rm\sc{#1}}}   % font for classes NP, SAT, etc
\def\O{\mathcal{O}}      %
\def\NO{\t{\#}}          % #
\def\XOR{\text{ {\raisebox{-2pt}{\ensuremath{\Hat{}}}} }}
\renewcommand{\le}{\leqslant}
\renewcommand{\ge}{\geqslant}
\newcommand{\q}[1]{\langle #1 \rangle}               % <x>
\newcommand\URL[1]{{\footnotesize{\url{#1}}}}        %
% \newcommand{\sfrac}[2]{{\scriptscriptstyle\frac{#1}{#2}}}  % Очень маленькая дробь
% \newcommand{\mfrac}[2]{{\scriptstyle\frac{#1}{#2}}}    % Небольшая дробь
\newcommand{\sfrac}[2]{{\scriptstyle\frac{#1}{#2}}}  % Очень маленькая дробь
\newcommand{\mfrac}[2]{{\textstyle\frac{#1}{#2}}}    % Небольшая дробь

\newcommand{\fix}[1]{{\color{fixcolor}{#1}}} % \underline
\def\bonus{\t{\red{(*)}}}
\def\ifbonus#1{\ifthenelse{\equal{#1}{}}{}{\bonus}}
\def\smallsquare{$\scalebox{0.5}{$\square$}$}

\newlength{\myItemLength}
\setlength{\myItemLength}{0.3em}
\def\ItemSymbol{\smallsquare}
\def\Item{\vspace*{\myItemLength}\ItemSymbol \ \ }

\newcommand{\LET}{%
  % [line width=0.6pt]
  \begin{tikzpicture}%
  \draw(0.8ex,0) -- (0.8ex,1.6ex);%
  \draw(0,1.6ex) -- (0.8ex,1.6ex);%
  \end{tikzpicture}%
  \hspace*{0.1em}%
}

% Отступы
\def\makeparindent{\hspace*{\parindent}\unskip}
\def\up{\vspace*{-0.5em}}%{\vspace*{-\baselineskip}}
\def\down{\vspace*{0.5em}}
\def\LINE{\vspace*{-1em}\noindent \underline{\hbox to 1\textwidth{{ } \hfil{ } \hfil{ } }}}
\def\BOX#1{\mbox{\fbox{\bf{#1}}}}
\def\Pagebreak{\pagebreak\vspace*{-1.5em}}

% Мелкий заголовок
\newcommand{\THEE}[1]{
  \vspace*{0.5em}
  \noindent{\bf \underline{#1}}%\hspace{0.5em}
  \vspace*{0.2em}
}
% Другой тип мелкого заголовка
\newcommand{\THE}[1]{
  \vspace*{0.5em} $\bullet$
  \noindent{\bf #1}%\hspace{0.5em}
  \vspace*{0.2em}
}

\newenvironment{MyTabbing}{
  \t\bgroup
  \vspace*{-\baselineskip}
  \begin{tabbing}
    aaaa\=aaaa\=aaaa\=aaaa\=aaaa\=aaaa\kill
}{
  \end{tabbing}
  \t\egroup
}

% Код с правильными отступами
\lstnewenvironment{code}{
  \lstset{}
%  \vspace*{-0.2em}
}%
{
%  \vspace*{-0.2em}
}
\lstnewenvironment{codep}{
  \lstset{language=python}
}%
{
}

% Формулы с правильными отступами
\newenvironment{smallformula}{
 
  \vspace*{-0.8em}
}{
  \vspace*{-1.2em}
  
}
\newenvironment{formula}{
 
  \vspace*{-0.4em}
}{
  \vspace*{-0.6em}
  
}

% Большая квадратная скобка
\makeatletter
\newenvironment{sqcases}{%
  \matrix@check\sqcases\env@sqcases
}{%
  \endarray\right.%
}
\def\env@sqcases{%
  \let\@ifnextchar\new@ifnextchar
  \left\lbrack
  \def\arraystretch{1.2}%
  \array{@{}l@{\quad}l@{}}%
}
\makeatother

% Определяем основные секции: \begin{Lm}, \begin{Thm}, \begin{Def}, \begin{Rem}
\renewcommand{\qedsymbol}{$\blacksquare$}
\theoremstyle{definition} % жирный заголовок, плоский текст
\newtheorem{Thm}{\underline{Теорема}}[subsection] % нумерация будет "<номер subsection>.<номер теоремы>"
\newtheorem{Lm}[Thm]{\underline{Lm}} % Нумерация такая же, как и у теорем
\newtheorem{Ex}[Thm]{Упражнение} % Нумерация такая же, как и у теорем
\newtheorem{Example}[Thm]{Пример} % Нумерация такая же, как и у теорем
\newtheorem{Code}[Thm]{Код} % Нумерация такая же, как и у теорем
\theoremstyle{plain} % жирный заголовок, курсивный текст
\newtheorem{Def}[Thm]{Def.} % Нумерация такая же, как и у теорем
\theoremstyle{remark} % курсивный заголовок, плоский текст
\newtheorem{Cons}[Thm]{Следствие} % Нумерация такая же, как и у теорем
\newtheorem{Conj}[Thm]{Гипотеза} % Нумерация такая же, как и у теорем
\newtheorem{Prop}[Thm]{Утверждение} % Нумерация такая же, как и у теорем
\newtheorem{Rem}[Thm]{Замечание} % Нумерация такая же, как и у теорем
\newtheorem{Remark}[Thm]{Замечание} % Нумерация такая же, как и у теорем
\newtheorem{Algo}[Thm]{Алгоритм} % Нумерация такая же, как и у теорем

% Определяем ЗАГОЛОВКИ
\def\SectionName{Алгебра}
\def\AuthorName{Илья Дудников}

\newlength{\sectionvskip}
\setlength{\sectionvskip}{0.5em}
\newcommand{\Section}[4][]{
  % Заголовок
  \pagebreak
%  \ifthenelse{\isempty{#1}}{
    \refstepcounter{section}
%  }{}
  \vspace{0.5em}
%  \ifthenelse{\isempty{#1}}{
%    \addtocontents{toc}{\protect\addvspace{-5pt}}%
    \addcontentsline{toc}{section}{\arabic{section}. #2}
%  }{}
  


  % Запомнили название и автора главы
  \gdef\SectionName{#2}
  \gdef\AuthorName{#4}

  % Заголовок страницы
  \lhead{Конспект, \NAME}
  \chead{}
  \rhead{\SectionName}
  \renewcommand{\headrulewidth}{0.4pt}
    
  \lfoot{}
  \cfoot{\thepage\t{/}\pageref*{LastPage}}
  \rfoot{Автор: \AuthorName}
  \renewcommand{\footrulewidth}{0.4pt}
}

\newcommand{\Subsection}[2][]{
  \refstepcounter{subsection}
  \vspace*{1em}
  \ifthenelse{\equal{#1}{}}
    {\addcontentsline{toc}{subsection}{\arabic{section}.\arabic{subsection}. #2}}
    {\addcontentsline{toc}{subsection}{\arabic{section}.\arabic{subsection}. \bonus\,#2}}
  {\color{blue}\bf\large \arabic{section}.\arabic{subsection}. \ifbonus{#1}\,{#2}} 
  \vspace*{0.5em}
  \makeparindent
}
\newcommand{\Subsubsection}[2][]{
  \refstepcounter{subsubsection}
  \vspace*{1em}
  \ifthenelse{\equal{#1}{}}
    {\addcontentsline{toc}{subsubsection}{\arabic{section}.\arabic{subsection}.\arabic{subsubsection}. #2}}
    {\addcontentsline{toc}{subsubsection}{\arabic{section}.\arabic{subsection}.\arabic{subsubsection}. \bonus\,#2}}
  {\color{blue}\bf\large \arabic{section}.\arabic{subsection}.\arabic{subsubsection}. \ifbonus{#1}\,#2}
  \vspace*{0.5em}
  \makeparindent
}

\newcommand{\Header}{
  \pagestyle{empty}
  \renewcommand{\dateseparator}{--}
  \begin{center}
    {\Large\bf 
     Алгебра 1 семестр ПИ,\\
    \vspace{0.3em}
     \NAME}\\
    \vspace{0.7em}
    {Собрано {\today} в {\currenttime}}
  \end{center}

  \LINE
  \vspace{0em}

  \renewcommand{\baselinestretch}{0.98}\normalsize
  \tableofcontents
  \renewcommand{\baselinestretch}{1.0}\normalsize
  \pagebreak
}

\newcommand{\BeginConspect}{
  \pagestyle{fancy}
  \setcounter{page}{1}
}

\definecolor{mygray}{rgb}{0.7,0.7,0.7}
\definecolor{ltgray}{rgb}{0.9,0.9,0.9}
\definecolor{fixcolor}{rgb}{0.7,0,0}
\definecolor{red2}{rgb}{0.7,0,0}
\definecolor{dkred}{rgb}{0.4,0,0}
\definecolor{dkblue}{rgb}{0,0,0.6}
\definecolor{dkgreen}{rgb}{0,0.6,0}
\definecolor{brown}{rgb}{0.5,0.5,0}

\newcommand{\green}[1]{{\color{green}{#1}}}
\newcommand{\black}[1]{{\color{black}{#1}}}
\newcommand{\red}[1]{{\color{red}{#1}}}
\newcommand{\dkred}[1]{{\color{dkred}{#1}}}
\newcommand{\blue}[1]{{\color{blue}{#1}}}
\newcommand{\dkgreen}[1]{{\color{dkgreen}{#1}}}

\begin{document}

\Header

\BeginConspect

\Section{Основы теории чисел}{}{Илья Дудников}
\Subsection{Делимость}
\begin{Def}
    $a \vdots b$ или $b | a \Leftrightarrow \exists q : a = b \cdot q, b \neq 0$ 
\end{Def}

Свойства:
\begin{MyList}
    \item Рефлексивность. $a \vdots a, a \neq 0$ 
    \item Антисимметричность на $\N$. $a \vdots b, b \vdots a \Rightarrow a = b$
    \item Транзитивность. $a \vdots b, \vdots c \Rightarrow a \vdots c$ 
    \item $a | b, a | c \Rightarrow a | (b \pm c)$.
    \begin{proof}
        $b = a \cdot q_1, c = a q_2 \Rightarrow b \pm c = a q_1 \pm a q_2 = a(q_1 \pm q_2)$ 
    \end{proof}
    \item $a | b \Rightarrow \forall c \to a | bc$ 
    \item Пусть $a | b_i, i = 1, ..., n, a | (b_1 + ... + b_n + c) \Rightarrow a | c$
    \begin{proof}
        $b_1 + ... + b_n + c = aq, a q_1 + a q_2 + ... + a q_n + c = aq \Rightarrow c = a(q - q_1 - ... - q_n) \\ \Rightarrow a | c$ 
    \end{proof}
    \item $a | b \Rightarrow \forall k \neq 0 \to ka | kb$ 
    \item $ka | kb \Rightarrow a | b$ 
\end{MyList}

\begin{Thm}[О делении с остатком]
    $$\forall a \wedge \forall b > 0 \ \exists ! q, r, 0 \leqslant r < b : a = bq + r$$
\end{Thm}

\begin{Def}
    $a$ - делимое, $b$ - делитель, $q$ - частное (неполное частное), $r$ - остаток
\end{Def}

\begin{proof}
    $\exists$-ние. Рассмотрим $a - bq$. Выберем $q$ так, чтобы $a - bq > 0$ было наименьшим. 
    Положим $r = a - bq \geqslant 0 \Rightarrow a = bq + r$. По выбору $q \to a - b(q + 1) < 0 \Rightarrow a < b(q + 1) \Rightarrow \\ 
    r = a - bq < b(q + 1) - bq = b$.

    Единственность. Преположим, что $a = b q_1  + r_1 = b q_2 + r_2, 0 \leqslant r_1, r_2 < b$ 
    \[|r_1 - r_2| < b, b q_1 + r_1 = b q_2 + r_2 \Rightarrow b(q_1 - q_2) = r_2 - r_1 \Rightarrow |b(q_1 - q_2)| \geqslant b\]
    Но $|r_1 - r_2| < b$ - противоречие.
\end{proof} 
\Pagebreak

\Subsection{Наибольший общий делитель}
\begin{Def}
    Общим делителем $a_1, a_2, ..., a_n$ называется $d: d | a_i, i = 1, ..., n$.
\end{Def}

\begin{Def}
    Наибольший общий делитель $a_1, a_2, ..., a_n$ называется $d$ такое, что 
    \begin{MyList}
        \item $d > 0$ 
        \item $d | a_i, i = 1, ..., n$ 
        \item если $d' | a_i, i = 1, ..., n$, то $d' | d$ 
    \end{MyList} 
    Обозначается $\gcd(a_1, a_2, ..., a_n) = (a_1, a_2, ..., a_n)$ 

    \begin{Rem}
        По определению $\gcd(0, 0) = 0$. $a \neq 0$, то $\gcd(a, 0) = 0$   
    \end{Rem}
\end{Def}

Свойства:
\begin{MyList}
    \item $b | a \Rightarrow (a, b) = b$ 
    \begin{proof}
        Докажем, что множество делителей $(a, b)$ совпадает с множество делителей $b$. \\
        $$d | (a, b) \Rightarrow d | b$$
        $$d | b \Rightarrow d | a \text{(по транзитивности)}\Rightarrow d | (a, b)$$    
    \end{proof}
    \item $a = bq + c \Rightarrow (a, b) = (b, c)$
    \item \begin{Algo}[Алгоритм Евклида]
        \[a = b q_1 + r_1, 0 \leqslant r_1 < b\]
        \[b = r_1 q_1 + r_2, 0 \leqslant r_2 < r_1\]
        \[r_1 = r_2 q_3 + r_3, 0 \leqslant r_3 < r_2\]
        \[...\]
        \[r_{n - 2} = r_{n - 1} q_n + r_n, 0 \leqslant r_n < r_{n - 1}\]
        \[r_{n - 1} = r_n q_{n + 1}\] 
    \end{Algo} 
    \begin{Thm}
        $r_n = \gcd (a, b)$ 
    \end{Thm}

    \begin{proof}
        $r_1 > r_2 > r_3 > ... \geqslant 0 \Rightarrow \exists r_{n + 1} = 0$. $r_n | r_{n - 1}$.  
        \[r_n = (r_n, r_{n - 1}) = (r_{n - 1}, r_{n - 2}) = ... = (r_2, r_1) = (b, r_1) = (a, b)\] 
    \end{proof}
    \Pagebreak
    \item $(ma, mb) = m \cdot (a, b)$ 
    \item $d | a, d | b \Rightarrow \left(\frac{a}{d}, \frac{b}{d}\right) = \frac{(a, b)}{d}$
    \begin{proof}
        $(a, b) = \left(d \cdot \frac{a}{d}, d \cdot \frac{b}{d}\right) = d \cdot \left(\frac{a}{d}, \frac{b}{d}\right)$ 
    \end{proof}
    \item $(a, b) = 1 \Rightarrow (a, bc) = (a, c)$ 
    \begin{proof}
        Докажем, что $(a, bc) | (a, c)$ 
        \[(a, bc) | a, (a, bc) | ac, (a, bc) | bc \Rightarrow (a, bc) | (ac, bc) \Rightarrow (a, bc) | (a, b) \cdot c = c\Rightarrow (a, bc) | (a, c)\]

        Теперь докажем, что $(a, c) | (a, bc)$
        \[(a, c) | a, (a, c) | c \Rightarrow (a, c) | bc \Rightarrow (a, c) | (a, bc) \Rightarrow (a, bc) = (a, c)\] 
    \end{proof}
    \item $(a, b) = 1, b | ac \Rightarrow b | c$
    \begin{proof}
        \[b | bc, b | ac \Rightarrow b | (bc, ac) = c\] 
    \end{proof}
    \item $(a, b) = (a - b, b)$
\end{MyList}

\begin{Thm}[Линейное представление НОД]
    $$(a, b) = d \Rightarrow \exists u, v : u \cdot a + v \cdot b = d$$
\end{Thm}

\begin{proof}
    Из алгоритма Евклида:
    \[r_{n - 2} = r_{n - 1} \cdot q_n + r_n \Rightarrow d = r_n = r_{n - 2} - r_{n - 1} \cdot q_n\]
    \[r_{n - 3} = r_{n - 2} \cdot q_{n - 1} + r_{n - 1} \Rightarrow d = r_{n - 2} - (r_{n - 3} - r_{n - 2} \cdot q_{n - 1}) q_n\]
    Из следующей строки выражаем $r_{n - 2}$ и т.д. $\Rightarrow$ останутся $a$ и $b$ $\Rightarrow d = u \cdot a + v \cdot b$  
\end{proof}
\end{document}