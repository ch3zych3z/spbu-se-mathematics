\documentclass[12pt]{article}

% Автор стиля: Сергей Копелиович
% Автор конспекта: Илья Дудников

\usepackage{cmap}
\usepackage[T2A]{fontenc}
\usepackage[utf8]{inputenc}
\usepackage[russian]{babel}
\usepackage{graphicx}
\usepackage{amsthm,amsmath,amssymb}
\usepackage{listings}
\usepackage{color}
\usepackage{xcolor}
\usepackage{array}
\usepackage{mathrsfs}
\usepackage{epigraph}

\usepackage[russian,colorlinks=true,urlcolor=red,linkcolor=blue]{hyperref}
\usepackage{enumerate}
\usepackage{datetime}
\usepackage{fancyhdr}
\usepackage{lastpage}
\usepackage{verbatim}
\usepackage{tikz}
\usetikzlibrary{arrows,decorations.markings,decorations.pathmorphing}
\usepackage{pgfplots}

\usepackage{ifthen}
\usepackage{mathtools}

%\usepackage{tabls}
%\usepackage{tabularx}
%\usepackage{xifthen}
%\listfiles

\usepackage{import}
\usepackage{xifthen}
\usepackage{pdfpages}
\usepackage{transparent}

\def\NAME{Лекция, 10/09/21}

\sloppy
\voffset=-20mm
\textheight=235mm
\hoffset=-22mm
\textwidth=180mm
\headsep=12pt
\footskip=20pt

\parskip=0em
\parindent=0em

\setlength\epigraphwidth{.8\textwidth}

\newlength{\tmplen}
\newlength{\tmpwidth}
\newcounter{listcounter}

% Список с маленькими отступами
\newenvironment{MyList}[1][4pt]{
  \begin{enumerate}[1.]
  \setlength{\parskip}{0pt}
  \setlength{\itemsep}{#1}
}{       
  \end{enumerate}
}
% Вложенный список с маленькими отступами
\newenvironment{InnerMyList}[1][0pt]{
  \vspace*{-0.5em}
  \begin{enumerate}[(a)]
  \setlength{\parskip}{-0pt}
  \setlength{\itemsep}{#1}
}{       
  \end{enumerate}
  \vspace*{-0.5em}
}
% Список с маленькими отступами
\newenvironment{MyItemize}[1][4pt]{
  \begin{itemize}
  \setlength{\parskip}{0pt}
  \setlength{\itemsep}{#1}
}{       
  \end{itemize}
}

% Основные математические символы
\def\TODO{{\color{red}\bf TODO}}
\def\N{\mathbb{N}}       %
\def\R{\mathbb{R}}       %
\def\F2{\mathbb{F}_2}    %
\def\Z{\mathbb{Z}}       %
\def\INF{\t{+}\infty}    % +inf
\def\EPS{\varepsilon}    %
\def\EMPTY{\varnothing}  %
\def\PHI{\varphi}        %
\def\SO{\Rightarrow}     % =>
\def\EQ{\Leftrightarrow} % <=>
\def\t{\texttt}          % mono font
\def\c#1{{\rm\sc{#1}}}   % font for classes NP, SAT, etc
\def\O{\mathcal{O}}      %
\def\NO{\t{\#}}          % #
\def\XOR{\text{ {\raisebox{-2pt}{\ensuremath{\Hat{}}}} }}
\renewcommand{\le}{\leqslant}
\renewcommand{\ge}{\geqslant}
\newcommand{\q}[1]{\langle #1 \rangle}               % <x>
\newcommand\URL[1]{{\footnotesize{\url{#1}}}}        %
% \newcommand{\sfrac}[2]{{\scriptscriptstyle\frac{#1}{#2}}}  % Очень маленькая дробь
% \newcommand{\mfrac}[2]{{\scriptstyle\frac{#1}{#2}}}    % Небольшая дробь
\newcommand{\sfrac}[2]{{\scriptstyle\frac{#1}{#2}}}  % Очень маленькая дробь
\newcommand{\mfrac}[2]{{\textstyle\frac{#1}{#2}}}    % Небольшая дробь

\newcommand{\fix}[1]{{\color{fixcolor}{#1}}} % \underline
\def\bonus{\t{\red{(*)}}}
\def\ifbonus#1{\ifthenelse{\equal{#1}{}}{}{\bonus}}
\def\smallsquare{$\scalebox{0.5}{$\square$}$}

\newlength{\myItemLength}
\setlength{\myItemLength}{0.3em}
\def\ItemSymbol{\smallsquare}
\def\Item{\vspace*{\myItemLength}\ItemSymbol \ \ }

\newcommand{\LET}{%
  % [line width=0.6pt]
  \begin{tikzpicture}%
  \draw(0.8ex,0) -- (0.8ex,1.6ex);%
  \draw(0,1.6ex) -- (0.8ex,1.6ex);%
  \end{tikzpicture}%
  \hspace*{0.1em}%
}

% Отступы
\def\makeparindent{\hspace*{\parindent}\unskip}
\def\up{\vspace*{-0.5em}}%{\vspace*{-\baselineskip}}
\def\down{\vspace*{0.5em}}
\def\LINE{\vspace*{-1em}\noindent \underline{\hbox to 1\textwidth{{ } \hfil{ } \hfil{ } }}}
\def\BOX#1{\mbox{\fbox{\bf{#1}}}}
\def\Pagebreak{\pagebreak\vspace*{-1.5em}}

% Мелкий заголовок
\newcommand{\THEE}[1]{
  \vspace*{0.5em}
  \noindent{\bf \underline{#1}}%\hspace{0.5em}
  \vspace*{0.2em}
}
% Другой тип мелкого заголовка
\newcommand{\THE}[1]{
  \vspace*{0.5em} $\bullet$
  \noindent{\bf #1}%\hspace{0.5em}
  \vspace*{0.2em}
}

\newenvironment{MyTabbing}{
  \t\bgroup
  \vspace*{-\baselineskip}
  \begin{tabbing}
    aaaa\=aaaa\=aaaa\=aaaa\=aaaa\=aaaa\kill
}{
  \end{tabbing}
  \t\egroup
}

% Код с правильными отступами
\lstnewenvironment{code}{
  \lstset{}
%  \vspace*{-0.2em}
}%
{
%  \vspace*{-0.2em}
}
\lstnewenvironment{codep}{
  \lstset{language=python}
}%
{
}

% Формулы с правильными отступами
\newenvironment{smallformula}{
 
  \vspace*{-0.8em}
}{
  \vspace*{-1.2em}
  
}
\newenvironment{formula}{
 
  \vspace*{-0.4em}
}{
  \vspace*{-0.6em}
  
}

% Большая квадратная скобка
\makeatletter
\newenvironment{sqcases}{%
  \matrix@check\sqcases\env@sqcases
}{%
  \endarray\right.%
}
\def\env@sqcases{%
  \let\@ifnextchar\new@ifnextchar
  \left\lbrack
  \def\arraystretch{1.2}%
  \array{@{}l@{\quad}l@{}}%
}
\makeatother

\DeclareMathOperator{\sort}{sort}

% Определяем основные секции: \begin{Lm}, \begin{Thm}, \begin{Def}, \begin{Rem}
\renewcommand{\qedsymbol}{$\blacksquare$}
\theoremstyle{definition} % жирный заголовок, плоский текст
\newtheorem{Thm}{\underline{Теорема}}[subsection] % нумерация будет "<номер subsection>.<номер теоремы>"
\newtheorem{Lm}[Thm]{\underline{Lm}} % Нумерация такая же, как и у теорем
\newtheorem{Ex}[Thm]{Упражнение} % Нумерация такая же, как и у теорем
\newtheorem{Example}[Thm]{Пример} % Нумерация такая же, как и у теорем
\newtheorem{Solution}[Thm]{Решение}
\newtheorem{Code}[Thm]{Код} % Нумерация такая же, как и у теорем
\theoremstyle{plain} % жирный заголовок, курсивный текст
\newtheorem{Def}[Thm]{Def.} % Нумерация такая же, как и у теорем
\theoremstyle{remark} % курсивный заголовок, плоский текст
\newtheorem{Cons}[Thm]{Следствие} % Нумерация такая же, как и у теорем
\newtheorem{Conj}[Thm]{Гипотеза} % Нумерация такая же, как и у теорем
\newtheorem{Prop}[Thm]{Утверждение} % Нумерация такая же, как и у теорем
\newtheorem{Rem}[Thm]{Замечание} % Нумерация такая же, как и у теорем
\newtheorem{Remark}[Thm]{Замечание} % Нумерация такая же, как и у теорем
\newtheorem{Algo}[Thm]{Алгоритм} % Нумерация такая же, как и у теорем

% Определяем ЗАГОЛОВКИ
\def\SectionName{Дискретная математика}
\def\AuthorName{Илья Дудников}

\newlength{\sectionvskip}
\setlength{\sectionvskip}{0.5em}
\newcommand{\Section}[4][]{
  % Заголовок
  \pagebreak
%  \ifthenelse{\isempty{#1}}{
    \refstepcounter{section}
%  }{}
  \vspace{0.5em}
%  \ifthenelse{\isempty{#1}}{
%    \addtocontents{toc}{\protect\addvspace{-5pt}}%
    \addcontentsline{toc}{section}{\arabic{section}. #2}
%  }{}
  


  % Запомнили название и автора главы
  \gdef\SectionName{#2}
  \gdef\AuthorName{#4}

  % Заголовок страницы
  \lhead{Конспект, \NAME}
  \chead{}
  \rhead{\SectionName}
  \renewcommand{\headrulewidth}{0.4pt}
    
  \lfoot{}
  \cfoot{\thepage\t{/}\pageref*{LastPage}}
  \rfoot{Автор: \AuthorName}
  \renewcommand{\footrulewidth}{0.4pt}
}

\newcommand{\Subsection}[2][]{
  \refstepcounter{subsection}
  \vspace*{1em}
  \ifthenelse{\equal{#1}{}}
    {\addcontentsline{toc}{subsection}{\arabic{section}.\arabic{subsection}. #2}}
    {\addcontentsline{toc}{subsection}{\arabic{section}.\arabic{subsection}. \bonus\,#2}}
  {\color{blue}\bf\large \arabic{section}.\arabic{subsection}. \ifbonus{#1}\,{#2}} 
  \vspace*{0.5em}
  \makeparindent
}
\newcommand{\Subsubsection}[2][]{
  \refstepcounter{subsubsection}
  \vspace*{1em}
  \ifthenelse{\equal{#1}{}}
    {\addcontentsline{toc}{subsubsection}{\arabic{section}.\arabic{subsection}.\arabic{subsubsection}. #2}}
    {\addcontentsline{toc}{subsubsection}{\arabic{section}.\arabic{subsection}.\arabic{subsubsection}. \bonus\,#2}}
  {\color{blue}\bf\large \arabic{section}.\arabic{subsection}.\arabic{subsubsection}. \ifbonus{#1}\,#2}
  \vspace*{0.5em}
  \makeparindent
}

\newcommand{\Header}{
  \pagestyle{empty}
  \renewcommand{\dateseparator}{--}
  \begin{center}
    {\Large\bf 
     Дискретная математика 1 семестр ПИ,\\
    \vspace{0.3em}
     \NAME}\\
    \vspace{0.7em}
    {Собрано {\today} в {\currenttime}}
  \end{center}

  \LINE
  \vspace{0em}

  \renewcommand{\baselinestretch}{0.98}\normalsize
  \tableofcontents
  \renewcommand{\baselinestretch}{1.0}\normalsize
  \pagebreak
}

\newcommand{\BeginConspect}{
  \pagestyle{fancy}
  \setcounter{page}{1}
}

\definecolor{mygray}{rgb}{0.7,0.7,0.7}
\definecolor{ltgray}{rgb}{0.9,0.9,0.9}
\definecolor{fixcolor}{rgb}{0.7,0,0}
\definecolor{red2}{rgb}{0.7,0,0}
\definecolor{dkred}{rgb}{0.4,0,0}
\definecolor{dkblue}{rgb}{0,0,0.6}
\definecolor{dkgreen}{rgb}{0,0.6,0}
\definecolor{brown}{rgb}{0.5,0.5,0}

\newcommand{\green}[1]{{\color{green}{#1}}}
\newcommand{\black}[1]{{\color{black}{#1}}}
\newcommand{\red}[1]{{\color{red}{#1}}}
\newcommand{\dkred}[1]{{\color{dkred}{#1}}}
\newcommand{\blue}[1]{{\color{blue}{#1}}}
\newcommand{\dkgreen}[1]{{\color{dkgreen}{#1}}}

\begin{document}

\Header

\BeginConspect

\Section{Теория вероятности}{}{Илья Дудников}
\Subsection{Основы теории вероятности}

\begin{Def}
    $\Omega = \{a_1, a_2, ..., a_n\}$ -- множество всех взаимо-исключающих исходов эксперимента (пространство элементарных событий) \\
    $X \subseteq \Omega$ -- событие
\end{Def}

\begin{Def}
    Дано $\Omega, \mathscr{A} \subset 2^\Omega$. Тогда $\mathscr{A}$ называется алгеброй, если 
    \begin{MyList}
        \item $\Omega \in \mathscr{A}$ 
        \item $A \in \mathscr{A}, B \in \mathscr{B} \SO A \cup B \in \mathscr{A}$
        \item $A \in \mathscr{A} \SO \overline{A} \in \mathscr{A}$  
    \end{MyList}   
\end{Def}

\begin{Prop}
    Если $\mathscr{A}$ -- алгебра, то 
    \begin{MyList}
        \item $\varnothing \in \mathscr{A}$ 
        \item $A, B \in \mathscr{A} \SO A \cap B \in \mathscr{A}$ 
        \item $A, B \in \mathscr{A} \SO A \setminus B \in \mathscr{A}$
        \item $A_i \in \mathscr{A} \SO \bigcup A_i \in \mathscr{A}, \bigcap A_i \in \mathscr{A}$  
    \end{MyList}
\end{Prop}

\begin{proof}
    \begin{MyList}
        \item $\Omega \in \mathscr{A} \SO \overline{\Omega} \in \mathscr{A} \SO \overline{\Omega} = \varnothing \SO \varnothing \in \mathscr{A}$
        \item $\overline{A \cap B} = \overline{A} \cup \overline{B} \in \mathscr{A}$. Тогда $\overline{\overline{A \cap B}} = A \cap B \in \mathscr{A}$
        \item $A \setminus B = A \cap \overline{B} \in \mathscr{A}$ 
        \item Доказывается по индукции.  
    \end{MyList}
\end{proof}

\begin{Def}
    $\mathscr{A}$ называется $\sigma$-алгеброй, если 
    \begin{MyList}
        \item $\Omega \in \mathscr{A}$ 
        \item $A_i \in \mathscr{A}, i = 1, .... \SO \bigcup_{i = 1}^\infty A_i \in \mathscr{A}$
        \item $A \in \mathscr{A} \SO \overline{A} \in \mathscr{A}$
    \end{MyList} 
\end{Def}

\begin{Def}
    Пусть есть пространство $\Omega$, определенная на нём $\mathscr{A}$ -- $\sigma$-алгебра и $f: \mathscr{A} \to \R$ -- функция над множеством.
    Тогда вероятностью называется функция из $\mathscr{A}$ в $\R$ такая, что
    \begin{MyList}
        \item $P(A) \geqslant 0 \ \forall A \in \mathscr{A}$ 
        \item $P(\Omega) = 1$
        \item $A_1, A_2, ... : A_i \cap A_j = \varnothing \ \forall i, j \SO P(\bigcup_{i = 1}^\infty A_i) = \sum_{i=1}^{\infty} P(A_i)$  
    \end{MyList}
    Перечисленные выше свойства называются \textbf{аксиомами теории вероятности}
    
    $(\Omega, \mathscr{A}, P)$ -- вероятностное пространство.
\end{Def}
\Pagebreak

Свойства вероятности:
\begin{MyList}
    \item $P(\Omega) = 1$
    \item $P(\varnothing) = 0$ 
    \item Если $A_1, A_2 \in \mathscr{A}, A_1 \cap A_2 = \varnothing$, то \[P(A_1 \cup A_2) = P(A_1) + P(A_2)\]
    \item Если $A_1, ..., A_n \in \mathscr{A}, A_i \cap A_j = \varnothing \ \forall i, j$, то \[P(\bigcup_{i = 1}^n A_i) = \sum_{i=1}^{n} P(A_i)\]
    \item $P(\overline{A}) = 1 - P(A)$ 
    \begin{proof}
        \[P(\overline{A} \cup A) = P(\Omega) = P(A) + P(\overline{A})\] 
    \end{proof}
    \item Если $A, B \in \mathscr{A}$, то \[P(A \cup B) = P(A) + P(B) - P(A \cap B)\]
    \begin{proof}
        \[A \cup B = (A \setminus B) \cup (B \setminus A) \cup (A \cap B)\]
        
        \[P(A \cup B) = P(A \setminus B) + P(B \setminus A) + P(A \cap B) = P((A \setminus B) \cup (A \cap B)) + P((B \setminus A) \cup (A \cap B)) - P(A \cap B) =\]
        \[= P(A) + P(B) - P(A \cap B)\]
    \end{proof}
    \item $P(\bigcup_{i = 1}^n A_i) = \sum_{i=1}^{n} P(A_i) - \sum_{i, j=1}^{n} (A_i \cap A_j) + ...$
    \item $A_1 \subset A_2 \subset ... \subset A_n \subset ...$
    \[\lim_{n \to \infty} P(A_n) = P(\bigcup_{i = 1}^\infty A_i)\]
    \begin{proof}
        $A_{k - 1} \subset A_k$. Рассмотрим $A_k \setminus A_{k - 1}$. Пусть $A_0 = \varnothing$.
        \[P(\bigcup_{i = 1}^\infty A_i) = \sum_{k=1}^{\infty} P(A_k \setminus A_{k - 1}) = \lim_{n \to \infty} \sum_{k=1}^{n} P(A_k \setminus A_{k - 1}) \lim_{n \to \infty} \sum_{k=1}^{n} P(A_k) - P(A_{k - 1}) =\]
        \[= \lim_{n \to \infty} P(A_n) - P(\varnothing) = \lim_{n \to \infty} P(A_n)\]
    \end{proof} 
    \item $A_1 \supset A_2 \supset ... \supset A_n \supset ...$. Тогда 
    \[\lim_{n \to \infty} P(A_n) = P(\bigcap_{i = 1}^\infty A_i)\]
\end{MyList}

\Pagebreak
\begin{Example}
    Два человека приходят на место в промежуток от 12 до 13ч и ждут 10 минут прежде чем уйти. Найти вероятность того, что они встретятся.
\end{Example}

\begin{Solution}
    Пусть $t_1$ -- время, когда приходит первый, $t_2$ -- время, когда приходит второй.
    \[|t_1 - t_2 \leqslant \frac{1}{6} \EQ \begin{cases}
        t_2 \geqslant t_1 - \frac{1}{6} \\
        t_2 \leqslant t_1 + \frac{1}{6}
    \end{cases}\] 
    \begin{figure*}[ht]
        \centering
        \import{./img}{first_example.pdf_tex}
    \end{figure*}

    Тогда вероятность -- площадь заштрихованной фигуры:
    \[S = 1 - 2 \cdot \frac{\frac{5}{6} \cdot \frac{5}{6}}{2} = 1 - \frac{25}{36} = \frac{11}{36}\]
\end{Solution}

\begin{Example}
    На $[0, 1]$ выбираются два числа $x, y$. Найти вероятность того, что их произведение меньше $\frac{1}{2}$  
\end{Example}

\begin{Solution}
    \[f(x) = \begin{cases}
        1, x \leqslant \frac{1}{2}, \\
        \frac{1}{2x}, x > \frac{1}{2}
    \end{cases}\]
    \begin{figure*}[ht]
        \centering
        \def\svgwidth{.3\columnwidth}
        \import{./img/}{second_example.pdf_tex}
    \end{figure*}

    Тогда искомая вероятность:
    \[P(x \cdot y < \frac{1}{2}) = \int_{0}^{1} f(x)dx = \int_{0}^{\frac{1}{2}} f(x)dx + \int_{\frac{1}{2}}^{1} f(x)dx = \frac{1}{2} + \int_{\frac{1}{2}}^{1} \frac{1}{2x}dx = \frac{1}{2} + \frac{\ln 2}{2}\]
\end{Solution}

\Pagebreak
\Subsection{Условная вероятность}
\begin{Def}
    Вероятность события $A$ при условии, что выполняется событие $B$ равна
    \[P(A | B) = \frac{P(A \cap B)}{P(B)}\] 
\end{Def}

\begin{Example}
    Есть урна, в которой лежит $m$ белых и $n$ черных шаров. Вытащим из неё два шара. Какова вероятность того, что они оба белые?
\end{Example}

\begin{Solution}
    \[P(\text{первый -- белый}) = \frac{m}{m + n}, P(\text{второй -- белый} | \text{первый -- белый}) = \frac{m - 1}{m + n - 1}\]
    \[P(\text{оба белые}) = \frac{m - 1}{m + n - 1} \cdot \frac{m}{m + n}\]
\end{Solution}

\Pagebreak
Свойства условной вероятности:
\begin{MyList}
    \item $P(\Omega | B) = 1$
    \item $P(\varnothing | B) 0 $ 
    \item $0 \leqslant P(A | B) \leqslant 1$ 
    \item $A \subset C \SO P(A | B) \leqslant P(C | B)$
    \item $P(\overline{A} | B) = 1 - P(A | B)$
    \item $P(A \cup C | B) = P(A | B) + P(C | B) - P(A \cap C | B)$
    \item $P(A \cap B) = P(A | B) \cdot P(B)$
    \[P(A_1 \cap A_2 \cap ... \cap A_n) = P(A_1) \cdot P(A_2 | A_1) \cdot P(A_3 | A_2 \cap A_1) \cdot ... \cdot P(A_n | \bigcap_{i = 1}^{n - 1} A_i)\]
    \[P((A_1 \cap ... \cap A_{n - 1}) \cap A_n) = P(A_n | A_1 \cap ... \cap A_{n - 1}) \cdot P(A_1 \cap ... \cap _{n - 1})\]
\end{MyList}

\begin{Example}
    Бросаем 3 кубика. Найти вероятность того, что хотя бы на одном из них выпадет $1$ при условии, что на всех выпали разные значения.
\end{Example}

\begin{Solution}
    \[P(A | B) = 1 - P(\overline{A} | B) = 1 - \frac{P(\overline{A} \cap B)}{P(B)} = 1 - \frac{1}{2} = \frac{1}{2}\]
\end{Solution}

\Subsection{Независимость событий}
\begin{Def}
    $A$ независимо от $B (P(B) \neq \varnothing)$, если $P(A | B) = P(A)$ 
\end{Def}

\begin{Prop}
    Если $A$ независимо от $B \SO B$ независимо от $A$.   
\end{Prop}

\begin{proof}
    \[P(B | A) = \frac{P(B \cap A)}{P(A)} = \frac{P(A \cap B) \cdot P(B)}{P(A) \cdot P(B)} = P(A | B) \cdot \frac{P(B)}{P(A)} = P(B)\]
\end{proof}

\begin{Def}
    $A, B$ -- независимые, если 
    \[P(A \cap B) = P(A) \cap P(B)\]
\end{Def}

\begin{Def}
    $A_1, ..., A_n$ -- независимы в совокупности, если 
    \[P(\bigcap_{i = 1}^n) = \prod_{i = 1}^n P(A_i)\]
\end{Def}

\begin{Def}
    $A_1, .., A_n$ -- попарно-независимы, если 
    \[\forall i, j \to P(A_i \cap A_j) = P(A_i) \cdot P(A_j)\]
\end{Def}

\begin{Rem}
    Если $A_1, ..., A_n$ попарно-независимы, то они необязательно независимы в совокупности.
\end{Rem}

\Pagebreak
\Subsection{Формула полной вероятности}

\begin{Def}
    Пусть $H_1, ..., H_n$ -- разбиение $\Omega$. Тогда $H_1 \cup ... \cup H_n = \Omega$ называется полной группой событий.
\end{Def}

\begin{Thm}
    $H_1, ..., H_n$ -- полная группа событий и $P(H_i) > 0 \ \forall i = 1, ..., n$. Тогда 
    \[\forall A \to P(A) = \sum_{i=1}^{n} P(H_i) \cdot P(A | H_i)\] 
\end{Thm}

\begin{proof}
    \[A = A \cap \Omega = A \cap (H_1 \cup ... \cup H_n) = (A \cap H_1) \cup (A \cap H_2) \cup ... \cup (A \cap H_n)\]
    \[P((A \cap H_1) \cup ... \cup (A \cap H_n)) = \sum_{i=1}^{n} P(A \cap H_i) = \sum_{i=1}^{n} P(A | H_i) \cdot P(H_i)\] 
\end{proof}

\end{document}