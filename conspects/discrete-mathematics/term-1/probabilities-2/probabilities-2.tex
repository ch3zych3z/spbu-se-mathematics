\documentclass[12pt]{article}

% Автор стиля: Сергей Копелиович
% Автор конспекта: Илья Дудников

\usepackage{cmap}
\usepackage[T2A]{fontenc}
\usepackage[utf8]{inputenc}
\usepackage[russian]{babel}
\usepackage{graphicx}
\usepackage{amsthm,amsmath,amssymb}
\usepackage{listings}
\usepackage{color}
\usepackage{xcolor}
\usepackage{array}
\usepackage{mathrsfs}
\usepackage{epigraph}

\usepackage[russian,colorlinks=true,urlcolor=red,linkcolor=blue]{hyperref}
\usepackage{enumerate}
\usepackage{datetime}
\usepackage{fancyhdr}
\usepackage{lastpage}
\usepackage{verbatim}
\usepackage{tikz}
\usetikzlibrary{arrows,decorations.markings,decorations.pathmorphing}
\usepackage{pgfplots}

\usepackage{ifthen}
\usepackage{mathtools}

%\usepackage{tabls}
%\usepackage{tabularx}
%\usepackage{xifthen}
%\listfiles

\usepackage{import}
\usepackage{xifthen}
\usepackage{pdfpages}
\usepackage{transparent}

\def\NAME{Лекция, 10/23/21}

\sloppy
\voffset=-20mm
\textheight=235mm
\hoffset=-22mm
\textwidth=180mm
\headsep=12pt
\footskip=20pt

\parskip=0em
\parindent=0em

\setlength\epigraphwidth{.8\textwidth}

\newlength{\tmplen}
\newlength{\tmpwidth}
\newcounter{listcounter}

% Список с маленькими отступами
\newenvironment{MyList}[1][4pt]{
  \begin{enumerate}[1.]
  \setlength{\parskip}{0pt}
  \setlength{\itemsep}{#1}
}{       
  \end{enumerate}
}
% Вложенный список с маленькими отступами
\newenvironment{InnerMyList}[1][0pt]{
  \vspace*{-0.5em}
  \begin{enumerate}[(a)]
  \setlength{\parskip}{-0pt}
  \setlength{\itemsep}{#1}
}{       
  \end{enumerate}
  \vspace*{-0.5em}
}
% Список с маленькими отступами
\newenvironment{MyItemize}[1][4pt]{
  \begin{itemize}
  \setlength{\parskip}{0pt}
  \setlength{\itemsep}{#1}
}{       
  \end{itemize}
}

% Основные математические символы
\def\TODO{{\color{red}\bf TODO}}
\def\C{\mathbb{C}}
\def\Q{\mathbb{Q}}
\def\N{\mathbb{N}}       %
\def\R{\mathbb{R}}       %
\def\F2{\mathbb{F}_2}    %
\def\Z{\mathbb{Z}}       %
\def\INF{\t{+}\infty}    % +inf
\def\EPS{\varepsilon}    %
\def\EMPTY{\varnothing}  %
\def\PHI{\varphi}        %
\def\SO{\Rightarrow}     % =>
\def\EQ{\Leftrightarrow} % <=>
\def\t{\texttt}          % mono font
\def\c#1{{\rm\sc{#1}}}   % font for classes NP, SAT, etc
\def\O{\mathcal{O}}      %
\def\NO{\t{\#}}          % #
\def\XOR{\text{ {\raisebox{-2pt}{\ensuremath{\Hat{}}}} }}
\renewcommand{\le}{\leqslant}
\renewcommand{\ge}{\geqslant}
\newcommand{\q}[1]{\langle #1 \rangle}               % <x>
\newcommand\URL[1]{{\footnotesize{\url{#1}}}}        %
% \newcommand{\sfrac}[2]{{\scriptscriptstyle\frac{#1}{#2}}}  % Очень маленькая дробь
% \newcommand{\mfrac}[2]{{\scriptstyle\frac{#1}{#2}}}    % Небольшая дробь
\newcommand{\sfrac}[2]{{\scriptstyle\frac{#1}{#2}}}  % Очень маленькая дробь
\newcommand{\mfrac}[2]{{\textstyle\frac{#1}{#2}}}    % Небольшая дробь

\newcommand{\fix}[1]{{\color{fixcolor}{#1}}} % \underline
\def\bonus{\t{\red{(*)}}}
\def\ifbonus#1{\ifthenelse{\equal{#1}{}}{}{\bonus}}
\def\smallsquare{$\scalebox{0.5}{$\square$}$}

\newlength{\myItemLength}
\setlength{\myItemLength}{0.3em}
\def\ItemSymbol{\smallsquare}
\def\Item{\vspace*{\myItemLength}\ItemSymbol \ \ }

\newcommand{\LET}{%
  % [line width=0.6pt]
  \begin{tikzpicture}%
  \draw(0.8ex,0) -- (0.8ex,1.6ex);%
  \draw(0,1.6ex) -- (0.8ex,1.6ex);%
  \end{tikzpicture}%
  \hspace*{0.1em}%
}

% Отступы
\def\makeparindent{\hspace*{\parindent}\unskip}
\def\up{\vspace*{-0.5em}}%{\vspace*{-\baselineskip}}
\def\down{\vspace*{0.5em}}
\def\LINE{\vspace*{-1em}\noindent \underline{\hbox to 1\textwidth{{ } \hfil{ } \hfil{ } }}}
\def\BOX#1{\mbox{\fbox{\bf{#1}}}}
\def\Pagebreak{\pagebreak\vspace*{-1.5em}}

% Мелкий заголовок
\newcommand{\THEE}[1]{
  \vspace*{0.5em}
  \noindent{\bf \underline{#1}}%\hspace{0.5em}
  \vspace*{0.2em}
}
% Другой тип мелкого заголовка
\newcommand{\THE}[1]{
  \vspace*{0.5em} $\bullet$
  \noindent{\bf #1}%\hspace{0.5em}
  \vspace*{0.2em}
}

\newenvironment{MyTabbing}{
  \t\bgroup
  \vspace*{-\baselineskip}
  \begin{tabbing}
    aaaa\=aaaa\=aaaa\=aaaa\=aaaa\=aaaa\kill
}{
  \end{tabbing}
  \t\egroup
}

% Код с правильными отступами
\lstnewenvironment{code}{
  \lstset{}
%  \vspace*{-0.2em}
}%
{
%  \vspace*{-0.2em}
}
\lstnewenvironment{codep}{
  \lstset{language=python}
}%
{
}

% Формулы с правильными отступами
\newenvironment{smallformula}{
 
  \vspace*{-0.8em}
}{
  \vspace*{-1.2em}
  
}
\newenvironment{formula}{
 
  \vspace*{-0.4em}
}{
  \vspace*{-0.6em}
  
}

% Большая квадратная скобка
\makeatletter
\newenvironment{sqcases}{%
  \matrix@check\sqcases\env@sqcases
}{%
  \endarray\right.%
}
\def\env@sqcases{%
  \let\@ifnextchar\new@ifnextchar
  \left\lbrack
  \def\arraystretch{1.2}%
  \array{@{}l@{\quad}l@{}}%
}
\makeatother

\DeclareMathOperator{\sort}{sort}

% Определяем основные секции: \begin{Lm}, \begin{Thm}, \begin{Def}, \begin{Rem}
\renewcommand{\qedsymbol}{$\blacksquare$}
\theoremstyle{definition} % жирный заголовок, плоский текст
\newtheorem{Thm}{\underline{Теорема}}[subsection] % нумерация будет "<номер subsection>.<номер теоремы>"
\newtheorem{Lm}[Thm]{\underline{Lm}} % Нумерация такая же, как и у теорем
\newtheorem{Ex}[Thm]{Упражнение} % Нумерация такая же, как и у теорем
\newtheorem{Example}[Thm]{Пример} % Нумерация такая же, как и у теорем
\newtheorem{Solution}[Thm]{Решение}
\newtheorem{Code}[Thm]{Код} % Нумерация такая же, как и у теорем
\theoremstyle{plain} % жирный заголовок, курсивный текст
\newtheorem{Def}[Thm]{Def.} % Нумерация такая же, как и у теорем
\theoremstyle{remark} % курсивный заголовок, плоский текст
\newtheorem{Cons}[Thm]{Следствие} % Нумерация такая же, как и у теорем
\newtheorem{Conj}[Thm]{Гипотеза} % Нумерация такая же, как и у теорем
\newtheorem{Prop}[Thm]{Утверждение} % Нумерация такая же, как и у теорем
\newtheorem{Rem}[Thm]{Замечание} % Нумерация такая же, как и у теорем
\newtheorem{Remark}[Thm]{Замечание} % Нумерация такая же, как и у теорем
\newtheorem{Algo}[Thm]{Алгоритм} % Нумерация такая же, как и у теорем

% Определяем ЗАГОЛОВКИ
\def\SectionName{Дискретная математика}
\def\AuthorName{Илья Дудников}

\newlength{\sectionvskip}
\setlength{\sectionvskip}{0.5em}
\newcommand{\Section}[4][]{
  % Заголовок
  \pagebreak
%  \ifthenelse{\isempty{#1}}{
    \refstepcounter{section}
%  }{}
  \vspace{0.5em}
%  \ifthenelse{\isempty{#1}}{
%    \addtocontents{toc}{\protect\addvspace{-5pt}}%
    \addcontentsline{toc}{section}{\arabic{section}. #2}
%  }{}
  


  % Запомнили название и автора главы
  \gdef\SectionName{#2}
  \gdef\AuthorName{#4}

  % Заголовок страницы
  \lhead{Конспект, \NAME}
  \chead{}
  \rhead{\SectionName}
  \renewcommand{\headrulewidth}{0.4pt}
    
  \lfoot{}
  \cfoot{\thepage\t{/}\pageref*{LastPage}}
  \rfoot{Автор: \AuthorName}
  \renewcommand{\footrulewidth}{0.4pt}
}

\newcommand{\Subsection}[2][]{
  \refstepcounter{subsection}
  \vspace*{1em}
  \ifthenelse{\equal{#1}{}}
    {\addcontentsline{toc}{subsection}{\arabic{section}.\arabic{subsection}. #2}}
    {\addcontentsline{toc}{subsection}{\arabic{section}.\arabic{subsection}. \bonus\,#2}}
  {\color{blue}\bf\large \arabic{section}.\arabic{subsection}. \ifbonus{#1}\,{#2}} 
  \vspace*{0.5em}
  \makeparindent
}
\newcommand{\Subsubsection}[2][]{
  \refstepcounter{subsubsection}
  \vspace*{1em}
  \ifthenelse{\equal{#1}{}}
    {\addcontentsline{toc}{subsubsection}{\arabic{section}.\arabic{subsection}.\arabic{subsubsection}. #2}}
    {\addcontentsline{toc}{subsubsection}{\arabic{section}.\arabic{subsection}.\arabic{subsubsection}. \bonus\,#2}}
  {\color{blue}\bf\large \arabic{section}.\arabic{subsection}.\arabic{subsubsection}. \ifbonus{#1}\,#2}
  \vspace*{0.5em}
  \makeparindent
}

\newcommand{\Header}{
  \pagestyle{empty}
  \renewcommand{\dateseparator}{--}
  \begin{center}
    {\Large\bf 
    Дискретная математика 1 семестр ПИ,\\
    \vspace{0.3em}
     \NAME}\\
    \vspace{0.7em}
    {Собрано {\today} в {\currenttime}}
  \end{center}

  \LINE
  \vspace{0em}

  \renewcommand{\baselinestretch}{0.98}\normalsize
  \tableofcontents
  \renewcommand{\baselinestretch}{1.0}\normalsize
  \pagebreak
}

\newcommand{\BeginConspect}{
  \pagestyle{fancy}
  \setcounter{page}{1}
}

\definecolor{mygray}{rgb}{0.7,0.7,0.7}
\definecolor{ltgray}{rgb}{0.9,0.9,0.9}
\definecolor{fixcolor}{rgb}{0.7,0,0}
\definecolor{red2}{rgb}{0.7,0,0}
\definecolor{dkred}{rgb}{0.4,0,0}
\definecolor{dkblue}{rgb}{0,0,0.6}
\definecolor{dkgreen}{rgb}{0,0.6,0}
\definecolor{brown}{rgb}{0.5,0.5,0}

\newcommand{\green}[1]{{\color{green}{#1}}}
\newcommand{\black}[1]{{\color{black}{#1}}}
\newcommand{\red}[1]{{\color{red}{#1}}}
\newcommand{\dkred}[1]{{\color{dkred}{#1}}}
\newcommand{\blue}[1]{{\color{blue}{#1}}}
\newcommand{\dkgreen}[1]{{\color{dkgreen}{#1}}}

\newcommand{\Mod}[1]{\ (\mathrm{mod}\ #1)}

\DeclareMathOperator{\Real}{Re}
\DeclareMathOperator{\Imag}{Im}
%\DeclareMathOperator{\tg}{tg}
%\DeclareMathOperator{\arctg}{arctg}
\DeclareMathOperator{\const}{const}

\begin{document}

\Header

\BeginConspect

\Section{Теория вероятности}{}{Илья Дудников}

\begin{Thm}[Формула Байеса]
    Пусть $H_1, H_2, ..., H_n$ -- полная группа событий. $A$ -- событие (считаем произошедшим). Тогда
    \[P(H_k | A) = \frac{P(H_k) \cdot P(A|H_k)}{\sum_{i=1}^{n} P(H_i) \cdot P(A|H_i)}\]
\end{Thm}

\begin{proof}
    \[P(H_k|A) = \frac{P(H_k \cap A)}{P(A)} = \frac{P(H_k) \cdot P(A|H_k)}{\sum_{i=1}^{n} P(H_i) \cdot P(A|H_i)}\] 
\end{proof}

\Subsection{Испытания Бернулли}

\begin{Def}
    Обозначим $P_n(m)$ -- вероятность получить $m$ успехов за $n$ испытаний. 
\end{Def} 

\begin{Thm}[Теорема Бернулли]
    Рассмотрим упорядоченный набор: $\underbrace{SSS...S}_n \underbrace{FFF...F}_{n - m}$, где $S$ обозначает успех, а $F$ -- неудачу.
    В силу независимости испытаний, вероятность получить конкретный упорядоченный набор равна $p^m (1 - p)^{n - m}$. Таких наборов, очевидно, $C_n^m$  
\end{Thm}

\begin{Thm}
    $0 \leqslant m_1 \leqslant m_2 \leqslant n$. $P_n(m_1, m_2)$ -- успех наступил от $m_1$ до $m_2$ раз.
    \[P_n(m_1, m_2) = \sum_{i=m_1}^{m_2} C_n^k p^k (1 - p)^{n - k}\] 
\end{Thm}

\begin{Def}
    Наивероятнейшее число событий -- число событий в испытаниях Бернулли с наибольшей вероятностью.
\end{Def}

\begin{Thm}
    Наивероятнейшее число успехов в $n$ испытаниях заключено между числами $np - (1 - p)$ и $np + p$  
\end{Thm}

\begin{proof}
    Рассмотрим следующее соотношение:
    \[\frac{P_n(m)}{P_n(m - 1)} = \frac{C_n^m p^m (1 - p)^{n - m}}{C_n^{m - 1} p^{m - 1} (1 - p)^{n - m + 1}} = \frac{p}{1 - p} \cdot \frac{n! (m - 1)!(n - m + 1)!}{n!m!(n - m)!} = \frac{p}{1 - p} \cdot \frac{n - m + 1}{m}\]

    Отсюда очевидно, что 
    \begin{gather*}
        P_n(m) > P_n(m - 1), m < (n + 1)p \\
        P_n(m) = P_n(m - 1), m = (n + 1)p \\
        P_n(m) < P_n(m - 1), m > (n + 1)p
    \end{gather*}
    Значит, при $m < (n + 1)p \ P_n(m)$ возрастает, при $m > (n + 1)p$ -- убывает. Тогда несложно найти $m$ такое, чтобы $P_n(m)$ было наибольшим:
    \[\begin{cases}
        P_n(m) > P_n(m - 1) \\
        P_n(m + 1) < P_n(m)
    \end{cases} \EQ \begin{cases}
        m < (n + 1)p \\
        m + 1 > (n + 1)p
    \end{cases} \EQ np + p - 1 < m < np + p\]  
\end{proof}

\Pagebreak
\Subsection{Предельные случаи испытаний Бернулли}

Рассмотрим ситуацию, когда вероятность какого-то события уменьшается пропорционально $n$, т.е. $p \thicksim \frac{1}{n}$ 
\begin{Thm}[Теорема Пуассона]
    Пусть $np = \const, \lambda = np$. 
    \[\forall m, \forall \lambda \ \lim_{n \to \infty} P_n(m) = \frac{\lambda^m}{m!}\cdot e^{-\lambda}\]
\end{Thm}

\begin{proof}
    \begin{align*}
        P_n(m) &= C_n^m p^m \cdot (1 - p)^{n - m} = \frac{n!}{m!(n - m)!} \cdot \left(\frac{\lambda}{n}\right)^m \left(1 - \frac{\lambda}{n}\right)^{n - m} = \\
        &= \frac{n(n - 1) ... (n - m) + 1}{m!} \cdot \left(1 - \frac{\lambda}{n}\right)^{n - m} \cdot \left(\frac{\lambda}{n}\right)^m = \\
        &= \frac{\lambda^m}{m!}\left(1 - \frac{\lambda}{n}\right)^n \left(1 - \frac{1}{n}\right) \left(1 - \frac{2}{n}\right)...\left(1 - \frac{m - 1}{n}\right)\left(1 - \frac{\lambda}{n}\right)^{-m} \SO \\
        &\SO \lim_{n \to \infty} P_n(m) = \lim_{n \to \infty} \frac{\lambda^m}{m!} \left(1 - \frac{\lambda}{n}\right)^n = \frac{\lambda^m}{m!}\cdot e^{-\lambda}
    \end{align*}
\end{proof}

\begin{Thm}[Локальная теорема Муавра-Лапласа]
    Пусть $x_n = \frac{m - np}{\sqrt{np(1 - p)}}$. Предположим, что $x_n$ ограничена при $n \to \infty$. Тогда
    \[\sqrt{np(1 - p)} \cdot P_n(m) \thicksim \frac{1}{\sqrt{2\pi}} \cdot e^{-\frac{x_n^2}{2}}\]   
\end{Thm}

\begin{proof}
    Вспомним, что $k! \thicksim \sqrt{2\pi k} \left(\frac{k}{e}\right)^k$. \\
    $n - m = n(1 - p) - x_n\sqrt{np(1 - p)}$. Тогда
    \begin{align*}
        &\sqrt{np(1 - p)}P_n(m) = \sqrt{np(1 - p)} C_n^m p^m (1 - p)^{n - m} = \frac{\sqrt{np(1 - p)} \cdot n!}{m!(n - m)!} \cdot p^m \cdot (1 - p)^{n - m} \\
        &\approx \frac{\sqrt{np(1 - p)} \cdot \sqrt{2\pi n} \left(\frac{n}{e}\right)^n}{\sqrt{2\pi m} \cdot \left(\frac{m}{e}\right)^m \cdot \sqrt{2\pi (n - m)} \cdot \left( \frac{n - m}{e}\right)^{n - m}} \cdot p^m \cdot (1 - p)^{n - m} \\
        &= \frac{\sqrt{np(1 - p)} \cdot \sqrt{n} \cdot n^n}{\sqrt{2\pi} \cdot \sqrt{m} \cdot m^m \cdot (n - m)^{n - m}} \cdot p^m (1 - p)^{n - m} = \frac{1}{\sqrt{2\pi}} \left( \frac{np}{m}\right)^m \cdot \left( \frac{n(1 - p)}{n - m}\right)^{n - m} \sqrt{ \frac{np}{m}} \cdot \sqrt{ \frac{n(1 - p)}{n - m}}\\
    \end{align*} 

    $m = np + x_n\sqrt{np(1 - p)}$
    \begin{align*}
        \frac{m}{np} = 1 + \frac{x_n\sqrt{1 - p}}{\sqrt{np}} &\xrightarrow[n \to \infty]{} 1 \\
        \frac{n - m}{n(1 - p)} = 1 - \frac{x_n\sqrt{p}}{\sqrt{n(1 - p)}} &\xrightarrow[n \to \infty]{} 1
    \end{align*} 

    Пусть, для удобства, $\exp(x) = e^x$. Тогда 
    \begin{gather*}
        \sqrt{np(1 - p)}P_n(m) \approx \frac{1}{\sqrt{2\pi}} \cdot \left(1 + \frac{x_n\sqrt{1 - p}}{\sqrt{np}}\right)^{-m} \left(1 - \frac{x_n\sqrt{p}}{\sqrt{n(1 - p)}}\right)^{-(n - m)} = \\
        = \frac{1}{\sqrt{2\pi}} \exp\left({-m \cdot \ln \left(1 + \frac{x_n\sqrt{1 - p}}{\sqrt{np}}\right) - (n - m) \cdot \ln \left(1 - \frac{x_n\sqrt{p}}{\sqrt{n(1 - p)}}\right)}\right)
    \end{gather*}

    Как мы знаем (откуда?)
    \[\ln(1 + y) \xrightarrow[y \to 0]{}y - \frac{y^2}{2}(1 + O(1))\]
    Следовательно $\sqrt{np(1 - p)}P_n(m) =$ 
    \[
        = \frac{1}{\sqrt{2\pi}} \exp \left(-m \left( \frac{x_n \sqrt{1 - p}}{\sqrt{np}} - \frac{x_n^2}{2np}\right)(1 + O(1)) - (n - m)\left(- \frac{x_n\sqrt{p}}{\sqrt{n(1 - p)}} - \frac{x_n^2p}{2n(1 - p)}\right)(1 + O(1))\right)
    \]
    \begin{gather*}
        x_n \left( \frac{(n - m)\sqrt{p}}{\sqrt{n(1 - p)}} - \frac{m\sqrt{1 - p}}{\sqrt{np}}\right) = \\
        = \frac{x_n}{\sqrt{np(1 - p)}}\left(np(1 - p) - x_n\sqrt{np(1 - p)} p - n(1 - p)\cdot p - x_n\sqrt{np(1 - p)}(1 - p)\right) = \\
        = -x_n^2 (p + (1 - p)) = -x_n^2
    \end{gather*}

    Таким образом:
    \[\sqrt{np(1 - p)}P_n(m) \approx \frac{1}{\sqrt{2\pi}}e^{\left(-x_n^2 + \frac{x_n^2}{2}\right)(1 + O(1))} \approx \frac{1}{\sqrt{2\pi}}e^{- \frac{x_n^2}{2}}\]
\end{proof}

\end{document}