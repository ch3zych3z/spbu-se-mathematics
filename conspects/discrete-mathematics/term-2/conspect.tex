\input{preamble.tex}

\begin{document}
	\Header

	\BeginConspect

	\Section{Кодирование информации}{}{Илья Дудников}

	\Subsection{Задача об оптимальном префиксном коде}

	Пусть $\Lambda$ -- произвольное конечное множество (алфавит), $a \in \Lambda$ -- символы. 
	Пусть $\forall a \in \Lambda \ \exists l(a) \in \N, \exists c(a) = \{0, 1\}^{l(a)}$ -- кодовая последовательность $a$, где $l(a)$ -- длина. 

	Очевидно, условие $\forall a, b \in \Lambda \to (a \neq b \SO c(a) \neq c(b))$ не является достаточным для однозначного распознавания символов.

	\begin{Def}
		Код называется префиксным, если $\forall a, b \in \Lambda \ c(a) = \omega \SO \not\exists m \in \N_0 : c(b) = \omega \gamma$, где $\gamma \in \{0, 1\}^m$  
	\end{Def}

	Пусть $\forall a \in \Lambda$ соответствует вероятность $p(a)$ появления этого символа в сообщении. $\sum_{a \in \Lambda} p(a) = 1$ и считаем $\forall a \in \Lambda \ p(a) > 0$. \\
	Введем дискретную случайную величину $l : \forall a \in \Lambda \ Pr\{l = l(a)\} = p(a)$ -- длина кодовой последовательности символа в сообщении.

	\begin{Def}
		Оптимальным называется префиксный код, минимизирующий математическое ожидание $l : \mathbb{E} l = \sum_{a \in \Lambda} l(a)p(a)$ 
	\end{Def}

	Чем чаще встречается символ, тем короче должна быть кодовая последовательность.

	Почему вообще ОПК существует? Известно, что $\mathbb{E} l \geqslant 1$ (в каждой кодовой последовательности должен быть хотя бы один символ).
	Всегда можно сделать префиксный код, в котором все символы имеют одинаковые длины кодовых последовательностей и эти последовательности различны
	($\forall a \in \Lambda \ l(a) = \lceil\log_2 (|\Lambda|)\rceil$), т.е. префиксный код существует и матожидание длины кодовой последовательности ограничено.

	\begin{Lm}
		Если в некотором коде $C$ существует $x \in \Lambda : c(x) = \omega \alpha$, где $\alpha \in \{0, 1\}$ и при этом $\not\exists y \in \Lambda, y \neq x : c(y) = \omega \gamma$, где $\gamma \in \{0, 1\}^k$ (то есть, если $\omega$ не является началом никакой другой кодовой последовательности, кроме $c(x)$ ),
		то код $C' : c'(x) = \omega, \forall y \in \Lambda, y \neq x \ c' = c(y)$ будет префиксным (по построению и условию леммы) и 
		$\mathbb{E}l' = \mathbb{E}l - p(x)l(x) + p(x)(l(x) - 1) = \mathbb{E} l - p(x) < \mathbb{E} l$. \\ 
		Тогда код $C$ точно не мог быть оптимальным.
	\end{Lm}

	\begin{Lm}[Лемма о кратчайшем префиксе]
		Если в префиксном коде $C \ \exists a, b \in \Lambda, a \neq b : p(a) < p(b), l(a) < l(b)$, то такой код не оптимален.
	\end{Lm}

	\begin{proof}
		Проверим, что для кода $C'$, в котором $c'(a) = c(b), c'(b) = c(a)$ и $\forall x \in \Lambda : x \neq a, x \neq b \ c'(x) = c(x)$ верно $\mathbb{E}l - \mathbb{E}l' > 0$.
		\[\mathbb{E} l - \mathbb{E} l' = p(a) l(a) + p(b) l(b) - p(a) l(b) - p(b) l(a) = (p(a) - p(b))(l(a) - l(b)) > 0\] 
	\end{proof}

	\begin{Lm}[Лемма о соседстве самых редких символов]
		Пусть $a, b \in \Lambda, a \neq b$ -- символы с намиеньшими вероятностями ( $\forall x \in \Lambda \ p(x) \geqslant p(b) \geqslant p(a)$ ). 
		Тогда $\exists $ ОПК $: c(a) = \omega 0, c(b) = \omega 1$, где $\exists k \in \N_0 : \omega \in \{0, 1\}^k$ и это самые длинные кодовые последовательности.
	\end{Lm}

	\begin{proof}
		Пусть $C'$ -- ОПК. По лемме о кратчайшем префиксе $a$ и $b$ имеют самые длинные кодовые последовательности в $C' : \forall x \in \Lambda, x \neq a, x \neq b \ l'(a) \geqslant l'(b) \geqslant l'(x)$
		
		Если $c(a) = \omega \gamma, \omega \in \{0, 1\}^{l'(b)}, \gamma \in \{0, 1\}^{l'(a) - l'(b)}$ и $\omega$ не является началом никакой кодовой последовательности
		(т.к. остальные кодовые последовательности не длиннее $\omega$ и $\not\exists $ символа с кодовой последовательностью $\omega$ в силу префиксности $C'$ ) $\SO$ 
		можно сократить кодовую последовательность $a$, создав более оптимальный код (?!).

		$\SO$ из оптимальности $C'$ следует $l(a) = l(b)$. Пусть $c'(b) = \omega 1$, тогда, если 
		$\exists x \in \Lambda : c'(x) = \omega 0$, то построим ОПК $C : c(a) = c'(x), c(x) = c'(a), \forall z \in \Lambda, z \neq a, z \neq x \ c(z) = c'(z)$. \\
		Если $\not\exists x \in \Lambda : c'(x) = \omega0$, то построим ОПК $C : c(a) = \omega0, \forall z \in \Lambda, z \neq a \ c(z) = c'(z)$.
	\end{proof}

	\begin{Lm}[Лемма об ОПК для расширенного алфавита]
		Пусть $a, b \in \Lambda, a \neq b$ -- символы с намиеньшими вероятностями.
		$\Lambda' = \Lambda \setminus \{a, b\} \cup \{\underbrace{ab}\}$, где $\underbrace{ab} \notin \Lambda$, $p(\underbrace{ab}) = p(a) + p(b)$.
		Пусть $C'$ -- ОПК для $\Lambda', c'(\underbrace{ab}) = \omega$. Тогда для $\Lambda$ код $C : c(a) = \omega 0, c(b) = \omega 1, \forall x \in \Lambda, x \neq a, x \neq b \ c(x) = c'(x)$ будет ОПК.
	\end{Lm}

	\begin{proof}
		$l(a) p(a) + l(b)p(b) = (l'(\underbrace{ab}) + 1)(p(a) + p(b)) = l'(\underbrace{ab})p(\underbrace{ab}) + p(\underbrace{ab})$. Тогда $\mathbb{E} l = \mathbb{E} l' + p(\underbrace{ab})$. \\
		Пусть $\overline{C}$ -- ОПК для $\Lambda$ и $\mathbb{E} \overline{l} < \mathbb{E} l$. По лемме о соседстве: $\overline{c}(a) = \gamma 0, \overline{c}(b) = \gamma 1$.
		Построим $\overline{C}'$ для $\Lambda' : \overline{c}'(\underbrace{ab}) = \gamma$ и $\forall x \in \Lambda, x \neq a, x \neq b \ \overline{c}'(x) = \overline{c}(x)$. 

		$\overline{C}'$ -- префиксный? По Лемме о кратчайшем префиксе $\not\exists $ символа с кодовой последовательностью длины $> \overline{l}(a)$.
		Никакой символ не мог иметь кодовую последовательность $\gamma$, т.к. $\overline{C}$ префиксный. Единственные две последовательности длины $\overline{l}(a)$, начинающиеся на $\gamma$ -- это коды $a$ и $b$.
		Но их нет в $\Lambda'$. При этом $\mathbb{E} \overline{l} = \mathbb{E} \overline{l}' = p(\underbrace{ab})$.
		По предположению $\mathbb{E}l' + p(\underbrace{ab}) = \mathbb{E}l > \mathbb{E} \overline{l} = \mathbb{E} \overline{l}' + p(\underbrace{ab})$ (?!) оптимальности $C'$ 
		$\SO \mathbb{E} \overline{l} \geqslant \mathbb{E}l$, но т.к. $\overline{C}$ -- ОПК $\SO \mathbb{E} \overline{l} = \mathbb{E} l$ и $C$ -- ОПК.
	\end{proof}

	Задача: нужно построить ОПК на алфавите $\Lambda, |\Lambda| = M$. По лемме об ОПК для расширенного алфавита задачу построения ОПК можно свести к такой же задаче, но с исходным алфавитом с числом букв на единицу меньше,
	и с набором вероятностей, получющимся из первоначального сложением двух наименьших вероятностей. \\
	Уменьшаем пока не получится алфавит из двух букв. ОПК для алфавита из 2-х букв -- $\{0, 1\}$.

	Строже: $\Lambda_0 := \Lambda$. $\forall k \in 0...(M - 3)$ берем $a_k, b_k \in \Lambda_k : \forall x \in \Lambda_k, x \neq a_k, x \neq b_k \ p(a_k) \leqslant p(b_k) \leqslant p(x)$ 
	и построим $\Lambda_{k + 1} = \Lambda_{k} \setminus \{a_k, b_k\} \cup \{\underbrace{a_kb_k}\}...$
	
	Для $\Lambda_{M - 2} = \{a_{M - 2}, b_{M - 2}\}$ оптимальным будет код $C_{M - 2} : c_{M - 2} (a_{M - 2}) = 0, c_{M - 2}(b_{M - 2}) = 1$,
	т.к. для него $\mathbb{E} l_{M - 2} = 1$. \\
	Теперь для $k \in 1 ... (M - 2)$ есть ОПК $C_k$ для $\Lambda_k$. По лемме об ОПК для расширенного алфавита строится ОПК $C_{k - 1}$ для $\Lambda_{k - 1}$ такой, что 
	$c_{k - 1} (a_{k - 1}) = c_k(a_{k - 1}b_{k - 1})0, c_{k - 1}(b_{k - 1}) = c_k (a_{k - 1}b_{k - 1})1, \forall x \in \Lambda_k, x \neq \underbrace{a_{k - 1}b_{k - 1}} \ c_{k - 1}(x) = c_k(x)$. \\
	Выполняем, пока не получится $C_0$ -- ОПК для $\Lambda_0 = \Lambda$.

	\begin{Example}
		$\Lambda_0 = \{a, b, c, d, e, f, g\}, p(a) = 0.13, p(b) = 0.08, p(c) = 0.25, p(d) = 0.18, p(e) = 0.03, p(f) = 0.12, p(g) = 0.21$. \\
		$a_0, = e, b_0 = b, \Lambda_1 = \{a, \underbrace{e, b}, c, d, f, g\}, p(a) = 0.13, p(\underbrace{eb}) = 0.11, p(c) = 0.25, p(d) = 0.18, p(f) = 0.12, p(g) = 0.21$. \\
		$a_1 = \underbrace{eb}, b_1 = f, \Lambda_2 = \{a, \underbrace{ebf}, c, d, g\}, p(a) = 0.13, p(\underbrace{ebf}) = 0.23, p(c) = 0.25, p(d) = 0.18, p(g) = 0.21$. \\ 
		$a_2 = a, b_2 = d, \Lambda_3 = \{\underbrace{ad}, \underbrace{ebf}, c, g\}, p(\underbrace{ad}) = 0.31, p(\underbrace{ebf}) = 0.23, p(c) = 0.25, p(g) = 0.21$. \\
		$a_3 = g, b_3 = \underbrace{ebf}, \Lambda_4 = \{\underbrace{ad}, \underbrace{gebf}, c\}, p(\underbrace{ad}) = 0.31, p(\underbrace{gebf}) = 0.44, p(c) = 0.25$.
		$a_4 = c, b_4 = \underbrace{ad}, \Lambda_5 \{\underbrace{cad}, \underbrace{gebf}\}, p(\underbrace{cad}) = 0.56, p(\underbrace{gebf}) = 0.44$.
		Тогда $c_5(\underbrace{gebf}) = 0, c(\underbrace{cad}) = 1$.
		
		Теперь раскрываем алфавит обратно:
		
		$c_4 (\underbrace{gebf}) = 0, c_4(c) = 10, c_4(\underbrace{ad}) = 11$. \\
		$c_3(g) = 00, c_3(\underbrace{ebf}) = 01, c_3(c) = 10, c_3(\underbrace{ad}) = 11$. \\
		$c_1(g) = 00, c_1(\underbrace{eb}) = 010, c_1(f) = 011, c_1(c) = 10, c_1(a) = 110, c_1(d) = 111$. \\
		$c_0(g) = 00, c_0(e) = 0100, c_0(b) = 0101, c_0(f) = 011, c_0(c) = 10, c_0(a) = 110, c_0(d) = 111$.
		
	\end{Example}

	\Subsection{Неравенство Крафта}

	Пусть задан набор длин $l_1, ..., l_m$, не все обязательно различны. Может ли такой набор оказаться набором длин некоторого префиксного кода.

	\begin{Thm}
		Для того, чтобы набор длин $l_1, ..., l_m$ мог быть набором длин кодовых последовательностей некоторого ПК для алфавита из $m$ символов необходимо и достаточно, чтобы $\sum_{i = 1}^m 2^{-l_i} \leqslant 1$. 
	\end{Thm}
\end{document}